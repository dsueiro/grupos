\chapter{El teorema de Maschke}

\index{Álgebra!de grupo}
Sea $K$ un cuerpo y sea $G$ un grupo finito. El \textbf{álgebra de grupo} $K[G]$ es el
$K$-espacio vectorial con base $\{g:g\in G\}$ con la estructura de álgebra dada
por el producto
\[
	\left(\sum_{g\in G}\lambda_gg\right)\left(\sum_{h\in G}\mu_hh\right)
	=\sum_{g,h\in G}\lambda_g\mu_h(gh).
\]
Observemos que el álgebra $K[G]$ es conmutativa si y sólo si $G$ es abeliano. 
Además $\dim K[G]=|G|$. 

\begin{example}
Sea $G=\{1,g,g^2\}$ el grupo cíclico de orden tres y sea $A=\C[G]$ el álgebra (compleja) del grupo $G$. Si 
$\alpha=a_11+a_2g+a_3g^2$ y $\beta=b_11+b_2g+b_3g^2\in A$, donde $a_1,a_2,a_3,b_1,b_2,b_3\in\C$, 
entonces la suma de $A$ está dada por
\begin{gather*}
\alpha+\beta=(a_1+b_1)1+(a_2+b_2)g+(a_3+b_3)g^2
\shortintertext{y el producto por}
\alpha\beta=(a_1b_1+a_2b_3+a_3b_2)1+(a_1b_2+a_2b_1+a_3b_3)g+(a_1b_3+a_2b_2+a_3b_1)g^2.
\end{gather*}
\end{example}

\index{Ideal!de aumentación}
Si $G$ es un grupo finito no trivial,  
entonces $K[G]$ posee ideales propios no triviales. 
Esto es porque el conjunto 
\[
	I(G)=\left\{\sum_{g\in G}\lambda_gg\in K[G]:\sum_{g\in G}\lambda_g=0\right\}
\]
es un ideal propio y no nulo de $K[G]$ (pues $\dim I(G)=\dim K[G]-1$). Este
conjunto se conoce como el \textbf{ideal de aumentación} de $K[G]$.

\begin{exercise}
	Sea $G=C_n$ el grupo ciclico de orden $n$ (escrito multiplicativamente).
	Demuestre que $K[G]\simeq K[X]/(X^n-1)$. 
\end{exercise}

\begin{proposition}
    Si $G$ es un grupo finito no trivial, entonces $K[G]$ tiene divisores de cero.	
\end{proposition}

\begin{proof}
    Sea $g\in G\setminus\{1\}$ y sea $n$ el orden de $g$. Para ver que $K[G]$ tiene divisores
    de cero alcanza con observar que 
    $(1-g)(1+g+\cdots+g^{n-1})=0$. 
\end{proof}

Si $A$ es un álgebra, entonces $\mathcal{U}(A)$ es el grupo de unidades del anillo $A$. 
La proposición que sigue se conoce como la propiedad universal del álgebra de grupo.

\begin{proposition}
Sean $A$ un álgebra y $G$ un grupo finito. Si $f\colon G\to\mathcal{U}(A)$ es un morfismo de grupos, entonces
existe un único morfismo $\varphi\colon K[G]\to A$ de álgebras tal que la restricción
$\varphi|_G$ de $\varphi$ al grupo $G$ es igual a $f$, es decir 	$\varphi|_G=f$. 
\end{proposition}

\begin{proof}
Como $G$ es base de $K[G]$, puede verificarse que 
el morfismo $\varphi$ de álgebras 
queda unívocametne determinado por 
\[
\varphi\left(\sum_{g\in G}\lambda_gg\right)=\sum_{g\in G}\lambda_gf(g).\qedhere
\]	
\end{proof}

La proposición anterior nos dice que si $G$ es un grupo finito y $A$ es un álgebra, 
para definir un morfismo de álgebras $K[G]\to A$ 
alcanza con tener un morfismo de grupos $G\to\mathcal{U}(A)$.  

\begin{exercise}
Sea $M$ un módulo. 
Si $p\colon M\to M$ es un morfismo tal que $p^2=p$, entonces 
$M=\ker p\oplus p(M)$.
\end{exercise}

% \begin{proof}
% 	Como $p$ es un morfismo, $\ker p$ y $p(M)$ son submódulos de $M$. 
% 	Para ver que $M=\ker p+p(M)$ alcanza con observar que todo $m\in M$ puede escribirse como
% 	$m=(m-p(m))+p(m)$ 
% 	y que $m-p(m)\in\ker p$ pues 
% 	\[
% 	p(m-p(m))=p(m)-p^2(m)=p(m)-p(m)=0. 
% 	\]
% 	Veamos ahora que $\ker p\cap p(M)=\{0\}$. Si $m\in\ker p\cap p(M)$, escribimos $m=p(m_1)$ para algún $m_1\in M$. Como entonces 
% 	$0=p(m)=p^2(m_1)=m_1$, se concluye que $m=0$.   
% \end{proof}

\index{Proyección}   
Recordemos que 
una \textbf{proyección} (o proyector) de un módulo $M$ es un   
morfismo $p\colon M\to M$ tal que $p^2=p$. 
 
%\begin{lemma}
%Si $A$ es un álgebra y $M$ es un $A$-módulo de dimensión finita tal que
%todo submódulo de $M$ se complementa, entonces $M$ es semisimple.
%\end{lemma}
%
%\begin{proof}
%Procederemos por inducción en $\dim M$. Si $M=\{0\}$ el resultado es trivial. Si $M\ne\{0\}$, 
%sea $S$ un submódulo no nulo de $M$ de dimensión minimal. En particular, $S$ es simple. Por hipótesis sabemos que existe un submódulo $T$ de $M$ tal que $M=S\oplus T$. Como $\dim T<\dim M$, la hipótesis inductiva implica que $T$ es suma directa de módulos simples. Luego $M$ también lo es. 
%\end{proof}
%
%El lema anterior vale también para módulos arbitrarios sobre anillos. Sin embargo, la demostración 
%requiere el uso del lema de Zorn.
%
%\begin{example}
%Sea $R=M_2(\C)$ y sea $M=\prescript{}{R}R$. Los subconjuntos
%\[
%I=\begin{pmatrix}
%\C&0\\
%\C&0
%\end{pmatrix},\quad
%J=\begin{pmatrix}
%0&\C\\
%0&\C
%\end{pmatrix}
%\]
%son submódulos de $M$ tales que $M\simeq I\oplus J$. Veamos que $M$ es semisimple, es decir que
%$I$ y $J$ son simples. 
%
%Si $S$ es un submódulo no nulo de $I$, sea 
%$\begin{pmatrix}
%a&0\\
%c&0
%\end{pmatrix}\in S$ no nulo. Supongamos que 
%$a\ne 0$, el caso $c\ne 0$ es similar. Entonces
%\[
%\begin{pmatrix}
%a^{-1} & 0\\
%0 & 0\end{pmatrix}
%\begin{pmatrix}
%a&0\\
%c&0
%\end{pmatrix}
%=\begin{pmatrix}
%1&0\\
%0&0
%\end{pmatrix}\in S.
%\]
%Análogamente se demuestra que $\begin{pmatrix}0&0\\1&0\end{pmatrix}\in S$. Luego $S=I$ y entonces $I$ es simple.  
%La misma técnica nos permite demostrar que $J$ es simple.
%\end{example}

%Es importante observar que si $A$ es un álgebra y $M$ es un módulo
%semisimple de dimensión finita, entones $M$ es suma directa de finitos simples. 
%
\begin{theorem}[Maschke]
\index{Teorema!de Maschke}
\index{Maschke!teorema de}
Sea $K$ un cuerpo de característica cero. 
Sea $G$ un grupo finito y sea $M$ un $K[G]$-módulo de dimensión finita. Entonces
$M$ es semisimple.
\end{theorem}

\begin{proof}
Alcanza con demostrar que todo submódulo $S$ de $M$ se complementa. 
Como, en particular, $S$ es un subespacio de $M$, existe un subespacio $T_0$ de $M$ 
tal que $M=S\oplus T_0$ (como espacios vectoriales). Vamos a usar el espacio vectorial
$T_0$ para construir un submódulo $T$ de $M$ que complementa a $S$. Como $M=S\oplus T_0$, 
cada $m\in M$ puede escribirse unívocamente como $m=s+t_0$ para ciertos $s\in S$ y $t_0\in T$. 
Podemos definir entonces la transformación lineal 
\[
p_0\colon M\to S,\quad
p_0(m)=s,
\]
donde $m=s+t_0$ con $s\in S$ y $t_0\in T$. 
Observemos que si $s\in S$, entonces $p_0(s)=s$. En particular, $p_0^2=p_0$ pues
$p_0(m)\in S$. 

El problema 
es que $p_0$ no es, en general, un morfismo de $\C[K]$-módulos. Promediamos
sobre el grupo $G$ para conseguir un morfismo de grupos: Sea 
\[
p\colon M\to S,\quad
p(m)=\frac{1}{|G|}\sum_{g\in G}g^{-1}\cdot p_0(g\cdot m).
\]

Primero demostramos que $p$ es un morfismo de $\C[K]$-módulos. Alcanza con ver que
$p(g\cdot m)=g\cdot p(m)$ para todo $g\in G$ y $m\in M$. En efecto,
\[
p(g\cdot m)=\frac{1}{|G|}\sum_{h\in G}h^{-1}\cdot p_0(h\cdot (g\cdot m))
=\frac{1}{|G|}\sum_{h\in G}(gh^{-1})\cdot p_0(h\cdot m)=g\cdot p(m).
\]

Veamos ahora que $p(M)=S$. La inclusión $\subseteq$ es trivial, pues $S$ es un submódulo de $M$ 
y además $p_0(M)\subseteq S$. Recíprocamente, si $s\in S$, entonces $g\cdot s\in S$, pues
$S$ es un submódulo. Luego 
$s=g^{-1}\cdot (g\cdot s)=g^{-1}\cdot p_0(g\cdot s)$ y en consecuencia
\[
s=\frac{1}{|G|}\sum_{g\in G}g^{-1}\cdot (g\cdot s)=\frac{1}{|G|}\sum_{g\in G}g^{-1}\cdot (p_0(g\cdot s))=p(s).
\]
Como $p(m)\in S$ para todo $m\in M$, entonces $p^2(m)=p(m)$, es decir que $p$ es un proyector en $S$. Luego $S$ se complementa en $M$, es decir $M=S\oplus\ker(p)$.
\end{proof} 

La misma demostración del teorema de Maschke vale para el álgebra de grupo real o racional. 
La descomposición de un módulo sobre el álgebra de grupo dependerá
fuertemente del cuerpo sobre el que se trabaje. 

\begin{example}
Sea $G=\langle g\rangle$ el grupo cíclico de orden cuatro y sea $\rho_g=\begin{pmatrix}
0&-1\\
1&0\end{pmatrix}$. 
Sea $M=\C^{2\times 1}$ con la estructura de $\C[G]$-módulo dada por 
\[
g\cdot\begin{pmatrix}u\\v\end{pmatrix}
%\begin{pmatrix}0&-1\\1&0\end{pmatrix}\begin{pmatrix}u\\v\end{pmatrix}
=\begin{pmatrix}-v\\u\end{pmatrix},
\]
es decir, si $a,b,c,d\in\C$, entonces 
\[
(a1+bg+cg^2+dg^3)\cdot\begin{pmatrix}u\\v\end{pmatrix}
=\begin{pmatrix}
(a-d)u+(c-b)v\\
(1-b)u+(a-d)v
\end{pmatrix}.
\]
Sabemos por el teorema de Maschke que $M$ es semisimple. Veamos cómo descomponer el módulo $M$ como suma directa de simples. 
Como $\dim M=2$, tendremos que $M$ es suma directa de dos submódulos de dimensión uno. 
Observemos que si $S$ es un submódulo tal que $\{0\}\subsetneq S\subsetneq M$, 
entonces $\dim S=1$. Además 
\[
S=\left\{\lambda\begin{pmatrix}
u_0\\
v_0
\end{pmatrix}:\lambda\in\C\right\}
\text{ es un submódulo de $M$}
\Longleftrightarrow
\begin{pmatrix}
u_0\\
v_0
\end{pmatrix}
\text{ es autovector de $\rho_g$}.
\]
Como la matriz $\rho_g$ tiene polinomio característico $X^2+1$, se sigue 
que  
$\begin{pmatrix}
i\\
1\end{pmatrix}$ es autovector de $\rho_g$ de autovalor $-i$ y que
$\begin{pmatrix}
-i\\
1\end{pmatrix}$ es autovector de autovalor $i$. 
Luego $M$ se descompone en suma directa de simples como 
\[
M=\C\begin{pmatrix}
i\\
1\end{pmatrix}
\oplus
\C
\begin{pmatrix}
-i\\
1\end{pmatrix}
\]
\end{example}

Observar que en ejemplo anterior pudimos descomponer a la matriz $\rho_g$ 
gracias a la existencia de autovectores, algo 
que no pasaría si consideramos módulos sobre el álgebra de grupo real.   

\begin{example}
Sea $G=\langle g\rangle$ el grupo cíclico de orden cuatro y sea $\rho_g=\begin{pmatrix}
0&-1\\
1&0\end{pmatrix}$. 
Sea $M=\R^{2\times 1}$ con la estructura de $\R[G]$-módulo dada por 
\[
g\cdot\begin{pmatrix}u\\v\end{pmatrix}
=\begin{pmatrix}-v\\u\end{pmatrix}.
\]
Tal como hicimos en el ejemplo anterior, 
como $\dim M=2$, si $S$ es un submódulo de $M$ tal que $\{0\}\subsetneq S\subsetneq M$, entonces $\dim S=1$. 
Pero como $\rho_g$ no tiene autovectores reales, $M$ no tendrá submódulos de dimensión uno.  
En consecuencia, $M$ es simple como $\R[G]$-módulo. 
\end{example}

Es posible dar una versión multiplicativa del teorema de Maschke. 

Un grupo $G$ \textbf{actúa por automorfismos} en $A$ si existe
un morfismo de grupos $G\to\Aut(A)$, es decir que se tiene una acción de $G$ en $A$ 
tal que $g\cdot 1_A=1_A$ y $g\cdot (ab)=(g\cdot a)(g\cdot b)$ para todo $g\in G$ y $a,b\in A$. 

\begin{theorem}
Sea $K$ un grupo finito de orden $m$ que actúa por automorfismos en $V=U\times W$, donde $W$ es un subgrupo de $V$ y 
$U$ es un subgrupo de $V$ abeliano y $K$-invariante. Si la función $u\mapsto u^m$ es biyectiva en $U$, 
entonces existe un subgrupo normal $K$-invariante $N$ de $V$ tal que $V=U\times N$.
\end{theorem}

\begin{proof}
Sea $\theta\colon U\times W\to U$, $(u,w)\mapsto u$. Entonces $\theta$ es un morfismo de grupos tal que 
$\theta(u)=u$ para todo $u\in U$. Como $U$ es $K$-invariante, entonces
$k^{-1}\cdot \theta(k\cdot v)\in U$ para todo $k\in K$ y $v\in V$. 
Como además $K$ es finito y $U$ es abeliano, 
queda bien definida la función
\[
\varphi\colon V\to U,\quad 
v\mapsto \prod_{k\in K}k^{-1}\cdot \theta(k\cdot v).
\]
Veamos que $\varphi$ es un morfismo de grupos. Si $x,y\in V$, entonces
\begin{align*}
    \varphi(xy) &= \prod_{k\in K}k^{-1}\cdot \theta(k\cdot (xy))\\
    &= \prod_{k\in K}k^{-1}\cdot (\theta(k\cdot x)\theta(k\cdot y))\\
%   &= \prod_{k\in K}(k^{-1}\cdot (\theta(k\cdot x))(k^{-1}\cdot \theta(k\cdot y))\\
    &= \prod_{k\in K}k^{-1}\cdot \theta(k\cdot x) \prod_{k\in K}k^{-1}\cdot \theta(k\cdot y)=\varphi(x)\varphi(y),
\end{align*}
pues $U$ es abeliano y $K$ actúa por automorfismos en $V$. 

Vamos a demostrar ahora que $N=\ker\varphi$ es $K$-invariante. Si $l\in K$, entonces
\begin{align*}
l^{-1}\cdot\varphi(l\cdot x)&=l^{-1}\cdot\left(\prod_{k\in K}k^{-1}\cdot \theta(k\cdot (l\cdot x))\right)=\prod_{k\in K}(kl)^{-1}\cdot\theta( (kl)\cdot x)=\varphi(x),
\end{align*}
pues $kl$ recorre todos los elementos de $K$ si $k$ recorre todos los elementos de $K$.

Nos falta demostrar que $V$ es el producto directo de $U$ y $N$. 
Veamos primero que $U\cap N=\{1\}$. Si $u\in U$, entonces $k\cdot u\in U$ para todo $k\in K$, lo 
que implica que $k^{-1}\cdot\theta(k\cdot u)=k^{-1}\cdot (k\cdot u)=u$. Luego $\varphi(u)=u^m$. Como por hipótesis 
esta función es biyectiva, se concluye que $U\cap N=U\cap\ker\varphi=\{1\}$. Veamos ahora que $V\subseteq UN$, ya que
la otra inclusión es trivial. Como $N=\ker\varphi$, entonces 
\[
\varphi(V)\subseteq U=\varphi(U)=\varphi(U)\varphi(N)=\varphi(UN) 
\]
y luego $V\subseteq (UN)N=UN$. 
% Si $v\in V$, escribimos $v=uw$ para $u\in U$ y $w\in W$. Como 
% $\varphi(v)=\varphi(uw)=\varphi(u)\varphi(w)=u\varphi(w)\in U$, entonces $v\in UN$. 
Luego $V$ es el producto directo de $U$ y $N$, pues $N$ es normal en $V$.
\end{proof}

\begin{corollary}
    Sean $p$ un primo, $K$ un grupo finito de orden coprimo con $p$ y $V$ un $p$-grupo elemental abeliano. 
    Si $K$ actúa por automorfismos en $V$ y $U$ es un subgrupo $K$-invariante de $V$, 
    existe un subgrupo $K$-invariante $N$ de $V$ tal que $V=U\times N$.
\end{corollary}

\begin{proof}
Sea $m=|K|$. Como $m$ y $|U|$ son coprimos, la función 
$u\mapsto u^m$ es biyectiva en $U$. Como $V$ es un espacio
vectorial sobre el cuerpo $\Z/p$, tenemos que
$V=U\times W$ para algún subgrupo $W$ de $V$. El corolario se obtiene
entonces al aplicar el teorema anterior.
\end{proof}

Supongamos que $G$ es un grupo finito. 
Sabemos por el teorema de Maschke que $\C[G]$ es un
álgebra semisimple. Por el teorema de Mollien, existe $r\in\N$ y
existen $n_1,\dots,n_r\in\N$ tales que 
\[
	\C[G]\simeq \prod_{i=1}^r M_{n_i}(\C),
\]
donde $r$ es la cantidad de módulos simples de $\C[G]$. 
Además 
\[
	|G|=\dim\C[G]=\sum_{i=1}^r n_i^2.
\]
Dado que $\C$ es un $\C[G]$-módulo de dimensión uno, es simple. 
Sin perder generalidad podemos suponer entonces
que $n_1=1$. 

\begin{theorem}
	Un grupo finito tiene tantas clases de isomorfismo de simples como clases
	de conjugación.
\end{theorem}

\begin{proof}
    Sea $G$ un grupo finito. Como $\C[G]\simeq\prod_{i=1}^rM_{n_i}(\C)$ 
    por el teorema de Wedderburn,
	entonces
	\[
		Z(\C[G])\simeq\prod_{i=1}^rZ(M_{n_i}(\C))\simeq\C^r.
	\]
	En particular, $\dim Z(\C[G])=r$. Por otro lado, si $\alpha=\sum_{g\in
	G}\lambda_gg\in Z(\C[G])$, entonces $h^{-1}\alpha h=\alpha$ para todo $h\in
	G$. Esto implica que
	\[
		\sum_{g\in G}\lambda_{hgh^{-1}}g=
		\sum_{g\in g}\lambda_g h^{-1}gh=\sum_{g\in G}\lambda_gg.
	\]
	Luego $\lambda_{g}=\lambda_{hgh^{-1}}$ para todo $g,h\in G$. Una base para
	$Z(\C[G])$ está dada entonces por los elementos de la forma
	\[
		\sum_{g\in K}g,
	\]
	donde $K$ es una clase de conjugación de $G$. Luego $\dim Z(\C[G])$ es
	igual a la cantidad de clases de conjugación de $G$.
\end{proof}

\begin{corollary}
    Si $G$ es un grupo finito de orden $n$ con $k$ clases de conjugación y 
    $m=(G:[G,G])$, entonces $n+3m\geq 4k$. 
\end{corollary}

\begin{proof}
    Sabemos que $G$ tiene $k$ clases de isomorfismos de módulos simples y que exactamente 
    $m$ son de dimensión uno. Luego $n=\sum_{i=1}^kn_i^2\geq m+4(m-k)$. 
\end{proof}

\begin{example}
    Como el grupo $C_4$ cíclico de orden cuatro es abeliano, 
    se tiene que $\C[C_4]\simeq\C\times\C\times\C\times\C$ como álgebras. 
\end{example}

\begin{example}
    Vimos en el ejemplo \ref{exa:S3deg2} 
    que $\Sym_3$ tiene una representación irreducible de grado dos. Como
    $6=1+n_2^2+\cdots+n_k^2$, se concluye que $k=3$ y $n_2=1$. Alternativamente podríamos haber
    obtenido $k=3$ al observar que $\Sym_3$ tiene tres clases de conjugación, de donde se sigue
    inmediatamente que $n_1=n_2=1$ y $n_3=2$. En conclusión, 
    \[
    \C[\Sym_3]\simeq \C\times\C\times M_2(\C)
    \]
    como álgebras. 
\end{example}

\begin{proposition}
	\label{pro:nunca_SS}
	Sea $K$ un cuerpo. 
	Si $G$ es un grupo infinito, entonces $K[G]$ no es semisimple.
\end{proposition}

\begin{proof}
	Sea $A=K[G]$ y supongamos que $A$ es semisimple.  Si $I$ es el ideal de
	aumentación de $A$, existe un ideal no nulo $J$ de $A$ tal que $A=I\oplus
	J$. Existen entonces $e\in I$, $f\in J$ tales que $1=e+f$. Si
	$x\in I$, entonces $x=xe+xf$ y luego $xf=x-xe\in I\cap J=\{0\}$. Como
	entonces $x=xe$ para todo $x\in I$, en particular $e=e^2$. Análogamente
	vemos que $f^2=f$. Además $ef=0$ pues $ef\in I\cap J=\{0\}$.  Como $I$
	es el ideal de aumentación y $If=(Ae)f=A(ef)=0$, se concluye que $(g-1)f=0$
	para todo $g\in G$ pues $g-1\in I$. Si suponemos que $f=\sum_{h\in
	G}\lambda_hh$, entonces 
	\[
	f=gf=\sum_{h\in G}\lambda_h(gh)=\sum_{h\in
	G}\lambda_{g^{-1}h}h.
	\]
	Luego $\lambda_h=\lambda_{g^{-1}h}$ para todo $g,h\in G$, una contradicción
	pues como $f\ne 0$ la suma que define a $f$ es infinita. 
\end{proof}
