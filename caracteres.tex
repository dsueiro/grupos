\chapter{Teoría de caracteres}

\begin{definition}
  \index{Carácter!de una representación}
  Sea $\phi\colon G\to\GL(V)$ una representación. El \textbf{carácter} de
  $\phi$ es la función $\chi_\phi\colon G\to\C$, $\chi_\phi(g)=\trace(\phi_g)$.
  Si $\phi$ es irreducible, $\chi_\phi$ se dice un \textbf{carácter
  irreducible}. El \textbf{grado} de $\chi_{\phi}$ es el número
  $\deg\chi_\phi=\deg\phi=\chi_{\phi}(1)=\dim V$. 
\end{definition}

\begin{proposition}
  \label{pro:chi(1)}
  Sea $\phi$ una representación con carácter $\chi$ y sea $g\in G$.
  Valen las siguientes afirmaciones:
  \begin{enumerate}
    \item $\chi(1)=\deg\phi$. 
    \item $\chi(g)=\chi(hgh^{-1})$ para todo $h\in G$.
    \item $\chi(g)$ es suma de $\chi(1)$ raíces $|g|$-ésimas de la unidad.
    \item $\chi(g^{-1})=\overline{\chi(g)}$.
    \item $|\chi(g)|\leq\chi(1)$.
  \end{enumerate}
\end{proposition}

\begin{proof}
  La primera propiedad es evidente pues $\phi_1=\id$. La segunda:
  \begin{align*}
    \chi(hgh^{-1})=\trace\phi_{hgh^{-1}}=\trace(\phi_h\phi_g\phi_h^{-1})=\trace\phi_g=\chi(g),
  \end{align*}
  pues $\trace(AB)=\trace(BA)$ para todo $A,B$. La tercera es fácil pues
  la traza de $\phi_g$ es la suma de los autovalores de $\phi_g$, que son
  raíces del polinomio $X^{|g|}-1$. Para demostrar la cuarta afirmación
  supongamos que $\chi(g)=\lambda_1+\cdots+\lambda_k$, donde los $\lambda_j$
  son raíces de la unidad. Entonces 
  \[
    \overline{\chi(g)}
    =\sum_{j=1}^k\overline{\lambda_j}=\sum_{j=1}^k\lambda_j^{-1}=\trace(\phi_{g^{-1}})=\trace\phi_{g^{-1}}=\chi(g^{-1}).
  \]
  La última afirmación es evidente pues $\chi(g)$ es suma de raíces de la
  unidad.
\end{proof}


Si $\chi$ y $\psi$ son caracteres de $G$, en particular son funciones $G\to\C$ y podemos entonces
definir suma, producto y producto por escalares como
\[
(\chi+\psi)(g)=\chi(g)+\psi(g),\quad
(\chi\psi)(g)=\chi(g)\psi(g),\quad
(\lambda\chi)(g)=\lambda\chi(g)
\]
para $\lambda\in\C$. Sin embargo, estas funciones no necesariamente dan caracteres. 

\begin{theorem}
Sea $G$ un grupo finito. Los caracteres irreducibles de $G$ son linealmente independientes.
\end{theorem}

\begin{proof}
Sean $S_1,\dots,S_k$ las clases de isomorfismos de los $\C[G]$-módulo simples y
sea $f\colon \C[G]\to M_{n_1}(\C)\times\cdots\times M_{n_k}(\C)$ el isomorfismo del teorema de Wedderburn. 
Para cada $j$ tenemos $n_j=\dim S_j$, pues $M_{n_j}(\C)\simeq S_j\oplus\cdots\oplus S_j$ ($n_j$-veces). 
Para cada $j\in\{1,\dots,k\}$ sea 
$e_j=f^{-1}(I_j)$, donde $I_j$ la matriz identidad de $M_{n_j}(\C)$. 
Supongamos que $\Irr(G)=\{\chi_1,\dots,\chi_k\}$. Vamos a demostrar que
\[
\chi_i(e_j)=\begin{cases}
\dim S_i & \text{si $i=j$},\\
0 & \text{en otro caso,}
\end{cases}
\]
para todo $i,j\in\{1,\dots,k\}$.
Cada $\chi_j(g)$ es la traza de la restricción de la acción de $g$ al simple $S_j$. En particular, 
como $e_ie_j=0$ si $i\ne j$, tenemos 
$\chi_i(e_j)=0$ si $i\ne j$. 
Como además $e_j$ actúa por la identidad en $S_j$, tenemos $\chi_j(e_j)=\trace(I_j)=\dim S_j$. 

Sean $\lambda_1,\dots,\lambda_k\in\C$ tales que $\sum_{i=1}^k\lambda_i\chi_i=0$. Al evaluar esta
expresión en cada $e_j$ vemos que $\lambda_j=0$ para todo $j$. 
\end{proof}

\begin{proposition}
    Si $U$ y $V$ son $\G[G]$-modulos, entonces $\chi_{U\oplus V}=\chi_U+\chi_V$.
\end{proposition}

\begin{proof}
    Sea $\{u_1,\dots,u_r\}$ una base de $U$ y sea $\{v_1,\dots,v_s\}$ una base de $V$. Entonces 
    $\{u_1,\dots,u_r,v_1,\dots,v_s\}$ es una base de $U\oplus V$. Si $g\in G$, en esta
    base 
    \[
    \rho_g=\begin{pmatrix}
      \rho_g|_U & *\\
      0 & \rho_g|_V
    \end{pmatrix}.
    \]
    Luego $\chi_{U\oplus V}(g)=\trace(\rho_g)=\trace\rho_g|_U+\trace\rho_g|_V=\chi_U(g)+\chi_V(g)$.
\end{proof}

\begin{theorem}
  Sea $G$ un grupo finito. Si $S_1,\dots,S_k$ son los representantes 
  de las clases de isomorfismos de los $\C[G]$-módulos simples y 
  $V=a_1S_1\oplus\cdots\oplus a_kS_k$, entonces
  \[
  \chi_V=a_1\chi_1+\cdots+a_k\chi_k.
  \]
  En particular, si $U$ y $V$ son $\C[G]$-módulos, 
  entonces $U\simeq V$ si y sólo si $\chi_U=\chi_V$.
\end{theorem}

\begin{proof}
    La primera afirmación se obtiene del lema anterior. 
    
    Supongamos que $U\simeq V$, es decir que existe un isomorfismo $f\colon U\to V$ de $\C[G]$-módulos. Si $\rho\colon G\to\GL(U)$ 
    es la representación que corresponde al módulo $U$ y $\psi\colon G\to\GL(V)$ es la que corresponde al módulo $V$, entonces
    que $f$ sea morfismo de módulos puede escribirse como 
    $f\circ\rho_g\circ f^{-1}=\psi_g$. Luego
    \[
    \chi_V(g)=\trace\psi_g=\trace(f\circ\rho_g\circ f^{-1})=\trace\rho_g=\chi_U(g).
    \]
    
    Supongamos ahora que $\chi_U=\chi_V$. Como $\C[G]$ es semisimple, podemos
    escribir $U\simeq \oplus_{i=1}^k a_iS_i$ y 
    también $V\simeq\oplus_{i=1}^k b_iS_i$ para ciertos $a_1,\dots,a_k,b_1,\dots,b_k\geq0$. 
    Como $0=\chi_U-\chi_V=\sum_{i=1}^k(a_i-b_i)\chi_i$ y además los $\chi_i$ son linealmente independientes, 
    se concluye que $a_i=b_i$ para todo $i\in\{1,\dots,k\}$, es decir $U\simeq V$.
\end{proof}

\begin{lemma}
	Si $G$ es un grupo finito y $V$ y $W$ son $\C[G]$-módulos, entonces 
	\begin{enumerate}
		\item $\chi_{V\otimes W}=\chi_V\chi_W$,
		\item $\chi_{V^*}=\overline{\chi_V}$.
	\end{enumerate}
\end{lemma}

\begin{proof}
	Demostremos la primera afirmación. 
	Sabemos que $\phi$ y $\psi$ son diagonalizables. Sea $g\in G$ y sea $\{v_1,\dots,v_n\}$
	una base de autovectores de $\phi_g$ con autovalores
	$\lambda_1,\dots,\lambda_n$ y sea $\{w_1,\dots,w_m\}$ una base de
	autovectores de $\psi_g$ con autovalores $\mu_1,\dots,\mu_m$. Cada
	$v_i\otimes w_j$ es autovectores de $\phi\otimes\psi$ de autovalor
	$\lambda_i\mu_j$ pues 
	\[
		g(v_i\otimes w_j)=gv_i\otimes gw_j=\lambda_iv_i\otimes \mu_jv_j=(\lambda_i\mu_j)v_i\otimes w_j.
	\]
	Luego 
	$\{v_i\otimes w_j:1\leq i\leq n,1\leq j\leq m\}$ es base de autovectores y
	los $\lambda_i\mu_j$ son los autovalores de $\phi\otimes\psi$. Se concluye que
	\[
	\chi_{V\otimes W}(g)=\sum_{i,j}\lambda_i\mu_j=(\sum_i\lambda_i)(\sum_j\mu_j)=\chi_V(g)\chi_W(g).
	\]

	Demostremos la segunda afirmación. Sea $g\in G$, sea $\{v_1,\dots,v_n\}$
	una base de autovectores de $\psi$ con autovalores
	$\lambda_1,\dots,\lambda_n$ y sea $\{f_1,\dots,f_n\}$ su base dual. Veamos que $\{f_1,\dots,f_n\}$ es base de autovectores
	con autovalores $\overline{\lambda_1},\dots,\overline{\lambda_n}$. En efecto, si $gv_j=\lambda_jv_j$, entonces
	$g^{-1}v_j=\lambda_j^{-1}v_j=\overline{\lambda_j}v_j$ (observemos que como $\psi_g$ es inversible, los $\lambda_j$ son no nulos). Luego
	\[
		(gf_i)(v_j)=f_i(g^{-1}v_j)=\overline{\lambda_j}f_i(v_j)=\overline{\lambda_j}\delta_{ij}.
	\]
	En conclusión
	\[
		\chi_{V^*}(g)=\sum_{i=1}^n\overline{\lambda_i}=\overline{\chi_V(g)}.\qedhere
	\]
\end{proof}

Como consecuencia del lema anterior tenemos que el producto de dos caracteres es un caracter. 
Esto nos permite demostrar que 
el conjunto de combinaciones lineales enteras de caracteres 
irreducibles es un anillo con las operaciones usuales. 

\begin{exercise}
	Demuestre que el carácter $\chi_{\Hom_{\C}(U,V)}$ del módulo $\Hom_{\C[G]}(U,V)$ es igual a
	$\overline{\chi_U}\chi_V$.
\end{exercise}

\begin{definition}
  \label{exercise:cf(G)}
  \index{Función!de clases}
  Una $f\colon G\to\C$ se dice una \textbf{función de clases} si
  $f(g)=f(hgh^{-1})$ para todo $g,h\in G$.  
\end{definition}

Vimos en la proposición~\ref{pro:chi(1)} que los caracteres son funciones de
clase. 

\begin{exercise}\
    \begin{enumerate}
      \item Demuestre que el conjunto $\cf(G)$ de funciones de clase $G\to\C$
	es un subespacio vectorial de $L(G)$. 
      \item Demuestre que las funciones
	\[
	  \delta_K\colon G\to\C,\quad
	  \delta_K(g)=\begin{cases}
	    1 & \text{si $g\in K$,}\\
	    0 & \text{si $g\not\in K$,}
	  \end{cases}
	\]
	donde $K\in\cl(G)$, forman una base del conjunto $\cf(G)$ de funciones de clases de 
	$G$. En particular $\dim\cf(G)=|\cl(G)|$.
	\end{enumerate}
\end{exercise}

\begin{proposition}
    Los caracteres irreducibles forman una base del espacio de funciones de clases.
\end{proposition}

\begin{proof}
    El conjunto $\Irr(G)$ de caracteres irreducibles de $G$ es linealmente independiente y además 
    $|\Irr(G)|=|\cl(G)|=|\cf(G)|$.
\end{proof}

Si $U$ es un $\C[G]$-módulo, definimos
\[
U^G=\{u\in U:g\cdot u=u\text{ para todo $g\in G$}\}.
\]

\begin{lemma}
\label{lem:invariantes}
  $\dim U^G=\frac{1}{|G|}\sum_{x\in G}\chi_U(x)$. 
\end{lemma}

\begin{proof}
  Sea $\alpha=\frac{1}{|G|}\sum_{x\in G}\rho_x\colon U\to U$. 
  Primero vemos que $\alpha^2=\alpha$. 
  Como $\rho_g\circ \alpha=\frac{1}{|G|}\sum_{x\in G}\rho_{gx}=\alpha$, pues  
  $gx$ recorre todo $G$ si $x$ recorre todo $G$, entonces
  \[
  \alpha(\alpha(v))=\frac{1}{|G|}\sum_{g\in G}\rho_g(\alpha(v))=\alpha(v) 
  \]
  para todo $v\in V$. En particular, $\alpha$ tiene autovalores 0 y 1. Sea $V$ 
  el autoespacio correspondiente al autovalor 1. 
  Afirmamos que $V=U^G$. Si $v\in V$, entonces
  \[
  \rho_g(v)=\rho_g(\alpha(v))=\frac{1}{|G|}\sum_{x\in G}\rho_g\rho_x(v)=\frac{1}{|G|}\sum_{y\in G}\rho_y(v)=\alpha(v)=v,
  \]
  pues si $x$ recorre todo $G$, también lo hace $gx$. 
  Recíprocamente, si $u\in U^G$ entonces
  $\rho_g(u)=u$ para todo $g\in G$. En particular, 
  \[
  \alpha(u)=\frac{1}{|G|}\sum_{x\in G}\rho_x(u)=\frac{1}{|G|}\sum_{x\in G}u=u.
  \]
  En consecuencia, 
  $\dim V=\trace(\alpha)=\frac{1}{|G|}\sum_{g\in G}\trace(\rho_g)=\frac{1}{|G|}\sum_{g}\chi(g)$.
\end{proof}

Sea $G$ un grupo finito. En el espacio de funciones $G\to\C$ definimos la operación
\[
\langle f,g\rangle=\frac{1}{|G|}\sum_{x\in G}f(x)\overline{g(x)},
\quad f,g\colon G\to\C.
\]
Es fácil verificar que esta operación es un producto interno. 

\begin{theorem}
Si $U$ y $V$ son $\C[G]$-módulos, entonces 
\[
\langle\chi_U,\chi_V\rangle=\dim\Hom_{\C[G]}(U,V).
\]
\end{theorem}

\begin{proof}
    Primero observamos que $\Hom_{\C[G]}(U,V)$ es un subespacio del conjunto $\Hom_{\C}(U,V)$ de transformaciones lineales $U\to V$.
    
    Veamos ahora $\Hom_{\C[G]}(U,V)=\Hom_{\C}(U,V)^G$. En efecto, si $f\in\Hom_{\C[G]}(U,V)$, entonces
    \[
    (g\cdot f)(u)=g\cdot f(g^{-1}\cdot u)=g\cdot (g^{-1}\cdot f(u))=(gg^{-1})\cdot f(u)=1\cdot f(u)=f(u)
    \]
    para todo $g\in G$ y $u\in U$. Recíprocamente, si $f\colon U\to V$ es una transformación lineal tal que 
    $g\cdot f=f$ para todo $g\in G$, entonces, en particular, $g^{-1}\cdot f=f$ y luego
    $(g^{-1}\cdot f)(u)=f(u)$ para todo $g\in G$ y $u\in U$, que es equivalente a 
    $g\cdot f(u)=f(g\cdot u)$. 

    Luego
    \begin{align*}
    \dim\Hom_{\C[G]}(U,V)&=\dim\Hom_{\C}(U,V)^G\\
    &=\frac{1}{|G|}\sum_{g\in G}\chi_{\Hom(U,V)}(g)
    =\frac{1}{|G|}\sum_{g\in G}\overline{\chi_U(g)}\chi_V(g)
    =\langle\chi_V,\chi_U\rangle.
    \end{align*}
    Para terminar la demostración solamente hay que observar 
    que 
    \[
    \langle\chi_U,\chi_V\rangle=\frac{1}{|G|}\sum_{g\in G}\chi_U(g)\overline{\chi_V(g)}=\overline{\langle\chi_V,\chi_U\rangle}.\qedhere
    \]
\end{proof}

\index{Caracteres!matriz de}
\index{Caracteres!tabla de}
Sea $G$ un grupo finito y sean $\chi_1,\dots,\chi_s$ los representantes de
caracteres irreducibles de $G$. Para abreviar simplemente diremos que
$\chi_1,\dots,\chi_k$ son los caracteres irreducibles de $G$ y escribiremos
\[
	\Irr(G)=\{\chi_1,\dots,\chi_k\}.
\]
Sean $g_1,\dots,g_k$ los representantes de las clases de
conjugación de $G$. Se define la \textbf{matriz de caracteres} de $G$ como la
matriz $X\in\C^{s\times s}$ dada por
\[
X_{ij}=\chi_i(g_j),\quad
1\leq i,j\leq k.
\]
Veremos a continuación dos resultados muy importantes. El primero es
sobre la ortogonalidad de las filas de la matriz de caracteres. 

\begin{example}
    En este ejemplo veremos cómo calcular computacionalmente 
    la tabla de caracteres de un grupo. Sabemos que $\Sym_3$ tiene tres clases de conjugación, por lo que
    $\Irr(\Sym_3)$ tendrá tres elementos:
\begin{lstlisting}
gap> S3 := SymmetricGroup(3);;
gap> irr := Irr(S3);;
gap> Size(irr);
3
gap> NrConjugacyClasses(S3);
3
\end{lstlisting}
\end{example}

\begin{theorem}[Schur]
\index{Teorema!primera ortogonalidad Schur}
Sea $G$ un grupo finito. 
Si $\chi,\psi\in\Irr(G)$, entonces 
\[
\langle\chi,\psi\rangle=\begin{cases}
1 & \text{si $\chi=\psi$},\\
0 & \text{en otro caso}.
\end{cases}
\]
\end{theorem}

\begin{proof}
Si $S_1,\dots,S_k$ los $\C[G]$-módulos simples, entonces
\[
\langle\chi_i,\chi_j\rangle=\dim\Hom_{\C[G]}(S_i,S_j)=
\begin{cases}
1 & \text{si $i=j$},\\
0 & \text{en otro caso},
\end{cases}
\]
ya que, como los $S_j$ son módulos simples, sabemos por el lema de Schur que 
$\Hom_{\C[G]}(S_i,S_i)\simeq\C$ y $\Hom_{\C[G]}(S_i,S_j)=\{0\}$ si $i\ne j$. 
\end{proof}

\begin{example}
Verifiquemos computacionalmente las relaciones de ortogonalidad 
de Schur en el caso del grupo simétrico $\Sym_3$. 
\begin{lstlisting}
gap> S3 := SymmetricGroup(3);;        
gap> irr := Irr(S3);              
[ Character( CharacterTable( Sym( [ 1 .. 3 ] ) ), [ 1, -1, 1 ] ), 
  Character( CharacterTable( Sym( [ 1 .. 3 ] ) ), [ 2, 0, -1 ] ), 
  Character( CharacterTable( Sym( [ 1 .. 3 ] ) ), [ 1, 1, 1 ] ) ]
gap> Display(irr);
[ [   1,  -1,   1 ],
  [   2,   0,  -1 ],
  [   1,   1,   1 ] ]
gap> ScalarProduct(irr[1],irr[1]);
1
gap> ScalarProduct(irr[1],irr[2]);
0
gap> ScalarProduct(irr[1],irr[3]);
0
gap> ScalarProduct(irr[2],irr[2]);
1
gap> ScalarProduct(irr[2],irr[3]);
0
gap> ScalarProduct(irr[3],irr[3]);
1

\end{lstlisting}
\end{example}

  
El teorema de Schur tiene muchas aplicaciones. Por ejemplo:
\begin{enumerate}
    \item $\Irr(G)$ es una base ortonormal del espacio $\cf(G)$ de funciones de clases de $G$. 
    \item Si $\alpha=\sum_{i=1}^ka_i\chi_i$ y $\beta=\sum_{i=1}^kb_i\chi_i$, entonces
    $\langle\alpha,\beta\rangle=\sum_{i=1}^ka_ib_i$. 
    \item Si $\alpha=\sum_{i=1}^ka_i\chi_i$, entonces $\alpha=\sum_{i=1}^k\langle\alpha,\chi_i\rangle\chi_i$.
\end{enumerate}

\begin{corollary}
Si $G$ es un grupo finito y $S_1,\dots,S_k$ son las clases
de isomorfismos de módulos simples, entonces 
\[
\C[G]\simeq (\dim S_1)S_1\oplus\cdots\oplus (\dim S_k)S_k.
\]
\end{corollary}

\begin{proof}
    Sabemos que la representación regular puede escribirse como
    \[
    \C[G]\simeq a_1S_1\oplus\cdots\oplus a_kS_k,
    \]
    para ciertos enteros no negativos $a_1,\dots,a_k$ 
    unívocamente determinados. 
    Supongamos que $G=\{g_1,\dots,g_n\}$. Sea $L$ la representación regular (a izquierda) de $G$, es decir
    $L_g(g_j)=gg_j$ para todo $j\in\{1,\dots,n\}$. La matriz de $L_g$ 
    en la base $g_1,\dots,g_n$ es
    \[
    (L_g)_{ij}=\begin{cases}
    1 & \text{si $g_i=gg_j$},\\
    0 & \text{en otro caso}.
    \end{cases}
    \]
    En particular, el caracter $\chi_L$ de la representación regular cumple 
    \[
    \chi_L(g)=\begin{cases}
    |G| & \text{si $g=1$},\\
    0 & \text{en otro caso}.
    \end{cases}
    \]
    La primera relación de ortogonalidad de Schur implica que $a_i=\langle\chi_L,\chi_i\rangle$ para todo $i$, es decir 
    $\chi_L=\sum_{i=1}^k\langle\chi_L,\chi_i\rangle\chi_i$. 
    Como para cada $j$ se tiene 
    \[
    \langle\chi_L,\chi_j\rangle=\frac{1}{|G|}\sum_{g\in G}\chi_L(g)\overline{\chi_i(g)}=\overline{\chi_j(1)}=\chi_j(1)=\dim S_j,
    \]
    se concluye que $\C[G]\simeq\oplus_{i=1}^k(\dim S_i)S_i$. 
\end{proof}

\begin{exercise}
    Sea $\alpha$ un caracter de $G$ y sea $n\in\{1,2,3\}$. Demuestre que $\alpha$ es suma
    de $n$ irreducibles si y sólo si $\langle\alpha,\alpha\rangle=n$. 
\end{exercise}

Veamos ahora la segunda relación de ortogonalidad de Schur.

\begin{theorem}
\index{Teorema!segunda ortogonalidad Schur}
  Sean $G$ un grupo finito y $g,h\in G$. 
  Entonces
  \[
	\sum_{\chi\in\Irr(G)}\chi(g)\overline{\chi(h)}=\begin{cases}
	  |C_G(g)| & \text{si $g$ y $h$ son conjugados,}\\
	  0 & \text{en otro caso.}
	\end{cases}
  \]
  En particular, las columnas de $X$ son ortogonales y la matriz $X$ es
  inversible.
\end{theorem}

\begin{proof}
  Supongamos que $g_1,\dots,g_r$ son los representantes de las clases de
  conjugación de $G$ y que $\Irr(G)=\{\chi_1,\dots,\chi_r\}$. Entonces
  \[
    \langle\chi_i,\chi_j\rangle
    =\frac{1}{|G|}\sum_{g\in G}\chi_i(g)\overline{\chi_j(g)}
    =\frac{1}{|G|}\sum_{k=1}^rc_k\chi_i(g_k)\overline{\chi_j(g_k)},
  \]
  donde cada $c_k$ es el tamaño de la clase de conjugación de $g_k$. Matricialmente,
  \[
    I=\frac{1}{|G|}XD X^*,
  \]
  donde $I$ es la matriz identidad de $r\times r$, $D$ es la matriz diagonal
  que tiene a $c_1,\dots,c_r$ en la diagonal principal y
  $X^*=\overline{X}^T$. Entonces\footnote{Si $A,B\in\C^{s\times s}$ son tales que  
  $AB=I$ entonces $BA=I$.}
  \[
	I=\frac{1}{|G|}X^*XD,
  \]
  es decir $|G|D^{-1}=X^*X$. Luego 
  \[
    \sum_{k=1}^r\overline{\chi_k(g_i)}\chi_k(g_j)=\begin{cases}
    |C_G(g_j)| & \text{si $i=j$},\\
    0 & \text{en otro caso},
    \end{cases}
  \]
  que es lo que queríamos demostrar.
\end{proof}

\begin{exercise}
Sea $G$ un grupo finito. Si $\chi$ es un caracter irreducible de $G$ y $\phi$ es un caracter de grado uno, entonces
$\chi\otimes\phi$ es un caracter irreducible de $G$. 
\end{exercise}

\begin{theorem}[Solomon]
  \label{theorem:Solomon}
  \index{Teorema de!Solomon}
  Sean $G$ un grupo finito, $\Irr(G)=\{\chi_1,\dots,\chi_r\}$ y $g_1,\dots,g_r$ los 
  representantes de las clases de conjugación de $G$. 
  Si $i\in\{1,\dots,r\}$, entonces 
  \[
	  \sum_{j=1}^r\chi_i(g_j)\in\N_0.
  \]
\end{theorem}

\begin{proof}
  Sea $V$ el espacio vectorial con base $\{e_g:g\in G\}$.  Hagamos actuar a $G$
  en $G$ por conjugación y sea $\rho\colon G\to\GL(V)$,
  $\rho_g(e_h)=e_{ghg^{-1}}$.  Observemos que en la base $\{e_g:g\in G\}$ la
  matriz de $\rho_g$ es
  \[
   (\rho_g)_{ij}=\begin{cases} 
    1 & \text{si $g_ig=gg_j$},\\
    0 & \text{en otro caso}.
  \end{cases}
  \]
  Sea $\chi$ el caracter de la
  representación $\rho$. Entonces 
  \[
    \chi(g)=\trace\rho_g=\sum_{k}(\rho_g)_{kk}=|\{k:g_kg=gg_k\}|=|C_G(g)|.
  \]
  Sean $m_1,\dots,m_r\in\N_0$ tales que 
  \[
	\chi=\sum_{i=1}^rm_i\chi_i.
  \]
  Entonces, si $c_j$ es el tamaño de la clase de conjugación de $g_j$, la
  cantidad $m_i$ de veces que la representación irreducible con caracter
  $\chi_i$ aparece en $\rho$ es igual a 
  \[
	  m_i=\langle\chi,\chi_i\rangle=\frac{1}{|G|}\sum_{g\in G}\chi(g)\overline{\chi_i(g)}
	  =\frac{1}{|G|}\sum_{j=1}^rc_j|C_G(g_j)|\overline{\chi_i(g_j)}
	  =\sum_{j=1}^r\overline{\chi_i(g_j)}.
  \]
  Luego $\sum_{j=1}^r\chi_i(g_j)=\overline{m_i}=m_i\in\N_0$.
\end{proof}

