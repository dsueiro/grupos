\chapter{Extensiones}
\label{extensiones}

\begin{definition}
	\index{Extensión}
	Sean $K$ y $Q$ grupos. Una \textbf{extension} de $K$ por $Q$ es un grupo
	$G$ que tiene un subgrupo normal $N$ isomorfo a $K$ tal que $G/N\simeq Q$.
	Equivalentemente, una \textbf{extensión} de $K$ por $Q$ es una sucesión
	exacta corta\footnote{$\iota$ es inyectiva, $p$ es sobreyectiva y $\ker
	p=\iota(N)$.}
\[
\begin{tikzcd}
	1 & K & G & Q & 1
	\arrow[from=1-1, to=1-2]
	\arrow["\iota", from=1-2, to=1-3]
	\arrow["p", from=1-3, to=1-4]
	\arrow[from=1-4, to=1-5]
\end{tikzcd}
\]
\end{definition}

\begin{example}
	$C_6$ y $\Sym_3$ son extensiones de $C_3$ por $C_2$.
\end{example}

\begin{example}
	$C_6$ es extensión de $C_2$ por $C_3$.
\end{example}

\begin{example}
	Sean $K$ y $Q$ grupos. El producto directo $K\times Q$ es extensión de $K$
	por $Q$. También es extensión de $Q$ por $K$.
\end{example}

Sea $G$ una extensión de $K$ por $Q$. Si $L$ es un subgrupo de $G$ que
contiene a $K$ entonces $L$ es una extensión de $K$ por $L/K$. 

\begin{definition}
	\index{Morfismo!de extensiones}
	Un \textbf{morfismo} entre las extensiones 
	\[
	1\to K\xrightarrow{\iota}G\xrightarrow{p} Q\to1,
	\quad
	1\to K_1\xrightarrow{\iota_1}G_1\xrightarrow{p_1} Q_1\to1,
	\]
	es una terna $(\alpha,\beta,\gamma)$ de morfismos tal que el siguiente
	diagrama conmuta:
\[\begin{tikzcd}
	1 & K & G & Q & 1 \\
	1 & {K_1} & {G_1} & {Q_1} & 1
	\arrow["\iota", from=1-2, to=1-3]
	\arrow["p", from=1-3, to=1-4]
	\arrow[from=1-4, to=1-5]
	\arrow[from=1-1, to=1-2]
	\arrow[from=2-1, to=2-2]
	\arrow["{\iota_1}", from=2-2, to=2-3]
	\arrow["{p_1}", from=2-3, to=2-4]
	\arrow[from=2-4, to=2-5]
	\arrow["\alpha", from=1-2, to=2-2]
	\arrow["\beta", from=1-3, to=2-3]
	\arrow["\gamma", from=1-4, to=2-4]
\end{tikzcd}\]
\end{definition}

\begin{definition}
	\index{Isomorfismo!de extensiones}
	\index{Equivalencia!de extensiones}
	Diremos que las extensiones 
	\[
	E\colon 1\to K\xrightarrow{\iota}G\xrightarrow{p} Q\to1,
	\quad
	E_1\colon 1\to K_1\xrightarrow{\iota_1}G_1\xrightarrow{p_1} Q\to1,
	\]
	son \textbf{isomorfas} si existe un morfismo $(\alpha,\beta,\id)$ entre $E$
	y $E_1$ con $\alpha$ isomorfismo. Las extensiones $E$ y $E_1$ se diran
	\textbf{equivalentes} si $K=K_1$ y $(\id,\beta,\id)$ es un isomorfismo
	entre $E$ y $E_1$.
\end{definition}

\begin{exercise}
	Demuestre que si $(\alpha,\beta,\id)$ es un isomorfismo de extensiones
	entonces $\beta$ es un isomorfismo de grupos.
\end{exercise}

% \begin{svgraybox}
% 	Veamos que $\ker\beta=1$. Sea $g\in G$ tal que $\beta(g)=1$. Como $g\in\ker
% 	p=\iota(K)$ pues $p(g)=p_1\beta(g)=p_1(1)=1$, existe $k\in K$ tal que
% 	$g=\iota(k)$. Entonces $1=\beta(g)=\beta\iota(k)=\iota_1\alpha(k)$. Como
% 	$\iota_1$ y $\alpha$ son morfismos inyectivos, $k=1$ y luego $g=1$.
	
% 	Veamos ahora que $\beta(G)=G_1$. Sea $g_1\in G_1$. Como $p$ es
% 	sobreyectiva, $p_1(g_1)=p(g)$ para algún $g\in G$. Como
% 	$\beta(g)g_1^{-1}\in \ker p_1=\iota_1(K_1)$ y $\alpha$ es epimorfismo,
% 	existe $k\in K$ tal que
% 	$\beta(g)g_1^{-1}=\iota_1(\alpha(k))=\beta(\iota(k))$. Luego
% 	$\beta(g\iota(k)^{-1})=g_1$.
% \end{svgraybox} 

%\begin{definition}
%	\index{Equivalencia!de extensiones}
%	Diremos que las extensiones 
%	\[
%	1\to K\xrightarrow{\iota}G\xrightarrow{p} Q\to1,
%	\quad
%	1\to K_1\xrightarrow{\iota_1}G_1\xrightarrow{p_1} Q\to1,
%	\]
%	son \textbf{equivalentes} si existe un morfismo 
%	$(\id,\beta,\id)$ entre esas extensiones. 
%\end{definition}
%

\begin{definition}
	\index{Levantamiento} 
	Sea $E\colon 1\to K\xrightarrow{\iota}G\xrightarrow{p} Q\to1$ una
	extensión.  Un \textbf{levantamiento} para $E$ es una función $\ell\colon
	Q\to G$ tal que $p(\ell(x))=x$ para todo $x\in Q$. 
\end{definition}

\begin{exercise}
	\label{exercise:lifting}
	Sea $E\colon 1\to K\xrightarrow{\iota}G\xrightarrow{p} Q\to1$ una
	extensión.  Demuestre las siguientes afirmaciones:
	\begin{enumerate}
		\item Si $\ell\colon Q\to G$ es un levantamiento, $\ell(Q)$
			es un transversal para $\ker p$ en $G$.
		\item Todo transversal a $\ker p$ en $G$ induce un levantamiento $\ell\colon
			Q\to G$.
		\item Si $\ell\colon Q\to G$ es un levantamiento entonces
			$\ell(xy)\ker p=\ell(x)\ell(y)\ker p$.
	\end{enumerate}
\end{exercise}

% \begin{svgraybox}
% 	Sea $N=\iota(K)=\ker p$. 
% 	\begin{enumerate}
% 		\item Veamos que las $\ell(x)N$ son disjuntas. 
% 			Si $\ell(x)N=\ell(y)N$, existe $n\in N$ tal que
% 			$\ell(x)=\ell(y)n$. Luego 
% 			\[
% 			x=p(\ell(x))=p(\ell(y)n)=p(\ell(y))p(n)=y.
% 			\]
% 			Sea $g\in G$ y sea $x=p(g)\in Q$. Como $x=p(\ell(x))$ y $p$ es
% 			morfismo de grupos, $x^{-1}=p(\ell(x)^{-1})$. Luego
% 			$g=\ell(x)\left(\ell(x)^{-1}g\right)$ y $\ell(x)^{-1}g\in N$ pues
% 			\[
% 			p(\ell(x)^{-1}g)=p(\ell(x)^{-1})p(g)=x^{-1}x=1.
% 			\]
% 		\item Sea $L$ un transversal a $N$ en $G$. Si $x\in Q$ entonces existe
% 			$g\in G$ tal que $x=p(g)$.  Sea $\ell(x)\in L\subseteq G$ tal que
% 			$gN=\ell(x)N$. Como $g^{-1}\ell(x)\in N=\ker p$, se concluye que
% 			$p(\ell(x))=x$ pues 
% 			\[
% 			x^{-1}p(\ell(x))=p(g)^{-1}p(\ell(x))=p(g^{-1}\ell(x))=1.
% 			\]
% 		\item Sean $x,y\in Q$ y sean $g,h\in G$ tales que $x=p(g)$, $y=p(h)$. 
% 			Como por definición $gN=\ell(x)N$, $hN=\ell(y)N$ y $p$ es morfismo de grupos, 
% 			\[
% 				\ell(xy)N=(gh)N=(gN)(hN)=\ell(x)\ell(y)N.
% 			\]
% 	\end{enumerate}
% \end{svgraybox}

%\begin{definition}
%	\index{Acoplamiento}
%	Un morfismo de grupos $\chi\colon Q\to\Out(K)$ se denomina un
%	\textbf{acoplamiento} de $Q$ en $K$.
%\end{definition}
%
%\begin{theorem}
%	\label{theorem:coupling}
%	Toda extensión $1\to K\xrightarrow{\iota}G\xrightarrow{p} Q\to1$ determina
%	un acoplamiento de $Q$ en $K$.
%\end{theorem}
%
%\begin{proof}
%	Sea $N=\ker p$. Como $N$ es normal en $G$, para cada $x\in Q$ se tiene un
%	automorfismo $\gamma_{\ell(x)}$ dado por $n\mapsto \ell(x)n\ell(x)^{-1}$ de
%	$N$. Como $\iota\colon K\to\iota(K)=N$ es un isomorfismo, tenemos un automorfismo
%	$\lambda_x\in\Aut(K)$ que hace conmutar al diagrama
%    \[
%    \xymatrix{
%    N
%	\ar[r]^-{\gamma_{\ell(x)}}
%    & N
%	\ar[d]^{\iota^{-1}}
%    \\
%    K
%	\ar[u]^{\iota}
%	\ar[r]_{\lambda_x}
%    & K
%    }
%    \]
%	es decir:	
%	\[
%	\iota(\lambda_x(k))=\ell(x)\iota(k)\ell(x)^{-1},\quad
%	k\in K.
%	\]
%	Estudiemos cómo depende 
%	esta función del levantamiento elegido. Si $\ell_1\colon Q\to G$
%	es un levantamiento, existe $n\in
%	N$ tal que $\ell(x)=\ell_1(x)n$. Si $k\in K$ y $l\in K$ es tal
%	que $\iota(l)=n$, entonces $\lambda_x$ y $(\lambda_1)_x$ pertenecen a la
%	misma coclase módulo $\Inn(K)$ pues 
%	\begin{align*}
%		\iota(\lambda_x(k))&=\ell(x)\iota(k)\iota(x)^{-1}
%		=\ell_1(x)n\iota(k)n^{-1}\ell_1(x)^{-1}\\
%		&=\ell_1(x)\iota(lkl^{-1})\ell_1(x)^{-1}
%		=\iota((\lambda_1)_x(lkl^{-1})).
%	\end{align*}
%	Queda bien definida entonces la función $\lambda\colon Q\to\Out(K)$,
%	$x\mapsto\lambda_x$. 
%	
%	Veamos que $\lambda$ es morfismo de grupos. Por el ejercicio~\ref{exercise:lifting}  
%	existe $n\in N$ tal que
%	$\ell(xy)=\ell(x)\ell(y)n$. Si escribimos $n=\iota(l)$ para algún $l\in K$
%	entonces vemos que $\lambda_x\lambda_y=\lambda_{xy}\gamma_l$ y luego
%	$\lambda(x)\lambda(y)=\lambda(xy)$. 
%\end{proof}
%
%\begin{exercise}
%	Demuestre que extensiones equivalentes dan el mismo acoplamiento.
%\end{exercise}
%
%\begin{svgraybox}
%	Si las extensiones 
%	\[
%	E\colon 1\to K\xrightarrow{\iota}G\xrightarrow{p} Q\to1,
%	\quad
%	E_1\colon 1\to K_1\xrightarrow{\iota_1}G_1\xrightarrow{p_1} Q\to1,
%	\]
%	son equivalentes entonces el diagrama 
%	\[
%	\xymatrix{
%	0\ar[r] 
%	& K
%	\ar@{=}[d]
%	\ar[r]^-{\iota}
%	& G
%	\ar[r]^-{p}
%	\ar[d]^\beta
%	& Q\ar[r]
%	\ar@{=}[d]
%	& 0
%	\\
%	0\ar[r] 
%	& K
%	\ar[r]^-{\iota_{1}}
%	& G_1
%	\ar[r]^-{p_{1}}
%	& Q\ar[r]
%	& 0
%	}
%	\]
%	es conmutativo. Si $\ell\colon Q\to G$ es un levantamiento para $E$
%	entonces $\beta\ell$ es un levantamiento para $E_1$ pues
%	$p_1(\beta\ell)=(p_1\beta)\ell=p\ell=\id$. 
%
%	Si $x\in Q$ y $k\in K$ entonces
%	$\iota(\lambda_x(k))=\ell(x)\iota(k)\ell(x)^{-1}$. Al aplicar $\beta$ y
%	usar la conmutatividad del diagrama:
%	\begin{align*}
%		\iota_1(\lambda_x(k))&=\beta\iota\lambda_x(k)
%		=\beta(\ell(x))\beta\iota(k)\beta(\ell(x)^{-1})\\
%		&=\beta\ell(x)\iota_1(k)(\beta\ell(x))^{-1}
%		=\iota_1((\lambda_1)_x(k)).
%	\end{align*}
%	Como $\iota_1$ es inyectiva,
%	$\chi(x)(k)=\lambda_x(k)=(\lambda_1)_x(k)=\chi_1(x)(k)$.
%\end{svgraybox}

\begin{definition}
	\index{Extensión!que se parte}
	Se dice que una extensión se \textbf{parte} si existe un levantamiento que
	es morfismo de grupos.
\end{definition}

El primer paso que daremos en este contexto es para entender las extensiones que se parten 
mediante "derivaciones". 

%\begin{lemma}
%	\label{lemma:split}
%	Si una extensión $E\colon 1\to K\xrightarrow{\iota}G\xrightarrow{p} Q\to1$
%	se parte entonces $N=\iota(K)$ admite un complemento en $G$.
%\end{lemma}
%
%\begin{proof}
%	
%\end{proof}

%\begin{definition}
%	Sea $E\colon 1\to K\xrightarrow{\iota}G\xrightarrow{p} Q\to1$ una
%	extensión. Diremos que un $\gamma\in\Aut(G)$ \textbf{estabiliza} a $E$ si 
%	el diagrama 
%	\[
%	\xymatrix{
%	0\ar[r] 
%	& K
%	\ar@{=}[d]
%	\ar[r]^-{\iota}
%	& G
%	\ar[r]^-{p}
%	\ar[d]^\gamma
%	& Q\ar[r]
%	\ar@{=}[d]
%	& 0
%	\\
%	0\ar[r] 
%	& K
%	\ar[r]^-{\iota}
%	& G
%	\ar[r]^-{p}
%	& Q\ar[r]
%	& 0
%	}
%	\]
%	es conmutativo. El \textbf{estabilizador} de la extensión $E$ es el
%	conjunto de $\gamma\in\Aut(G)$ que estabilizan a $E$.
%\end{definition}
%
%\begin{theorem}
%	\label{theorem:estabilizador_abeliano}
%	El estabilizador de una extensión 
%	$E\colon 1\to K\xrightarrow{\iota}G\xrightarrow{p} Q\to1$ 
%	es un grupo abeliano.
%\end{theorem}
%
%\begin{proof}
%	Sea $S$ el estabilizador de la extensión $E$ y sea $\gamma\in S$. 
%	Sea $T$ un transversal a $N=\ker p$ en $G$. Si $g\in G$ existen $n,n_1\in N$ y
%	$t,t_1\in T$ tales que $g=nt$ y $\gamma(g)=n_1t_1$. 
%	Como entonces $tN=t_1N$ pues 
%	\[
%		p(t_1)=p(n_1t_1)=p(\gamma(g))=p(g)=p(nt)=p(n)p(t)=p(t),
%	\]
%	se concluye que $\gamma(g)=n_1t$. Luego 
%	\[
%	g\gamma(g)^{-1}=nt(n_1t)^{-1}=ntt^{-1}n_1=nn_1\in N.
%	\]
%
%	Veamos ahora que $g\gamma(g)^{-1}\in Z(N)$. Si $n\in N$ entonces
%	$\gamma(n)=n$ pues $n=\iota(k)$ para
%	algún $k\in K$ y entonces $\gamma(n)=\gamma\iota(k)=\iota(k)=n$. Luego 
%	\[
%	[g\gamma(g)^{-1},n]=g\gamma(g)^{-1}n\gamma(g)g^{-1}n^{-1}=g\gamma(g^{-1}ng)g^{-1}n^{-1}=1.
%	\]
%
%	Sea 
%	\[
%	\Psi\colon S\to\prod_{g\in G}Z(N),
%	\quad
%	\gamma\mapsto (g^{-1}\gamma(g))_{g\in G}. 
%	\]
%	
%	
%
%\end{proof}

%\section{Derivaciones y complementos}

\begin{definition}
	\index{Derivación}
	\index{$1$-cociclo}
	Sean $Q$ y $K$ grupos. Supongamos que $Q$ actúa por automorfismos en $K$.
	Una función $\varphi\colon Q\to K$ se dice un  $1$-\textbf{cociclo} (o una
	\textbf{derivación}) si 
	\[
		\varphi(xy)=\varphi(x)(x\cdot\varphi(y))
	\]
	para todo $x,y\in Q$.  El conjunto de \textbf{derivaciones} de $Q$ en $K$
	se define como
	\[
	\Der(Q,K)=Z^1(Q,K)=\{\delta\colon Q\to K:\text{$\delta$ es $1$-cociclo}\}.
	\]
\end{definition}

\begin{example}
	Sea $Q$ un grupo que actúa por automorfismos en $K$. Para cada $k\in K$, la
	función $Q\to K$, $x\mapsto [k,x]=kxk^{-1}x^{-1}$, es una derivación.
\end{example}

% \begin{svgraybox}
% 	Para $k\in K$ y $x\in Q$ escribimos $\delta_k(x)=[k,x]$. Entonces 
% 	\[
% 	\delta_k(x)(x\delta_k(y)x^{-1})
% 	=kxk^{-1}x^{-1}xkyk^{-1}y^{-1}x^{-1}
% 	=k(xy)k^{-1}(xy)^{-1}
% 	=\delta_k(xy).
% 	\]
% \end{svgraybox}

\begin{exercise}
	\label{exercise:1cocycle}
	Sea $\varphi\colon Q\to K$ un $1$-cociclo. Demuestre las siguientes afirmaciones:
	\begin{enumerate}
		\item $\varphi(1)=1$.
		\item $\varphi(y^{-1})=(y^{-1}\cdot\phi(y))^{-1}=y^{-1}\cdot\phi(y)^{-1}$.
		\item El conjunto $\ker\varphi=\{x\in Q:\varphi(x)=1\}$ 	es un
			subgrupo de $Q$. 
	\end{enumerate}
\end{exercise}

% \begin{svgraybox}
% 	La primera afirmación es fácil pues, como 
% 	\[
% 		\varphi(1)=\varphi(11)=\varphi(1)(1\cdot \varphi(1))=\varphi(1)^2,
% 	\]
% 	se concluye que $\varphi(1)=1$. 
	
% 	Veamos que $\ker\varphi$ es un subgrupo. 
% 	Como $\varphi(1)=1$, $K$ es no vacío. Sean
% 	$x,y\in \ker\varphi$. Como $1=\varphi(y^{-1}y)=\varphi(y^{-1})(y^{-1}\cdot
% 	\varphi(y))$, se tiene que
% 	\[
% 	\varphi(y^{-1})=(y^{-1}\cdot\phi(y))^{-1}.
% 	\]
% 	Similarmente se demuestra que 
% 	\[
% 		\varphi(y^{-1})=y^{-1}\cdot\phi(y)^{-1}.
% 	\]
% 	De estas fórmulas se deduce que si $x\in \ker\varphi$ entonces
% 	$x^{-1}\in \ker\varphi$. Luego $\ker\varphi$ es un subgrupo pues si $x,y\in \ker\varphi$, 
% 	$\varphi(xy)=\varphi(x)(x\cdot \varphi(y))=1(x\cdot 1)=1$. 
% \end{svgraybox}

Recordemos que un subgrupo $K$ de un grupo $G$ admite un complemento $Q$ si 
$G$ se factoriza como 
$G=KQ$ y $K\cap Q=\{1\}$. 
El ejemplo típico es el siguiente, el producto semidirecto $G=K\rtimes Q$, donde $K$ es un subgrupo
normal de $G$ y $Q$ es un subgrupo de $G$ tales que $K\cap Q=\{1\}$. 

\begin{theorem}
	\label{theorem:complementos}
	Sea $Q$ un grupo que actúa por automorfismos en el grupo $K$.  Existe una
	biyección entre el conjunto de complementos de $K$ en $K\rtimes Q$ y el
	conjunto $\Der(Q,K)$.
\end{theorem}

\begin{proof}
	El grupo $Q$ actúa en $K$ por conjugación, entonces $\delta\in\Der(Q,K)$ si
	y sólo si $\delta(xy)=\delta(x)x\delta(y)x^{-1}$, $x,y\in Q$. En este caso,
	las fórmulas del ejercicio anterior quedan así:
	$\delta(1)=1$, $\delta(x^{-1})=x^{-1}\delta(x)^{-1}x$.
	
	Sea $\mathcal{C}$ el conjunto de complementos de $K$ en $K\rtimes Q$.  Sea
	$C\in\mathcal{C}$. Si $x\in Q$, sabemos que 
	existen únicos $k\in K$ y $c\in C$ tales que $x=k^{-1}c$. Queda bien
	definida entonces la función $\delta_C\colon Q\to K$, $x\mapsto k$. Vale
	que $\delta(x)x=c\in C$. 
	
	Veamos que $\delta_C\in\Der(Q,K)$. Si $x,x_1\in Q$, escribimos $x=k^{-1}c$
	y $x_1=k_1^{-1}c_1$, donde $k,k_1\in K$ y $c,c_1\in C$. Como $K$ es normal
	en $K\rtimes Q$, podemos escribir a $xx_1$ como $xx_1=k_2c_2$, donde
	$k_2=k^{-1}(ck_1^{-1}c^{-1})\in K$, $c_2=cc_1\in C$. Luego 
	\[
		\delta(xx_1)xx_1=cc_1=\delta(x)x\delta(x_1)x_1
	\]
	implica que $\delta(xx_1)=\delta(x)x\delta(x_1)x^{-1}$. 
	Tenemos así una función $F\colon\mathcal{C}\to\Der(Q,K)$, $F(C)=\delta_C$.

	Vamos a construir ahora $G\colon\Der(Q,K)\to\mathcal{C}$. 
	Para
	cada $\delta\in\Der(Q,K)$ vamos a definir un complemento $\Delta$ de $K$ en $K\rtimes Q$: 
	\[
	\Delta=\{\delta(x)x:x\in Q\}.
	\]

	Veamos que $\Delta$ es un subgrupo de $K\rtimes Q$. Como $\delta(1)=1$,
	$1\in X$. Si $x,y\in Q$ entonces
	$\delta(x)x\delta(y)y=\delta(x)x\delta(y)x^{-1}xy=\delta(xy)xy\in \Delta$.
	Por último si $x\in Q$ entonces
	$(\delta(x)x)^{-1}=x^{-1}\delta(x)^{-1}xx^{-1}=\delta(x^{-1})x^{-1}$.
	
	
	Veamos que $\Delta\cap K=\{1\}$. Si $x\in Q$ es tal que $\delta(x)x\in K$
	entonces, como $\delta(x)\in K$, $x\in K\cap Q=1$. Si $g\in G$ entonces
	existen únicos $k\in K$, $x\in Q$ tales que $g=kx$. Escribimos
	$g=k\delta(x)^{-1}\delta(x)x$. Como $k\delta(x)^{-1}\in K$ y $\delta(x)x\in
	\Delta$, se concluye que $G=K\Delta$. Queda bien definida entonces la
	función $G\colon\Der(Q,K)\to\mathcal{C}$, $G(\delta)=\Delta$.

	Veamos ahora que $G\circ F=\id_{\mathcal{C}}$. 
	Sea $C\in\mathcal{C}$. Entonces 
	\[
	G(F(C))=G(\delta_C)=\{\delta_C(x)x:x\in
	Q\}=C,
	\]
	por construcción. (Vimos que $\delta_C(x)x\in C$. Recíprocamente,  si $c\in
	C$, escribimos $c=kx$ para únicos $k\in K$, $x\in Q$ y luego $x=k^{-1}c$
	que implica $c=\delta_c(x)x$.)

	Por último veamos que $F\circ G=\id_{\Der(Q,K)}$. Sea $\delta\in\Der(Q,K)$.
	Entonces 
	\[
	F(G(\delta))=F(\Delta)=\delta_{\Delta}.
	\]
	Queremos demostrar que $\delta_\Delta=\delta$.  Sea $x\in Q$. Existe
	$\delta(y)y\in\Delta$ para algún $y\in Q$ tal que $x=k^{-1}\delta(y)y$.
	Luego $\delta_{\Delta}(x)x=\delta(y)y$ y luego $\delta(x)=\delta(y)$ por la
	unicidad de la escritura.
\end{proof}

\begin{definition}
	\index{Derivación!interior}
	\index{$1$-coborde}
	Sean $Q$ y $K$ grupos. Supongamos que $Q$ actúa por automorfismos en $K$.
	Un $\delta\in\Der(Q,K)$ se dice \textbf{interior} si existe $k\in K$ tal
	que $\delta(x)=[k,x]$ para todo $x\in Q$. El conjunto de
	\textbf{derivaciones interiores} será denotado por
	\[
		\Inn(Q,K)=B^1(Q,K)=\{\delta\in\Der(Q,K):\text{$\delta$ es interior}\}.
	\]
	Una derivación interior también se llama \textbf{$1$-coborde}.
\end{definition}

\begin{theorem}[Sysak]
	\index{Teorema!de Sysak}
	\index{Sysak, Y.}
	\label{theorem:Sysak}
	Sean $Q$ y $K$ grupos tales que $Q$ actúa por automorfismos en $K$. Sea
	$\delta\in\Der(Q,K)$.
	\begin{enumerate}
		\item $\Delta=\{\delta(x)x:x\in Q\}$ es un complemento para $K$ en $K\rtimes Q$.
		\item $\delta\in\Inn(Q,K)$ si y sólo si $Q$ y $\Delta$ son conjugados en
			$K$.
		\item $\ker\delta=Q\cap\Delta$.
		\item $\delta$ es sobreyectiva si y sólo si $K\rtimes Q=\Delta Q$.
	\end{enumerate}
\end{theorem}

\begin{proof}
	Vimos en la demostración del teorema~\ref{theorem:complementos} que
	el conjunto $\Delta$ es un complemento para $K$ en $K\rtimes Q$. 

	Demostremos la segunda afirmación. Si suponemos que $\delta$ es interior,
	existe $k\in K$ tal que $\delta(x)=[k,x]=kxk^{-1}x^{-1}$ para todo $x\in
	Q$. Como $\delta(x)x=kxk^{-1}$ para todo $x\in Q$,  $\Delta=kQk^{-1}$.
	Recíprocamente, si existe $k\in K$ tal que $\Delta=kQk^{-1}$, para cada
	$x\in Q$ existe $y\in Q$ tal que $\delta(x)x=kyk^{-1}$. Como
	$[k,y]=kyk^{-1}y^{-1}\in K$, $\delta(x)\in K$ y $\delta(x)x=[k,y]y\in KQ$,
	se concluye que $x=y$ y luego $\delta(x)=[k,x]$. 

	Demostremos la tercera afirmación. Si $x\in Q$ es tal que $\delta(x)x=y\in
	Q$ entonces $\delta(x)=yx^{-1}\in K\cap Q=\{1\}$. Recíprocamente, si $x\in Q$
	es tal que $\delta(x)=1$ entonces $x=\delta(x)x\in Q\cap\Delta$. 

	Demostremos la cuarta afirmación. Si $\delta$ es sobreyectiva, para cada
	$k\in K$ existe $y\in Q$ tal que $\delta(y)=k$. Luego $K\rtimes Q\subseteq
	\Delta Q$ pues $kx=\delta(y)x=(\delta(y)y)y^{-1}x\in \Delta Q$. Además
	$\Delta Q\subseteq K\rtimes Q$ pues si $\delta(x)\in K$ para todo $x\in Q$.
	Recíprocamente, si $k\in K$ y $x\in Q$ existen 
	$y,z\in Q$ tales que $kx=\delta(y)yz$; en particular, 
	por la unicidad de la escritura de $K\rtimes Q$,
	$k=\delta(y)$. 
\end{proof}

Un caso importante de grupos que admiten factorización
es el siguiente: 

\begin{definition}
	Un grupo $G$ admite una \textbf{factorización triple} si tiene subgrupos
	$A$, $B$ y $M$ tales que $G=MA=MB=AB$ y $A\cap M=B\cap M=\{1\}$.
\end{definition}

%%% TODO:
%%% ejemplos de factorizacion triple
%%% caracterizacion de 1-cociclos biyectivos
%%% ejemplo con anillo radical
%%% braces?

Una consecuencia inmediata del teorema de Sysak:

\begin{corollary}
	Supongamos que el grupo $Q$ actúa por automorfismos en $K$. Sea
	$\delta\in\Der(Q,K)$ sobreyectivo. Entonces $G=K\rtimes Q$ admite una
	factorización triple.
\end{corollary}

% \begin{proof}
% 	Es consecuencia inmediata del teorema~\ref{theorem:Sysak}. 
% \end{proof}
Otra consecuencia: 

\begin{exercise}
	\label{xca:ker1cocycle}
	Sea $\delta\in\Der(Q,K)$. 
	\begin{enumerate}
	\item Demuestre que $\delta$ es inyectiva si y sólo si
	$\ker\delta=\{1\}$.
	\item Si $\delta$ es biyectivo, demuestre que 
	$K$ admite un complemento
	$\Delta$ en $K\rtimes Q$ tal que $K\rtimes Q=K\rtimes\Delta=\Delta Q$ y
	$Q\cap\Delta=\{1\}$.
	\end{enumerate}
\end{exercise}

% \begin{proof}
% 	Sean $x,y\in Q$ tales que $\delta(x)=\delta(y)$. Como $\delta(x^{-1}y)=1$
% 	pues 
% 	\[
% 	\delta(x^{-1}y)=\delta(x^{-1})(x^{-1}\delta(y)x)=\delta(x^{-1})x^{-1}\delta(x)x=\delta(x^{-1}x)=\delta(1)=1
% 	\]
% 	y $\delta$ es inyectiva, $x^{-1}y=1$. La afirmación recíproca es trivial.
% \end{proof}

% \begin{corollary}
% 	Si $\delta\in\Der(Q,K)$ es biyectivo entonces $K$ admite un complemento
% 	$\Delta$ en $K\rtimes Q$ tal que $K\rtimes Q=K\rtimes\Delta=\Delta Q$ y
% 	$Q\cap\Delta=1$.
% \end{corollary}

% \begin{proof}
% 	Vimos en el teorema de Sysak que $\delta$ es sobreyectiva si y
% 	sólo si $K\rtimes Q=\Delta Q$ y que $\ker\delta=Q\cap\Delta$.
% \end{proof}

%\section{Aplicación: subespacios invariantes}
%
%Sea $A$ un grupo que actúa por automorfismos en un grupo $G$. Definimos
%\[
%C_G(A)=\{g\in G:g\cdot a=a\text{ para todo $a\in A$}\}.
%\]
%
%Como aplicación de la teoría de Schur--Zassenhaus vamos a demostrar los
%teoremas de Sylow para subespacios $A$-invariantes.
%Necesitamos el siguiente lema:
%
%\begin{lemma}
%	\label{lemma:Glauberman}
%	Sean $A$ y $G$ grupos finitos de órdenes coprimos. Supongamos que $A$ actúa
%	por automorfismos en $G$ y que $A$ o $G$ es resoluble. Supongamos que $A$
%	actúa en un conjunto $X$ y que $G$ actúa transitivamente en $X$ de forma tal que
%	\begin{equation}
%		\label{equation:Glauberman:compatibilidad}
%		a\cdot (g\cdot x)=(aga^{-1})\cdot (a\cdot x)
%	\end{equation}
%	para todo $a\in A$, $g\in G$, $x\in X$. Valen las siguientes afirmaciones:
%	\begin{enumerate}
%		\item Existe un $x\in X$ invariante por la acción de $A$.
%		\item Si $x,y\in X$ son invariantes por la acción de $A$ entonces
%			existe $c\in C_G(A)$ tal que $c\cdot x=y$.
%	\end{enumerate}
%\end{lemma}
%
%\begin{proof}
%	Sea $\Gamma=G\rtimes A$ el producto semidirecto. Todo $\gamma$ se escribe
%	en forma única como $\gamma=ga$ con $g\in G$, $a\in A$. Veamos que $\Gamma$
%	actúa en $X$ por
%	\[
%		\gamma\cdot x=(ga)\cdot x=g\cdot (a\cdot x).
%	\]
%	Es fácil ver que es una acción pues la igualdad
%	\[
%	(ga)\cdot ((hb)\cdot x)=((ga)(hb))\cdot x=(gaha^{-1})\cdot ((ab)\cdot x)
%	\]
%	es consecuencia de la relación de
%	compatibilidad~\eqref{equation:Glauberman:compatibilidad}.\framebox{completar}
%
%\end{proof}
%\begin{theorem}
%	\label{theorem:Sylow_Ainv}
%\end{theorem}
%
%
\section*{Cohomología}

\begin{definition}
	\index{$G$-módulo}
	Sea $G$ un grupo. Un $G$-módulo es un grupo abeliano $A$ con una acción por
	automorfismos de $G$. 
\end{definition}

En esta sección $G$ es un grupo y $A$ es un $G$-módulo. El grupo $A$ será
escrito aditivamente. 

\begin{definition}
	\index{Cocadena}
	Sea $n\geq0$. Una \textbf{cocadena}
	de grado $n$ (o $n$-cocadena) de $G$ con valores en $A$ es una función
	$f\colon G\times\cdots\times G\to A$, $(s_1,\dots,s_n)\mapsto
	f(s_1,\dots,s_n)$. 
\end{definition}

	El conjunto $C^n(G,A)$ de $n$-cocadenas de $G$ con valores en
	$A$ es un grupo abeliano. 

\begin{definition}
	\index{Coborde}
	Sea $f\in C^n(G,A)$. Se define el \textbf{coborde} de $f$ como el elemento $df\in
	C^{n+1}(G,A)$ dado por 
	\begin{align*}
		df(s_1,\dots,&s_{n+1})=s_1\cdot f(s_2,\dots,s_{n+1})\\
		&+\sum_{i=1}^n (-1)^i f(s_1,\dots,s_{i-1},s_is_{i+1},s_{i+2},\dots,s_{n+1})
		+(-1)^{n+1}f(s_1,\dots,s_n).
	\end{align*}
\end{definition}

\begin{example}
	Veamos la función $d\colon C^0(G,A)\to C^1(G,A)$.  Si $a\in C^0(G,A)=A$
	entonces 
	\[
	da(s)=s\cdot a-a.
	\]
	Luego $da=0$ si y sólo si $s\cdot a=0$ para
	todo $s\in G$.
\end{example}

\begin{example}
	Veamos ahora la función $d\colon C^1(G,A)\to C^2(G,A)$. Si $f\in C^1(G,A)$
	entonces
	\[
	df(s,t)=s\cdot f(t)-f(st)+f(s).
	\]
\end{example}

\begin{example}
	Veamos ahora la función $d\colon C^2(G,A)\to C^3(G,A)$. Si $f\in C^2(G,A)$
	entonces
	\[
	df(u,v,w)=u\cdot f(v,w)-f(uv,w)+f(u,vw)-f(u,v).
	\]
\end{example}

\begin{lemma}
	\label{lemma:dd=0}
	La composición $C^n(G,A)\xrightarrow{d} C^{n+1}(G,A)\xrightarrow{d}
	C^{n+2}(G,A)$ es cero.
\end{lemma}

La demostración del lema anterior es fácil pero tediosa, por eso queda como ejercicio. 

\begin{definition}
	\index{Cociclo}
	\index{Coborde}
	Sea $f$ una cocadena $f$ de grado $n$. Se dice que $f$ es un
	\textbf{cociclo} de grado $n$ si $df=0$. Denotaremos por $Z^n(G,A)$ al
	conjunto de $n$-cociclos de $G$ en $A$. Se dice que $f$ es un
	\textbf{coborde} de grado $n$ si existe una cocadena $g$ de grado $(n-1)$
	tal que $f=dg$. Denotaremos por $B^n(G,A)$ al conjunto de $n$-cobordes de
	$G$ en $A$.
\end{definition}

Obviamente $Z^n(G,A)$ es un grupo abeliano y $B^n(G,A)\subseteq Z^n(G,A)$.

\begin{definition}
	\index{Cohomología}
	El $n$-ésimo \textbf{grupo de cohomología} de $G$ con valores en $A$
	es el grupo abeliano
	\[
	H^n(G,A)=Z^n(G,A)/B^n(G,A).
	\]
\end{definition}

\begin{exercise}
	Demuestre que $H^0(G,A)=A^G$, donde 
	\[
	A^G=\{a\in A:s\cdot a=a\text{ para todo $s\in G$}\}. 
	\]
\end{exercise}

% \begin{svgraybox}
% 	Pues para todo $a\in A=C^0(G,A)$ se tiene que $da=0$ si y sólo si $a\in
% 	A^G$.
% \end{svgraybox}

\begin{example}
	$f\in Z^1(G,A)$ si y sólo si $f(st)=s\cdot f(t)+f(s)$ para todo $s,t\in G$.
	En particular, si $G$ actúa trivialmente en $A$, $Z^1(G,A)=\hom(G,A)$. Como
	en este caso $B^1(G,A)=0$, se concluye que $H^1(G,A)\simeq\hom(G,A)$. 
\end{example}

\begin{example}
	$f\in Z^2(G,A)$ si y sólo si $f$ es un \textbf{factor}, es decir:
	\begin{equation}
		\label{eq:2cociclo}
		u\cdot f(v,w)-f(uv,w)+f(u,vw)-f(u,v)=0
	\end{equation}
	para todo $u,v,w\in G$.
\end{example}

\begin{exercise}
	Sea $f\in Z^2(G,A)$ tal que $f(1,1)=0$. Demuestre que 
	\[
	f(1,x)=f(x,1)=0
	\]
	para todo $x\in G$.
\end{exercise}

% \begin{svgraybox}
% 	Al usar la fórmula~\eqref{eq:2cociclo} con $(u,v,w)=(1,1,x)$ se obtiene
% 	$f(1,x)=f(1,1)$. Con $(u,v,w)=(x,1,1)$ se obtiene $x\cdot f(1,1)=f(x,1)$. 
% \end{svgraybox}

\begin{definition}
	\index{Cohomólogos!cociclos}
	\index{Cociclos!cohomólogos}
	Sean $f,g\in Z^2(G,A)$. Se dice que $f$ y $g$ son \textbf{cohomólogos} si
	existe algún coborde $h$ tal que $f-g=dh$. 
\end{definition}

Un factor $f\in Z^2(G,A)$ se dice \textbf{normalizado} si $f(1,1)=0$. 

\begin{lemma}
	\label{lemma:normalizado}
	Todo $f\in Z^2(G,A)$ es cohomólogo a un factor normalizado. 
\end{lemma}

\begin{proof}
	Sea $\gamma\colon G\to A$ tal que $\gamma(1)=-f(1,1)$. Sea
	$g=f+d\gamma$. Entonces $g\in Z^2(G,A)$, $g(1,1)=0$ y $dg=d(f+d\gamma)=df$ pues
	$d^2=0$. 
\end{proof}

\begin{theorem}
	\label{theorem:|G|H^2=0}
	Sea $G$ un grupo finito y sea $A$ un $G$-módulo. Entonces 
	\[
	|G|H^n(G,A)=0.
	\]
\end{theorem}

\begin{proof}
	Sea $m=|G|$. Sea $f\in Z^n(G,A)$. Vamos a demostrar que $mf$ es un coborde. Sea 
	\[
	F(s_1,\dots,s_{n-1})=\sum_{s\in G}f(s_1,\dots,s_{n-1},s).
	\]
	Como $f\in Z^n(G,A)$, 
	\begin{align*}
		0=s_1\cdot f(s_2,\dots,s_{n+1})-f(s_1s_2,s_3,\dots,s_{n+1})+\cdots&+(-1)^nf(s_1,\dots,s_ns_{n+1})\\
		&+(-1)^{n+1}f(s_1,\dots,s_n).
	\end{align*}
	%Si $s_{n+1}$ recorre todo $G$ entonces $s_ns_{n+1}$ también lo hace. Luego, 
	Al sumar estas igualdades sobre todo $s_{n+1}\in G$ obtenemos 
	\begin{align*}
		0=s_1\cdot F(s_2,\dots,s_n)-F(s_1s_2,\dots,s_n)+\cdots&+(-1)^nF(s_1,\dots,s_{n-1})\\
		&+(-1)^{n+1}mf(s_1,\dots,s_n).
	\end{align*}
	Luego $0=dF(s_1,\dots,s_n)-(-1)^nmf(s_1,\dots,s_n)$, es decir $mf=d(
	(-1)^nF)$.
\end{proof}

\begin{corollary}
	\label{corollary:a->|G|a}
	Sea $G$ un grupo finito y $A$ un $G$-módulo. Si $a\mapsto |G|a$ es un
	automorfismo de $A$ entonces $H^n(G,A)=0$ para todo $n\geq1$. 
\end{corollary}

\begin{proof}
	Como la función $x\mapsto |G|x$ es un automorfismo de $C^n(G,A)$ que
	conmuta con $d$, induce un automorfismo $H^n(G,A)\to H^n(G,A)$, $x\mapsto
	|G|x$. Como $|G|H^n(G,A)=0$ por el teorema~\ref{theorem:|G|H^2=0}, se
	concluye que $H^n(G,A)=0$.
\end{proof}

\begin{corollary}
	\label{corollary:H^n=0}
	Sea $G$ un grupo finito y $A$ un $G$-módulo. Si $A$ es finito y de orden
	coprimo con $|G|$ entonces $H^n(G,A)=0$ para todo $n\geq1$. 
\end{corollary}

\begin{proof}
	Como $|G|H^n(G,A)=0$ por el teorema~\ref{theorem:|G|H^2=0} y $a\mapsto
	|G|a$ es un automorfismo de $A$, $H^n(G,A)$ por el
	corolario~\ref{corollary:a->|G|a}.
\end{proof}

\begin{corollary}
	Sea $G$ un grupo finito y $A$ un $G$-módulo finitamente generado.  Entonces
	$H^n(G,A)$ es finito para todo $n\geq1$. 	
\end{corollary}

\begin{proof}
	Como $C^n(G,A)$ es finitamente generado, $H^n(G,A)$ es finitamente
	generado.  Luego $H^n(G,A)$ es finito por ser un grupo abeliano finitamente
	generado y de torsión.
\end{proof}

\section*{Extensiones abelianas}

%En esta sección estudiaremos extensiones $G$ de $K$ por $Q$ en el caso en que
%$K$ sea un grupo abeliano.	En este caso $\Der(Q,A)$ es un grupo abeliano con
%la operación 
%\[
%(\delta_1\delta_2)(x)=\delta_1(x)\delta_2(x).
%\]
%
%\begin{exercise}
%	Sea $Q$ un grupo que actúa por automorfismos en un grupo abeliano $K$.
%	Demuestre que $\Inn(Q,A)$ es un subgrupo del grupo abeliano $\Der(Q,A)$. 
%\end{exercise}
%
%\begin{svgraybox}
%	Si $\delta_k,\delta_l\in\Inn(Q,K)$ entonces para $x\in Q$ se tiene
%	\[
%	\delta_k\delta_l(x)=k(xk^{-1}x^{-1})lxl^{-1}x^{-1}=(kl)x(kl)^{-1}x^{-1}=\delta_{kl}(x)
%	\]
%	pues $xk^{-1}x^{-1}\in K$ y $K$ es abeliano. Además 
%	\[
%	\delta_k(x)^{-1}
%	=(kxk^{-1}x^{-1})^{-1}
%	=k(k^{-1}xkx^{-1})k^{-1}
%	=k\delta_{k^{-1}}(x)k^{-1}
%	=\delta_{k^{-1}}
%	\]
%	pues $K$ es abeliano.
%\end{svgraybox}
%
%\begin{theorem}
%	\label{theorem:semidirecto:Kabeliano}
%	Sea $K$ un $G$-módulo y sea $Q$ un grupo que actúa por automorfismos en
%	$K$. Existe una correspondencia biyectiva entre clases de conjugación de
%	complementos de $K$ en $K\rtimes Q$ y $\Der(Q,K)/\Inn(Q,K)$.
%\end{theorem}
%
%\begin{proof}
%	Sean $C$ y $D$ complementos de $K$ y supongamos que existe $g\in K\rtimes
%	Q$ tal que $C=gDg^{-1}$. Como $G=KQ$, existe $k\in K$ tal que $C=kDk^{-1}$.
%	Si $x\in Q$, existe $d\in D$ tal que
%	\[
%		\delta_C(x)x=kdk^{-1}.
%	\]
%	Como $K$ es abeliano,
%	\[
%		[x^{-1}d,k]=(x^{-1}d)k(x^{-1}d)^{-1}k^{-1}=1. 
%	\]
%	Esto implica que $[k,d]=[k,x]$ y entonces $[k,x]^{-1}\delta_C(x)x=d$. Luego 
%	\[
%	\delta_D(x)x=[k,d]d=[k,x]d\dots
%	\]
%
%	Sean $\delta_C,\delta_D\in\Der(Q,K)$ tales que
%	$\delta_D(x)=\delta_C(x)[k,x]^{-1}$ para algún $k\in K$. Como $K$ es
%	abeliano y $xkx^{-1}\in K$, se concluye que  
%	$D=k^{-1}Ck$ ya que $C=\{\delta_C(x)x:x\in Q\}$, $D=\{\delta_D(x)x:x\in
%	Q\}$ y 
%	\[
%	\delta_D(x) 
%	= \delta_C(x)[k,x]^{-1}
%	=\delta_C(x)xkx^{-1}k^{-1}
%	=\delta_C(x)k^{-1}xkx^{-1}
%	=k^{-1}\delta_C(x)xkx^{-1}.
%	\]
%	\framebox{completar}
%\end{proof}

\begin{definition}
	\index{Dato!para una extensión}
	Un \textbf{dato} es un par $(Q,K)$, donde $Q$ es un grupo y $K$ es
	$Q$-módulo.  Se dice que un grupo $G$ \textbf{realiza} el dato $(Q,K)$ si
	$G$ es una extensión de $K$ por $Q$ y para todo levantamiento $\ell\colon
	Q\to G$ se tiene  
	\[
	x\cdot a=\ell(x)a\ell(x)^{-1},\quad a\in K,x\in Q.
	\]
\end{definition}

%\begin{exercise}
%	\label{exercise:extension}
%	Sea $G$ una extensión de $K$ por $Q$. Supongamos que $K$ es abeliano. Si
%	$\ell\colon Q\to G$ es un levantamiento demuestre que existe un morfismo
%	$\chi\colon Q\to\Aut(K)$ tal que el grupo $G$ realiza el dato $(Q,K,\chi)$.
%\end{exercise}
%
%\begin{svgraybox}
%	Vimos en el teorema~\ref{theorem:coupling} que toda extensión determina un
%	acoplamiento $\chi\colon Q\to\Out(K)\simeq\Aut(K)$, donde
%	$\Out(K)\simeq\Aut(K)$ pues $K$ es abeliano.  Luego $G$ realiza el dato
%	$(Q,K,\chi)$, donde $\chi\colon Q\to\Aut(K)$.
%\end{svgraybox}

\begin{definition}
	\index{Factor}
	Sea $Q$ un grupo y sea $K$ un $Q$-módulo. Sea $1\to K\to G\to Q\to 1$ una
	extensión de $K$ por $Q$ que realiza el dato $(K,Q)$ y sea $\ell\colon Q\to
	G$ un levantamiento tal que $\ell(1)=1$. Un \textbf{factor} para $\ell$ es
	una función $f\colon Q\times Q\to K$ tal que 
	\begin{equation}
		\label{eq:ell}
		\ell(x)\ell(y)=f(x,y)\ell(xy)
	\end{equation}
	para todo $x,y\in Q$. 
\end{definition}

\begin{lemma}
	Sea $Q$ un grupo y sea $K$ un $Q$-módulo. Sea $1\to K\to G\to Q\to 1$ una
	extensión de $K$ por $Q$ que realiza el dato $(K,Q)$. Si $f$ es un factor
	para $\ell$ entonces
	\begin{align}
		\label{eq:f(1x)}&f(1,x)=f(x,1)=1,\\
		\label{eq:fcocycle}&f(x,y)f(xy,z)=(x\cdot f(y,z))f(x,yz)
	\end{align}
	para todo $x,y,z\in Q$.
\end{lemma}

\begin{proof}
	Si hacemos $x=1$ en~\eqref{eq:f(1x)} obtenemos $1=f(1,y)$. De la misma
	forma se obtiene la otra igualdad. Para demostrar la
	igualdad~\eqref{eq:fcocycle} calculamos 
	\begin{align*}
		&(\ell(x)\ell(y))\ell(z)=(f(x,y)\ell(xy))\ell(z)
		=f(x,y)(\ell(xy)\ell(z))
		=f(x,y)f(xy,z)\ell(xyz).
	\end{align*}
	Por otro lado, como 
	la extensión realiza el dato $(K,Q)$, 
	\begin{align*}
		\ell(x)(\ell(y)\ell(z))&=\ell(x)(f(y,z)\ell(yz))\\
		&=(x\cdot f(y,z))\ell(x)\ell(yz)=(x\cdot f(y,z))f(x,yz)\ell(xyz).
	\end{align*}
	La igualdad~\eqref{eq:fcocycle} se obtiene al observar que el producto de
	$G$ es asociativo.
\end{proof}

\begin{lemma}
	\label{lemma:G(K,Q,f)}
	Sea $Q$ un grupo y sea $K$ un $Q$-módulo. Sea $f\colon Q\times Q\to K$ una
	función que satisface \eqref{eq:f(1x)} y~\eqref{eq:fcocycle}. Entonces
	existe una extensión $1\to K\to G\to Q\to 1$ de $K$ por $Q$ que realiza el
	dato $(K,Q)$ y existe un levantamiento $\ell\colon Q\to G$ cuyo factor es
	$f$.
\end{lemma}

\begin{proof}
	Sea $G=K\times Q$ con el producto
	\[
		(a,x)(b,y)=(a(x\cdot b)f(x,y),xy).
	\]
	Veamos que el producto es asociativo:
	\begin{align*}
		((a,x)(b,y))(c,z)&=(a(x\cdot b)f(x,y),xy)(c,z)\\
		&=(a(x\cdot b)f(x,y)((xy)\cdot c))f(xy,z),xyz)
	\end{align*}
	Por otro lado
	\begin{align*}
		(a,x)((b,y)(c,z)) &= (a,x)(b(y\cdot c)f(y,z),yz)\\
		&=(a(x\cdot (b(y\cdot c)f(y,z))f(x,yz),xyz)\\
		&=(a(x\cdot b)(xy)\cdot c)(x\cdot f(y,z))f(x,yz),xyz).
	\end{align*}
	Es fácil verificar que el neutro es $(1,1)$ y el inverso de $(a,x)$ es 
	\[
	(a,x)^{-1}=(x^{-1}\cdot (f(x,x^{-1})a)^{-1},x^{-1}).
	\]
	Sea $\ell\colon Q\to G$ un levantamiento. Si $x\in Q$ existe $b\in K$ tal
	que $\ell(x)=(b,x)$. Veamos que la extensión realiza el dato $(K,Q)$. Como
	$f(x,1)=1$ y $K$ es abeliano, 
	\begin{align*}
		\ell(x)(a,1)\ell(x)^{-1}
		&=(b,x)(a,1)(b,x)^{-1}\\
		&=(b(x\cdot a)f(x,1),x)(x^{-1}\cdot (f(x,x^{-1})b)^{-1},x^{-1})\\
		&=(b(x\cdot a)b^{-1}f(x,x^{-1})^{-1}f(x,x^{-1}),1)\\
		&=(x\cdot a,1).
	\end{align*}
	Por último, para ver que existe un levantamiento con factor $f$ basta
	considerar $\ell\colon Q\to G$, $\ell(x)=(1,x)$ pues 
	\[
		\ell(x)\ell(y)(\ell(xy)^{-1}=(1,x)(1,y)(1,xy)^{-1}=(f(x,y),1).\qedhere
	\]
\end{proof}

Si $Q$ es un grupo, $K$ es un $Q$-módulo y $f\colon Q\times Q\to K$ es una
función que satisface~\eqref{eq:f(1x)} y~\eqref{eq:fcocycle}. El grupo $G$ del
lema~\ref{lemma:G(K,Q,f)} será denotado por $G(K,Q,f)$. 

\begin{lemma}
	\label{lemma:existe_f}
	Sea $Q$ un grupo y sea $K$ un $Q$-módulo. Sea $1\to K\to G\to Q\to 1$ una
	extensión de $K$ por $Q$ que realiza el dato $(K,Q)$. Entonces existe
	un factor $f\colon Q\times Q\to K$ tal que $G\simeq G(K,Q,f)$. 
\end{lemma}

\begin{proof}
	Sea $\ell\colon Q\to G$ un levantamiento y sea $f$ su factor. Como $G$ es
	unión disjunta de coclases
	\[
		G=\bigcup_{x\in Q}K\ell(x),
	\]
	para cada $g\in G$ existen únicos $a\in K$ y $x\in Q$ tales que
	$g=a\ell(x)$. Queda entonces bien definida una función biyectiva
	$\phi\colon G\to G(K,Q,f)$, $a\ell(x)\mapsto (a,x)$. Veamos que $\phi$ es
	un morfismo de grupos: 
	\begin{align*}
		\phi(a\ell(x)b\ell(y)) 
		&=\phi(a\ell(x)b\ell(x)^{-1}\ell(x)\ell(y))\\
		&=\phi(a\ell(x)b\ell(x)^{-1}f(x,y)\ell(xy))\\
		&=\phi(a(x\cdot b)f(x,y)\ell(xy))\\
		&=(a(x\cdot b)f(x,y),xy)\\
		&=\phi(a\ell(x))\phi(b\ell(y)).\qedhere
	\end{align*}
\end{proof}

\begin{lemma}
	\label{lemma:coborde}
	Sean $Q$ un grupo, $K$ un $Q$-módulo y $1\to K\to G\to Q\to 1$ una
	extensión de $K$ por $Q$ que realiza el dato $(K,Q)$. Para $j\in\{1,2\}$
	sea $\ell_j\colon Q\to G$ un levantamiento con factor $f_j$ tal que
	$\ell_j(1)=1$.  Entonces existe $\gamma\colon Q\to K$ tal que $\gamma(1)=1$
	y 
	\[
		f_2(x,y)=\gamma(x)(x\cdot \gamma(y))f_1(x,y)\gamma(xy)^{-1}
	\]
	para todo $x,y\in Q$.
\end{lemma}

\begin{proof}
	Como $\ell_1(x)$ y $\ell_2(x)$ están en la misma coclase de
	$K$, existe $\gamma(x)\in K$ tal que $\ell_2(x)=\gamma(x)\ell_1(x)$. Como
	$\ell_1(1)=\ell_2(1)=1$, $\gamma(1)=1$. Además 
	\begin{align*}
		f_2(x,y)\ell_2(xy)&=\ell_2(x)\ell_2(y) = \gamma(x)\ell_1(x)\gamma(y)\ell_1(y)\\
		&=\gamma(x)\ell_1(x)\gamma(y)\ell_1(x)^{-1}\ell_1(x)\ell_1(y)\\
		&=\gamma(x)(x\cdot \gamma(y))f_1(x,y)\ell_1(xy)
		=\gamma(x)(x\cdot \gamma(y))f_1(x,y)\gamma(xy)^{-1}\ell_2(xy),
	\end{align*}
	que implica lo que se quería demostrar.
\end{proof}

\begin{definition}
	Sean $Q$ un grupo y $K$ un $Q$-módulo. Una función $g\colon Q\times Q\to K$
	se dice un \textbf{coborde} si existe una función $\gamma\colon Q\to K$ con
	$\gamma(1)=1$ tal que
	\[
	g(x,y)=(x\cdot \gamma(y))\gamma(xy)^{-1}\gamma(x)
	\]
	para todo $x,y\in Q$.
\end{definition}

\begin{lemma}
	\label{lemma:equivalencia}
	Sean $Q$ un grupo y $K$ un $Q$-módulo. Dos extensiones $G_1$ y $G_2$ de $K$ por $Q$ 
	que realizan el dato $(K,Q)$ son
	equivalentes si y sólo existe un factor  $f_1$ de $G_1$ y un factor $f_2$
	de $G_2$ tales que $f_1f_2^{-1}$ es un coborde.
\end{lemma}

\begin{proof}
	Para cada $j\in\{1,2\}$ sea $\ell_j\colon Q\to G$ un levantamiento con
	factor $f_j$ y tal que $\ell_j(1)=1$. 
	Como $G$ es unión disjunta de coclases
	\[
	G=\bigcup_{x\in Q}K\ell_1(x)
	\]
	todo $g_1\in G_1$ se escribe unívocamente como $g_1=a\ell_1(x)$ para $a\in
	K$ y $x\in Q$. Sea $\phi\colon G_1\to G_2$, $a\ell_1(x)\mapsto
	a\gamma(x)\ell_2(x)$. Es evidente que $\phi$ hace conmutar al diagrama
	\begin{equation}
		\label{eq:diagrama}
% 	\xymatrix{
% 	0\ar[r] 
% 	& K
% 	\ar@{=}[d]
% 	\ar[r]%@{^{(}->}[r]
% 	& G_1
% 	\ar[r]^{p_1}
% 	\ar[d]^\phi
% 	& Q\ar[r]
% 	\ar@{=}[d]
% 	& 0
% 	\\
% 	0\ar[r] 
% 	& K
% 	\ar[r]
% 	& G_2
% 	\ar[r]^{p_2}
% 	& Q\ar[r]
% 	& 0
% 	}
\begin{tikzcd}
	1 & K & {G_1} & Q & 1 \\
	1 & K & {G_1} & {Q_1} & 1
	\arrow[from=1-2, to=1-3]
	\arrow["{p_1}", from=1-3, to=1-4]
	\arrow[from=1-4, to=1-5]
	\arrow[from=1-1, to=1-2]
	\arrow[from=2-1, to=2-2]
	\arrow[from=2-2, to=2-3]
	\arrow["{p_2}", from=2-3, to=2-4]
	\arrow[from=2-4, to=2-5]
	\arrow[equal, no head, from=1-2, to=2-2]
	\arrow["\phi", from=1-3, to=2-3]
	\arrow[equal, from=1-4, to=2-4]
    \end{tikzcd}
	\end{equation}
	Veamos que $\phi$ es morfismo. Por un lado tenemos
	\[
		\phi(a\ell_1(x)b\ell_1(y))=\phi(a(x\cdot b)f_1(x,y)\ell_1(xy))=a(x\cdot b)f_1(x,y)\gamma(xy)\ell_2(xy).
	\]
	Por otro lado, se tiene 
	\begin{align*}
		\phi(a\ell_1(x))\phi(b\ell_1(y))
		&=a\gamma(x)\ell_2(x)b\gamma(y)\ell_2(y)\\
		&=a\gamma(x)(x\cdot b)\ell_2(x)\gamma(y)\ell_2(y)\\
		&=a\gamma(x)(x\cdot b)(x\cdot \gamma(y))\ell_2(x)\ell_2(y)\\
		&=a\gamma(x)(x\cdot b)(x\cdot \gamma(y))f_2(x,y)\ell_2(xy)\\
		&=a(x\cdot b)\gamma(x)(x\cdot \gamma(y))f_2(x,y)\ell_2(xy)
	\end{align*}
	pues $K$ es abeliano.  Por el lema~\ref{lemma:coborde}, existe una función
	$\gamma\colon Q\to K$ con $\gamma(1)=1$ y tal que  
	$f_1(x,y)=\gamma(x)(x\cdot \gamma(y))f_2(x,y)\gamma(xy)^{-1}$ 
	para todo $x,y\in Q$. Luego $\phi$ es morfismo de grupos. 

	Recíprocamente, supongamos que existe $\gamma$ tal que el
	diagrama~\eqref{eq:diagrama} conmuta. En particular, $\gamma(a)=a$ para
	todo $a\in K$ y la función $\phi\ell_1\colon Q\to G_2$ es un levantamiento pues 
	$x=p_1\ell_1(x)=p_2\phi\ell_1(x)$ 
	para todo $x\in Q$. Como
	\[
	\phi\ell_1(x)\phi\ell_1(y)=\phi f_1(x,y)\phi\ell_1(xy)=f_1(x,y)\phi\ell_1(xy)
	\]
	para todo $x,y\in Q$, $f_1$ es también un factor para la extensión $G_2$.
	Si $f_2$ es un factor para $G_2$, entonces $f_1f_2^{-1}$ es un coborde por
	el lemma~\ref{lemma:coborde}. 
\end{proof}

\begin{example}
	Sea $p$ un número primo impar. Sean $K=\langle a\rangle\simeq C_p$, $G=\langle
	g\rangle\simeq C_{p^2}$ y $Q=\langle gK\rangle=G/K\simeq C_p$.  Veamos que
	las extensiones
	\[
	\begin{tikzcd}
	1 & K & {G_1} & Q & 1 \\
	1 & K & {G_1} & {Q_1} & 1
	\arrow["\iota_1", from=1-2, to=1-3]
	\arrow["{p_1}", from=1-3, to=1-4]
	\arrow[from=1-4, to=1-5]
	\arrow[from=1-1, to=1-2]
	\arrow[from=2-1, to=2-2]
	\arrow["\iota_2", from=2-2, to=2-3]
	\arrow["{p_2}", from=2-3, to=2-4]
	\arrow[from=2-4, to=2-5]
	\arrow[equal, no head, from=1-2, to=2-2]
	\arrow[dashed, "\phi", from=1-3, to=2-3]
	\arrow[equal, from=1-4, to=2-4]
    \end{tikzcd}
    \]	
    donde $\iota_1(a)=g^p$, $\iota_2(a)=g^{2p}$ y $p_1(g)=p_2(g)=gK$, no son
	equivalentes. Si existe $\phi$ entonces
	$\phi(g^p)=\phi\iota_1(a)=\iota_2(a)=g^{2p}$. Esto implica que
	$\phi(g)=g^2$ y luego $g\in K$ pues $gK=p_1(g)=p_2\phi(g)=g^2K$, una
	contradicción. 
\end{example}

Sea $Q$ un grupo y sea $K$ un $Q$-módulo. Sea $E(Q,K)$ el conjunto de clases de
equivalencia de extensiones $1\to K\to G\to Q\to 1$ que realizan el dato $(Q,K)$. 

\begin{theorem}[Schreier]
	\label{theorem:Schreier:extensiones_abelianas}
	Sean $Q$ un grupo y $K$ un $Q$-módulo.  Existe una correspondencia
	biyectiva entre $H^2(Q,K)$ y el conjunto $E(Q,K)$ de clases de equivalencia
	de extensiones que realizan el dato $(Q,K)$. Bajo esta correspondencia, el
	factor nulo se corresponde con la clase de equivalencia de extensiones que
	se parten.
\end{theorem}

\begin{proof}
	Sea $[G]$ la clase de equivalencia de $1\to K\to G\to Q\to 1$. Sea
	$\phi\colon H^2(Q,K)\to E(Q,K)$ dado por $f+B^2(Q,K)\mapsto [G(K,Q,f)]$,
	donde $[G(K,Q,f)]$ es la clase de una extensión que realiza el dato $(Q,K)$
	(existe gracias al lemma~\ref{lemma:G(K,Q,f)}). El
	lemma~\ref{lemma:equivalencia} implica que $\phi$ está bien definida y es
	una función inyectiva. La función $\phi$ es sobreyectiva pues el
	lemma~\ref{lemma:existe_f} implica que si $[G]\in E(Q,K)$ entonces
	$[G]=[G(K,Q,f)]=\phi(f+B^2(Q,K))$ para algún $f$. %La última afirmación es evidente. 
\end{proof}

\begin{remark}
	Como aplicación del teorema de Schreier podemos dar una demostración breve
	de la existencia del complemento en el teorema de
	Schur--Zassenhaus~\ref{theorem:SchurZassenhaus:abeliano}.  Para $Q=G/N$
	consideramos la extensión $1\to N\to G\to Q\to 1$. Como $|N|$ y $|Q|$ son
	coprimos, $H^2(Q,N)=0$ por el corolario~\ref{corollary:H^n=0}.  Por el
	teorema de Schreier~\ref{theorem:Schreier:extensiones_abelianas}, $E(Q,N)$
	contiene un único elemento y luego la extensión $1\to N\to G\to Q\to 1$ se
	parte.
\end{remark}

% TODO: corolario sobre extensiones centrales

%\begin{definition}
%	Sea $G$ una extensión de $K$ por $Q$ y sea $\pi\colon G\to Q$ un morfismo sobreyectivo. 
%	%Un \textbf{levantamiento} 
%	%de $x\in Q$ es un elemento $\ell(x)\in G$ tal que $\pi(\ell(x))=x$. 
%	Un \textbf{levantamiento} es una función $\ell\colon Q\to G$ tal que
%	$\pi(\ell(x))=x$ para todo $x\in Q$. 
%\end{definition}
%
%\begin{exercise}
%	Si $G$ es una extensión de $K$ por $Q$, $\pi\colon G\to Q$ es un morfismo
%	sobreyectivo y $\ell\colon Q\to G$ es un levantamiento, entonces la imagen
%	$\ell(Q)$ es un transversal para $\ker\pi$ en $G$.
%\end{exercise}
%
%\begin{svgraybox}
%	Si $\ell(x)\ker\pi=\ell(y)\ker\pi$, existe $k\in\ker\pi$ tal que
%	$\ell(x)=\ell(y)k$. Luego 
%	\[
%	x=\pi(\ell(x))=\pi(\ell(y)k)=\pi(\ell(y))\pi(k)=y.
%	\]
%	Sea $g\in G$ y sea $x=\pi(g)\in Q$. Como $x=\pi(\ell(x))$ y $\pi$ es
%	morfismo de grupos, $x^{-1}=\pi(\ell(x)^{-1})$. Luego
%	$g=\ell(x)\left(\ell(x)^{-1}g\right)$ y $\ell(x)^{-1}g\in\ker\pi$ pues
%	\[
%		\pi(\ell(x)^{-1}g)=\pi(\ell(x)^{-1})\pi(g)=x^{-1}x=1.
%	\]
%\end{svgraybox}
%
%\begin{theorem}
%	\label{theorem:extensions}
%	Sea $G$ una extensión de $K$ por $Q$. Supongamos que $K$ es abeliano y sea
%	$\ell\colon Q\to G$ un levantamiento. Existe un morfismo $\theta\colon
%	Q\to\Aut(K)$ tal que
%	\[
%		\theta_x(a)=\ell(x)a\ell(x)^{-1}
%	\]
%	para todo $a\in K$, $x\in Q$.
%\end{theorem}
%
%\begin{proof}
%	Para $g\in G$ sea $\gamma_g\colon G\to G$, $\gamma_g(x)=gxg^{-1}$.  Como
%	$K$ es normal en $G$, $\gamma_g|_K\in\Aut(K)$.  Sea $\mu\colon
%	G\to\Aut(K)$, $\mu(g)=\gamma_g|_K$. Entonces $K\subseteq\ker\mu$ y luego
%	$\mu$ induce un morfismo $G/K\to\Aut(K)$ dado por $gK\mapsto\mu(g)$. 
%	
%	Sea $\lambda\colon Q\to G/K$, $\lambda(x)=\ell(x)K$. La función $\lambda$
%	es morfismo de grupos pues si $x,y\in Q$ entonces
%	\[
%	\lambda(xy)=\ell(xy)K=\ell(x)\ell(y)K=\lambda(x)\lambda(y)
%	\]
%	pues $\pi(\ell(xy))=\pi(\ell(x)\ell(y))$. 
%
%	Además $\lambda$ es inyectiva ya que $\ell(x)=\ell(y)K$
%
%
%
%
%\end{proof}
