\chapter{Ejemplos de tablas de caracteres}

\index{Tabla de caracteres}
Sea $G$ un grupo finito y sean $\chi_1,\dots,\chi_r$ los caracteres
irreducibles de $G$. Sin pérdida de generalidad podemos suponer que $\chi_1$ es
el carácter trivial.  Sabemos que $r$ es igual a la cantidad de clases de
conjugación de $G$. Como además cada $\chi_j$ es constante en las clases de
conjugación de $G$, los caracteres de $G$ quedan completamente determinados si
conocemos el valor de cada $\chi_j$ en los representantes de las $r$ clases de
conjugación de $G$. Consideramos entonces la \textbf{tabla de caracteres} de
$G$
\begin{center}
\begin{tabular}{|c|cccc|}
\hline 
 & $1$ & $k_{2}$ & $\cdots$ & $k_{r}$\tabularnewline
 & $1$ & $g_{2}$ & $\cdots$ & $g_{r}$\tabularnewline
\hline 
$\chi_{1}$ & $1$ & $1$ & $\cdots$ & $1$\tabularnewline
$\chi_{2}$ & $n_{2}$ & $\chi_{2}(g_{2})$ & $\cdots$ & $\chi_{2}(g_{r})$\tabularnewline
$\vdots$ & $\vdots$ & $\vdots$ & $\ddots$ & $\vdots$\tabularnewline
$\chi_{r}$ & $n_{r}$ & $\chi_{r}(g_{2})$ & $\cdots$ & $\chi_{r}(g_{r})$\tabularnewline
\hline
\end{tabular}
\end{center}
donde los $n_j$ son los grados de las representaciones irreducibles de $G$ y $k_j$ es el tamaño de 
la clase de conjugación del elemento $g_j$ en $G$ para todo $j\in\{1,\dots,r\}$. 

\begin{example}
	Sea $G=\langle g\rangle$ el grupo cíclico de $n$ elementos. Sea $\lambda$ una raíz primitiva
	$n$-ésima de la unidad. Para cada $i$ sea $V_i$ un espacio vectorial de
	dimensión uno con base $\{v\}$. Cada $V_i$ es un $\C[G]$-módulo con 
	\[
		g\cdot v=\lambda^{i-1}v.
	\]
	Además cada $V_i$ es simple pues $\dim V_i=1$. El carácter $\chi_i$ de
	$V_i$ está dado por $\chi_i(g^m)=\lambda^{m(i-1)}$ para todo
	$m\in\{1,\dots,n\}$. Como los $\chi_1,\dots,\chi_n$ son todos distintos y
	$G$ admite $n$ representaciones irreducibles, los $\chi_j$ son los
	caracteres de todas las representaciones irreducibles de $G$. La tabla de
	caracteres es fácil de calcular:
	\begin{center}
		\begin{tabular}{|c|ccccc|}
			\hline 
			& 1 & 1 & 1 & $\cdots$ & 1\tabularnewline
			& $1$ & $g$ & $g^2$ & $\cdots$ & $g^{n-1}$\tabularnewline
			\hline 
			$\chi_{1}$ & $1$ & $1$ & $1$ & $\cdots$ & $1$\tabularnewline
			$\chi_{2}$ & $1$ & $\lambda$ & $\lambda^2$ & $\cdots$ & $\lambda^{n-1}$\tabularnewline
			$\chi_{3}$ & $1$ & $\lambda^2$ & $\lambda^4$ & $\cdots$ & $\lambda^{n-2}$\tabularnewline
			$\vdots$ & $\vdots$ & $\vdots$ & $\vdots$ & $\ddots$ & $\vdots$\tabularnewline
			$\chi_{n}$ & $1$ & $\lambda^{n-1}$ & $\lambda^{n-2}$ & $\cdots$ & $\lambda$\tabularnewline
			\hline
		\end{tabular}
	\end{center}
	Para dar un ejemplo computacional concreto expondremos la tabla de caracteres del grupo cíclico de orden cuatro. 
\begin{lstlisting}
gap> C4 := CyclicGroup(4);;                       
gap> T := CharacterTable(C4);;
gap> Display(T);
CT1

     2  2  2  2  2

       1a 4a 2a 4b

X.1     1  1  1  1
X.2     1 -1  1 -1
X.3     1  A -1 -A
X.4     1 -A -1  A

A = E(4)
  = Sqrt(-1) = i
\end{lstlisting}
Hay varias observaciones que debemos hacer:
\begin{enumerate}
    \item El símbolo \lstinline{E(4)} denota a una raíz cuarta primitiva de la unidad. 
    \item Si bien la función \lstinline{CharacterTable} se usa para calcular la tabla de caracteres de un grupo, 
    esta función calcula algunas otras cosas. Por ejemplo:
\end{enumerate}
\begin{lstlisting}
gap> OrdersClassRepresentatives(T);
[ 1, 4, 2, 4 ]
gap> SizesCentralizers(T);
[ 4, 4, 4, 4 ]
gap> SizesConjugacyClasses(T);
[ 1, 1, 1, 1 ]
\end{lstlisting}
\end{example}

% \begin{proposition}
% 	Sean $G$ y $H$ grupos abelianos finitos. Sean $\chi_1,\dots,\chi_{|G|}$ son las
% 	representaciones irreducibles de $G$ y sean $\eta_1,\dots,\eta_{|H|}$ las
% 	representaciones irreducibles de $H$.  Entonces las 
% 	\[
% 		\alpha_{ij}\colon G\times H\to\C^{\times},\quad
% 		(g,h)\mapsto\chi_i(g)\eta_j(h),
% 	\]
% 	$i\in\{1,\dots,|G|\}$, $j\in\{1,\dots,|H|\}$, forman un conjunto completo
% 	de representantes de las clases de equivalencia de representaciones de
% 	$G\times H$.
% \end{proposition}

% \begin{proof}
% Es fácil demostrar que las $\alpha_{i,j}$ son representaciones. 
% Son todas
% distintas pues si $\alpha_{ij}=\alpha_{kl}$ entonces, en particular,
% \[
% \chi_i(g)=\alpha_{ij}(g,1)=\alpha_{kl}(g,1)=\chi_k(g)
% \]
% para todo $g\in G$, y análogamente $\eta_j=\eta_l$.  Como $|G\times H|=|G||H|$,
% las $\alpha_{ij}$ son todas las representaciones irreducibles de $G\times H$.
% \end{proof}

Como todo grupo abeliano finito es producto directo de grupos abelianos finitos 
y vimos que todo caracter irreducible de un producto directo es producto de caraceres irreducibles, 
es posible calcular tablas de caracteres de grupos abelianos finitos. 

\begin{example}
	Calculemos ahora la tabla de caracteres de $C_2\times C_2=\{1,a,b,ab\}$:
	\begin{center}
		\begin{tabular}{|c|rrrr|}
			\hline 
			& 1 & 1 & 1 & 1\tabularnewline
			& $1$ & $a$ & $b$ & $ab$\tabularnewline
			\hline 
			$\chi_{1}$ & $1$ & $1$ & $1$ & $1$\tabularnewline
			$\chi_{2}$ & $1$ & $1$ & $-1$ & $-1$\tabularnewline
			$\chi_{3}$ & $1$ & $-1$ & $1$ & $-1$\tabularnewline
			$\chi_{4}$ & $1$ & $-1$ & $-1$ & $1$\tabularnewline
			\hline
		\end{tabular}
	\end{center}
	Computacionalmente:
\begin{lstlisting}
gap> Display(CharacterTable(AbelianGroup([2,2])));
CT2

     2  2  2  2  2

       1a 2a 2b 2c

X.1     1  1  1  1
X.2     1 -1  1 -1
X.3     1  1 -1 -1
X.4     1 -1 -1  1
\end{lstlisting}
Obviamente la forma en la que~\GAP~ordena a los caracteres irreducibles de un grupo 
no tiene por qué coincidir con la forma en la que nosotros los ordenamos. 
\end{example}

\begin{example}
	Vimos que el grupo simétrico $\Sym_3$ tiene tres clases de conjugación 
	con representantes $\id$, $(12)$ y $(123)$. La tabla de caracteres es entonces
	\begin{center}
		\begin{tabular}{|c|rrr|}
			\hline
			& $1$ & $3$ & $2$\tabularnewline
			& $1$ & $(12)$ & $(123)$ \tabularnewline
			\hline 
			$\chi_{1}$ & $1$ & $1$ & $1$\tabularnewline
			$\chi_{2}$ & $1$ & $-1$ & $1$ \tabularnewline
			$\chi_{3}$ & $2$ & $0$ & $-1$ \tabularnewline
			\hline
		\end{tabular}
	\end{center}
	¿Cómo fue que calculamos esta tabla de caracteres? Los caracteres de grado
	uno fueron muy fáciles de calcular. Para calcular la tercera fila de la
	tabla podemos utilizar la representación irreducible 
	\[
	(12)\mapsto \begin{pmatrix}-1&1\\0&1\end{pmatrix},
	\quad
	(123)\mapsto \begin{pmatrix}0&-1\\1&-1\end{pmatrix}
	\]
	pues es irreducible y además 
	\begin{align*}
		&\chi_3\left( (12) \right)=\trace\begin{pmatrix}-1&1\\0&1\end{pmatrix}=0,\\
		&\chi_3\left( (123) \right)=\chi_3\left( (12)(23)\right)=\trace\begin{pmatrix}0&-1\\1&-1\end{pmatrix}=-1.
	\end{align*}

	Es importante mencionar que podríamos haber calculado la tercera fila de la
	tabla sin conocer explícitamente la representación irreducible. Podriamos
	por ejemplo usar la representación regular.  
	Sabemos que el carácter de la representación
	regular $L$ está dado por
	\[
		\chi_L(g)=\begin{cases}
			6 & \text{si $g=\id$},\\
			0 & \text{si $g\ne\id$}.
		\end{cases}
	\]
	Luego la ecuación $0=\chi_L\left( (12) \right)=1-1+2\chi_3( (12))$ nos dice
	que $\chi_3( (12))=0$ y la ecuación $0=\chi_L( (123))=1+1+2\chi_3( (123))$
	nos dice que $\chi_3\left( (123) \right)=-1$. 

	Alternativamente podríamos haber usado alguna de las relaciones de
	ortogonalidad. Por ejemplo si $\chi_3( (12) )=a$ y $\chi_3( (123))=b$,
	entonces obtenemos $a=0$ y $b=-1$ al resolver 
	\begin{align*}
		0&=\langle \chi_3,\chi_1\rangle=\frac16(2+3a+2b),\\
		0&=\langle \chi_3,\chi_2\rangle=\frac16(2-3a+2b).
	\end{align*}
	
	Para ver qué es lo que puede obtenerse con la función \lstinline{CharacterTable} hacemos lo siguiente:
\begin{lstlisting}
gap> S3 := SymmetricGroup(3);;
gap> T := CharacterTable(S3);;
gap> Display(T);
CT3

     2  1  1  .
     3  1  .  1

       1a 2a 3a
    2P 1a 1a 3a
    3P 1a 2a 1a

X.1     1 -1  1
X.2     2  . -1
X.3     1  1  1
\end{lstlisting}
Tal como hicimos antes, podemos extraer otra información de la tabla de caracteres calculada:
\begin{lstlisting}
gap> SizesConjugacyClasses(T);
[ 1, 3, 2 ]
gap> SizesCentralizers(T);
[ 6, 2, 3 ]
gap> SizesConjugacyClasses(T);
[ 1, 3, 2 ]
gap> OrdersClassRepresentatives(T);
[ 1, 2, 3 ]
\end{lstlisting}
\end{example}

\begin{example}
	Vamos a calcular la tabla de caracteres de $\Sym_4$. Sabemos que $\Sym_4$ 
	tiene orden $24$ y cinco clases de conjugación 
	\begin{center}
		\begin{tabular}{c|ccccc}
			representante & $\id$ & $(12)$ & $(12)(34)$ & $(123)$ & $(1234)$\tabularnewline
			\hline
			tamaño & $1$ & $6$ & $3$ & $8$ & $6$
		\end{tabular}
	\end{center}

	Como el conmutador $[\Sym_4,\Sym_4]\simeq\Alt_4$ entonces
	$\Sym_4/[\Sym_4,\Sym_4]$ tiene dos elementos y luego $\Sym_4$ tiene
	solamente dos representaciones de grado uno: una es el signo y la otra es
	la representación trivial. Tenemos entonces dos filas de la tabla de caracteres:
	\begin{center}
		\begin{tabular}{|c|rrrrr|}
			\hline
			& $\id$ & $(12)$ & $(12)(34)$ & $(123)$ & $(1234)$\tabularnewline
%			& $1$ & $6$ & $3$ & $8$ & $6$\tabularnewline
			\hline
			$\chi_1$ & $1$ & $1$ & $1$ & $1$ & $1$\tabularnewline
			$\chi_2$ & $1$ & $-1$ & $1$ & $1$ & $-1$\tabularnewline
			\hline
		\end{tabular}
	\end{center}
	Sabemos que existen $n_3,n_4,n_5\in\{2,3,4\}$ tales que
	$24=1+1+n_3^2+n_4^2+n_5^2$. Es fácil ver que $(n_3,n_4,n_5)=(2,3,3)$ es la
	única solución con $n_3\leq n_4\leq n_5$.

	Encontraremos otra representación al usar la acción de $\Sym_4$ en el
	espacio vectorial
	\[
		V=\{(x_1,x_2,x_3,x_4)\in\R^4:x_1+x_2+x_3+x_4=0\},
	\]
	es decir: $g\cdot (x_1,x_2,x_3,x_4)=(x_{g^{-1}(1)},x_{g^{-1}(2)},x_{g^{-1}(3)},x_{g^{-1}(4)})$. 
	Sean 
	\[
		v_1=(1,0,0,-1),
		\quad
		v_2=(0,1,0,-1),
		\quad
		v_3=(0,0,1,-1).
	\]
	Entonces $\{v_1,v_2,v_3\}$ es base de $V$ y 
	\begin{align*}
		&(12)\cdot v_1=v_2,&&
		(12)\cdot v_2=v_1,&&
		(12)\cdot v_3=v_3,\\
		&(1432)\cdot v_1=-v_3,&&
		(1432)\cdot v_2=v_1-v_3,&&
		(1432)\cdot v_3=v_2-v_3.
	\end{align*}
	Como $\Sym_4=\langle (12),(1432)\rangle$ esto es suficiente para conocer la
	acción de cualquier $g\in\Sym_4$ en cualquier $v\in V$.  Esta acción nos da
	una representación $\rho$ de $\Sym_4$ en $V$:
	\[
		\rho_{(12)}=\begin{pmatrix}
			0 & 1 & 0\\
			1 & 0 & 0\\
			0 & 0 & 1
		\end{pmatrix},\quad
		\rho_{(1432)}=\begin{pmatrix}
			0 & 1 & 0\\
			0 & 0 & 1\\
			1 & -1 & -1
		\end{pmatrix}.
	\]
	Calculemos el caracter $\chi$ de $\rho$. Tenemos hasta ahora que
	$\chi(\id)=3$, $\chi\left( (12) \right)=1$, $\chi\left( (1234) \right)=-1$.
	Para calcular el valor de $\chi$ es los $3$-ciclos hacemos por ejemplo 
	\begin{align*}
		&\chi\left( (234) \right)=\chi\left( (12)(1234) \right)=\trace(\rho_{(12)}\rho_{(1234)})=\trace\begin{pmatrix}
			0 & 0 & 1\\
			0 & 1 & 0\\
			1 & -1 & -1
		\end{pmatrix}
		=0.
	\end{align*}
	Similarmente, para calcular el valor de $\chi$ en el producto de dos
	trasposiciones alcanza con observar que 
	\begin{align*}
		&\chi\left( (13)(24) \right)=\chi\left( (1234)(1234) \right)=\trace(\rho_{(1234)}^2)=\trace\begin{pmatrix}
			0 & 0 & 1\\
			1 & -1 & -1\\
			-1 & 2 & 0
		\end{pmatrix}
		=-1.
	\end{align*}
	Veamos que $\chi$ es un carácter irreducible: 
	\[
		\langle \chi,\chi\rangle=\frac{1}{24}(3^2+6+0+6+3)=1.
	\]
	Con lo que tenemos es fácil construir el caracter de otra representación irreducible pues
	$\sgn\otimes\chi$ es irreducible:
	\[
		\langle \sgn\otimes\chi,\sgn\otimes\chi\rangle=\frac{1}{24}(3^2+(-1)^26+(-1)^23+6)=1.
	\]
	Tenemos así cuatro de los cinco caracteres irreducibles de $G$. Nos falta
	uno, digamos $\chi_5$.  Para calcular $\chi_5$ usamos el carácter de la
	representación regular $L$:
	\begin{align*}
		0 &= \chi_L\left( (12) \right)=1+(-1)+3+3(-1)+2\chi_5\left( (12) \right),\\
		0 &= \chi_L\left( (12)(34) \right)=1+1+3(-1)+3(-1)+2\chi_5\left( (12)(34) \right),\\
		0 &= \chi_L\left( (123) \right)=1+1+0+0+2\chi_5\left( (123) \right),\\
		0 &= \chi_L\left( (1234) \right)=1+(-1)+3(-1)+3+2\chi_5\left( (1234) \right)=0,
	\end{align*}
	de donde obtenemos los valores de $\chi_5$. Nos queda así la siguiente tabla:
	\begin{center}
		\begin{tabular}{|c|rrrrr|}
			\hline
			& $\id$ & $(12)$ & $(12)(34)$ & $(123)$ & $(1234)$\tabularnewline
			\hline
			$\chi_1$ & $1$ & $1$ & $1$ & $1$ & $1$\tabularnewline
			$\sgn$ & $1$ & $-1$ & $1$ & $1$ & $-1$\tabularnewline
			$\chi$ & $3$ & $1$ & $-1$ & $0$ & $-1$\tabularnewline
			$\sgn\otimes\chi$ & $3$ & $-1$ & $-1$ & $0$ & $1$\tabularnewline
			$\chi_5$ & $2$ & $0$ & $2$ & $-1$ & $0$\tabularnewline
			\hline
		\end{tabular}
	\end{center}
% 	Dejamos como ejercicio comparar la tabla de caracteres que obtuvimos con la tabla 
% 	caracteres que nos devuelve~\GAP:
% \begin{lstlisting}
% gap> Display(CharacterTable(SymmetricGroup(4)));
% CT4

%      2  3  2  3  .  2
%      3  1  .  .  1  .

%       1a 2a 2b 3a 4a
%     2P 1a 1a 1a 3a 2b
%     3P 1a 2a 2b 1a 4a

% X.1     1 -1  1  1 -1
% X.2     3 -1 -1  .  1
% X.3     2  .  2 -1  .
% X.4     3  1 -1  . -1
% X.5     1  1  1  1  1
% \end{lstlisting}
\end{example}

Dejamos como ejercicio calcular computacionalmente 
la tabla de caracteres de $\Sym_4$ y compararla con el resultado obtenido en el ejemplo anterior. 

\begin{example}
	Calculemos ahora la tabla de caracteres de $\Alt_4$. Este grupo tiene orden $12$ y 
	cuatro clases de conjugación:
	\begin{center}
		\begin{tabular}{c|cccc}
			representante & $\id$ & $(123)$ & $(132)$ & $(123)$\tabularnewline
			\hline
			tamaño & $1$ & $4$ & $4$ & $3$ 
		\end{tabular}
	\end{center}
	Como $[\Alt_4,\Alt_4]=\{\id,(12)(34),(13)(24),(14)(23)\}$,
	$\Alt_4/[\Alt_4,\Alt_4]$ tiene tres elementos. Luego $\Alt_4$ tiene tres
	caracteres irreducibles de grado uno y uno de grado tres. Sea
	$\omega=\exp(2\pi i/3)$ una raíz cúbica primitiva de la unidad. Si $\chi$
	es un carácter no trivial de grado uno, $\chi\left( (123) \right)=\omega^j$
	para algún $j\in\{1,2\}$ y $\chi\left( (132) \right)=\omega^{2j}$. Como 
	$(132)(134)=(12)(34)$ y además 
	las permutaciones $(134)$ y $(123)$ son conjugadas, entonces 
	\[
	\chi_i((12)(34))=\chi_i((132)(134))=\chi_i((132))\chi_i((134))=\omega^3=1
	\]
	para todo $i\in\{1,2\}$. 
	
	Para calcular $\chi_4$ usamos el truco de la representación regular $L$, 
	\begin{align*}
		0&=\chi_L\left( (12)(34) \right)=1+1+1+3\chi_4\left( (12)(34) \right),\\
		0&=\chi_L\left( (123) \right)=1+\omega+\omega^2+3\chi_4\left( (123) \right),\\
		0&=\chi_L\left( (132) \right)=1+\omega+\omega^2+3\chi_4\left( (132) \right),
	\end{align*}
	de donde obtenemos que $\chi_4\left( (123) \right)=\chi_4\left( (132)
	\right)=0$ y $\chi_4\left( (12)(34) \right)=-1$. Logramos así calcular la tabla de caracteres del grupo $\Alt_4$:
	\begin{center}
		\begin{tabular}{|c|rrrr|}
			\hline
			& $\id$ & $(123)$ & $(132)$ & $(12)(34)$\tabularnewline
			\hline
			$\chi_1$ & $1$ & $1$ & $1$ & $1$\tabularnewline
			$\chi_2$ & $1$ & $\omega$ & $\omega^2$ & $1$\tabularnewline
			$\chi_3$ & $1$ & $\omega^2$ & $\omega$ & $1$\tabularnewline
			$\chi_4$ & $3$ & $0$ & $0$ & $-1$\tabularnewline
			\hline
		\end{tabular}
	\end{center}
	Si bien ya sabemos cómo calcular computacionalmente una tabla de caracteres, el caso del grupo $\Alt_4$ 
	involucra raíces cúbicas de la unidad. Por eso conviene ver qué obtendremos al utilizar la computadora:
\begin{lstlisting}
gap> A4 := AlternatingGroup(4);;
gap> T := CharacterTable(A4);;
gap> Display(T);
CT5

     2  2  2  .  .
     3  1  .  1  1

       1a 2a 3a 3b
    2P 1a 1a 3b 3a
    3P 1a 2a 1a 1a

X.1     1  1  1  1
X.2     1  1  A /A
X.3     1  1 /A  A
X.4     3 -1  .  .

A = E(3)^2
  = (-1-Sqrt(-3))/2 = -1-b3
\end{lstlisting}
Como imaginamos, el símbolo \lstinline{E(3)} denota una raíz cubica primitiva de la unidad $\omega$. 
Para ahorrar espacio se utiliza la variable \lstinline{A} para denotar al complejo $\omega^2$ (que vemos escrito como el simbolo \lstinline{E(3)^2}) y
el símbolo \lstinline{/A} para denotar al complejo $\omega$, el inverso multiplicativo de $\omega^2$. 
\end{example}

\begin{example}
	Vamos a calcular la tabla de caracteres de los grupos no abelianos de orden
	$8$. (Salvo isomorfismo hay dos grupos no abelianos de ocho elementos, el
	grupo de cuaterniones y el diedral, pero no vamos a utilizar esta
	información.) Sea $G$ un grupo no abeliano tal que $|G|=8$. Como $Z(G)\ne
	1$ y $G/Z(G)$ no es cíclico (pues $G$ no es abeliano), $|Z(G)|=2$. Como
	además $G/Z(G)$ es abeliano (porque $|G/Z(G)|=4$), tenemos que $1\ne
	[G,G]\subseteq Z(G)$) y luego $[G,G]=Z(G)$. Como $|G/[G,G]|=4$, $G$ admite
	exactamente cuatro representaciones de grado uno. Como además
	$8=1+1+1+1+n_5^2+\cdots+n_r^2$, se concluye que $r=5$ y $n_5=2$.  Sabemos
	entonces que $G$ tiene cinco clases de conjugación, digamos con
	representantes $1,x,a,b,c$, donde $[G,G]=Z(G)=\langle x\rangle$.  La
	ecuación de clases nos dice que las clases de conjugación de $a$, $b$ y $c$
	tienen tienen dos elementos. 

	Sabemos que $G/[G,G]\simeq C_2\times C_2$. Por la
	proposición~\ref{proposition:Lin(G)}, toda representación de grado uno de
	$G$ es de la forma $\chi_j\circ\pi$, donde $\chi_j$ es una representación
	de grado uno de $C_2\times C_2$ y $\pi\colon G\to G/[G,G]$ es el morfismo
	canónico. Esto nos permite calcular gran parte de los valores de los
	caracteres de grado uno:
	\begin{center}
		\begin{tabular}{|c|rrrrr|}
			\hline
			& $1$ & $x$ & $a$ & $b$ & $c$\tabularnewline
			\hline
			$\chi_1$ & $1$ & $1$ & $1$ & $1$ & $1$\tabularnewline
			$\chi_2$ & $1$ & $?$ & $-1$ & $1$ & $-1$\tabularnewline
			$\chi_3$ & $1$ & $?$ & $1$ & $-1$ & $-1$\tabularnewline
			$\chi_4$ & $1$ & $?$ & $-1$ & $-1$ & $1$\tabularnewline
			\hline
		\end{tabular}
	\end{center}
	Como $0=\langle \chi_1,\chi_2\rangle=\frac18(1+x+2+2(-1)+2(-1))$, se
	concluye que $\chi_2(x)=1$. De la misma forma probamos que $\chi_j(x)=1$
	para todo $j\in\{3,4\}$. 
	
	Nos falta calcular el valor del caracter de grado dos. Para eso usamos la
	representación regular $L$. Al resolver el sistema 
	\begin{align*}
		0&=\chi_L(x)=1+1+1+1+2\chi_5(x),\\
		0&=\chi_L(a)=1+1+-1-1+2\chi_5(a),\\
		0&=\chi_L(b)=1-1+1-1+2\chi_5(b),\\
		0&=\chi_L(c)=1-1-1+1+2\chi_5(c),
	\end{align*}
	obtenemos $\chi_5(x)=-2$ y $\chi_5(a)=\chi_5(b)=\chi_5(c)=0$. Luego la
	tabla de caracteres de $G$ es 
	\begin{center}
		\begin{tabular}{|c|rrrrr|}
			\hline
			& $1$ & $x$ & $a$ & $b$ & $c$\tabularnewline
			\hline
			$\chi_1$ & $1$ & $1$ & $1$ & $1$ & $1$\tabularnewline
			$\chi_2$ & $1$ & $1$ & $-1$ & $1$ & $-1$\tabularnewline
			$\chi_3$ & $1$ & $1$ & $1$ & $-1$ & $-1$\tabularnewline
			$\chi_4$ & $1$ & $1$ & $-1$ & $-1$ & $1$\tabularnewline
			$\chi_5$ & $2$ & $-2$ & $0$ & $0$ & $0$\tabularnewline
			\hline
		\end{tabular}
	\end{center}
	
	Para terminar con los ejemplos, primero listamos
	la tabla de caracteres del grupo de cuaterniones:
\begin{lstlisting}
gap> Q8 := QuaternionGroup(8);;
gap> Display(CharacterTable(Q8));
CT6

     2  3  2  2  3  2

       1a 4a 4b 2a 4c
    2P 1a 2a 2a 1a 2a
    3P 1a 4a 4b 2a 4c

X.1     1  1  1  1  1
X.2     1 -1 -1  1  1
X.3     1 -1  1  1 -1
X.4     1  1 -1  1 -1
X.5     2  .  . -2  .
\end{lstlisting}
Ahora listamos la tabla de caracteres del grupo diedral de ocho elementos. Es importante 
observar que la notación utilizada por~\GAP~no coincide con nuestra notación, ya que para nosotros $\D_n$ 
es el diedral de $2n$ elementos. 
\begin{lstlisting}
gap> D4 := DihedralGroup(8);;
gap> Display(CharacterTable(D8));
CT7

     2  3  2  2  3  2

       1a 2a 4a 2b 2c

X.1     1  1  1  1  1
X.2     1 -1  1  1 -1
X.3     1  1 -1  1 -1
X.4     1 -1 -1  1  1
X.5     2  .  . -2  .
\end{lstlisting}
\end{example}


% \begin{example}
%     Supongamos que el grupo $G$ actúa en un conjunto finito $X$. 
%     Si $V$ es el espacio vectorial con base en $\{x:x\in X\}$ tenemos un morfismo
%     $\rho\colon G\to\GL(V)$, $g\mapsto\rho_g$. Observemos que para cada $g\in G$ 
%     la matriz de $\rho_g$ en la base $\{x:x\in X\}$ es
%     &\begin{align*}
%     (\rho_g)_{ij}=\begin{cases}
%     1 & \text{si $g\cdot x_j=x_i$},\\
%     0 & \text{en otro caso}.
%     \end{cases}
%     \shortintertext{En particular,}
%     (\rho_g)_{ii}=\begin{cases}
%     1 & \text{si $x_i\in\Fix(g)$},\\
%     0 & \text{en otro caso}.
%     \end{cases}
%     \end{align*}
%   Descomponemos $V=\oplus_{i=1}^km_iV_i$ para ciertos $m_1,\dots,m_k\geq0$ y los $\C[G]$-módulos simples no isomorfos $V_1,\dots,V_k$. Sin perder
%   generalidad podemos suponer que $V_1$ es la representación trivial. 

