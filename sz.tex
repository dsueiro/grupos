\chapter{El teorema de Schur--Zassenhaus}

\begin{lemma}
	\label{lemma:1cocycle}
	Si $\varphi\colon G\to N$ es un $1$-cociclo con núcleo $K$ entonces
	$\varphi(x)=\varphi(y)$ si y sólo si $xK=yK$. En particular,
	$(G:K)=|\varphi(G)|$. 
\end{lemma}

\begin{proof}
	Si $\varphi(x)=\varphi(y)$ entonces, como 
	\[
		\varphi(x^{-1}y)
		=\varphi(x^{-1})(x^{-1}\cdot\varphi(y))
		=\varphi(x^{-1})(x^{-1}\cdot\varphi(x))
		=\varphi(x^{-1}x)=\varphi(1)
		=1,
	\]
	tenemos $xK=yK$. Recíprocamente, si $x^{-1}y\in K$, entonces, como 
	\[
	1=\varphi(x^{-1}y)=\varphi(x^{-1})(x^{-1}\cdot \varphi(y)),
	\]
	tenemos que $\varphi(y)=x\cdot\varphi(x^{-1})^{-1}$. De acá obtenemos
	$\varphi(x)=\varphi(y)$.

	La segunda afirmación resulta ahora evidente pues $\varphi$ es constante en
	cada coclase de $K$ y toma $(G:K)$ valores distintos. 
\end{proof}

\begin{lemma}
	\label{lemma:d}
	Sea $G$ un grupo finito, $N$ un subgrupo normal abeliano de $G$ y $S,T,U$
	transversales de $N$ en $G$. Sea 
	\[
	d(S,T)=\prod st^{-1}\in N,
	\]
	donde el producto se hace sobre todos los $s\in S$ y $t\in T$ tales que
	$sN=tN$. Valen las siguientes afirmaciones:
	\begin{enumerate}
		\item $d(S,T)d(T,U)=d(S,U)$.
		\item $d(gS,gT)=gd(S,T)g^{-1}$ para todo $g\in G$.
		\item $d(nS,S)=n^{(G:N)}$ para todo $n\in N$.
	\end{enumerate}
\end{lemma}

\begin{proof}
	Si $s\in S$, $t\in T$, $u\in U$ con $sN=tN=uN$ entonces, como $N$ es
	abeliano y $(st^{-1})(tu^{-1})=su^{-1}$, 
	\[
		d(S,T)d(T,U)=\prod (st^{-1})(tu^{-1})=\prod su^{-1}=d(S,U).
	\]

	Como $sN=tN$ si y sólo si $gsN=gtN$ para todo $g\in G$, 
	\[
	g\left(\prod st^{-1}\right)g^{-1}=\prod gst^{-1}g^{-1}=\prod (gs)(gt)^{-1}=d(gS,gT).
	\]

	Por último, como $N$ es normal, $nsN=sN$ para todo $n\in N$. Luego
	\[
		d(nS,S)=\prod (ns)s^{-1}=n^{(G:N)}.
	\]
\end{proof}

\begin{theorem}[Schur--Zassenhaus]
	\index{Schur--Zassenhaus!teorema de}
	\index{Teorema!de Schur--Zassenhaus}
	\label{theorem:SchurZassenhaus:abeliano}
	Sea $G$ un grupo finito y sea $N$ un subgrupo normal abeliano de $G$. Si
	$|N|$ y $(G:N)$ son coprimos, $N$ se complementa en $G$. Si $N$ se
	complementa en $G$, todos los complementos son conjugados.  
\end{theorem}

\begin{proof}
	Sea $T$ un transversal de $N$ en $G$. Sea $\theta\colon G\to N$,
	$\theta(g)=d(gT,T)$. Como $N$ es abeliano, el lema~\ref{lemma:d} implica
	que $\theta$ es un $1$-cociclo, donde $G$ actúa en $N$ por conjugación:
	\begin{align*}
		\theta(xy)&=d(xyT,T)
		=d(xyT,xT)d(xT,T)\\
		&=(xd(yT,T)x^{-1})d(xT,T)=(x\cdot\theta(y))\theta(x).
	\end{align*}

	\begin{claim}
		$\theta|_N\colon N\to N$ es sobreyectiva.
	\end{claim}

	Si $n\in N$, el lema~\ref{lemma:d} implica que
	$\theta(n)=d(nT,T)=n^{(G:N)}$. Pero como los números $|N|$ y $(G:N)$ son
	coprimos, existen $r,s\in\Z$ tales que $r|N|+s(G:N)=1$. Luego
	\[
		n=n^{r|N|+s(G:N)}=(n^s)^{(G:N)}=\theta(n^s).
	\]

	Sea $H=\ker\theta$. Vamos a demostrar que $H$ es un complemento para $N$. 
	Por el ejercicio~\ref{exercise:1cocycle}, $H$ es un
	subgrupo de $G$. Como
	\[
		|N|=|\theta(G)|=(G:H)
	\]
	por el lema~\ref{lemma:1cocycle}, se concluye que $|H|$ divide a $(G:N)$.
	Como $N\cap H$ es un subgrupo de $N$ y de $H$, entonces $N\cap H=1$ pues
	$|N|$ y $(G:N)=|H|$ son coprimos.  Como $|NH|=|N||H|=|G|$, se concluye que
	$G=NH$ y entonces $H$ es un complemento para $N$.

	Veamos ahora que dos complementos para $N$ son conjugados. 
	Sea $K$ un complemento de $N$ en $G$. Como $NK=G$ y $N\cap K=1$, $K$ es un
	transversal para $N$. Sea $m=d(T,K)\in N$. Como $\theta|_N$ es
	sobreyectiva, existe $n\in N$ tal que $\theta(n)=m$. Por el
	lema~\ref{lemma:d}, para todo $k\in K$ se tiene
	\[
	kmk^{-1}=kd(T,K)k^{-1}=d(kT,kK)=d(kT,K)=d(kT,T)d(T,K)=\theta(k)m,
	\]
	y luego $\theta(k)\in N$. Entonces, como $N$ es abeliano,
	$\theta(n^{-1})=m^{-1}$ y luego 
	\begin{align*}
		\theta(nkn^{-1})&=\theta(n)n\theta(kn^{-1})n^{-1}
		=m\theta(kn^{-1})\\
		&=m\theta(k)k\theta(n^{-1})k^{-1}
		=m\theta(k)km^{-1}k^{-1}=1.
	\end{align*}
	Queda demostrado entonces que $nKn^{-1}\subseteq H=\ker\theta$. Como
	$|K|=(G:N)=|H|$, se concluye que $nKn^{-1}=H$.
\end{proof}

En el siguiente teorema vemos que no es necesario suponer que el subgrupo
normal $N$ es abeliano.

\begin{theorem}[Schur--Zassenhaus]
	\index{Schur--Zassenhaus!teorema de}
	\index{Teorema!de Schur--Zassenhaus}
	\label{theorem:SchurZassenhaus}
	Sea $G$ un grupo finito y sea $N$ un subgrupo normal de $G$. Si $|N|$ y
	$(G:N)$ son coprimos entonces $N$ se complementa en $G$.
\end{theorem}

\begin{proof}
	Procederemos por inducción en $|G|$. Si existe un subgrupo propio $K$ de
	$G$ tal que $NK=G$ entonces, como $(K:K\cap N)=(G:N)$ es coprimo con $|N|$,
	es también coprimo con $|K\cap N|$. Como además $K\cap N$ es normal en $K$,
	por hipótesis inductiva, $K\cap N$ se complementa en $K$, y luego existe un
	subgrupo $H$ de $K$ tal que $|H|=(K:K\cap N)=(G:N)$. 
	
	Supongamos entonces que no existe un subgrupo propio $K$ de $G$ tal que
	$NK=G$.  Podemos suponer que $N\ne1$ (de lo contrario, basta tomar $G$ como
	complemento de $N$ en $G$).  Como $N$ está contenido en todo subgrupo
	maximal de $G$ (pues si existe un maximal $M\subsetneq G$ tal que
	$N\not\subseteq M$ entonces $NM=G$), se tiene que $N\subseteq\Phi(G)$. Por
	el teorema de Frattini ~\ref{theorem:Frattini}, $\Phi(G)$ es nilpotente y
	luego $N$ es nilpotente; en particular, $Z(N)\ne1$. Sea $\pi\colon G\to
	G/Z(N)$ el morfismo canónico. Como $N$ es normal en $G$ y $Z(N)$ es
	característico en $N$, $Z(N)$ es normal en $G$.  Además 
	\[
	(\pi(G):\pi(N))=\frac{|\pi(G)|}{|\pi(N)|}=\frac{|G/Z(N)|}{|N/N\cap Z(N)|}=(G:N)
	\]
	es coprimo con $|N|$, y entonces es también coprimo con $|\pi(N)|$. Por hipótesis
	inductiva, $\pi(N)$ admite un complemento en $G/Z(N)$, digamos $\pi(K)$
	para algún subgrupo $K$ de $G$. Luego $G=NK$ pues 
	$\pi(G)=\pi(N)\pi(K)=\pi(NK)$. 
	Como entonces $K=G$ (pues sabíamos que no existe $K$ tal que $G=NK$), 
	$\pi(N)$ es abeliano pues 
	\[
		\pi(Z(N))=\pi(N)\cap\pi(K)=\pi(N)\cap\pi(G)=\pi(N).
	\]
	Luego $N\subseteq Z(N)$ es abeliano y entonces, por el
	teorema~\ref{theorem:SchurZassenhaus:abeliano}, el subgrupo $N$ admite un
	complemento. 
\end{proof}

\begin{theorem}
	\label{theorem:SchurZassenhaus:conjugacion}
	Sea $G$ un grupo finito y sea $N$ un subgrupo normal de $G$ tal que $|N|$ y
	$(G:N)$ son coprimos. Si $N$ es resoluble o $G/N$ es resoluble, todos los
	complementos de $N$ en $G$ son conjugados.
\end{theorem}

%\begin{proof}
%	Sea $G$ un contraejemplo minimal, es decir: existen complementos $K_1$ y
%	$K_2$ a $N$ en $G$ que no son conjugados.
%
%	\begin{claim}
%		$N$ es minimal en $G$.
%	\end{claim}
%
%	Si $M\subseteq N$ es minimal normal en $G$, $M\ne1$ pues $N\ne1$. Sea
%	$\pi\colon G\to G/M$ el morfismo canónico. El grupo $\pi(G)$ contiene un
%	subgrupo normal $\pi(N)$ de índice coprimo con $|\pi(N)|$. Además
%	$\pi(K_1)$ y $\pi(K_2)$ complementan a $\pi(N)$. Como $|G|$ es minimal,
%	$\pi(K_1)$ y $\pi(K_2)$ son conjugados en $\pi(G)$, es decir: existe $x\in G$ tal que 
%	$\pi(K_1)=\pi(xK_2x^{-1})$.
%
%\end{proof}

\begin{proof}
	Sea $G$ un contraejemplo minimal, es decir: existen complementos $K_1$ y
	$K_2$ a $N$ en $G$ tales que $K_1$ y $K_2$ no son conjugados, y $|G|$ toma
	el menor valor posible.

	\begin{claim}
		Todo subgrupo $U$ de $G$ satisface las hipótesis del teorema con
		respecto al subgrupo normal $U\cap N$.
%		Sea $U$ un subgrupo de $G$. Entonces $U$ satisface las hipótesis del
%		teorema con respecto al subgrupo normal $U\cap N$. Si $U$ contiene un
%		complemento $H$ para $N$ en $G$, entonces $H$ complementa a $U\cap N$
%		en $U$.
	\end{claim}
	
	Como $N$ es normal en $G$, $U\cap N$ es normal en $U$. Además $|U\cap N|$ y
	$(U:U\cap N)$ son coprimos pues $|U\cap N|$ divide a $|N|$ y $(U:U\cap
	N)=(UN:N)$ divide a $(G:N)$.  Si $G/N$ es resoluble, entonces $U/U\cap N$
	es resoluble pues $U/U\cap N$ es isomorfo a un subgrupo de $G/N$. Si $N$ es
	resoluble, $U\cap N$ es resoluble.
%
%	Como $|H|$ divide a $|U|$ y $|H|$ es coprimo con $|U\cap N|$, se tiene que
%	$|H|$ divide a $(U:U\cap N)$. Como además $(U:U\cap N)$ divide a
%	$(G:N)=|H|$, se concluye que $|H|=|U:U\cap N|$. Luego $H$ complementa a
%	$U\cap N$ en $U$.

	\begin{claim}
		Si existe un subgrupo normal $L$ de $G$ tal que $\pi(N)$ es normal en
		$\pi(G)$, donde $\pi\colon G\to G/L$ es el morfismo canónico, entonces
		$\pi(G)$ satisface las hipótesis del teorema con respecto a $\pi(N)$.
		En este caso, si $H$ es un complemento para $N$ en $G$, $\pi(H)$ es un
		complemento para $\pi(N)$ en $\pi(G)$.
	\end{claim}

	Si $N$ es resoluble, $\pi(N)$ es resoluble. Si $G/N$ es resoluble,
	$\pi(G)/\pi(N)\simeq G/NL$ es resoluble. Además
	$(\pi(G):\pi(N))=\frac{|G/L|}{|N/N\cap L|}$ divide a $(G:N)$. 
	
	Si $H$ es un complemento para $N$ en $G$, $|\pi(H)|$ y $|\pi(N)|$ son
	coprimos. Luego $\pi(H)$ es un complemento para $\pi(N)$ pues
	$\pi(G)=\pi(N)\pi(H)=\pi(NH)$ y la intersección
	$\pi(N)\cap\pi(H)$ es trivial.

	\begin{claim}
		$N$ es minimal-normal en $G$.
	\end{claim}

	Sea $M\ne1$ normal tal que $M\subseteq N$. Sea $\pi\colon G\to G/M$ el
	morfismo canónico. Vimos que $\pi(G)$ satisface las hipótesis del teorema
	con respecto al subgrupo normal $\pi(N)$. Por la minimalidad de $G$, existe
	$x\in G$ tal que $\pi(xK_1x^{-1})=\pi(K_2)$. Sabemos que el subgrupo
	$U=MK_2$ también satisface las hipótesis del teorema con respecto al
	subgrupo normal $U\cap N$. Como además $xK_1x^{-1}\cup K_2\subseteq U$,
	podemos concluir que $xK_1x^{-1}$ y $K_2$ complementan a $U\cap N$ en $U$.
	Luego $MK_2=G$ pues $xK_1x^{-1}$ y $K_2$ no son conjugados y $G$ es
	minimal. Esto implica que $M=N$ pues 
	\[
		\frac{|K_2|}{|M\cap K_2|}=|MK_2|=|G|=|NK_2|=|N||K_2|.
	\]

	\begin{claim}
		$N$ no es resoluble y $G/N$ es resoluble. 
	\end{claim}
	
	En caso contrario, por el lema~\ref{lemma:minimal_normal} tendríamos que
	$N$ es abeliano ya que $N$ es minimal-normal, y luego tendríamos una
	contradicción al utilizar el teorema~\ref{theorem:SchurZassenhaus:abeliano}
	que implicaría que $K_1$ y $K_2$ son conjugados.

	\medskip
	Sea $p\colon G\to G/N$ el morfismo canónico y sea $S$ tal que $p(S)$
	minimal-normal en $p(G)=G/N$.  Por el lema~\ref{lemma:minimal_normal},
	$p(S)$ un $p$-grupo para algún primo $p$.  Como $G=NK_1=NK_2$ y $N\subseteq
	S$, el lema de Dedekind~\ref{lemma:Dedekind} implica que
	\[
	S=N(S\cap K_1)=N(S\cap K_2).
	\]
	Luego $S\cap K_1$ y $S\cap K_2$
	complementan a $N$ en $S$. En particular $S\cap K_1$ y $S\cap K_2$ son
	$p$-subgrupos de Sylow de $S$ pues 
	\[
		|S\cap K_1|=|S:N|=|S\cap K_2|,
	\]
	y $p$ no divide a $|N|$. 	
	Por el teorema de Sylow, existe $s\in
	S$ tal que \[
	S\cap sK_1s^{-1}=S\cap K_2.
	\]
	En particular $S\ne G$ gracias a la minimalidad de $G$.
	Sea 
	\[
		L=S\cap K_2=S\cap sK_1s^{-1}\ne1.
	\]
	Como $S$ es normal en $G$, $sK_1s^{-1}\cup K_2\subseteq N_G(L)$ (pues $L$
	es normal en $sK_1s^{-1}$ y en $K_2$). Los subgrupos $sK_1s^{-1}\subseteq
	N_G(L)$ y $K_2\subseteq N_G(L)$ complementan a $N\cap N_G(L)$ en $N_G(L)$, y luego
	$N_G(L)=G$ por la minimalidad de $G$ (si $N_G(L)\ne G$ entonces
	$sK_1s^{-1}$ y $K_2$ serían conjugados en $G$ por serlo en $N_G(L)$). Luego
	$L$ es normal en $G$. 
	
	Sea $\pi_L\colon G\to G/L$ el morfismo canónico. Como
	$\pi_L(K_1)$ y $\pi_L(K_2)$ complementan a $\pi_L(N)$ en $\pi_L(G)$, la minimalidad
	de $|G|$ implica que existe $g\in G$ tal que $\pi_L(gK_1g^{-1})=\pi_L(K_2)$, es
	decir: existe $g\in G$ tal que $(gK_1g^{-1})L=K_2L$.  Luego $gK_1g^{-1}\cup
	K_2\subseteq \langle K_2,L\rangle=K_2$ pues $L\subseteq K_2$. Tenemos entonces que
	$gK_1g^{-1}=K_2$, una contradicción por la minimalidad de $G$. 
%	Sea $L$ un subgrupo maximal normal de $G$ tal que $N\subseteq L$. Por
%	definición $L\ne G$. Como $L\cap K_1$ y $L\cap K_2$ complementan a $N$ en
%	$L$, la minimalidad de $G$ implica que existe $x\in G$ tal que 
%	\[ 
%	L\cap K_2=x(L\cap K_1)x^{-1}=L\cap xK_1x^{-1}.
%	\]
%	Sea $D=L\cap K_2$. Como $L$ es normal en $G$, $D$ es normal en $K_2$ y en
%	$xK_1x^{-1}$. Como $K_2$ y $xK_1x^{-1}$ son complementos para 
%	$N$ en $G$ y además
%	$xK_1x^{-1}\cup K_2\subseteq N_G(D)$, la minimalidad de $G$ implica que $N_G(D)=G$.
%	Si $N$ es resoluble, $N\ne1$ (pues de lo contrario $G=H=K$ y no hay nada
%	para demostrar). Sea $L\subseteq N$ un subgrupo minimal normal de $G$. Por
%	el lema~\ref{lemma:minimal_normal}, $L$ es abeliano\dots
%
%	Si $G/N$ es resoluble,\dots 
\end{proof}


	Por el teorema de Feit--Thompson, no es necesario suponer que $N$ o $G/N$
	es resoluble en el teorema~\ref{theorem:SchurZassenhaus:conjugacion}: como
	todo grupo de order impar es resoluble y $|N|$ es coprimo con $(G:N)$,
	alguno de estos grupos tiene orden impar.


Veamos una aplicación a grupos resolubles finitos.

\begin{theorem}
	\label{theorem:solvable_maximal}
	Sea $G$ un grupo finito resoluble y sea $p$ un primo que divide al orden de $G$.
	Entonces existe un maximal $M$ de indice una potencia de $p$.
\end{theorem}

\begin{proof}
	Procederemos por inducción en $|G|$. Si $G$ es un $p$-grupo, el resultado
	es verdadero. Veamos el caso general. Sea $p$ un primo que divide al orden
	de $G$, sea $N$ un subgrupo minimal-normal y sea $\pi\colon G\to G/N$ el
	morfismo canónico. Como $G$ es resoluble, Por el lema~\ref{lemma:minimal_normal}, $N$ es un
	$q$-grupo para algún primo $q$. Como $G/N$ es resoluble, si $p$ divide a
	$(G:N)$ entonces, por hipótesis inductiva, $G/N$ tiene un subgrupo maximal
	$M_1$ de índice una potencia de $p$. Por el teorema de la correspondencia,
	$M=\pi^{-1}(M_1)$ es un subgrupo maximal de $G$ de índice una potencia de
	$p$. Si $p$ no divide a $(G:N)$, $p$ divide a $|N|$ y luego
	$N\in\Syl_p(G)$. Como $N$ es normal en $G$ y $|N|$ es coprimo con $|G/N|$,
	el teorema de Schur--Zassenhaus~\ref{theorem:SchurZassenhaus} implica que
	existe un complemento $K$ a $N$ en $G$, es decir $G=NK$ y $N\cap K=1$. Sea
	$M$ un subgrupo maximal que contiene a $K$. Entonces $(G:M)$ es una
	potencia de $p$.
\end{proof}

Veamos ahora una aplicación a grupos superresolubles finitos.

\begin{definition}
	\index{Grupo!lagrangiano}
	Un grupo finito $G$ se dice \textbf{lagrangiano} si para cada $d$ que
	divide a $|G|$ existe un subgrupo de $G$ de orden $d$.
\end{definition}

\begin{example}
	El grupo $\Alt_4$ no es lagrangiano pues no tiene subgrupos de orden seis.
\end{example}

\begin{theorem}
	Todo grupo finito superresoluble es lagrangiano. 
\end{theorem}

\begin{proof}
	Sea $p$ un primo que divide al orden de $G$.  Como los subgrupos de un
	grupo superresoluble son superresolubles, basta ver que existe un subgrupo
	índice $p$. Como $G$ es resoluble, existe un subgrupo maximal $M$ de índice
	$p^{\alpha}$ por el teorema~\ref{theorem:solvable_maximal}. Como los
	maximales de superresolubles tienen índice primo
	(teorema~\ref{theorem:super_structure}), se concluye que $\alpha=1$.
\end{proof}
