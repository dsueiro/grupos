\chapter{Teoría de Hall para grupos resolubles}
\label{Hall}

Como aplicación del teorema de Schur--Zassenhaus visto en el capítulo \ref{SchurZassenhaus} 
expondremos la teoría de Hall para grupos resolubles. Necesitaremos además 
algunos de los resultados básicos sobre grupos resolubles que vimos en el capítulo 
\ref{HallWielandt}. 
Una presentación elemental
de la teoría de Hall puede consultarse por ejemplo 
en el libro de Hungerford \cite{MR600654}. 

\medskip

\index{$\pi$-número}
\index{$\pi$-grupo}
\index{$\pi$-subgrupo}
Sea $G$ un grupo finito y sea $\pi$ un conjunto de números primos. Diremos que
$G$ es un $\pi$-grupo si todo primo que divide a $|G|$ pertenece a $\pi$.
Obviamente un $\pi$-subgrupo de $G$ es un subgrupo de $G$ que además es un
$\pi$-grupo. Un $\pi$-número es un entero tal que sus divisores primos están en
$\pi$. El complemento de $\pi$ en el conjunto de los números primos será
denotado por $\pi'$. Luego un $\pi'$-número será un entero no divisible por los
primos de $\pi$.

\begin{definition}
	\index{Hall!subgrupo de}
	\index{Subgrupo!de Hall}
	Sea $\pi$ un conjunto de números primos. Un subgrupo $H$ de $G$ se dice un
	\textbf{$\pi$-subgrupo de Hall} si $H$ es $\pi$-subgrupo de $G$ y el índice $(G:H)$
	es un $\pi'$-número.
\end{definition}

\begin{theorem}[Hall]
	\index{Teorema!de Hall}
	\index{Hall!teorema de}
	\label{theorem:HallE}
	Sea $\pi$ un conjunto de primos y sea $G$ un grupo finito resoluble. Entonces
	$G$ tiene un $\pi$-subgrupo de Hall.
%	Si $G$ es un grupo resoluble de orden $nm$ con $(n:m)=1$ entonces $G$
%	contiene un subgrupo de orden $m$.
\end{theorem}

\begin{proof}
	Supongamos que $G$ tiene orden $nm>1$ con $(n:m)=1$. Demostraremos por
	inducción en $|G|$ que 	existe un subgrupo de orden $m$. Sea $K$ un
	subgrupo de $G$ minimal-normal. Sea $\pi\colon G\to G/K$ el morfismo
	canónico.  Como $G$ es resoluble, $K$ es un $p$-grupo por el
	lema~\ref{lemma:minimal_normal}.
	
	Hay dos casos a considerar.  Supongamos primero que $p$ divide a $m$. Como
	$|G/K|<|G|$, por hipótesis inductiva y por la correspondencia, existe un
	subgrupo $J$ de $G$ que contiene a $K$ tal que $\pi(J)$ es un subgrupo de
	$\pi(G)=G/K$ de orden $m/|K|$. Entonces $J$ tiene orden $m$ pues 
	\[
	m/|K|=|\pi(J)|=\frac{|J|}{|K\cap J|}=(J:K).
	\]

	Supongamos ahora que $p$ no divide a $m$. Por hipótesis inductiva y por la
	correspondencia, existe un subgrupo $H$ de $G$ que contiene a $K$ tal que
	$\pi(H)$ es un subgrupo de $G/K$ de orden $m$.  Como $|H|=m|K|$, $K$ es
	normal en $H$ y $|K|$ es coprimo con $|H:K|$, el teorema de
	Schur--Zassenhaus~\ref{theorem:SchurZassenhaus} implica que existe un
	complemento $J$ de $K$ en $H$. Luego $J$ es un subgrupo de $G$ de orden
	$|J|=m$.
\end{proof}

\begin{example}
	El grupo $\Alt_5$ contiene un un $\{2,3\}$-subgrupo de Hall isomorfo a
	$\Alt_4$.
\end{example}

\begin{example}
	El grupo simple $\PSL_3(2)$ de orden $168$ no contiene $\{2,7\}$-subgrupos
	de Hall.
\end{example}

El teorema~\ref{theorem:HallE} dice que para todo conjunto de primos $\pi$
todo grupo finito contiene $\pi$-subgrupos de Hall.

\begin{theorem}[Hall]
	\index{Teorema!de Hall}
	\index{Hall!teorema de}
	\label{theorem:HallC}
	Sea $G$ un grupo finito resoluble y sea $\pi$ un conjunto de primos. Todos los
	$\pi$-subgrupos de Hall de $G$ son conjugados.
\end{theorem}

\begin{proof}
	Podemos suponer que $G\ne1$. Procederemos por inducción en $|G|$.  Sean $H$
	y $K$ dos $\pi$-subgrupos de Hall de $G$. Sea $M$ un subgrupo de $G$
	minimal-normal y sea $\pi\colon G\to G/M$ el morfismo canónico. Como $G$ es
	resoluble, el lema~\ref{lemma:minimal_normal} implica que  $M$ es un
	$p$-grupo para algún primo $p$.  Como $\pi(H)$ y $\pi(K)$ son
	$\pi$-subgrupos de Hall de $G/M$, los subgrupos $\pi(H)$ y $\pi(K)$ con
	conjugados en $G/M$. Luego existe $g\in G$ tal que $gHMg^{-1}=KM$. 

	Hay dos casos a considerar. Supongamos primero que $p\in\pi$. Como $|HM|$ y
	$|KM|$ son $\pi$-números y $|H|=|K|$ es el mayor $\pi$-número que divide al
	orden de $G$, se concluye que $H=HM$ y $K=KM$. En particular, $gHg^{-1}=K$. 

	Supongamos ahora que $p\not\in\pi$. Es evidente que $K$ complementa a $M$ en
	$KM$ pues $K\cap M=1$. Veamos que $gHg^{-1}$ complementa a $M$ en $KM$:
	como $M$ es normal en $G$, 
	\[
	(gHg^{-1})M=gHMg^{-1}=KM,
	\]
	y $gHg^{-1}\cap M=1$ ya que $p\not\in\pi$. Estos complementos tienen que
	ser conjugados por el teorema de
	Schur--Zassenhaus~\ref{theorem:SchurZassenhaus:conjugacion}.
\end{proof}

\begin{corollary}
	Sea $G$ un grupo finito y sea $N$ un subgrupo normal de $G$ de orden $n$.
	Supongamos que $N$ o $G/N$ es resoluble. Si $|G:N|=m$ es coprimo con $n$ y
	$m_1$ divide a $m$, todo subgrupo de $G$ de orden $m_1$ está contenido en
	algún subgrupo de orden $m$.
\end{corollary}

\begin{proof}
	Sea $H$ un complemento para $N$ en $G$. Entonces $|H|=m$. Sea $H_1$
	subgrupo de $G$ tal que $|H_1|=m_1$. 
	%Entonces,
	%como  $G=NH$ y $N\subseteq NH_1$, el lema de Dedekind~\ref{lemma:Dedekind}
	%implica que 
	%\[
	%NH_1=NH_1\cap NH=N(H\cap NH_1).
	%\]
	Como $n$ y $m$ son coprimos, $m_1=|H_1|=|H\cap NH_1|$ pues
	\[
	\frac{|H||N||H_1|}{|H\cap NH_1|}=
	\frac{|H||NH_1|}{|H\cap NH_1|}=|H(NH_1)|=|G|=|NH|=|N||H|.
	\]
	Como $H_1$ y $H\cap NH_1$ son complementos para $N$ en $NH_1$, ambos de
	orden coprimo con $n$, existe $g\in G$ tal que $H_1=g(H\cap NH_1)g^{-1}$. Luego 
	$H_1\subseteq gHg^{-1}$ y entonces $|gHg^{-1}|=m$. 
\end{proof}

