\chapter{El radical de Jacobson}

% todo: agregar las cuentas de las notas viejas 
% arreglar la prueba del otro teorema de Wedderburn, en teoría de álgebras tiene un gap
% arreglar el lema que sigue
% definición de J(A) como la intersección de los ideales a izquierda maximales
% ponemos después la equivalencia con los inversos y vemos que es ideal y que puedo tomar maximales a derecha
% en las notas viejas está la equivalencia con idelaes nilpotentes

\begin{definition}
\index{Radical!de Jacobson}
Si $A$ es un álgebra, el \textbf{radical de Jacobson} $J(A)$ es la intersección de 
los ideales a izquierda maximales de $A$. 
\end{definition}

Como $A$ es un álgebra unitaria, 
$A$ contiene al menos un ideal maximal a izquierda, es decir $J(A)\ne A$.

\begin{proposition}
	\label{pro:radical}
	Sea $A$ un álgebra y sea $a\in A$. Las siguientes afirmaciones son equivalentes:
	\begin{enumerate}
		\item $a\in J(A)$.
		\item Para todo $b\in A$, $1-ab$ tiene inversa a derecha.
		\item Para todo $b\in A$, $1-ab$ es inversible.
		\item $a$ pertenece a la intersección de los ideales a derecha maximales de $A$.
		\item Para todo $b\in A$, $1-ba$ tiene inversa a izquierda.
		\item Para todo $b\in A$, $1-ba$ es inversible.
	\end{enumerate}
\end{proposition}

\begin{proof}
	Demostremos que $(1)\implies(5)$. Sea $b\in A$ tal que $1-ba$ no tiene
	inversa a izquierda. Existe entonces un ideal a izquierda maximal $I$ tal
	que $1-ba\in I$. Como por definición $J(A)\subseteq I$, se
	concluye que $1\in I$, una contradicción.

	Veamos ahora que $(5)\implies (6)$. Si existe $c\in A$ tal que
	$c(1-ba)=1$ entonces 
	\[
	c=1+cba=1-(-cb)a.
	\]
	Como por hipótesis este elemento tiene
	inversa a izquierda, existe $d\in A$ tal que
	$1=dc$. Luego $d=1-ba$ es inversible a derecha. 
	
	La implicación $(6)\implies (5)$ es trivial. 
	
	Veamos que
	$(5)\implies(1)$. Si $a\not\in J(A)$ sea $I$ un ideal a izquierda maximal
	tal que $a\not\in I$. Por maximalidad, $A=I+Aa$. Entonces
	existen $x\in I$ y $b\in A$ tales que $1=x+ba$. Luego
	$x=1-ba\in I$ no tiene inversa a izquierda, pues de lo contrario
	tendríamos $yx=1\in I$ para algún $y\in A$. 

	Análogamente se demuestra que $(2)\Longleftrightarrow (3)\Longleftrightarrow (4)$.

	Para finalizar demostremos que $(3)\Longleftrightarrow (6)$. 
	Si $1-ab$ tiene inversa $c$ entonces, como $(1-ab)c=1$, 
	\[
	1=1-ba+ba=1-ba+b(1-ab)ca=1-ba+bca-babca=(1-ba)(1+bca).
	\]
	Similarmente, si $c(1-ab)=1$, entonces $1=(1+bca)(1-ba)$. 
\end{proof}

La proposición anterior implica que $J(A)$ es un ideal a derecha, pues es también 
la intersección de los ideales a derecha de $A$ maximales. En consecuencia,  
$J(A)$ es un ideal. 

\begin{definition}
\index{Ideal!nilpotente}
Un ideal $I$ de un álgebra se dice \textbf{nilpotente} si $I^m=\{0\}$ para algún $m\in\N$, es decir
si $x_1\cdots x_m=0$ para todo $x_1,\dots,x_m\in I$.
\end{definition}

\begin{proposition} 
	Si $A$ y $B$ son álgebras, valen las siguientes propiedades: 
	\begin{enumerate}
	\item $J(A\times B)=J(A)\times J(B)$.
	\item $J(A/J(A))=\{0\}$.
	\item Si $I$ es un ideal nilpotente de $A$, entonces $I\subseteq J(A)$.
	\end{enumerate}
\end{proposition}

\begin{proof}
Para la primera afirmación:
\begin{align*}
(a,b)\in &J(A\times B)
\Longleftrightarrow
(1,1)-(x,y)(a,b)\text{ es inversible para todo $(x,y)\in A\times B$}\\
&\Longleftrightarrow   
(1-xa,1-yb)\text{ es inversible para 
todo $(x,y)\in A\times B$}\\
&\Longleftrightarrow 
1-xa\text{ y } 
1-yb\text{ son inversibles todo $x\in A$, $y\in B$}\\
&\Longleftrightarrow (a,b)\in J(A)\times J(B).
\end{align*}

Demostremos ahora la segunda afirmación. Sea $\pi\colon A\to A/J(A)$ es el morfismo canónico. 
Sea $\pi(a)\in J(A/J(A))$. Si $a\not\in J(A)$, entonces
existe un ideal a izquierda de $A$ maximal tal que $a\not\in I$. Como por el teorema de la correspondencia 
$\pi(I)$ es un ideal a izquierda
maximal de $A/J(A)$, entonces $\pi(a)\in\pi(I)$, lo que implica en particular 
que existe $y\in I$ tal que
$a-y\in J(A)\subseteq I$, una contradicción pues $a\not\in I$.  

Demostremos la tercera afirmación. Sea $I$ un ideal de $A$ 
tal que $I^m=\{0\}$. Si $a\in A$ y $x\in I$, entonces
$(ax)^m\in I^m=\{0\}$. Entonces $x\in J(A)$ pues $1-ax$ es inversible, ya que 
\[
1=1-(ax)^m=(1+ax+(ax)^2+\cdots+(ax)^{m-1})(1-ax)\qedhere.
\]
\end{proof}

\begin{lemma}
Sea $A$ un álgebra de dimensión finita. Existen finitos ideales a izquierda maximales
$I_1,\dots,I_k$ tales que $J(A)=I_1\cap\cdots \cap I_k$.
\end{lemma}

\begin{proof}
Sea $X$ el conjunto de ideales a izquierda 
formados por intersecciones finitas de ideales a izquierda maximales de $A$. 
Como $A$ tiene ideales maximales a izquierda, $X$ es no vacío. Sea $J=I_1\cap\cdots\cap I_k$ 
un elemento de $X$ de dimensión minimal. Veamos que $J=J(A)$. Como $J(A)$ es la intersección
de los ideales a izquierda maximales de $A$, solamente hay que demostrar que $J(A)\supseteq J$. Si existe
$a\in J\setminus J(A)$, entonces sea $M$ un ideal a izquierda maximal de $A$ tal que 
$a\not\in M$. Pero $J\cap M$ es un ideal a izquierda de $A$ que es intersección 
finita de ideales a izquierda maximales  
y tal que $M\cap J\subsetneq J$, una contradicción a la minimalidad de $\dim J$. 
\end{proof}

\begin{lemma}[Nakayama] 
	\index{Lema!de Nakayama}
	Sea $A$ un álgebra y sea $M$ un $A$-módulo finitamente generado. Si
	$I\subseteq A$ es un ideal tal que $I\subseteq J(A)$ y $I\cdot M=M$
	entonces $M=\{0\}$.
\end{lemma}

\begin{proof}
	Supongamos que $M\ne\{0\}$ y sea $\{m_1,\dots,m_k\}$ un conjunto minimal de generadores del
	módulo $M$. Como $m_k\in M=I\cdot M$, existen $a_1,\dots,a_k\in I$ tales
	que 
	\[
	m_k=a_1\cdot m_1+\cdots+a_k\cdot m_k,
	\]
	es decir: 
	$(1-a_k)\cdot m_k=a_1\cdot m_1+\cdots+a_{k-1}\cdot m_{k-1}$. Como $I\subseteq J(A)$, el elemento 
	$1-a_k$ es inversible. Luego $m_k$ pertenece al submódulo generado
	por $m_1,\dots,m_{k-1}$, una contradicción.
\end{proof}

\begin{proposition}
Si $A$ es un álgebra de dimensión finita, entonces el radical $J(A)$ es un ideal nilpotente. 	
\end{proposition}

\begin{proof}
	Como $A$ tiene dimensión finita, la sucesión de ideales
	\[
	A\supseteq J(A)\supseteq J(A)^2\supseteq\cdots
	\]
	se estabiliza, es decir que existe $m\in\N$ tal que $J(A)^{m+k}=J(A)^m$ para todo $k\in N$. 
	En particular, 
	$J(A)J(A)^m=J(A)^{m+1}\subseteq J(A)^m$. El lema de Nakayama con $I=J(A)$ y el módulo $M=J(A)^m$, 
	que es finitamente generado, 
	implica que $J(A)^m=\{0\}$.  
\end{proof}

\begin{theorem}
Sea $A$ un álgebra de dimensión finita. Las siguientes afirmaciones son equivalentes.
\begin{enumerate}
\item $A$ es semisimple.
\item $J(A)=\{0\}$.
\item $A$ no tiene ideales nilpotentes no nulos.
\end{enumerate}
\end{theorem}

\begin{proof}
	Demostremos que $(1)\implies(2)$. 
	Si $A$ es semisimple, entonces, por el teorema de Wedderburn, 
	$A\simeq M_{n_1}(D_1)\times\cdots\times M_{n_k}(D_k)$ para ciertos
	$n_1,\dots,n_k\in\N$ y ciertas álgebras de división $D_1,\dots,D_k$. 
	Entonces 
	\begin{align*}
	J(A)&\simeq J(M_{n_1}(D_1)\times\cdots\times M_{n_k}(D_k))\\
	&\simeq J(M_{n_1}(D_1))\times\cdots\times J(M_{n_k}(D_k))\simeq\{0\},
	\end{align*}
	pues cada $M_{n_j}(D_j)$ es un álgebra simple.
	
	Demostremos $(2)\implies(1)$. Supongamos entonces que $J(A)=\{0\}$. 
	Vimos que $J(A)=I_1\cap\cdots\cap I_k$ para
	finitos ideales a izquierda maximales $I_1,\dots,I_k$. 
	Como cada $A/I_j$ es un $A$-módulo 
	simple, entonces $(A/I_1)\oplus\cdots\oplus (A/I_1)$ es un $A$-módulo semisimple. 
	El morfismo de $A$-módulos
	\[
	A\to (A/I_1)\oplus\cdots\oplus (A/I_1),
	\quad
	a\mapsto (a+I_1,\dots,a+I_k),
	\]
	tiene núcleo $I_1\cap\dots\cap I_k=J(A)=\{0\}$ y luego 
	es inyectivo. En consecuencia,  
	la representación regular de $A$ es un módulo 
	semisimple, por ser isomorfo a un submódulo de un módulo semisimple. 
	Esto implica que el álgebra $A$ es también semisimple.  
	
	La equivalencia entre $(2)$ y $(3)$ es ahora fácil pues vimos que $J(A)$ es un ideal nilpotente
	que contiene a todo ideal nilpotente de $A$. 	
%	La implicación $(2)\implies(3)$ es trivial, pues vimos que 
%	$J(A)$ contiene a todo ideal nilpotente de $A$. 
%	
%	La implicación $(3)\implies(2)$ también es trivial ahora que
%	sabemos que $J(A)$ es un ideal nilpotente. 
\end{proof}

