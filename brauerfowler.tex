\chapter{El teorema de Brauer--Fowler}

En este capítulo vamos a demostrar el teorema de Brauer--Fowler. 

El resultado es fundamental
en la clasificación de grupos simples finitos. Daremos dos demostraciones, una basada en 
teoría de caracteres y una demostración alternativa completamente elemental. 

Comenzaremos con la demostración que usa teoría de caracteres.

\index{Parte!simétrica}
\index{Parte!antisimétrica}
Sea $\rho\colon G\to\GL(V)$ 
una representación
con caracter $\chi$. Vimos que el $\C[G]$-módulo $V\otimes V$ tiene caracter $\chi^2$. Sea 
$v_1,\dots,v_n$ una base de $V$ y sea 
\[
T\colon V\to V,\quad
v_i\otimes v_j\mapsto v_j\otimes v_i.
\]
Dejamos como ejercicio verificar que $T(v\otimes w)=w\otimes v$ para todo 
$v,w\in V$. Luego la transformación lineal 
$T$ no depende de la base elegida. Observemos que
además $T$ es morfismo de $\C[G]$-módulos, pues
\[
T(g\cdot (v\otimes w))=T((g\cdot v)\otimes (g\cdot w))=(g\cdot w)\otimes (g\cdot v)=g\cdot T(w\otimes v)
\]
para todo $g\in G$ y $v,w\in V$. 
En particular, la \textbf{parte simétrica} 
\begin{gather*}
S(V\otimes V)=\{x\in V\otimes V:T(x)=x\}
\shortintertext{la \textbf{parte antisimétrica}}
A(V\otimes V)=\{x\in V\otimes V:T(x)=-x\}
\end{gather*}
de $V\otimes V$ son ambas 
$\C[G]$-submódulos de $V\otimes V$. Estos nombres están motivados por la siguiente observación 
\[
V\otimes V=S(V\otimes V)\oplus A(V\otimes V).
\]
En efecto, 
$S(V\otimes V)\cap A(V\otimes V)=\{0\}$ pues 
si $x\in S(V\otimes V)\cap A(V\otimes V)$, entonces $x=T(x)$ y $x=-T(x)$ y luego $x=0$. Además 
$V\otimes V=S(V\otimes V)+ A(V\otimes V)$ pues todo $x\in V\otimes V$ puede escribirse como 
\[
x=\frac12(x+T(x))+\frac12(x-T(x))
\]
con $\frac12(x+T(x))\in S(V\otimes V)$ y $\frac12(x-T(x))\in A(V\otimes V)$. 

Veamos que el conjunto $\{v_i\otimes v_j+v_j\otimes v_i:1\leq i,j\leq n\}$ es
base de $S(V\otimes V)$ 
y que el conjunto 
\[
\{v_i\otimes v_j-v_j\otimes v_i:1\leq i<j\leq n\}
\]
es base de $A(V\otimes V)$. Como ambos conjuntos son linealmente independientes, 
entonces 
$\dim S(V\otimes V)\geq n(n+1)/2$ y también 
$\dim A(V\otimes V)\geq n(n-1)/2$. Como además 
\[
n^2=\dim (V\otimes V)=\dim S(V\otimes V)+\dim A(V\otimes V),
\]
se concluye que $\dim S(V\otimes V)=n(n+1)/2$ y que $\dim A(V\otimes V)=n(n-1)/2$. 

\begin{proposition}
    Sea $G$ un grupo finito y  
    sea $V$ un $\C[G]$-módulo de dimensión finita con caracter $\chi$. Si el módulo $S(V\otimes V)$ 
    tiene caracter $\chi_S$ y el módulo $A(V\otimes V)$ tiene caracter $\chi_A$, entonces 
    \begin{align*}
        &\chi_S(g)=\frac12(\chi^2(g)+\chi(g^2)),\\
        &\chi_A(g)=\frac12(\chi^2(g)-\chi(g^2)).
    \end{align*}
\end{proposition}

\begin{proof}
    Sea $g\in G$. Sea $\rho\colon G\to\GL(V)$ la representación asociada al módulo $V$, es decir $\rho(g)(v)=\rho_g(v)=g\cdot v$. 
    Sabemos que $\rho_g$ es diagonalizable. Sea $\{e_1,\dots,e_n\}$ una base de autovectores de $\rho_g$, digamos
    $g\cdot e_i=\lambda_ie_i$ con $\lambda_i\in\C$ para $i\in\{1,\dots,n\}$. En particular, $\chi(g)=\sum_{i=1}^n\lambda_i$. 
    
    Como $\{e_i\otimes e_j-e_j\otimes e_i:1\leq i<j\leq n\}$ es base de $A(V\otimes V)$ y además 
    \[
    g\cdot (e_i\otimes e_j-e_j\otimes e_i)=\lambda_i\lambda_j(e_i\otimes e_j-e_j\otimes e_i),
    \]
    tenemos $\chi_A(g)=\sum_{1\leq i<j\leq n}\lambda_i\lambda_j$. Por otro lado, como $g^2\cdot e_i=\lambda_i^2e_i$ para todo $i$,
    $\chi(g^2)=\sum_{i=1}^n\lambda_i^2$. Luego
    \[
    \chi^2(g)=\chi(g)^2=\sum_{i=1}^n\sum_{j=1}^n\lambda_i\lambda_j=2\sum_{1\leq i<j\leq n}\lambda_i\lambda_j+\sum_{i=1}^n\lambda_i^2=2\chi_A(g)+\chi(g^2).
    \]
    Como además $V\otimes V=S(V\otimes V)\oplus A(V\otimes V)$, se tiene 
    $\chi^2(g)=\chi_S(g)+\chi_A(g)$, es decir 
    $\chi_S(g)=\frac12(\chi^2(g)+\chi(g^2))$.
\end{proof}

\index{Involución}
Una \textbf{involución} en un grupo es un elemento $x\ne 1$ tal que $x^2=1$. 
Es posible la cantidad de involuciones 
con la tabla de caracteres:

\begin{proposition}
Si $G$ es un grupo finito con $t$ involuciones, entonces 
\[
1+t=\sum_{\chi\in\Irr(G)}\langle\chi_S-\chi_A,\chi_1\rangle\chi(1).
\]
\end{proposition}

\begin{proof}
Supongamos que $\Irr(G)=\{\chi_1,\dots,\chi_k\}$, donde $\chi_1$ es el caracter trivial de $G$. 
Para $x\in G$ sea 
\[
\theta(x)=|\{y\in G:y^2=x\}|.
\]
Como $\theta$ es una función de clases
$\theta$ puede escribirse como combinación lineal de los $\chi_j$, digamos
\[
\theta=\sum_{\chi\in\Irr(G)}\langle\theta,\chi\rangle\chi.
\]
Calculamos
\begin{align*}
    \langle\chi_S-\chi_A,\chi_1\rangle 
    &=\frac{1}{|G|}\sum_{g\in G}\chi(g^2)\\
    &=\frac{1}{|G|}\sum_{x\in G}\sum_{\substack{g\in G\\g^2=x}}\chi(g^2)
    =\frac{1}{|G|}\sum_{x\in G}\theta(x)\chi(x)=\langle\theta,\chi\rangle.
\end{align*}
Luego $\theta(x)=\sum_{\chi\in\Irr(G)}\langle\chi_S-\chi_A,\chi_1\rangle\chi$ y el resultado se obtiene
al evaluar esta expresión en $x=1$. 
\end{proof}

\index{Desigualdad!de Cauchy--Schwartz}
Necesitamos un lema:
% Recordemos la desigualdad de Cauchy--Schwartz. Si $x_1,\dots,x_n\in\R$, entonces
% $\sum x_i^2\geq\frac{1}{n}(\sum x_i)^2$. 

\begin{lemma}
Sea $G$ un grupo finito con $k$ clases de conjugación. 
Si $t$ es la cantidad de involuciones de $G$, entonces 
$t^2\leq (k-1)(|G|-1)$. 
\end{lemma}

\begin{proof}
Supongamos que $\Irr(G)=\{\chi_1,\dots,\chi_k\}$, donde $\chi_1$ es el 
carácter trivial de $G$. Primero vamos a demostrar que 
$t\leq\sum_{i=2}^k\chi_i(1)$. En efecto, 
como 
\[
|\langle\chi_S-\chi_A,\chi_1\rangle|
=\frac{1}{|G|}\left|\sum_{g\in G}\chi(g^2)\right|
\leq\frac{1}{|G|}\sum_{g\in G}|\chi(g^2)|\leq 1,
\]
entonces 
\begin{align*}
1+t=\theta(1)
&=\left|\sum_{\chi\in\Irr(G)}\langle\chi_S-\chi_A,\chi_1\rangle\chi(1)\right|\\
&\leq\sum_{\chi\in\Irr(G)}|\langle\chi_S-\chi_A,\chi_1\rangle|\chi(1)
\leq\sum_{\chi\in\Irr(G)}\chi(1),
\end{align*}
de donde se obtiene inmediatamente que $t\leq\sum_{i=2}^k\chi_i(1)$. 
Si utilizamos ahora 
la desigualdad de Cauchy--Schwartz, 
\[
t^2\leq\left(\sum_{i=2}^k\chi_i(1)\right)^2
\leq(k-1)\sum_{i=2}^k\chi(1)^2=(k-1)(|G|-1).\qedhere
\]
\end{proof}

Ahora sí estamos en condiciones de dar la primera demostración del teorema de
Brauer--Fowler. 

\begin{theorem}[Brauer--Fowler]
\index{Teorema!de Brauer--Fowler}
Sea $G$ un grupo finito y simple y sea $x$ una involución. Si $|C_G(x)|=n$, entonces $|G|\leq (n^2)!$	
\end{theorem}

\begin{proof}
Supongamos primero que existe un subgrupo propio $H$ de $G$ tal que
$(G:H)\leq n^2$. En ese caso, hacemos actuar a $G$ en $G/H$ por multiplicación a izquierda 
y tenemos un morfismo de grupos $\rho\colon G\to\Sym_{n^2}$. Como $G$ es un grupo simple, 
$\ker\rho=\{1\}$ o bien $\ker\rho=G$. Si $\ker\rho=G$, entonces $\rho(g)(yH)=yH$ para todo
$g\in G$ e $y\in G$, lo que implica que $g\in H$, una contradicción. Luego $\rho$ es inyectiva
y entonces $G$ es isomorfo a un subgrupo de $\Sym_{n^2}$. En particular, $|G|$ divide a $(n^2)!$

Sea $m=(|G|-1)/t$. 
Como $|C_G(x)|=n$, el grupo $G$ tiene al menos $|G|/n$ involuciones (pues la clase de conjugación
de $x$ tiene tamaño $|G|/n$ y todos sus elementos son involuciones), es decir $t\geq |G|/n$. Luego
$m=(|G|-1)/t<n$. Basta demostrar entonces que $G$ contiene un subgrupo de índice $\leq m^2$. 

Sean $C_1,\dots,C_k$ las clases de conjugación de $G$, donde $C_1=\{1\}$. 
Como $G$ es simple, $|C_i|>1$ 
para todo $i\in\{2,\dots,k\}$. Notar que 
\[
|G|-1\leq\frac{(k-1)(|G|-1)^2}{t^2}\Longleftrightarrow t^2\leq(k-1)(|G|-1),
\]
que vale gracias al lema anterior. 
Si $|C_i|>m$ para todo $i\in\{2,\dots,k\}$, entonces, como
\[
|G|-1\leq\frac{(k-1)(|G|-1)^2}{t^2}=(k-1)m^2,
\]
tendríamos 
\[
|G|-1=\sum_{i=2}^k|C_i|>(k-1)m^2,
\]
una contradicción. Luego existe una clase de conjugación $C$ de $G$ tal que $|C|\leq m^2$. Si $g\in C$, entonces
$C_G(g)$ es un subgrupo de $G$ de índice $|C|\leq m^2$.
\end{proof}

La cota del teorema de Brauer--Fowler no es importante, ya que
para considerar una forma posible de atacar la clasificación de grupos simples 
solamente es necesario saber que existen  
finitos grupos simples finitos con un cierto centralizador de invouciones.

\begin{corollary}
Sea $n\in\N$. Existe (a lo sumo) una cantidad finita de grupos simples 
finitos con una involución con centralizador de orden $n$. 
\end{corollary}

Veamos un ejemplo sencillo que da una idea de cómo es que pueden clasificarse grupos simples
una vez que se tiene fija la estructura del centralizador de una involución. 

\begin{exercise}
Si $G$ es un grupo simple finito y $x$ es una involución con centralizador de orden dos, entonces 
$G\simeq\Z/2$. 
\end{exercise}

% \begin{proof}
% Supongamos que $\Irr(G)=\{\chi_1,\chi_2,\dots,\chi_k\}$. No conocemos la tabla de caracteres del grupo $G$ 
% pero sabemos que...
% \end{proof}

\subsubsection*{Una demostración elemental del teorema de Brauer--Fowler}

% Supongamos primero que existe un subgrupo propio $H$ de $G$ tal que
% $(G:H)\leq n^2$. En ese caso, hacemos actuar a $G$ en $G/H$ por multiplicación a izquierda 
% y tenemos un morfismo de grupos $\rho\colon G\to\Sym_{n^2}$. Como $G$ es un grupo simple, 
% $\ker\rho=\{1\}$ o bien $\ker\rho=G$. Si $\ker\rho=G$, entonces $\rho(g)(yH)=yH$ para todo
% $g\in G$ e $y\in G$, lo que implica que $g\in H$, una contradicción. Luego $\rho$ es inyectiva
% y entonces $G$ es isomorfo a un subgrupo de $\Sym_{n^2}$. En particular, $|G|$ divide a $(n^2)!$
Tal como hicimos en el primer párrafo de la demostración del teorema de Brauer--Fowler, alcanza con
encontrar un subgrupo de índice $\leq 2n^2$. 
Sea $X$ la clase de conjugación de $x$. Para $g\in G$ definimos
\[
J(g)=\{z\in X:zgz^{-1}=g^{-1}\}.
\]
Primero veamos que $|J(g)|\leq|C_G(g)|$. La función $J(g)\to C_G(g)$, $z\mapsto gz$, está bien definida, 
pues 
\[
(gz)g(gz)^{-1}=g(xgx^{-1})g^{-1}=g^{-1}\in C_G(g),
\]
y es inyectiva, pues $gz=gz_1$ implica $z=z_1$.

Sea $\{(g,z)\in G\times X:zgz^{-1}=g^{-1}\}$.  
Como la función $X\times X\to J$, $(y,z)\mapsto (yz,z)$, 
está bien definida, pues $z(yz)z^{-1}=zy=(yz)^{-1}$, y es trivialmente una función inyectiva, 
tenemos entonces que
\[
|X|^2\leq |J|=\sum_{(g,z)\in J}1\leq\sum_{g\in G}|J(g)|=\sum_{g\in G}|C_G(g)|=k|G|,
\]
donde $k$ es la cantidad de clases de conjugación de $G$, 
pues $(g,z)\in J$ si y sólo si $z\in J(g)$. Luego $|G|\leq kn^2$, pues
\[
\left(\frac{|G|}{|C_G(x)|}\right)^2=|X|^2=\frac{|G|}{n^2}\leq k|G|.
\]

\begin{claim}
Existe alguna clase de conjugación que tiene $\leq 2n^2$ elementos.
\end{claim}

De lo contrario, si $C_1,\dots,C_k$ son las clases de conjugación de $G$, donde 
$C_1=\{1\}$ y $|C_i|>2n^2$ para todo $i\in\{2,\dots,k\}$, entonces 
\[
|G|=1+\sum_{i=2}^k|C_i|>1+\sum_{i=2}^kn^2=1+(k-1)2n^2\geq |G|,
\]
una contradicción. 

\begin{claim}
Existe un subgrupo $H$ de $G$ tal que $(G:H)\leq 2n^2$.
\end{claim}

Sea $C$ alguna clase de conjugación de $G$ tal que $|C|\leq 2n^2$ y sea $g\in C$.  
Entonces $H=C_G(g)$ es un subgrupo de $G$ tal que $(G:H)\leq 2n^2$.\qed

\medskip
Este resultado es uno de los primeros pasos hacia la clasificación de grupos simples finitos. 