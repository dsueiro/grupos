\chapter{El teorema del matrimonio de Hall}

Supongamos que tenemos un conjunto finito de personas $p_1,p_2,\dots,p_n$ y cada uno de esas 
personas, digamos la persona $p_i$, se postuló en varios trabajos, digamos $T_i$. 
Nos interesa saber bajo qué condiciones todas 
las personas podrán obtener un trabajo al que se postularon. 

\begin{theorem}[Hall]
El problema tiene solución si y sólo si para 
cada $k\in\{1,\dots,n\}$ cada conjunto de $k$ personas 
se postula en al menos $k$ trabajos.
\end{theorem}

\begin{proof}
    Demostremos primero la implicación fácil. Si existe un conjunto de $k$ personas que se postuló a menos de $k$ trabajos, entonces alguna de esas personas 
    no podrá conseguir trabajo. 
    
    Para demostrar la afirmación recíproca procederemos por 
    inducción en $n$, la cantidad de personas. El caso $n=1$ es trivial. Supongamos entonces que el teorema
    es válido si hay $<n$ personas. Si hay $n$ personas, hay dos casos a considerar. 
    
    Si todo conjunto de $k$ personas, $k<n$, se postula colectivamente al menos a $k+1$ trabajos, entonces la condición de Hall se verifica (y sobrará un trabajo). Elegimos
    cualquier persona para que trabaje donde se haya postulado. Como la condición de Hall vale para las $n-1$ personas restantes, esas personas conseguirán un trabajo al que se postularon 
    gracias a la hipótesis inductiva. 
    
    Si, en cambio, existe un conjunto de $k$ personas que se postula colectivamente a exactamente $k$ trabajos, estas $k$ personas, por hipótesis inductiva, podrán conseguir trabajo. 
    Quedan ahora $n-k$ personas sin trabajo. Para cada $l\leq n-k$, toda colección de $l$ de estas personas se postula al menos a $l$ trabajos 
    (pues de lo contrario, estas $l$ personas junto con las $k$ personas anteriores se hubieran postulado colectivamente a $<l+k$ trabajos, una contradicción). Podemos 
    aplicar entonces la hipótesis inductiva a esas $n-k$ personas y vemos que también podrán conseguir trabajo. 
\end{proof}

El teorema fue demostrado por Hall en 1935, aunque con una prueba distinta. La demostración que
presentamos es básicamente la de Halmos y Vaughan~\cite{MR33330}, aunque allí el teorema se presenta en términos de 
hombres, mujeres y matrimonios. 

\begin{theorem}[Hall]
Sea $G$ un grupo finito y $H\leq G$ tal que $(G:H)=n$. Existen $g_1,\dots,g_n\in G$ tales que
$\{g_1H,\dots,g_nH\}=\{Hg_1,\dots,Hg_n\}$.
\end{theorem}

\begin{proof}
     Supongamos que $\{x_1,\dots,x_n\}$ es un sistema completo de representantes de coclases de $H$ a derecha y 
     $\{y_1,\dots,y_n\}$ es un sistema completo de representantes de coclases de $H$ a izquierda. Para
     cada $i\in\{1,\dots,n\}$ sea 
     \[
     T_i=\{j:y_jH\cap Hx_i\ne\emptyset\}.
     \]
     Si $I\subseteq\{1,\dots,n\}$ es un subconjunto no vacío, sea $J=\cup_{i\in I}T_i$. Si $i\in I$ y $g\in Hx_i$, entonces
     $y\in y_jH$ para algún $j\in\{1,\dots,n\}$. En particular, $j\in A_i$ y además 
     \[
     \bigcup_{i\in I}Hx_i\subseteq\bigcup_{j\in J}y_jH.
     \]
     Como las uniones son disjuntas, al tomar cardinalidad en esta última inclusión y observar que $|H|=|Hx_i|=|y_jH|$ para todo $i,j$, se concluye que
     $|I|\leq |J|$. 

    Por el teorema de Hall, existen elementos distintos $t_1\in T_1,\dots,t_n\in T_n$ tales que $Hx_i\cap y_{t_i}H\ne\emptyset$. Para cada $i\in\{1,\dots,n\}$ sea 
    $g_i\in Hx_i\cap y_{t_i}H$. Entonces $g_1,\dots,g_n$ es un sistema completo de representantes de coclases de $H$ a derecha y a izquierda.
\end{proof}

Para poder demostrar el teorema de Weiss 
necesitamos unos resultados auxiliares sobre coclases dobles. 

\begin{lemma}
     Sean $G$ un grupo finito, 
     $H$ y $K$ subgrupos de $G$ del mismo índice y $x\in G$. Si $\alpha_1,\dots,\alpha_m$ es
     un sistema completo de representantes de $H/(xKx^{-1}\cap H)$, entonces
     \begin{equation}
         \label{eq:Weiss}
         HxK=\bigcup_{i=1}^m\alpha_ixK\quad(\text{unión disjunta}).
     \end{equation}
     En particular, 
     \[
     |HxK|=\frac{|H||K|}{|H\cap xKx^{-1}|}.
     \]
\end{lemma}

\begin{proof}
    Sea $L=xKx^{-1}\cap H$. 
    Primero observemos que la unión es disjunta. Si $\alpha_ixK=\alpha_jxK$, entonces 
    \[
    x^{-1}\alpha_j^{-1}\alpha_ix=(\alpha_jx)^{-1}(\alpha_ix)\in K.
    \]
    Luego $\alpha_j^{-1}\alpha_i\in xKx\cap H=L$, es decir $\alpha_iL=\alpha_jL$, lo que implica $i=j$. 
    
    Veamos ahora
    que $HxK\subseteq\cup_{i=1}^m\alpha_ixK$, ya que la otra inclusión es trivial. 
    Como $H=\cup_{i=1}^m\alpha_iL$, entonces 
    \[
    HxK\subseteq\bigcup_{i=1}^m\alpha_iLxK=\bigcup_{i=1}^m\alpha_ixK,
    \]
    pues $LxK=xK$. 
    
    El tomar cardinalidad en~\eqref{eq:Weiss} obtenemos $|HxK|=m|K|$. 
\end{proof}

\begin{lemma}
Sean $G$ un grupo finito y $x\in G$. Si $H$ y $K$ son subgrupos de $G$, entonces
\[
\#\{yK:yK\subseteq HxK\}=(H:xKx^{-1}\cap H).
\]
%la cantidad de coclases a izquierda de $K$ contenidas en $HxK$ es igual a
%$(H:xKx^{-1}\cap H)$. 
\end{lemma}

\begin{proof}
    Sea $L=xKx^{-1}\cap H$. Consideremos la función 
    \[
    \varphi\colon H/L\to\{yK:yK\subseteq HxK\},
    \quad 
    hL\mapsto hxK.
    \]
    
    Veamos primero que $\varphi$ está bien definida. Si $hL=h_1L$ para $h,h_1\in H$, entonces
    $h_1^{-1}h\in L=xKx^{-1}\cap H$, es decir $h_1^{-1}h=xkx^{-1}$ para algún $k\in K$. Como entonces $(h_1x)^{-1}(hx)=x^{-1}h_1^{-1}hx\in K$, se concluye que
    $(hx)K=(h_1x)K$. 
    
    Claramente, $\varphi$ es sobreyectiva, pues $hxK=\varphi(hL)$ para todo $h\in H$ y $k\in K$. 
    Veamos entonces que $\varphi$ es inyectiva. Si $hxK=h_1xK$, entonces $x^{-1}h_1^{-1}hx\in K$. Luego
    $h_1^{-1}h\in xKx^{-1}\cap H=L$, es decir $h_1L=hL$. 
\end{proof}

Análogamente puede demostrarse que bajo las hipótesis del lema, también se tiene que 
$\#\{Hz:Hz\subseteq HxK\}=(K:x^{-1}Hx\cap K)$.

\begin{theorem}[Weiss]
Sea $G$ un grupo finito y sean $H$ y $K$ subgrupos de $G$ del mismo índice. Entonces
existe un sistema común de representantes de coclases a izquierda de $H$ en $G$ y 
de coclases a derecha de $K$ en $G$. 
\end{theorem}

\begin{proof}
    Primero observamos que las coclases $Hy$ y $zK$ tienen un representante en común si y sólo si 
    $Hy\cap zK\ne\emptyset$ pues 
    \[
    Hx=Hy\text{ y }zK=xK
    \Longleftrightarrow xy^{-1}\in H\text{ y }z^{-1}x\in K
    \Longleftrightarrow x\in Hy\cap zK.
    \]
    Sabemos que $G$ es unión disjunta de $(H,K)$-coclases dobles. 
    
    \begin{claim} 
    Si 
    \[
    HxK=\bigcup_{i=1}^kHy_i=\bigcup_{j=1}^lz_jK,
    \]
    donde las uniones son disjuntas, entonces $k=l$ (pues $H$ y $K$ tienen el mismo orden)  
    y para cada $i\in\{1,\dots,k\}$ se tiene que $Hy_i\cap z_jK\ne\emptyset$ para todo $j\in\{1,\dots,l\}$. 
    \end{claim}
    
    Fijemos $i_0\in\{1,\dots,k\}$. Sin perder generalidad (reordenando, si fuera necesario) podemos suponer que 
    $Hy_{i_0}\cap z_jK\ne\emptyset$ para todo $j\in\{1,\dots,m\}$. Como 
    \[
    Hy_{i_0}\subseteq \bigcup_{i=1}^kHy_i=Hx K=\bigcup_{j=1}^lz_jK,
    \]
    entonces, en particular, $k=\#\{Hy_i:Hy_i\subseteq  HxK\}=(H:xHx^{-1}\cap K)$. 
    Como además $Hy_{i_0}K\subseteq\cup_{j=1}^mz_jK$, 
    entonces 
    \[
    \frac{|H||K|}{|H\cap xKx^{-1}|}=|Hy_{i_0}K|\leq m|K|.
    \]
    Luego se concluye que $k=m$, pues $k=(H:L)\leq m\leq k$. 
\end{proof}