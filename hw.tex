\chapter{Los teoremas de Hall y Wielandt}

\begin{lemma}[Argumento de Frattini]
	\label{lemma:Frattini_argument}
	\index{Frattini!argumento de}
	Sea $G$ un grupo finito y sea $K$ un subgrupo normal de $G$. Si
	$P\in\Syl_p(K)$ para algún primo $p$, entonces $G=KN_G(P)$.
\end{lemma}

\begin{proof}
	Sea $g\in G$. Como $gPg^{-1}\subseteq gKg^{-1}=K$ pues $K$ es normal en $G$
	y además $gPg^{-1}\in\Syl_p(K)$, existe $k\in K$ tal que
	$kPk^{-1}=gPg^{-1}$. Luego $k^{-1}g\in N_G(P)$ pues
	$P=(k^{-1}g)P(k^{-1}g)^{-1}$. Tenemos entonces que $g=k(k^{-1}g)\in
	KN_G(P)$.
\end{proof}

\begin{theorem}[Hall]
	\label{theorem:Hall}
	Sea $G$ un grupo finito tal que todo subgrupo maximal de $G$ tiene índice
	primo o el cuadrado de un primo. Entonces $G$ es resoluble.
\end{theorem}

\begin{proof}
	Procederemos por inducción en $|G|$. Sea $N$ un subgrupo minimal-normal de
	$G$ y sea $p$ el mayor divisor primo de $|N|$. Sean $P\in\Syl_p(N)$ y
	$L=N_G(P)$. Si $L=G$ entonces $P$ es normal en $G$ y luego, como $P$ y
	$G/P$ son resoluble por hipótesis inductiva, $G$ es resoluble. Supongamos
	entonces que $L\ne G$ y sea $M$ un subgrupo maximal que contiene a $L$. 
	Por el argumento de Frattini (lemma~\ref{lemma:Frattini_argument}),  
	$G=NL=NM$. Como $M$ es maximal, 
	existe un primo $q$ tal que 
	\[
	(N:N\cap M)=(G:M)\in\{q,q^2\}
	\]
	pues $(G:M)=|G|/|M|=|NM|/|M|=|N|/|N\cap M|$.  
	Luego $q$ divide a $|N|$ y entonces $q\leq p$; en particular $q\not\equiv
	1\bmod p$. Si $g\in G$ entonces 
	\[
	gPg^{-1}\subseteq gNg^{-1}=N
	\]
	y luego $gPg^{-1}\in\Syl_p(N)$. Al hacer actuar a $G$ por conjugación en
	$\Syl_p(P)$, vemos que la cantidad de $p$-subgrupos de $N$ es entonces
	igual a 
	\[
		(G:N_G(P))=(G:L)\equiv 1\bmod p.	
	\]
	Como $L\subseteq M$ se tiene que $L=N_M(P)$. 
	
	Supongamos que $|P|=p^{\alpha}$.  Como $P\subseteq M$, podemos hacer actuar
	a $P$ en el conjunto $X=\{mPm^{-1}:m\in M\}$. Veamos que $\{P\}$ es la
	única órbita que contiene un único elemento. Si $\{P_1\}$ es una órbita con
	un único elemento, $P_1$ es un subgrupo normal de $\langle P,P_1\rangle$ y
	luego $P_1P$ es un subgrupo de $M$ de orden $p^{\beta}$ con
	$\beta\leq\alpha$. Como $P\subseteq P_1P$, se concluye que $P_1P=P$ y luego
	$P=P_1$. Podemos descomponer al conjunto $X$ como unión
	disjunta de órbitas
	\[
	X=\{P\}\cup O(P_1)\cup\cdots\cup O(P_k),
	\]
	donde $\{P\}$ es la única órbita que contiene solamente un elemento y cada
	$O(P_j)$ tiene cardinal divisible por $p$. 
	Luego 
	\[
		(M:N_M(P))=(M:L)=|X|\equiv1\bmod p.
	\]
	De la igualdad $(G:L)=(G:M)(M:L)$ se concluye que $(G:M)\equiv1\bmod p$.
	Luego $(G:M)=q^2\equiv 1\bmod p$. Como esto implica que $q\equiv -1\bmod
	p$, se concluye que $q=2$ y $p=q+1=3$.

	Como $(N:N\cap M)=4$, al hacer actuar a $N$ en $N/N\cap M$ por
	multiplicación a izquierda tenemos un morfismo no trivial $\rho\colon
	N\to\Sym_4$. Como $[N,N]$ es característico en $N$ y $N$ es normal en $G$,
	por la minimalidad de $N$ hay dos posibilidades: $[N,N]=1$ o
	$[N,N]=N$. Si $[N,N]=N$ entonces
	\[
	\rho(N)=\rho([N,N])=[\rho(N),\rho(N)].
	\]
	Como $\Sym_4$ es resoluble, $\rho(N)$ es resoluble y luego $\rho(N)=1$, una
	contradicción. Luego $[N,N]=1$. Como $N$ es resoluble por ser abeliano
	y $G/N$ es resoluble por hipótesis inductiva, $G$ es resoluble.
%	El grupo $N$ tiene orden $2^a3^b$, y entonces $N$ es resoluble por el
%	teorema de Burnside.  Como $N$ es resoluble y $G/N$ es resoluble por
%	hipótesis inductiva, $G$ es resoluble por el
%	teorema~\ref{theorem:resoluble}. 
\end{proof}

Para terminar esta sección veremos dos resultados que permiten detectar
resolubilidad. Primero necesitamos un lema.

\begin{lemma}
	\label{lemma:4Wielandt}
	Sea $G$ un grupo finito. Sean $H$ y $K$ subgrupos de $G$ tales que de
	índices coprimos. Entonces $G=HK$ y $(H:H\cap K)=(G:K)$.
\end{lemma}

\begin{proof}
	Sea $D=H\cap K$. Como
	\[
	(G:D)=\frac{|G|}{|H\cap K|}=(G:H)(H:H\cap K),
	\]
	$(G:H)$ divide a $(G:D)$. Similarmente obtenemos que $(G:K)$ divide a
	$(G:D)$. Como $(G:H)$ y $(G:K)$ son coprimos, se concluye que $(G:H)(G:K)$
	divide a $(G:D)$. En particular, 
	\[
	\frac{|G|}{|H|}\frac{|G|}{|K|}=(G:H)(G:K)\leq (G:D)=\frac{|G|}{|H\cap K|},
	\]
	que implica $|G|=|HK|$. Como entonces $|G|=|HK|=|H||K|/|H\cap K|$, se
	concluye que $(G:K)=(H:H\cap K)$.
\end{proof}

\index{Clausura normal}
Recordemos que si $H$ es un subgrupo de $G$, la
\textbf{clausura normal} $H^G$ de $H$ en $G$ se define como el subgrupo
\[
	H^G=\langle xHx^{-1}:x\in G\rangle.
\]

\begin{exercise}
	Sea $G$ un grupo y $H$ un subgrupo.  Demuestre que $H^G$ es normal en $G$ y
	que $H^G$ es el (único) menor subgrupo normal de $G$ que contiene a $H$.
\end{exercise}

\begin{svgraybox}
	Es trivial demostrar que $H^G$ es normal en $G$.  Sea $N$ un subgrupo
	normal de $G$ tal que $H\subseteq N$. Como $xHx^{-1}\subseteq xNx^{-1}=N$
	para todo $x\in G$, $H\subseteq H^G\subseteq N$. 
\end{svgraybox}

\begin{example}
	Sea $G=\Alt_4$ y sea $H=\{\id,(12)(34)\}$. La clausura normal de $H$ en $G$
	es el grupo $H^G=\{id,(12)(34),(13)(24),(14)(23)\}\simeq C_2\times C_2$. El
	código:
	\begin{lstlisting}
gap> G := AlternatingGroup(4);;
gap> NormalClosure(G, Subgroup(G, [(1,2)(3,4)]));
Group([ (1,2)(3,4), (1,3)(2,4) ])
gap> StructureDescription(last);
"C2 x C2"
	\end{lstlisting}
\end{example}

\begin{theorem}[Wielandt]
	\label{theorem:Wielandt:solvable}
	Sea $G$ un grupo finito y sean $H,K,L$ subgrupos de $G$ con índices
	coprimos dos a dos. Si $H$, $K$, $L$ son resolubles, entonces $G$ es
	resoluble.
\end{theorem}

%%% TODO: revisar la demostración para escribirla mejor!

\begin{proof}
	Podemos suponer que $G\ne1$. Procederemos por inducción en $|G|$. Sea $N$
	un subgrupo de $G$ minimal-normal y sea $\pi\colon G\to G/N$ el morfismo
	canónico.  Los subgrupos $\pi(H)$, $\pi(K)$, $\pi(L)$ de $\pi(G)=G/N$ son
	resolubles. Los índices de $\pi(H)$, $\pi(K)$ y $\pi(L)$ en $\pi(G)$ son
	coprimos dos a dos pues por ejemplo\footnote{El núcleo de la restricción
	$\ker(\pi|_H)=\ker \pi\cap N$ y entonces $\pi(H)\simeq H/N\cap H$.}
	\[
	(\pi(G):\pi(H))=(G/N:H/N\cap H)=(G:NH)
	\]
	divide a $(G:N)$. Por hipótesis inductiva, $\pi(G)$ es resoluble. Si $H=1$
	entonces $|G|=(G:H)$ es coprimo con $(G:K)$ y luego $G=K$ es resoluble. Si
	$H\ne 1$ sea $M$ un subgrupo minimal-normal de $H$. Por el
	lema~\ref{lemma:minimal_normal}, $M$ es un $p$-grupo para algún primo $p$.
	Sin pérdida de generalidad podemos suponer que el primo $p$ no divide a
	$(G:K)$. Existe entonces $P\in\Syl_p(G)$ tal que $P\subseteq K$. Como los
	subgrupos de Sylow son conjugados, existe $g\in G$ tal que $M\subseteq
	gKg^{-1}$. Como $(G:gKg^{-1})=(G:K)$ es coprimo con $(G:H)$, el
	lema~\ref{lemma:4Wielandt} implica que $G=(gKg^{-1})H$. 
	
	Veamos que todos los conjugados de $M$ están en $gKg^{-1}$. 
	Si $x\in G$ escribimos $x=uv$ con $u\in 
	gKg^{-1}$, $v\in H$ y luego, como $M$ es normal en $H$, 
	\[
	xMx^{-1}=(uv)M(uv)^{-1}=uMu^{-1}\subseteq gKg^{-1}.
	\]
	En particular, $1\ne M^G\subseteq gKg^{-1}$ es resoluble pues $gKg^{-1}$ es
	resoluble. Como por hipótesis inductiva $G/M^G$ es resoluble, se concluye
	que $G$ es resoluble al aplicar el teorema~\ref{theorem:resoluble}.
\end{proof}

\begin{definition}
	\index{$p$-complemento}
	Sea $G$ un grupo finito de orden $p^{\alpha}m$ con $p$ coprimo con $m$. Un
	subgrupo $H$ de $G$ se dice un \textbf{$p$-complemento} si $|H|=m$. 
\end{definition}

\begin{example}
	Sea $G=\Sym_3$. El subgrupo $H=\langle (123)\rangle$ es un $2$-complemento
	y el subgrupo $K=\langle (12)\rangle$ es un $3$-complemento.
\end{example}

Recordemos que un teorema de Burnside afirma que todo grupo finito $G$ de orden
divisible por exactamente dos números primos es resoluble. Este resultado es
necesario para demostrar el siguiente teorema de Hall:

\begin{theorem}[Hall]
	\label{theorem:Hall:solvable}
	Sea $G$ un grupo finito tal que admite un $p$-complemento para todo primo
	que divide al orden de $G$. Entonces $G$ es resoluble.
\end{theorem}

\begin{proof}
	Sea $|G|=p_1^{\alpha_1}\cdots
	p_k^{\alpha_k}$ con los $p_j$ primos distintos. Procederemos por inducción
	en $k$. Si $k=1$ el resultado es cierto pues $G$ es un $p$-grupo. Si $k=2$
	el resultado es válido gracias al teorema de Burnside. Supongamos entonces
	que $k\geq3$. Para cada $j\in\{1,2,3\}$ sea $H_j$ un $p_j$-complemento en
	$G$. Como $|H_j|=|G|/p_j^{\alpha_j}$, los $H_j$ tienen índices coprimos.

	Veamos que $H_1$ es resoluble. Observemos que $|H_1|=p_2^{\alpha_2}\cdots
	p_k^{\alpha_k}$. Sea $p$ un primo que divide a $|H_1|$ y sea $Q$ un
	$p$-complemento en $G$. 
	Como $(G:H_1)$ y $(G:Q)$ son
	coprimos, el lema~\ref{lemma:4Wielandt} implica que 
	\[
	(H_1:H_1\cap Q)=(G:Q)
	\]
	y luego $H_1\cap Q$ es un $p$-complemento en $H_1$.  Luego $H_1$ es
	resoluble por hipótesis inductiva. De la misma forma se demuestra que $H_2$
	y $H_3$ son resolubles.

	Como $H_1$, $H_2$ y $H_3$ son resolubles y tiene índices coprimos, el
	resultado se obtiene al aplicar el teorema de
	Wielandt~\ref{theorem:Wielandt:solvable}.
\end{proof}
