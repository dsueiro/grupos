\chapter{Los teoremas de Hall y Wielandt}

\begin{definition}
	\index{Subgrupo!elemental abeliano}
	Sea $p$ un número primo. Un $p$-grupo $P$ se dice \textbf{elemental
	abeliano} si $x^p=1$ para todo $x\in P$.
\end{definition}

\begin{definition}
	\index{Subgrupo!minimal-normal}
	Un subgrupo $M$ de $G$ se dice \textbf{minimal-normal} si $M\ne1$,
	$M$ es normal en $G$ y el único subgrupo normal de $G$ contenido
	propiamente en $M$ es el trivial. 
\end{definition}

\begin{example}
	Si un subgrupo normal $M$ es minimal (con respecto a la inclusión),
	entonces es minimal-normal. Sin embargo, la recíproca no es cierta.  El
	subgrupo de $\Alt_4$ generado por $(12)(34)$, $(13)(24)$ y $(14)(23)$ es
	normal-minimal en $\Alt_4$ pero no es minimal. 
\end{example}

\begin{exercise}
	Demuestre que todo grupo finito contiene un subgrupo minimal-normal. 
\end{exercise}

\begin{example}
	Sea $G=\D_{6}=\langle r,s:r^6=s^2=1,\,srs=r^{-1}\rangle$ el grupo diedral
	de doce elementos. Los subgrupos $S=\langle r^2\rangle$ 
	y $T=\langle r^3\rangle$ son minimal-normales en $G$.
	% son los únicos!
\end{example}

\begin{example}
	Sea $G=\SL_2(3)$. No es difícil demostrar que el único subgrupo
	minimal-normal de $G$ es el centro $Z(\SL_2(3))\simeq C_2$:
\begin{lstlisting}
gap> List(MinimalNormalSubgroups(SL(2,3)), \
StructureDescription);
[ "C2" ]
\end{lstlisting}
\end{example}

\index{Subgrupo!característico}
Un subgrupo $H$ de $G$ se dice \textbf{característico} si 
$f(H)\subseteq H$ para todo $f\in\Aut(G)$. 
El centro y el conmutador de un grupo son ejemplos de subgrupos característicos. 
Es fácil demostrar que
un subgrupo característico es normal.  

\begin{exercise}
    Si $H$ es característico en $K$ y $K$ es normal en $G$, entonces 
    $H$ es normal en $G$. 
\end{exercise}

El resultado que sigue es muy útil. 

\begin{lemma}
	\label{lemma:minimal_normal}
	Sea $M$ un subgrupo minimal-normal de $G$. Si $M$ es resoluble y finito
	entonces $M$ es un $p$-grupo elemental abeliano para algún primo $p$.
\end{lemma}

\begin{proof}
	Como $M$ es resoluble, $[M,M]\subsetneq M$. Además $[M,M]$ es normal en $G$
	pues $[M,M]$ es característico en $M$ y $M$ es normal en $G$. La 
	minimal-normalidad del subgrupo $M$ implica que $[M,M]=\{1\}$ y luego $M$ es abeliano. 
	
	Si $M$ es finito, existe un primo $p$ tal que $1\ne P=\{x\in
	M:x^p=1\}\subseteq M$.  Como $P$ es característico en $M$, $P$ es normal en
	$G$. Por minimalidad $P=M$.
\end{proof}

\begin{theorem}
	Sea $G$ un grupo finito no trivial resoluble. 
	\begin{enumerate}
		\item Todo subgrupo maximal tiene índice $p^\alpha$ para algún primo $p$. 
		\item Existe un primo $p$ tal que $G$ contiene un $p$-subgrupo
			minimal-normal.
	\end{enumerate}
\end{theorem}

\begin{proof}
	Para demostrar la primera afirmación procederemos por inducción en $|G|$.
	Si $|G|$ es una potencia de un primo no hay nada para demostrar. Supongamos
	entonces que $|G|\geq6$ y sea $M$ un subgrupo maximal de $G$. Sea $N$ un
	subgrupo minimal-normal de $G$ y sea $\pi\colon G\to G/N$ el morfismo
	canónico.  Si $N=G$, entonces $N=G$ es un $p$-grupo y el resultado ya está demostrado. 
	Supongamos entonces que $N\ne G$. 
	Como $M\subseteq NM\subseteq G$,
	se tiene que $M=NM$ o bien $NM=G$ (por maximalidad de $M$).  Si
	$M=NM\supseteq N$, entonces $\pi(M)$ es un subgrupo maximal de $\pi(G)=G/N$
	y luego 
	\[
	(G:M)=(\pi(G):\pi(M))
	\]
	es una potencia de un primo por hipótesis inductiva. 
	Si en cambio $NM=G$ entonces 
	\[
	(G:M)=\frac{|G|}{|M|}=\frac{|NM|}{|N|}=\frac{|N|}{|N\cap M|}
	\]
	es una potencia de un primo pues $N$ es un $p$-grupo por el
	lema anterior. 
	
	La demostración de la segunda afirmación queda como ejercicio
\end{proof}

\begin{example}
	Sea $G=\Sym_4$. El $2$-subgrupo 
	\[
	K=\{\id,(12)(34),(13)(24),(14)(23)\}\simeq
	C_2\times C_2
	\]
	es minimal-normal. Sin embargo, $G$ no posee $3$-subgrupos
	minimal-normales. 
\end{example}

\begin{theorem}
	Sea $G$ un grupo finito no trivial. Entonces $G$ es resoluble si y sólo si
	todo cociente no trivial de $G$ contiene un subgrupo normal abeliano no
	trivial.
\end{theorem}

\begin{proof}
	Todo cociente de $G$ es resoluble y por lo tanto contiene un subgrupo
	minimal-normal, que resulta abeliano. Para demostrar la recíproca
	procederemos por inducción en $|G|$. Sea $N$ un subgrupo normal abeliano de
	$G$. Si $N=G$ entonces $G$ es resoluble por ser abeliano. Si $N\ne G$,
	entonces $|G/N|<|G|$. Como todo cociente de $G/N$ es un cociente de $G$, el
	grupo $G/N$ satisface las hipótesis del teorema. Luego $G/N$ es resoluble
	por hipótesis inductiva y entonces, como $N$ y $G/N$ son resolubles, $G$ es
	resoluble.
\end{proof}

Como aplicación del teorema~\ref{theorem:SchurZassenhaus} de
Schur--Zassenhaus, en el capítulo~\ref{extensiones},
teorema~\ref{theorem:solvable_maximal}, demostraremos que si $G$ es un
grupo finito resoluble no trivial y $p$ es un primo que divide al orden de
$G$, existe un subgrupo maximal de índice una potencia de $p$. 
Otra aplicación del teorema de Schur--Zassenhaus: la teoría de Hall, 
una generalización de
la teoría de Sylow para grupos resolubles. 

\begin{exercise}
	\label{exercise:resoluble}
	Sea $G$ un grupo.  Demuestre que $G$ es resoluble si y sólo si existe una
	sucesión de subgrupos normales
	\[
		1=N_0\subseteq N_1\subseteq\cdots\subseteq N_k=G
	\]
	tales que cada cociente $N_i/N_{i-1}$ es abeliano.
\end{exercise}

% \begin{svgraybox}
% 	Si existe una sucesión de subgrupos normales $1=N_0\subseteq
% 	N_1\subseteq\cdots\subseteq N_k=G$ tales que cada cociente $N_i/N_{i-1}$ es
% 	abeliano, entonces $[N_i,N_i]\subseteq N_{i-1}$ para cada $i$. Demostremos
% 	por inducción que $G^{(m)}\subseteq N_{n-m}$ para todo $m\leq n$. El caso
% 	$m=0$ es trivial pues $G^{(0)}=G\subseteq N_{n}=G$; si suponemos que el
% 	resultado vale para $m$ entonces, por hipótesis inductiva, 
% 	\[
% 	G^{(m+1)}=[G^{(m)},G^{(m)}]\subseteq [N_{n-m},N_{n-m}]\subseteq N_{n-m-1}.
% 	\]
% 	Luego $G^{(n)}\subseteq N_{0}=1$ y $G$ es resoluble.

% 	Supongamos ahora que $G$ es resoluble. Entonces existe $n\in\N$ tal que
% 	$G^{(n)}=1$. Como cada $G^{(i)}$ es característico en $G$, en particular
% 	cada $G^{(i)}$ es normal en $G$. Además 
% 	\[
% 		1=G^{(n)}\subseteq G^{(n-1)}\subseteq\cdots\subseteq G^{(0)}=G.
% 	\]
% \end{svgraybox}


\begin{lemma}[Argumento de Frattini]
	\label{lemma:Frattini_argument}
	\index{Frattini!argumento de}
	Sea $G$ un grupo finito y sea $K$ un subgrupo normal de $G$. Si
	$P\in\Syl_p(K)$ para algún primo $p$, entonces $G=KN_G(P)$.
\end{lemma}

\begin{proof}
	Sea $g\in G$. Como $gPg^{-1}\subseteq gKg^{-1}=K$ pues $K$ es normal en $G$
	y además $gPg^{-1}\in\Syl_p(K)$, existe $k\in K$ tal que
	$kPk^{-1}=gPg^{-1}$. Luego $k^{-1}g\in N_G(P)$ pues
	$P=(k^{-1}g)P(k^{-1}g)^{-1}$. Tenemos entonces que $g=k(k^{-1}g)\in
	KN_G(P)$.
\end{proof}

\begin{theorem}[Hall]
	\label{theorem:Hall}
	Sea $G$ un grupo finito tal que todo subgrupo maximal de $G$ tiene índice
	primo o el cuadrado de un primo. Entonces $G$ es resoluble.
\end{theorem}

\begin{proof}
	Procederemos por inducción en $|G|$. Sea $N$ un subgrupo minimal-normal de
	$G$ y sea $p$ el mayor divisor primo de $|N|$. Sean $P\in\Syl_p(N)$ y
	$L=N_G(P)$. Si $L=G$ entonces $P$ es normal en $G$ y luego, como $P$ y
	$G/P$ son resoluble por hipótesis inductiva, $G$ es resolubles. Supongamos
	entonces que $L\ne G$ y sea $M$ un subgrupo maximal que contiene a $L$. 
	Por el argumento de Frattini (lemma~\ref{lemma:Frattini_argument}),  
	$G=NL=NM$. Como $M$ es maximal, 
	existe un primo $q$ tal que 
	\[
	(N:N\cap M)=(G:M)\in\{q,q^2\}
	\]
	pues $(G:M)=|G|/|M|=|NM|/|M|=|N|/|N\cap M|$.  
	Luego $q$ divide a $|N|$ y entonces $q\leq p$; en particular $q\not\equiv
	1\bmod p$ (pues si $q\equiv1\mod p$, entonces $q\mid p-1$ y luego $p\leq q-1<q\leq p$, una contradicción). 
	Si $g\in G$ entonces 
	\[
	gPg^{-1}\subseteq gNg^{-1}=N
	\]
	y luego $gPg^{-1}\in\Syl_p(N)$. Al hacer actuar a $G$ por conjugación en
	$\Syl_p(P)$, vemos que la cantidad de $p$-subgrupos de $N$ es entonces
	igual a 
	\[
		(G:N_G(P))=(G:L)\equiv 1\bmod p.	
	\]
	Como $L\subseteq M$ se tiene que $L=N_M(P)$. 
	
	%Supongamos que $|P|=p^{\alpha}$.  
	Como $P\subseteq M$, podemos hacer actuar
	a $P$ en el conjunto $X=\{mPm^{-1}:m\in M\}$. Veamos que $\{P\}$ es la
	única órbita que contiene un único elemento. Si $\{P_1\}$ es una órbita con
	un único elemento, digamos $P_1=mPm^{-1}$ para algún $m\in M$, entonces
 	$P=m^{-1}P_1m=P_1$. 
% 	es un subgrupo normal de $\langle P,P_1\rangle$ y
% 	luego $P_1P$ es un subgrupo de $M$ de orden $p^{\beta}$ con
% 	$\beta\leq\alpha$. Como $P\subseteq P_1P$, se concluye que $P_1P=P$ y luego
% 	$P=P_1$. 

    Descomponemos ahora al conjunto $X$ como unión
	disjunta de órbitas
	\[
	X=\{P\}\cup O(P_1)\cup\cdots\cup O(P_k),
	\]
	donde $\{P\}$ es la única órbita que contiene solamente un elemento y cada
	$O(P_j)$ tiene cardinal divisible por $p$. 
	Luego 
	\[
		(M:N_M(P))=(M:L)=|X|\equiv1\bmod p.
	\]
	De la igualdad $(G:L)=(G:M)(M:L)$ se concluye que $(G:M)\equiv1\bmod p$.
	Luego $(G:M)=q^2\equiv 1\bmod p$. Como esto implica que $q\equiv -1\bmod
	p$, se concluye que $q=2$ y $p=q+1=3$.

	Como $(N:N\cap M)=4$, al hacer actuar a $N$ en $N/N\cap M$ por
	multiplicación a izquierda tenemos un morfismo no trivial $\rho\colon
	N\to\Sym_4$. Como $[N,N]$ es característico en $N$ y $N$ es normal en $G$,
	por la minimal-normalidad del subgrupo $N$ hay dos posibilidades: $[N,N]=\{1\}$ o bien 
	$[N,N]=N$. Si $[N,N]=N$ entonces
	\[
	\rho(N)=\rho([N,N])=[\rho(N),\rho(N)].
	\]
	Como $\Sym_4$ es resoluble, $\rho(N)$ es resoluble y luego $\rho(N)=\{1\}$, una
	contradicción. Luego $[N,N]=\{1\}$. Como $N$ es resoluble por ser abeliano
	y $G/N$ es resoluble por hipótesis inductiva, $G$ es resoluble.
%	El grupo $N$ tiene orden $2^a3^b$, y entonces $N$ es resoluble por el
%	teorema de Burnside.  Como $N$ es resoluble y $G/N$ es resoluble por
%	hipótesis inductiva, $G$ es resoluble por el
%	teorema~\ref{theorem:resoluble}. 
\end{proof}

Para terminar esta sección veremos dos resultados que permiten detectar
resolubilidad. Primero necesitamos un lema.

\begin{lemma}
	\label{lemma:4Wielandt}
	Sea $G$ un grupo finito. Sean $H$ y $K$ subgrupos de $G$ tales que de
	índices coprimos. Entonces $G=HK$ y $(H:H\cap K)=(G:K)$.
\end{lemma}

\begin{proof}
	Sea $D=H\cap K$. Como
	\[
	(G:D)=\frac{|G|}{|H\cap K|}=(G:H)(H:H\cap K),
	\]
	$(G:H)$ divide a $(G:D)$. Similarmente obtenemos que $(G:K)$ divide a
	$(G:D)$. Como $(G:H)$ y $(G:K)$ son coprimos, se concluye que $(G:H)(G:K)$
	divide a $(G:D)$. En particular, 
	\[
	\frac{|G|}{|H|}\frac{|G|}{|K|}=(G:H)(G:K)\leq (G:D)=\frac{|G|}{|H\cap K|},
	\]
	que implica $|G|=|HK|$. Como entonces $|G|=|HK|=|H||K|/|H\cap K|$, se
	concluye que $(G:K)=(H:H\cap K)$.
\end{proof}

\index{Clausura normal}
Recordemos que si $H$ es un subgrupo de $G$, la
\textbf{clausura normal} $H^G$ de $H$ en $G$ se define como el subgrupo
\[
	H^G=\langle xHx^{-1}:x\in G\rangle.
\]

\begin{exercise}
	Sea $G$ un grupo y $H$ un subgrupo.  Demuestre que $H^G$ es normal en $G$ y
	que $H^G$ es el (único) menor subgrupo normal de $G$ que contiene a $H$.
\end{exercise}

\begin{svgraybox}
	Es trivial demostrar que $H^G$ es normal en $G$.  Sea $N$ un subgrupo
	normal de $G$ tal que $H\subseteq N$. Como $xHx^{-1}\subseteq xNx^{-1}=N$
	para todo $x\in G$, $H\subseteq H^G\subseteq N$. 
\end{svgraybox}

\begin{example}
	Sea $G=\Alt_4$ y sea $H=\{\id,(12)(34)\}$. La clausura normal de $H$ en $G$
	es el grupo $H^G=\{id,(12)(34),(13)(24),(14)(23)\}\simeq C_2\times C_2$. El
	código:
	\begin{lstlisting}
gap> G := AlternatingGroup(4);;
gap> NormalClosure(G, Subgroup(G, [(1,2)(3,4)]));
Group([ (1,2)(3,4), (1,3)(2,4) ])
gap> StructureDescription(last);
"C2 x C2"
	\end{lstlisting}
\end{example}

\begin{theorem}[Wielandt]
	\label{theorem:Wielandt:solvable}
	Sea $G$ un grupo finito y sean $H,K,L$ subgrupos de $G$ con índices
	coprimos dos a dos. Si $H$, $K$, $L$ son resolubles, entonces $G$ es
	resoluble.
\end{theorem}

%%% TODO: revisar la demostración para escribirla mejor!

\begin{proof}
	Podemos suponer que $G\ne1$. Procederemos por inducción en $|G|$. Sea $N$
	un subgrupo de $G$ minimal-normal y sea $\pi\colon G\to G/N$ el morfismo
	canónico.  Los subgrupos $\pi(H)$, $\pi(K)$, $\pi(L)$ de $\pi(G)=G/N$ son
	resolubles. Los índices de $\pi(H)$, $\pi(K)$ y $\pi(L)$ en $\pi(G)$ son
	coprimos dos a dos pues por ejemplo\footnote{El núcleo de la restricción
	$\ker(\pi|_H)=\ker \pi\cap N$ y entonces $\pi(H)\simeq H/N\cap H$.}
	\[
	(\pi(G):\pi(H))=(G/N:H/N\cap H)=(G:NH)
	\]
	divide a $(G:N)$. Por hipótesis inductiva, $\pi(G)$ es resoluble. Si $H=1$
	entonces $|G|=(G:H)$ es coprimo con $(G:K)$ y luego $G=K$ es resoluble. Si
	$H\ne 1$ sea $M$ un subgrupo minimal-normal de $H$. Por el
	lema~\ref{lemma:minimal_normal}, $M$ es un $p$-grupo para algún primo $p$.
	Sin pérdida de generalidad podemos suponer que el primo $p$ no divide a
	$(G:K)$. Existe entonces $P\in\Syl_p(G)$ tal que $P\subseteq K$. Como los
	subgrupos de Sylow son conjugados, existe $g\in G$ tal que $M\subseteq
	gKg^{-1}$. Como $(G:gKg^{-1})=(G:K)$ es coprimo con $(G:H)$, el
	lema~\ref{lemma:4Wielandt} implica que $G=(gKg^{-1})H$. 
	
	Veamos que todos los conjugados de $M$ están en $gKg^{-1}$. 
	Si $x\in G$ escribimos $x=uv$ con $u\in 
	gKg^{-1}$, $v\in H$ y luego, como $M$ es normal en $H$, 
	\[
	xMx^{-1}=(uv)M(uv)^{-1}=uMu^{-1}\subseteq gKg^{-1}.
	\]
	En particular, $1\ne M^G\subseteq gKg^{-1}$ es resoluble pues $gKg^{-1}$ es
	resoluble. Como por hipótesis inductiva $G/M^G$ es resoluble, se concluye
	que $G$ es resoluble al aplicar el teorema~\ref{theorem:resoluble}.
\end{proof}

\begin{definition}
	\index{$p$-complemento}
	Sea $G$ un grupo finito de orden $p^{\alpha}m$ con $p$ coprimo con $m$. Un
	subgrupo $H$ de $G$ se dice un \textbf{$p$-complemento} si $|H|=m$. 
\end{definition}

\begin{example}
	Sea $G=\Sym_3$. El subgrupo $H=\langle (123)\rangle$ es un $2$-complemento
	y el subgrupo $K=\langle (12)\rangle$ es un $3$-complemento.
\end{example}

Recordemos que un teorema de Burnside afirma que todo grupo finito $G$ de orden
divisible por exactamente dos números primos es resoluble. Este resultado es
necesario para demostrar el siguiente teorema de Hall:

\begin{theorem}[Hall]
	\label{theorem:Hall:solvable}
	Sea $G$ un grupo finito tal que admite un $p$-complemento para todo primo
	que divide al orden de $G$. Entonces $G$ es resoluble.
\end{theorem}

\begin{proof}
	Sea $|G|=p_1^{\alpha_1}\cdots
	p_k^{\alpha_k}$ con los $p_j$ primos distintos. Procederemos por inducción
	en $k$. Si $k=1$ el resultado es cierto pues $G$ es un $p$-grupo. Si $k=2$
	el resultado es válido gracias al teorema de Burnside. Supongamos entonces
	que $k\geq3$. Para cada $j\in\{1,2,3\}$ sea $H_j$ un $p_j$-complemento en
	$G$. Como $|H_j|=|G|/p_j^{\alpha_j}$, los $H_j$ tienen índices coprimos.

	Veamos que $H_1$ es resoluble. Observemos que $|H_1|=p_2^{\alpha_2}\cdots
	p_k^{\alpha_k}$. Sea $p$ un primo que divide a $|H_1|$ y sea $Q$ un
	$p$-complemento en $G$. 
	Como $(G:H_1)$ y $(G:Q)$ son
	coprimos, el lema~\ref{lemma:4Wielandt} implica que 
	\[
	(H_1:H_1\cap Q)=(G:Q)
	\]
	y luego $H_1\cap Q$ es un $p$-complemento en $H_1$.  Luego $H_1$ es
	resoluble por hipótesis inductiva. De la misma forma se demuestra que $H_2$
	y $H_3$ son resolubles.

	Como $H_1$, $H_2$ y $H_3$ son resolubles y tiene índices coprimos, el
	resultado se obtiene al aplicar el teorema de
	Wielandt~\ref{theorem:Wielandt:solvable}.
\end{proof}
