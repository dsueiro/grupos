\chapter{La cantidad de grupos finitos}

En este capítulo mencionaremos algunos problemas y algunos resultados relacionados con la
cantidad de clases de isomorfismo de grupos finitos de un orden dado. 
Este problema de clasificación 
es obviamente casi tan antiguo como la teoría de grupos. Al dar los primeros pasos en teoría de grupos
nos encontramos con algunos resultados fáciles de demostrar:
\begin{itemize}
    \item Existe un único grupo finito de orden primo y es cíclico. 
    \item Existen dos grupos de orden cuatro, ambos abelianos. 
    \item Existen dos grupos de orden seis, uno de ellos es no abeliano. 
    \item Los grupos de orden $p^2$ son abelianos. 
\end{itemize}

Con los teoremas de Sylow se puede ir un poco más lejos. Es fácil demostrar, por ejemplo,
que existe un único grupo de orden 15 y es cíclico. El mismo resultado puede
demostrarse para otros órdenes, por ejemplo 455 y 615. 

Una pregunta surge naturalmente. ¿Para qué valores de $n$ 
existe un único grupo (que obviamente resultará isomorfo a $C_n$) de orden $n$? 
La respuesta fue dada por Burnside. 

\begin{definition}
\index{Número!cíclico}
Un número $n\in\N$ 
se dice \textbf{cíclico} si $C_n$ es el único grupo (salvo isomorfismo) de
orden $n$.
\end{definition}

Algunos ejemplos de números cíclicos: $2$, $3$, $15$ y 
$615=3\cdot 5\cdot 41$. 

\begin{theorem}[Burnside]
\index{Teorema!de Burnside}
	Sea $n\in\N$. Entonces $n$ es cíclico si y sólo si 
	$n$ y $\phi(n)$ son coprimos.
\end{theorem}

\begin{proof}
	Supongamos que $n$ es cíclico.  Sin perder generalidad podemos suponer que
	$n$ es libre de cuadrados (pues de lo contrario, si $n=p^am$ con $m\in\N$,
	$p$ primo tal que $\gcd(p,m)=1$ y $a\geq2$, el grupo $C_m\times C_p^a$ tiene
	orden $n$ y no es cíclico).  Escribimos entonces 
	\[
	n=p_1\cdots p_k
	\]
	con los
	$p_j$ primos distintos y $\phi(n)=(p_1-1)\cdots(p_k-1)$. Si
	$\gcd(n,\phi(n))\ne1$, existen primos distintos $p$ y $q$ tales que $p$ divide
	a $q-1$. El grupo $G= C_m\times (C_p\rtimes C_q)$ tiene orden $n=pqm$ y no
	es cíclico.  

	Supongamos que $\gcd(n,\phi(n))=1$ y que $n$ no es cíclico. Sea $G$ un grupo de
	mínimo orden $n$ no cíclico. 

	Sin perder generalidad podemos suponer que $n$ es libre de cuadrados:
	si $n=p^\alpha m$, $p$ primo, $m\in\N$ coprimo con $p$ y $\alpha\geq2$,
	entonces, como $\phi(n)=p^{\alpha-1}(p-1)\phi(m)$, tendríamos que $p$
	divide a $\gcd(n,\phi(n))$. Luego
	\[
		n=p_1\cdots p_k,
	\]
	con los $p_j$ primos distintos.

	\begin{claim}
	  Todo subgrupo de $G$ y todo cociente de $G$ es cíclico.
	\end{claim}

	Si $m$ divide a $n$ entonces $\gcd(m,\phi(m))=1$ pues $n$ y
	$\phi(n)=(p_1-1)\cdots(p_k-1)$ son coprimos.  Luego todo subgrupo y
	todo cociente propio es cíclico por la minimalidad de $n$.

	\begin{claim}
	  $Z(G)=\{1\}$.
	\end{claim}

	Para cada $i\in\{1,\dots,k\}$ sea $x_i\in G$ un elemento de orden
	$p_i$. Si $G$ fuera abeliano, $G$ sería cíclico: $x_1\cdots x_k$ sería
	un elemento de orden $n$. Luego $Z(G)\ne G$.  Ahora bien, si
	$1<|Z(G)|<n$, entonces $G/Z(G)$ sería cíclico (pues todo cociente de $G$
	lo es) y luego $G$ sería abeliano. 

	\begin{claim}
	  Si $M$ es un subgrupo maximal de $G$ y $x\in M\setminus\{1\}$, entonces 
	  $M=C_G(x)$. En particular, si $M$ y $N$ son subgrupos
	  maximales distintos, entonces $M\cap N=\{1\}$. 
	\end{claim}

	Como $Z(G)\ne\{1\}$, $C_G(x)\ne G$. Y como $M$ es cíclico,
	$M\subseteq C_G(x)$. Luego, por maximalidad, $M=C_G(x)$. 
	Si $M$ y $N$ son dos subgrupos maximales y $x\in M\cap
	N\setminus\{1\}$, entonces $M=N=C_G(x)$. 

	\begin{claim}
	  Si $M$ es un subgrupo maximal, $M=N_G(M)$.   
	\end{claim}

	Sea $x\in N_G(M)\setminus\{1\}$ y sea $\alpha\in\Aut(M)$ dado por
	$y\mapsto xyx^{-1}$. Como $M$ es cíclico, si $m=|M|$ entonces
	$|\Aut(M)|$ tiene orden $\phi(m)$. Por otro lado, como $|x|$ divide a
	$n$, $|\alpha|$ divide a $n$. Luego $|\alpha|$ divide a
	$\gcd(n,\phi(m))=1$. Esto significa que $x\in C_G(M)$, es decir:
	$N_G(M)\subseteq C_G(M)$. Como
	\[
	M\subseteq N_G(M)\subseteq C_G(M)
	\]
	$M$ es maximal y $Z(G)\ne\{1\}$, obtenemos que $M=N_G(M)=C_G(M)$. 

	\medskip
	Sean $M_1,\dots,M_l$ los representantes de las clases de conjugación de
	subgrupos maximales de $G$. Para cada $j\in\{1,\dots,l\}$ sea
	$m_j=|M_j|$.  Como $M_{j}=N_G(M_j)$ para cada $j$, la órbita de $M_j$
	tiene $n/m_j$ elementos. 
	
	Como para cada $g\in G\setminus\{1\}$ existe un único subgrupo maximal
	$M$ tal que $g\in M$, 
	\begin{equation}
	  \label{eq:particion}
	  n=1+\sum_{j=1}^l \frac{n}{m_j}(m_j-1).
	\end{equation}
	Si $l=1$ entonces $n=m_1$, una contradicción. Si $l>1$ entonces, como
	para cada $j$ se tiene que $m_j\geq2$, al reescribir~\eqref{eq:particion}, tenemos
	\begin{align*}
	  \frac{1}{n}+l-1=\sum_{j=1}^l\frac{1}{m_j}\leq\frac{l}{2}.
	\end{align*}
	De esta desigualdad obtenemos $nl\leq 2n-2<2n$ y entonces $l<2$, absurdo. 
%	\begin{claim*}
%	  $G$ es simple.
%	\end{claim*}
%
%	Sea $H$ un subgrupo propio normal de $G$ y sea $m=|H|$. Por la minimalidad de $n$,
%	$H$ es cíclico.  Como $G$ actúa por conjugación en $H$, tenemos un
%	morfismo de grupos $\rho\colon G\to\Aut(H)$, 
%	\begin{align*}
%	  g\mapsto\rho_g\colon H&\to H,\\
%	  h&\mapsto ghg^{-1}.
%	\end{align*}
%	Por el primer teorema de isomorphismos, $G/\ker\rho$ es isomorfo a un
%	subgrupo de $\Aut(H)$. Luego $|G|/|\ker\rho|$ divide a $\phi(m)$. Como
%	además $|G|/|\ker\rho|$ divide a $n$, entonces $|G|/|\ker\rho|$ divide
%	a $(n:\phi(m))=1$. Luego $\rho$ es trivial y entonces $H\subseteq Z(G)=1$. 
%
%	\begin{claim*}
%	  Sean $K$ y $L$ dos subgrupos propios de $G$ maximales y distintos. Entonces 
%	  $K\cap L=1$.
%	\end{claim*}
%
%	Como $K$ y $L$ son cíclicos, son abelianos. Entonces $K,L\subseteq
%	C_G(K\cap L)$. Por maximalidad, $C_G(K\cap L)=G$. Luego $K\cap
%	L\subseteq Z(G)=1$. 
%
%	\begin{claim*}
%		Si $g\in G\setminus\{1\}$ entonces existe un único subgrupo
%		maximal $M$ de $G$ tal que $g\in M$. 
%	\end{claim*}
%
%	Sean $M_1,\dots,M_l$ los subgrupos maximales de $G$. El grupo $G$ actúa por conjugación
%	en $\{M_1,\dots,M_l\}$. Sean $N_1,\dots,N_r$ representantes de las órbitas de la acción. 
%
\end{proof}

Análogamente pueden definirse números abelianos y nilpotentes. Estos números 
están clasificados y una demostración elemental puede consultarse en \cite{MR1786236}. 
También existe la noción de número resoluble. Gracias al teorema de Feit--Thompson 
todo número impar es un número resoluble. 
Estos números también están clasificados, aunque
la demostración es bastante más difícil ya que depende de un teorema muy profundo de Thompson y del 
famoso teorema de Feit--Thompson. 

\medskip
En \cite{MR2410121} se define la función $\gnu(n)$, que devuelve la cantidad de clases de isomorfismo 
de grupos de orden $n$. Por ejemplo, $\gnu(1)=\gnu(2)=\gnu(3)=\gnu(5)=1$ y $\gnu(4)=\gnu(6)=2$. 
El nombre viene de \emph{\textbf{g}roups \textbf{nu}mber}. 

El teorema de Burnside puede reformularse así: 
\[
\gnu(n)=1\Longleftrightarrow\text{$n$ es cíclico}\Longleftrightarrow\gcd(n,\phi(n))=1.
\]

En \cite{MR2410121} Conway, Dietrich y O'Brien caracterizaron los $n\in\N$ tales que
$\gnu(n)=2$, $\gnu(n)=3$, $\gnu(n)=4$. 

En la enciclopedia de sucesiones, 
la sucesión 
\[
\gnu(1),\gnu(2),\gnu(3),\gnu(4)\dots
\]
es \lstinline{A000001}, ver \url{http://oeis.org/A000001} para más información. 

\GAP~contiene una base de datos con todos los grupos de orden $\leq2000$, excepto los 
grupos de orden 1024. La base de datos contiene además otros grupos, por ejemplo
aquellos de orden $p^n$ para todo número primo $p$ y todo $n\leq 6$. La base de datos
fue escrita por Besche, Eick y O'Brien y hoy en día es una herramienta fundamental 
en teoría de grupos \cite{MR1935567}. En particular, esta base de datos nos permite
calcular fácilmente algunos valores de la función $\gnu$. 

La función \lstinline{NrSmallGroups} devuelve la cantidad de clases de isomorfismo de grupos
de un cierto orden. Definimos entonces la función $\gnu$ y calculamos algunos ejemplos:

\begin{lstlisting}
gap> gnu := NrSmallGroups;;
gap> gnu(16);
14
gap> gnu(32);
51
gap> gnu(64);
267
gap> gnu(27);
5
gap> gnu(81);
15
gap> gnu(128);
2328
gap> gnu(512);
10494213
\end{lstlisting}

Se sabe que $\gnu(1024)=49487365422$, aunque este valor no puede obtenerse  
con \GAP~ya que, para ahorrar memoria, 
la base de datos no incluye la inmensa lista de grupos de orden 1024. Los grupos
de orden 1024 fueron clasificados por
Besche, Eick y O'Brien, el anuncio fue hecho en \cite{MR1826989}. 

Más del 99\% de los grupos de orden $<2000$ es de orden $1024$. 
De hecho, como vimos, hay 49487365422 grupos de orden 1024 y la cantidad de clases de isomorfismo 
de grupos de orden $n\ne1024$ con $n<2016$ es 423164131. 

\begin{lstlisting}
gap> Sum([1..1023], gnu)+Sum(List([1025..2015], gnu));
423164131
\end{lstlisting}

Estos números nos dan aproximadamente 99,15\%. 

Estas observaciones sugieren naturalmente la siguiente conjetura,
que parece ser parte del folclore matemático:

\begin{conjecture}
Casi todo grupo finito es un $2$-grupo. 
\end{conjecture}

La numerología que hicimos nos permite evitar tener que hacer precisiones sobre 
qué significa \guillemotleft casi todo grupo finito\guillemotright. Problemas
similares aparecen en el capítulo 22 del libro \cite{MR2382539}.

\begin{problem}
Calcular $\gnu(2048)$. 
\end{problem}

Se sabe que $\gnu(2048)>1774274116992170$, que es la cantidad de subgrupos de orden 2048 
de una cierta clase. 

\begin{conjecture}
Sea $n\in\N$. La sucesión
\[
\gnu(n),\gnu^2(n)=\gnu(\gnu(n)),\gnu^3(n)\dots
\]
se estabiliza en 1. 
\end{conjecture}

No es difícil verificar que la conjetura es verdadera para $n<2000$.

La siguiente conjetura apareció en forma independiente en varios lugares. 
Aparentemente la primera aparición más o menos explícita 
fue alrededor de 1930 y se debe Miller. 
Independientemente MacHale la formuló cuarenta años más tarde. 

\begin{conjecture}
La función $\N\to\N$, $n\mapsto\gnu(n)$ es sobreyectiva. 
\end{conjecture}

Para más información sobre esta conjetura referimos a \cite[\S21.6]{MR2382539}.

Si bien hay muchas conjeturas sobre el comportamiento de la función que cuenta la cantidad de 
clases de isomorfismos de grupos finitos, existen varios 
resultados. 

\begin{theorem}
Si $n\in\N$, entonces $\gnu(n)\leq n^{n\log n}$.
\end{theorem}

La demostración es sencilla y elemental y puede consultarse en el segundo 
capítulo del libro \cite{MR2382539}. 

En el caso de $p$-grupos puede probarse elementalmente que 
\[
\gnu(p^n)\leq p^{\frac16(n^3-n)},
\]
ver \cite[Theorem 5.1]{MR2382539}. Se conocen 
cotas mucho más sofisticada: 

\begin{theorem}[Higman--Sims]
\index{Teorema!de Higman--Sims}
Si $p$ es un número primo y $n\in\N$, entonces 
\[
p^{\frac{2}{27}n^3-O(n^2)}\leq\gnu(p^n)\leq p^{\frac{2}{27}n^3+O(n^{\frac{8}{3}})}.
\]
\end{theorem}

La demostración del teorema anterior resulta de combinar los trabajos de Higman \cite{MR113948} 
y Sims \cite{MR169921}. Una presentación moderna puede encontrarse en el libro
\cite{MR2382539}. 

En \cite{MR123605} Higman conjeturó 
que $\gnu(p^n)$ es una función polinomial de $p$ y $p$ módulo $N$ para una cierta
cantidad finita de enteros $N$. Se conoce como la conjetura PORC.

\begin{conjecture}[Higman]
Sea $n\in\N$. Existen entonces $N$ polinomios 
\[
P_{0}(X),P_{1}(X),\dots,P_{N-1}(X)
\]
tales que
si $p\equiv i\bmod N$, entonces $\gnu(p^n)=P_{i}(p)$. 
\end{conjecture}

PORC viene de \emph{\textbf{P}olynomial \textbf{O}n \textbf{R}esidue \textbf{C}lasses}. 

Se sabe que la conjetura es cierta para $n\leq7$, aunque el problema permanece abierto para $n\geq8$. 
En \cite{MR2921623}, un trabajo de más de setenta páginas, 
du Sautoy y Vaughan--Lee construyeron una familia de grupos
de orden $p^{10}$ que sugiere que la conjetura PORC podría no ser verdadera. De todas formas, la conjetura
PORC de Higman sigue abierta. 

\begin{theorem}[Pyber]
\index{Teorema!de Pyber}
Si $n\in\N$, entonces 
$\gnu(n)\leq n^{\frac{2}{27}\mu(n)^2+O(\mu(n)^{\frac{5}{3}})}$, donde $\mu(n)$ es el mayor exponente
que aparece en la factorización en primos de $n$. 
\end{theorem}

La demostración aparece en \cite{MR1200081} y en el caso de grupos no resolubles utiliza la clasificación de grupos simples. Una presentación detallada puede consultarse en el libro 
\cite{MR2382539} de Blackburn, Neumann y Venkataraman.

