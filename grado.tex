\chapter{El grado de un caracter}

Nuestro objetivo ahora es demostrar el teorema de Frobenius, que afirma
que el grado de una representación irreducible es un divisor del orden del grupo. 

\begin{definition}
	\index{Entero algebraico}
	Sea $\alpha\in\C$. Se dice que $\alpha$ es un \textbf{entero algebraico} si
	$\alpha$ es raíz de un polinomio mónico con coeficientes en $\Z$.  
\end{definition}

Escribiremos $\A$ para denotar al conjunto de enteros algebraicos.  

Toda raíz $n$-ésima de la
unidad es un entero algebraico. Los autovalores de una matriz
$A\in\Z^{n\times n}$ son enteros algebraicos. 

\begin{proposition}
  \label{pro:Z}
  $\Q\cap\A=\Z$.
\end{proposition}

\begin{proof}
  Sea $m/n\in\Q$ con $(m:n)=1$ y $n>0$. Supongamos que $m/n$ es raíz del
  polinomio $t^k+a_{k-1}t^{k-1}+\cdots+a_1t+a_0\in\Z[t]$. Entonces
  \[
	0=(m/n)^k+a_{k-1}(m/n)^{k-1}+\cdots+a_1m/n+a_0.
  \]
  Al multiplicar por $n^k$,
  \[
	0=m^k+a_{k-1}m^{k-1}n+\cdots+a_1mn^{k-1}+a_0n^{k},
  \]
  y entonces $n$ divide a $m^k$ pues podemos escribir 
  \[
	 m^k=-n(a_{k-1}m^{k-1}+\cdots+a_1mn^{k-2}+a_0n^{k-1}). 
	\]
  Como $m$ y $n$ son coprimos se concluye así que $m/n\in\Z$ pues $n\in\{-1,1\}$. 
\end{proof}

\begin{lemma}
  \label{lem:matriz_entera}
  Sea $x\in\C$. Entonces $x\in\A$ si y sólo si $x$ es autovalor de una matriz
  entera. 
\end{lemma}

\begin{proof}
  Supongamos que $x\in\A$, digamos que $x$ es raíz de
  \[
    f=t^n+a_{n-1}t^{n-1}+\cdots+a_1t+a_0\in\Z[t].
  \]
  Entonces $x$ es autovalor de la matriz compañera de $f$:
  \[
    C(f)=\begin{pmatrix}
      0 & 0 & \cdots &0 & -a_0\\
      1 & 0 & \cdots &0 & -a_1\\
      0 & 1 & \cdots &0 & -a_2\\
      \vdots &\vdots & \ddots & \vdots & \vdots\\
      0 & 0 & \cdots &1 & -a_{n-1}
    \end{pmatrix}
  \]
  Recíprocamente, si $x$ es autovalor de una matriz $A\in\Z^{n\times n}$,
  entonces $x$ es raíz del polinomio mónico $f(t)=\det(tI-A)\in\Z^{n\times n}$. 
\end{proof}

\begin{theorem}
  \label{theorem:subanillo}
  $\A$ es un subanillo de $\C$.
\end{theorem}

\begin{proof}
  Sean $\alpha,\beta\in\A$. Gracias el lema anterior podemos suponer que $\alpha$ es autovalor de
  $A\in\Z^{n\times n}$ y que $\beta$ es autovalor de $B\in\Z^{m\times m}$, digamos
  $Av=\alpha v$ y $Bw=\beta w$.  Como 
  \[
	(A\otimes I_{m\times m}+I_{n\times n}\otimes B)(v+w)=(\alpha+\beta)(v+w),
	\quad
	(A\otimes B)(v\otimes w)=\alpha\beta(v\otimes w),
  \]
  se concluye que $\alpha+\beta\in\A$ y que $\alpha\beta\in\A$. 
\end{proof}

\begin{theorem}
  \label{theorem:chi(g)inA}
  Si $\chi$ es un caracter de un grupo finito $G$, entonces $\chi(g)\in\A$ para todo
  $g\in G$.
\end{theorem}

\begin{proof}
  Sea $\rho$ una representación con caracter $\chi$. Sabemos que $\rho_g$
  es diagonalizable con autovalores $\lambda_1,\dots,\lambda_k$. Los
  $\lambda_j$ son enteros algebraicos por ser raíces de la unidad. Luego
  $\chi(g)=\trace\rho_g=\lambda_1+\cdots+\lambda_k\in\A$.
\end{proof}

\begin{lemma}
  \label{lem:combinacion_lineal}
  Sea $x\in\C$. Entonces $x\in\A$ si y sólo si existen $z_1,\dots,z_k\in\C$ no
  todos cero tales que $xz_i=\sum_{j=1}^ka_{ij}z_j$, $a_{ij}\in\Z$, para todo
  $i\in\{1,\dots,k\}$.
\end{lemma}

\begin{proof}
  Supongamos que $x\in\A$ es raíz del polinomio 
  \[
    t^k+a_{k-1}t^{k-1}+\cdots+a_1t+a_0\in\Z[t].
  \]
  Sean $z_i=x^{i-1}$, $i\in\{1,\dots,k\}$. Entonces $xz_i=x^i=z_{i+1}$ para
  todo $i\in\{1,\dots,k-1\}$ y $xz_{k-1}=x^k=-a_{0}-\cdots-a_{k-1}x^{k-1}$.

  Demostremos la otra implicación. Sean $A=(a_{ij})$ y $Z=(z_1,\dots,z_k)^T$. 
  Entonces $AZ=xZ$ pues para cada $i\in\{1,\dots,k\}$ se tiene
  \[
    (AZ)_i=\sum_{j=1}^ka_{ij}z_j=xz_i=(xZ)_i.
  \]
  Como $Z\ne0$, $x$ es autovalor de la matriz $A\in\Z^{k\times k}$. Luego
  $x\in\A$.
\end{proof}

Necesitaremos la siguiente aplicación del lema de Schur.

\begin{lemma}
    Si $V$ es un $\C[G]$-módulo simple y $T\colon V\to V$ es un morfismo de $\C[G]$-módulos, entonces
    $T=\lambda\id$ para algún $\lambda\in\C$. 
\end{lemma}

\begin{proof}
    Como estamos sobre los números complejos, existe un autovalor 
    $\lambda\in\C$ de $T$. Luego $T-\lambda\id$ no es inversible y entonces, 
    como $V$ es simple, el lema de Schur implica que $T-\lambda\id=0$. 
\end{proof}

\begin{theorem}
  \label{theorem:algebraic}
  Sean $G$ un grupo finito, $g\in G$ y $\chi\in\Irr(G)$. Si $K$ es la clase de
  conjugación de $g$ en $G$, entonces
  \[
    \frac{|K|\chi(g)}{\chi(1)}\in\A.
  \]
\end{theorem}

\begin{proof}
  Sea $\phi$ una representación con caracter $\chi$.  Sean $C_1,\dots,C_r$ las
  clases de conjugación de $G$. Para cada $i\in\{1,\dots,r\}$ 
  definimos 
  \[
    T_i=\sum_{x\in C_i}\phi_x.
  \]

  Vamos a demostrar que $T_i=\left(\frac{|C_i|\chi(C_i)}{\chi(1)}\right)\id$,
  donde $\chi(C_i)$ denota el valor de $\chi$ en la clase de conjugación $C_i$. 
  Cada $T_i$
  es un morfismo de representaciones, pues 
  \[
    \phi_g\circ T_i\circ \phi_{g^{-1}}
	=\sum_{x\in C_i}\phi_g
	=\sum_{x\in C_i}\phi_{gxg^{-1}}
	=\sum_{x\in C_i}\phi_x
	=T_i,
  \]
  y entonces el lema de Schur implica que $T_i=\lambda\id$ para algún
  $\lambda\in\C$. 
  
  Calculemos ahora $\lambda$:
  \[
    \chi(1)\lambda 
	=\trace(\lambda\id)
	=\trace(T_i)
	=\sum_{x\in C_i}\trace(\phi_x)
	=\sum_{x\in C_i}\chi(x)
	=\chi(C_i)|C_i|.
  \]


  Veamos ahora que $T_iT_j=\sum_{k=1}^r a_{ijk}T_k$, donde $a_{ijk}\in\N_0$.
  Calculamos
  \[
	T_iT_j=\sum_{x\in C_i}\sum_{y\in C_j}\phi_x\phi_y=\sum_{\substack{x\in C_i\\y\in C_j}}\phi_{xy}=\sum_{g\in G}a_{ijg}\phi_g,
  \]
  donde $a_{ijg}$ es la cantidad de veces que $g$ puede escribirse como $g=xy$
  con $x\in C_i$, $y\in C_j$. 
  Veamos que los $a_{ijg}$ dependen únicamente de la clase de conjugación de
  $g$. En efecto, sea 
  \[
    X_g=\{(x,y)\in C_i\times C_j:xy=g\}.
  \]
  Si $h=kgk^{-1}$, entonces la función 
  \[
	 X_g\to X_h, \quad(x,y)\mapsto
  (kxk^{-1},kyk^{-1}) 
   \]
   está bien definida y es biyectiva con inversa $X_h\to
  X_g$, $(a,b)\mapsto (k^{-1}ak,k^{-1}bk)$. En particular, $|X_g|=|X_h|$. 

  Como los $a_{ijg}$ dependen de la clase de conjugación de $g$,
  \[
	T_iT_j=\sum_{g\in G}a_{ijg}\phi_g
	=\sum_{k=1}^r\sum_{g\in C_k}a_{ijg}\phi_g
	=\sum_{k=1}^ra_{ijk}\sum_{g\in C_k}\phi_g
	=\sum_{k=1}^ra_{ijk}T_k,
  \]
  tal como queríamos demostrar. De esta igualdad obtenemos:
  \begin{equation}
  \label{eq:omega}
    \left(\frac{|C_i|}{\chi(1)}\chi(C_i)\right)
    \left(\frac{|C_j|}{\chi(1)}\chi(C_j)\right)
    =\sum_{k=1}^sa_{ijk}\left(\frac{|C_k|}{\chi(1)}\chi(C_k)\right),
  \end{equation}
  y se concluye que $|C_i|\chi(C_i)/\chi(1)\in\A$ gracias al
  lema~\ref{lem:combinacion_lineal}.
\end{proof}

\begin{theorem}[Frobenius]
  \label{theorem:chi(1)||G|}
  \index{Teorema de!Frobenius}
  \index{Frobenius!teorema de}
  Sean $G$ un grupo finito y $\chi\in\Irr(G)$. Entonces $\chi(1)$ divide al
  orden de $G$.
\end{theorem}

\begin{proof}
  Sea $\phi$ una representación irreducible con caracter $\chi$. Como $\chi$ es
  irreducible, $1=\langle\chi,\chi\rangle$ y entonces 
  \[
    \frac{|G|}{\chi(1)}=\frac{|G|}{\chi(1)}\langle\chi,\chi\rangle=\sum_{g\in G}\frac{\chi(g)}{\chi(1)}\overline{\chi(g)},
  \]
  Sean $C_1,\dots,C_r$ las clases de conjugación de $G$. Entonces
  \[
    \frac{|G|}{\chi(1)}
    =\sum_{i=1}^r\sum_{g\in C_i}\frac{\chi(g)}{\chi(1)}\overline{\chi(g)}
    =\sum_{i=1}^r\left(\frac{|C_i|}{\chi(1)}\chi(C_i)\right)\overline{\chi(C_i)}\in\A,
  \]
  por los teoremas~\ref{theorem:subanillo},~\ref{theorem:chi(g)inA}
  y~\ref{theorem:algebraic}.  Luego $|G|/\chi(1)\in\Q\cap\A=\Z$
  (proposición~\ref{pro:Z}). En particular, $\chi(1)$ divide al orden de $G$.
\end{proof}

\begin{exercise}
	\label{xca:p2_abeliano}
  Sea $p$ un primo. Demuestre que todo grupo de orden $p^2$ es abeliano.
\end{exercise}

\begin{exercise}
	\label{xca:pq}
  Sean $p<q$ primos tales que $q\centernot\equiv1\mod p$. Demuestre que todo grupo de
  orden $pq$ es abeliano.
\end{exercise}

Veamos una aplicación.

\begin{theorem}
    Si $G$ es un grupo finito simple, $\chi(1)\ne 2$ para todo $\chi\in\Irr(G)$. 
\end{theorem}

\begin{proof}
    Sea $\chi\in\Irr(G)$ tal que $\chi(1)=2$ y sea
    $\rho\colon G\to\GL_2(\C)$ una representación de $G$ con caracter $\chi$. Como $G$ es simple y $\ker\rho$ es normal en $G$, 
    $\ker\rho=\{1\}$, es decir $\rho$ es inyectiva. 
    
    Como $G$ tiene un caracter irreducible de grado dos, $G$ es no abeliano, entonces 
    $[G,G]=G$ pues $[G,G]\ne\{1\}$. Sabemos que $G$ tiene exactamente $(G:[G,G])$ caracteres irreducible de grado uno, entonces 
    el único caracter irreducible de $G$ de grado uno es el trivial. La función
    \[
    G\to\C^{\times},\quad
    g\mapsto\det(\rho_g),
    \]
    es un morfismo de grupos, luego es un caracter de grado uno. Como tiene que ser el caracter trivial, 
    $\det(\rho_g)=1$ para todo $g\in G$. 
    
    Por el teorema de Frobenius, 
    $\chi(1)$ divide al orden de $G$ y luego 
    $G$ tiene orden par. Sea $x\in G$ un elemento de orden dos. Como $\rho$ es inyectiva, 
    $\rho_x$ tiene orden dos en $\GL_2(\C)$. Como $\rho_x$ es diagonalizable, existe $C\in\GL_2(\C)$ tal que
    \[
    C\rho_xC^{-1}=\begin{pmatrix}
    \lambda & 0\\
    0 & \mu 
    \end{pmatrix},
    \]
    donde $\lambda,\mu\in\{-1,1\}$ pues $\rho_x^2$ es la matriz identidad. Como $1=\det(\rho_g)=\lambda\mu$, entonces
    $\lambda=\mu=-1$, lo que implica que la matriz $\rho_x$ es central en $\GL_2(\C)$. Como $\rho$ es inyectiva, 
    $x$ es también central en $G$, es decir $xg=gx$ para todo $g\in G$. Luego $\langle x\rangle$ es un subgrupo propio 
    normal no trivial de $G$.
\end{proof}

Vamos a demostrar una mejora del teorema de Frobenius. 

\begin{proposition}
Sean $G$ y $G_1$ dos grupos finitos. 
Si $\rho$ es una representación irreducible de $G$ y $\rho_1$ es
  una representación irreducible de $G_1$, entonces $\rho\otimes\rho_1$ 
  es una representación irreducible de $G\times G_1$. 
%     \item Toda representación irreducible de $G\times G_1$ es equivalente a una representación $\rho\otimes\rho_1$, donde
%     $\rho$ es una representación irreducible de $G$ y $\rho_1$ es una representación irreducible de $G_1$. 
%   \end{enumerate}
\end{proposition}

\begin{proof}
    Sea $\chi$ el caracter de $\rho$ y $\chi_1$ el caracter de $\rho_1$. 
    Como $\rho$ y $\rho_1$ son irreducibles, $\langle\chi,\chi\rangle=\langle\chi_1,\chi_1\rangle=1$. Entonces
    \begin{align*}
    \langle\chi\otimes\chi_1,\chi\otimes\chi_1\rangle
    &=\frac{1}{|G\times G_1|}\sum_{(g,g_1)\in G\times G_1}\chi(g)\chi_1(g_1)\overline{\chi(g)}\overline{\chi_1(g_1)}\\
    &=\left(\frac{1}{|G|}\sum_{g\in G}\chi(g)\overline{\chi(g)}\right)\left(\frac{1}{|G_1|}\sum_{g_1\in G_1}\chi_1(g_1)\overline{\chi_1(g_1)}\right)\\
    &=\langle\chi,\chi\rangle\langle\chi_1,\chi_1\rangle=1.
    \end{align*}
    Luego $\rho\otimes\rho_1$ es irreducible. 
\end{proof}

\begin{exercise}
    Sean $G$ y $G_1$ grupos finitos. 
    Demuestre que toda representación irreducible de $G\times G_1$ es de la forma $\rho\otimes\rho_1$, donde $\rho$ es una representación
    irreducible de $G$ y $\rho_1$ es una representación irreducible de $G_1$. 
\end{exercise}

\begin{theorem}[Schur]
    \index{Teorema!de Schur}
    \index{Schur!teorema de}
    Sean $G$ un grupo finito y $\chi\in\Irr(G)$. Entonces $\chi(1)$ divide al índice $(G:Z(G))$.
\end{theorem}

\begin{proof}
    Sea $\rho\colon G\to\GL(V)$, $g\mapsto\rho_g$, una representación con caracter $\chi$. 
    Si $z\in Z(G)$, entonces $\rho_z$ conmuta con $\rho_g$ para todo $g\in G$. Por el lema de Schur, 
    $\rho_z(v)=\lambda(z)v$ para todo $v\in V$. Sea $\lambda\colon Z(G)\to\C^\times$, $z\mapsto \lambda(z)$. Como 
    \[
    \lambda(z_1z_2)v=\rho_{z_1z_2}(v)=\rho_{z_1}\rho_{z_2}(v)=\lambda(z_1)\lambda(z_2)v
    \]
    para todo $v\in V$, $\lambda$ es morfismo de grupos. 
    
    Para $n\in\N$ sea $G^n=G\times\cdots\times G$ ($n$-veces) y 
    sea 
    \[
    \sigma\colon G^n\to\GL(V^{\otimes n}),
    \quad 
    (g_1,\dots,g_n)\mapsto \rho_{g_1}\otimes\cdots\otimes\rho_{g_n}.
    \]
    La representación $\sigma$ tiene
    caracter $\chi^{n}$ y es irreducible. 
    Como 
    \begin{align*}
    \sigma(z_1,\dots,z_n)(v_1\otimes\cdots\otimes v_n)
    &=z_1\cdot v_1\otimes\cdots\otimes z_n\cdot v_n\\
    &=\lambda(z_1)\cdots\lambda(z_n)(v_1\otimes\cdots\otimes v_n)\\
    &=\lambda(z_1\cdots z_n)(v_1\otimes\cdots\otimes v_n),
    \end{align*}
    el subgrupo 
    \[
    H=\{(z_1,\dots,z_n)\in Z(G)\times\cdots\times Z(G):z_1\cdots z_n=1\}
    \]
    de $G^n$ actúa trivialmente
    en $V^{\otimes n}$, lo que nos da una representación 
    \[
    \tau\colon G^n/H\to\GL(V^{\otimes n}),
    \]
    es decir una
    estructura de $\C[G^n/H]$-módulo sobre $V^{\otimes n}$. Como $V^{\otimes n}$ 
    es un $\C[G^n]$-módulo simple, $V^{\otimes n}$ también es simple como $\C[G^n/H]$-módulo. 
    Por el teorema de Frobenius aplicado al $\C[G]$-módulo $V$ sabemos que 
    el grado $\chi(1)$ divide a $|G|$, digamos $|G|=\chi(1)s$ para algún $s\in\Z$. 
    Ese mismo teorema, ahora 
    aplicado al $\C[G^n/H]$-módulo $V^{\otimes n}$, nos dice que
    el grado $\chi(1)^n$ de $\tau$ 
    divide al entero $(G^n:H)=(G:Z(G))^{n-1}|G|$, digamos $|G|(G:Z(G))^{n-1}=\chi(1)^nr$ para algún $r\in\Z$. 
    Sean $a,b\in\Z$ tales que $\gcd(a,b)=1$ y 
    \[
    \frac{a}{b}=\frac{(G:Z(G))}{\chi(1)}.
    \]
    Como 
    \[
    s\frac{a^{n-1}}{b^{n-1}}=s\frac{(G:Z(G))^{n-1}}{\chi(1)^{n-1}}=\frac{|G|(G:Z(G))^{n-1}}{\chi(1)^n}\in\Z,
    \]
    Luego $b^{n-1}$ divide a $s$. Como $n$ es arbitrario, se sigue que $b=1$. 
    % \[
    % \frac{(G:Z(G))}{\chi(1)}=\frac{r}{s}\in\A\cap\Q=\Z.\qedhere
    % \]
    % %$\deg\sigma=(\dim V)^m$ divide al entero 
    %\[
    % \frac{|G|^m}{|Z(G)|^{m-1}}=\frac{(\dim V)^mt^m|Z(G)|}{|Z(G)|^{m}},
    % \]
    % lo que equivale a decir que $\dim V$ divide a $(G:Z(G))$. 
\end{proof}

% Antes de demostrar una generalización del teorema anterior encontrada por It\^o, necesitamos un lema. 

% \begin{lemma}
% Sea $G$ un grupo finito y sea $A$ un subgrupo normal abeliano de $G$. Si $\rho\colon G\to\GL(V)$ 
% es una representación irreducible, entonces
% \begin{enumerate}
%     \item existe un subgrupo $H$ de $G$ tal que $A\subseteq H\subsetneq G$ y una 
%     representación irreducible $\sigma$ de $H$ que induce a la representación $\rho$, o bien 
%     \item la restricción $\rho|_A$ es la suma directa finitas representaciones irreducibles isomorfas.
% \end{enumerate}
% \end{lemma}

% \begin{proof}
%     Sea $V=\oplus_{i=1}^nV_i$ la descomposición de la restricción $\rho|_A$ en irreducibles. Como $\rho(s)$ permuta los $V_i$, 
%     los permuta transitivamente pues $V$ es un módulo simple. 
% \end{proof}

% \begin{theorem}[It\^o]
% \index{Teorema!de It\^o}
% \index{It\^o|teorema de}
% Sea $G$ un grupo finito y sea $A$ un subgrupo normal y abeliano de $G$. Si $\chi\in\Irr(G)$, entonces
% $\chi(1)$ divide a $(G:A)$. 
% \end{theorem}

% \begin{proof}
%     Sea $\rho\colon G\to\GL(V)$ una representación irreducible 
%     de $G$ con caracter $\chi$. Si nos encontramos en el primer caso del lema anterior, entonces
%     $\deg\sigma$ divide al índice $(H:A)$ y entonces, al multiplicar en ambos miembros por $(G:H)$ obtememos que 
%     \[
%     \deg\rho=\deg\sigma(G:H)\text{ divide al entero }(G:H)(H:A)=(G:A).
%     \]
    
%     Si estamos en el segundo caso del lema anterior, sea $G_1=\rho(G)$ y sea $A_1=\rho(A)$. El morfismo canónico $G/A\to G_1/A_1$
%     es sobreyectivo y entonces $(G_1:A_1)$ divide a $(G:A)$. Como para cada $a\in A$ se tiene $\rho_a=\lambda(a)\id$ 
%     para $\lambda(a)\in\C$. En particular, $\rho(A)\subseteq Z(G_1)$ y luego, por el teorema de Schur, 
%     $\deg\rho$ divide al índice $(G_1:A_1)$, que a su vez divide al índice $(G:A)$. 
% \end{proof}

La demostración anterior fue descubierta por Tate y está basada en 
el "truco del producto tensorial". Para más información 
sobre este truco referimos al blog de Terence Tao: 
\url{https://terrytao.wordpress.com}. Allí encontraremos una 
entrada dedicada exclusivamente muchas de las aplicaciones
de este poderoso truco. 

\medskip
El teorema de Schur también puede generalizarse. 
En 1951 It\^o demostró el siguiente resultado: 

\begin{theorem}[It\^o]
\index{Teorema!de It\^o}
Si $G$ es un grupo finito y $\chi\in\Irr(G)$, entonces
$\chi(1)$ divide a $(G:A)$ para todo subgrupo normal abeliano $A$. 
\end{theorem}

La demostración, que no es más difícil que la demostración del 
teorema de Frobenius o del teorema de Schur que vimos en este capítulo, puede consultarse por ejemplo 
en~\cite[\S8.1]{MR0450380}. 

\medskip
Para terminar con el capítulo vamos a mencionar algunas de las conjeturas de conteo más famosas. 
En 1971 McKay hizo la siguiente conjetura:

\begin{conjecture}[McKay]
\index{Conjetura!de McKay}
Sea $p$ un primo. 
Si $G$ es un grupo finito y $P\in\Syl_p(G)$, entonces
\[
|\{\chi\in\Irr(G):p\nmid \chi(1)\}|
=|\{\psi\in\Irr(N_G(P))|:p\nmid\psi(1)\}|.
\]
\end{conjecture}

En la versión original de McKay el grupo $G$ es simple y el primo es $p=2$. La versión general 
de la conjetura fue en realidad formulada por Alperin en~\cite{MR0404417} e independientemente
por Isaacs en~\cite{MR332945}.

La conjeetura de McKay permanece abierta y es uno de los problemas abiertos más importantes 
en la teoría de representaciones de grupos finitos sobre los números complejos. 

\index{Teorema!de Malle-Sp\"ath}
Se sabe que la conjetura de McKay es verdadera para varias clases de grupos. 
Isaacs la demostró para grupos resolubles, ver por ejemplo~\cite{MR332945,MR3791517}. 
Malle y Sp\"ath demostraron que 
la conjetura de McKay es cierta para el primo $p=2$. 

\begin{theorem}[Malle--Sp\"ath]
\index{Teorema!de Malle--Sp\"ath}
Si $G$ es un grupo finito y $P\in\Syl_2(G)$, entonces
\[
|\{\chi\in\Irr(G):2\nmid \chi(1)\}|
=|\{\psi\in\Irr(N_G(P))|:2\nmid\psi(1)\}|.
\]
\end{theorem}

La demostración aparece en~\cite{MR3549625} y utiliza 
la clasificación de grupos simples finitos. Se basa en 
demostrar que todo grupo simple cumple con ciertas propiedades 
bastante más complicadas que la conjetura de McKay, 
un resultado de Isaacs, Malle y Navarro~\cite{MR2336079}. 

Podemos verificar computacionalmente algunos casos pequeños 
de la conjetura de McKay con la siguiente función:

\begin{lstlisting}
gap> McKay := function(G, p)
> local N, n, m;
> N := Normalizer(G, SylowSubgroup(G, p));
> n := Number(Irr(G), x->Degree(x) mod p <> 0);
> m := Number(Irr(N), x->Degree(x) mod p <> 0);
> if n = m then
> return true;
> else
> return false;
> fi;
> end;
function( G, p ) ... end
\end{lstlisting}

Como ejemplo vamos a verificar computacionalmente la conjetura de McKay para el grupo 
de Mathieu $M_{11}$. Se sabe que $M_{11}$ es un grupo simple de orden 7920. 

\begin{lstlisting}
gap> M11 := MathieuGroup(11);;
gap> PrimeDivisors(Order(M11));
[ 2, 3, 5, 11 ]
gap> McKay(M11,2);
true
gap> McKay(M11,3);
true
gap> McKay(M11,5);
true
gap> McKay(M11,11);
true
\end{lstlisting}

% \begin{exercise}
% Verifique la conjetura de McKay para el grupo $\Sym_3$.
% \end{exercise}
La siguiente conjetura 
es un refinamiento de la conjetura de McKay y 
fue formulada por Isaacs y Navarro a principios del siglo XXI:

\begin{conjecture}[Isaacs--Navarro]
\index{Conjetura!de Isaacs--Navarro}
Sean $p$ un primo y $k\in\Z$. 
Si $G$ es un grupo finito y $P\in\Syl_p(G)$, entonces
\begin{align*}
|\{\chi\in\Irr(G):&p\nmid \chi(1)\text{ y }\chi(1)\equiv\pm k\bmod p\}|\\
&=|\{\psi\in\Irr(N_G(P))|:p\nmid\psi(1)\text{ y }\psi(1)\equiv\pm k\bmod p\}|.
\end{align*}
\end{conjecture}

Aunque la conjetura de Isaacs--Navarro en general permanece abierta, 
se sabe que es verdadera para varias clases de grupos, por ejemplo 
para grupos resolubles, para los 26 grupos simples esporádicos y para el grupo simétrico, 
ver por ejemplo~\cite{MR1935849}. 

Para verificar la conjetura de Isaacs--Navarro en 
ejemplos de orden pequeño podemos utilizar el siguiente código:

\begin{lstlisting}
gap> IsaacsNavarro := function(G, k, p)
> local mG, mN, N;
> N := Normalizer(G, SylowSubgroup(G, p));
> mG := Number(Filtered(Irr(G), x->Degree(x)\
> mod p <> 0), x->Degree(x) mod p in [-k,k] mod p);
> mN := Number(Filtered(Irr(N), x->Degree(x)\
> mod p <> 0), x->Degree(x) mod p in [-k,k] mod p);
> if mG = mN then
> return mG;
> else
> return false;
> fi;
> end;
function( G, k, p ) ... end
\end{lstlisting}

Dejamos como ejercicio verificar la conjetura de Isaacs--Navarro 
por ejemplo para el grupo de Mathieu $M_{11}$. 