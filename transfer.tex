\chapter{El morfismo de transferencia}

%\section{El morfismo de transferencia}

\index{Morfismo de transferencia}
Sea $G$ un grupo y sea $H$ un subgrupo de índice finito. Vamos a definir un
morfismo de grupos $G\to H/[H,H]$, el \textbf{morfismo de transferencia} de $G$
en $H$. Fijemos un \textbf{transversal a izquierda}\footnote{Un transversal a
izquierda de $H$ en $G$ es simplemente un conjunto de representantes de coclases
a izquierda de $H$ en $G$.} \index{Transversal} $T$ de $H$ en $G$.

\begin{lemma}
	\label{lemma:sigma}
	Sean $G$ un grupo y $H$ un subgrupo de índice $n$. Sean
	$S=\{s_1,\dots,s_n\}$ y $T=\{t_1,\dots,t_n\}$ transversales de $H$ en $G$.
	Dado $g\in G$, existen únicos $h_1,\dots,h_n\in H$ y una permutación
	$\sigma\in\Sym_n$  
	tales que
	\[
		gt_i=s_{\sigma(i)}h_i,\quad
		i\in\{1,\dots,n\}.
	\]
\end{lemma}

\begin{proof}
	Si $i\in\{1,\dots,n\}$ existe un único $j\in\{1,\dots,n\}$ tal que $gt_i\in
	s_jH$. Luego existe un único $h_i\in H$ tal que $gt_i=s_jh_i$. Al tomar
	$\sigma(i)=j$ queda entonces definida una función
	$\sigma\colon\{1,\dots,n\}\to\{1,\dots,n\}$.  Para ver que
	$\sigma\in\Sym_n$ basta ver que $\sigma$ es inyectiva. Si
	$\sigma(i)=\sigma(k)=j$, como $gt_i=s_jh_i$ y $gt_k=s_jh_k$, tenemos que
	$t_i^{-1}t_k=h_i^{-1}h_k\in H$ y luego $i=k$ pues $t_iH=t_kH$.
\end{proof}

%\begin{exercise} 
%	Demuestre que la acción de $G$ en el conjunto de coclases $H\backslash G$
%	dada por $(Hx)\cdot g=H(xg)$ induce una acción a derecha de $G$ en $T$. 
%\end{exercise}
%
%\begin{svgraybox}
%	Si $t\in T$ y $g\in G$ existe un único $t\cdot g\in T$ tal que $(Ht)\cdot
%	g=H(t\cdot g)$. Como $G$ actúa en $H\backslash G$ por multiplicación a
%	derecha, $H(t\cdot (g_1g_2))=H((t\cdot g_1)\cdot g_2)$ para todo $t\in T$,
%	$g_1,g_2\in G$.
%\end{svgraybox}
%
%\begin{exercise} 
%	Demuestre que si $t\in T$ y $g\in G$ entonces $tg(t\cdot g)^{-1}\in H$.
%\end{exercise}
%
%\begin{svgraybox}
%	Sean $t\in T$ y $g\in G$. Entonces $t\cdot g\in H$ es el único elemento de
%	$H$ tal que $H(tg)=H(t\cdot g)$. Luego $(tg)(t\cdot g)^{-1}\in H$.
%\end{svgraybox}

\begin{definition}
	\label{definition:nu_T}
	Sea $G$ un grupo y sea $H$ un subgrupo de $G$ de índice finito $n$. Si
	$T=\{t_1,\dots,t_n\}$ es un transversal de $H$ en $G$, se define la función 
	\[
		\nu_T\colon G\to H/[H,H],\quad
		\nu_T(g)=\prod_{i=1}^n h_i
	\]
	donde $gt_i=t_jh_i$.
\end{definition}

\begin{lemma}
	\label{lemma:nu_T}
	Sea $G$ un grupo y sea $H$ un subgrupo de $G$ de índice finito. Si $T$ y
	$S$ son transversales de $H$ en $G$, entonces $\nu_T=\nu_S$.
\end{lemma}

\begin{proof}
	Supongamos que $gs_i=s_{\sigma(i)}h_i$ para todo $i$ y escribamos
	$s_i=t_ik_i$, $k_i\in H$. Entonces, si $l_i=k_{\sigma(i)}h_ik_i^{-1}$, 
	\[
	gt_i=gs_ik_i^{-1}=s_{\sigma(i)}h_ik_i^{-1}=t_{\sigma(i)}k_{\sigma(i)}h_ik_i^{-1}=t_{\sigma(i)}l_i
	\]
	para todo $i\in\{1,\dots,n\}$. Además 
	\[
			s_{\sigma(i)}^{-1}gs_i=k_{\sigma(i)}^{-1}t_{\sigma(i)}^{-1}gt_ik_i.
	\]
	Como $H/[H,H]$ es un grupo abeliano, tenemos 
	\begin{align*}
		\nu_S(g)
		&=\prod_{i=1}^n s_{\sigma(i)}^{-1}gs_i
		=\prod_{i=1}^n k_{\sigma(i)}^{-1}t_{\sigma(i)}^{-1}gt_ik_i\\
		&=\prod_{i=1}^n k_{\sigma(i)}^{-1}\prod_{i=1}^n k_i\prod_{i=1}^n t_{\sigma(i)}^{-1}gt_i
		=\prod_{i=1}^n t_{\sigma(i)}^{-1}gt_i
		=\nu_T(g).\qedhere
	\end{align*}
\end{proof}

El lema~\ref{lemma:nu_T} demuestra que si $H$ es un subgrupo de $G$ de índice
finito, queda bien definida la función
\[
\nu\colon G\to H/[H,H],
\quad
\nu(g)=\nu_T(g),
\]
donde $T$ es algún transversal de $H$ en $G$. 

\begin{theorem}
	\label{theorem:transfer}
	Sea $G$ un grupo y sea $H$ un subgrupo de $G$ de índice finito. Entonces
	$\nu(xy)=\nu(v)\nu(y)$ para todo $x,y\in G$.
\end{theorem}

\begin{proof}
	Sea $T=\{t_1,\dots,t_n\}$ un transversal de $H$ en $G$. Sean $x,y\in G$. Por el
	lema~\ref{lemma:sigma} existen únicos $h_1,\dots,h_n,k_1,\dots,k_n\in H$ y
	existen $\sigma,\tau\in\Sym_n$ tales que $xt_i=t_{\sigma(i)}h_i$,
	$yt_i=t_{\tau(i)}k_i$. Como
	\[
	xyt_i=xt_{\tau(i)}k_i=t_{\sigma\tau(i)}h_{\tau(i)}k_i,
	\]
	y el grupo $H/[H,H]$ es abeliano, 
	\[
		\nu(xy)=\prod_{i=1}^n h_{\tau(i)}k_i=\prod_{i=1}^n h_{\tau(i)}\prod_{i=1}^n k_i=\nu(x)\nu(y).\qedhere
	\]
\end{proof}

El teorema~\ref{theorem:transfer} afirma que $\nu$ es un morfismo de grupos.
Queda entonces justificada la siguiente definición:

\begin{definition}
	Sea $G$ un grupo y sea $H$ un subgrupo de índice finito. El
	\textbf{morfismo de transferencia} es el morfismo $\nu\colon G\to H/[H,H]$,
	$\nu(g)=\nu_T(g)$, donde $T$ es algún transversal de $H$ en $G$.
\end{definition}

\begin{example}
	% explicar mucho mejor!
	% ademas vamos a hacer reciprocidad como aplicación de Fourier
	\index{Gauss!lema de}
	Sea $p$ un número primo. Sean $G=\F_p^\times$ y $H=\{-1,1\}$. Entonces
	$(G:H)=\frac{p-1}{2}$. Calculemos el morfismo de transferencia:
	\[
		\nu\colon G\to H,\quad
		\nu(x)=x^{\frac{p-1}{2}}=\legendre{x}{p}=\begin{cases}
			1 & \text{si $x$ es un cuadrado},\\
			-1 & \text{en otro caso}.
		\end{cases}.
	\]
	Elegimos un transversal $T=\{1,2,\dots,\frac{p-1}{2}\}$. Para $x\in G$, $t\in T$ definimos
	\[
	\epsilon(x,t)=\begin{cases}
		1 & \text{si $xt\in T$},\\
		-1 & \text{si $xt\not\in T$}.
	\end{cases}
	\]
	Al calcular el morfismo de transferencia obtenemos el \textbf{lema de
	Gauss}:
	\[
	\legendre{x}{p}=\prod_{t\in T}\epsilon(x,t).
	\]
\end{example}

\begin{theorem}
	\label{theorem:P_noabeliano}
	Sea $G$ un grupo finito. Sea $p$ un primo que divide al orden del subgruo $[G,G]\cap
	Z(G)$. Si $P\in\Syl_p(G)$ entonces $P$ es no abeliano.
\end{theorem}

\begin{proof}
	Supongamos que $P$ es abeliano y sea $T=\{t_1,\dots,t_n\}$ un transversal
	de $P$ en $G$. Como $[G,G]\cap Z(G)$ es un subgrupo normal de $G$, podemos
	suponer que $P\cap [G,G]\cap Z(G)\ne1$. Sea $z\in P\cap [G,G]\cap Z(G)$ tal
	que $z\ne1$. 

	Sea $\nu\colon G\to P$ el morfismo de transferencia. Vamos a calcular
	$\nu(z)$ con el lema~\ref{lemma:sigma}. Para cada $i\in\{1,\dots,n\}$ sean
	$x_1,\dots,x_n\in P$ y sea $\sigma\in\Sym_n$ tales que
	$zt_i=t_{\sigma(i)}x_i$. Como $z\in Z(G)$, se tiene
	$t_i=t_{\sigma(i)}x_iz^{-1}$ y luego la unicidad del lema~\ref{lemma:sigma}
	implica que $\sigma=\id$ y $x_i=z$ para todo $i$. Luego 
	\[
	\nu(z)=z^{|T|}=z^{(G:P)}. 
	\]

	Como $P$ es abeliano, $[G,G]\subseteq\ker\nu$. Luego $\nu(z)=1$. Esto es
	una contradicción pues $1\ne z\in P$ y $z^{(G:P)}=1$ implica que $z$ tiene
	orden no divisible por $p$. 
\end{proof}

% rotman 7.47, 7.48

\begin{lemma}
	\label{lemma:evaluation}
	Sea $G$ un grupo y sea $H$ un subgrupo de índice $n$. Sea
	$T=\{t_1,\dots,t_n\}$ un transversal de $H$ en $G$.  Para cada $g\in G$ existe
	$m\in\N$ y 
	existen $s_{1},\dots,s_{m}\in T$ y enteros positivos $n_1,\dots,n_m$
	tales que 
	$s_i^{-1}g^{n_i}s_i\in H$,
	$n_1+\cdots+n_m=n$ y 
	\[
		\nu(g)=\prod_{i=1}^m s_i^{-1}g^{n_i}s_i.
	\]
\end{lemma}

% TODO: explicar mejor!

\begin{proof}
	Para cada $i$ existen $h_1,\dots,h_n\in H$ y $\sigma\in\Sym_n$ tales que
	$gt_i=t_{\sigma(i)}h_i$. Escribimos $\sigma$ como producto de ciclos
	disjuntos
	\[
		\sigma=\alpha_1\cdots\alpha_m.
	\]

	Para cada $i\in\{1,\dots,n\}$, escribamos 
	$\alpha_i=(j_{1}\cdots j_{n_i})$. Como 
	\[
		g t_{j_k}=t_{\sigma(j_k)}h_{j_k}=\begin{cases}
			t_{j_1}h_{n_i} & \text{si $i=n_i$},\\
			t_{j_{k+1}}h_{k} & \text{en otro caso},
		\end{cases}
	\]
	entonces
	\[
	t_{j_1}^{-1}g^{n_i}t_{j_1}
	=t_{j_1}^{-1}gg^{n_i-1}t_{j_1}
	=t_{j_1}^{-1}gt_{j_r}h_{j_{r-1}}\cdots h_{j_1}
	=h_{j_r}\cdots h_{j_1}\in H,
	\]
	y definimos $s_i=t_{j_1}$. Como $\nu(g)=h_1\cdots h_{n}$, 
	de aquí se deduce inmediatamente el lema.
\end{proof}

\begin{proposition}
	\label{proposition:v(g)=g^n}
	Sea $G$ un grupo y sea $H$ un subgrupo central de índice $n$. 
	Entonces $\nu(g)=g^n$ para todo $g\in G$.
\end{proposition}

\begin{proof}
	Sea $g\in G$. Por el lema anterior (lema~\ref{lemma:evaluation}) existen $s_1,\dots,s_m\in
	H$ tales que $s_i^{-1}g^{n_i}s_i\in H$ y $\nu(g)=\prod_{i=1}^m
	s_i^{-1}g^{n_i}s_i$.  Como $H$ es normal en $G$ por ser central,
	\[
	g^{n_i}=s_i(s_i^{-1}g^{n_i}s_i)s_i^{-1}\in H\subseteq Z(G).
	\]
	Luego 
	\[
		\nu(g)
		=\prod_{i=1}^m s_i^{-1}g^{n_i}s_i
		=\prod_{i=1}^m g^{n_i}
		=g^{\sum_{i=1}^m n_i}
		=g^n.\qedhere 
	\]
\end{proof}

\begin{corollary}
	Si un grupo $G$ tiene un subgrupo $H$ de índice $n$ tal que $H\subseteq
	Z(G)$ entonces $g\mapsto g^n$ es un morfismo de grupos.
\end{corollary}

\begin{proof}
	La función $g\mapsto g^n$ es el morfismo de transferencia, ver la
	proposición~\ref{proposition:v(g)=g^n} y el teorema~\ref{theorem:transfer}.
\end{proof}

\begin{corollary}
	\label{corollary:[x,y]^n=1}
	Sea $G$ un grupo tal que $(G:Z(G))=n$. Si $x,y\in G$ entonces $[x,y]^n=1$. 
\end{corollary}

\begin{proof}
	Como $Z(G)$ es abeliano, el núcleo del morfismo de transferencia $\nu\colon
	G\to Z(G)$, $\nu(g)=g^n$, contiene al conmutador $[G,G]$.
\end{proof}

\begin{corollary}
	\label{corollary:semidirecto}
	Sea $G$ un grupo finito y sea $H$ un subgrupo abeliano de
	índice $n$, donde $n$ es coprimo con $|H|$.  Sea
	$N=\ker(\nu\colon G\to H)$. Entonces $G\simeq N\rtimes H$.
\end{corollary}

% todo: no entiendo un paso de la demostración, por qué quito los s_i

\begin{proof}
	Como $H$ es abeliano, $H=H/[H,H]$ y el morfismo de transferencia es
	$\nu\colon G\to H$. Por el lema~\ref{lemma:evaluation}, podemos escribir
	\[
		\nu(h)
		=\prod_{i=1}^m s_i^{-1}h^{n_i}s_i
		=\prod_{i=1}^m h^{n_i}
		=h^{\sum_{i=1}^m n_i}=h^n.
	\]
	La composición $H\hookrightarrow G\xrightarrow{\nu} H$ es morfismo de
	grupos. 
	
	Vamos a demostrar que es un isomorfismo. Es inyectiva pues si $h^n=1$
	entonces $|h|$ divide a $|H|$ y divide a $n$; y como $n$ y $|H|$ son
	coprimos, $h=1$. Veamos que es sobreyectiva. Como $n$ y $|H|$ son coprimos,
	existe $m\in\Z$ tal que $nm\equiv 1\bmod |H|$. Si $h\in H$ entonces $h^m\in
	H$ y $\nu(h^m)=h^{nm}=h$. 

	Tenemos entonces que $G\simeq N\rtimes H$ pues $N$ es normal en $G$, $N\cap
	H=\{1\}$ y $G=NH$ (pues $|NH|=|N||H|$ y $G/N\simeq H$).
\end{proof}

\begin{corollary}[Frobenius]
	Sea $H$ un subgrupo central de un grupo finito $G$. Si $|H|$
	y $|G/H|$ son coprimos entonces $G\simeq H\times G/H$.
\end{corollary}

\begin{proof}
	Es consecuencia inmediata del corolario~\ref{corollary:semidirecto} pues
	$H$ es normal por ser un subgrupo central.
\end{proof}

