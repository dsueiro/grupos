\chapter{Conmutadores}

Sea $G$ un grupo finito y sean $C_1,\dots,C_s$ sus clases de conjugación. 
Vimos en el teorema~\ref{theorem:algebraic} que 
\[
\omega_\chi(C_i)=\frac{|C_i|\chi(C_i)}{\chi(1)}
\]
es un número algebraico para todo $\chi\in\Irr(G)$ y todo $i\in\{1,\dots,s\}$. 
Vimos además en la fórmula~\eqref{eq:omega} 
que
\begin{gather*}
    \left(\frac{|C_i|}{\chi(1)}\chi(C_i)\right)
    \left(\frac{|C_j|}{\chi(1)}\chi(C_j)\right)
    =\sum_{k=1}^sa_{ijk}\left(\frac{|C_k|}{\chi(1)}\chi(C_k)\right),
\shortintertext{es decir}
\omega_{\chi}(C_i)\omega_{\chi}(C_j)=\sum_{k=1}^sa_{ijk}\omega_{\chi}(C_k),
\end{gather*}
donde $a_{ijk}$ es la cantidad de soluciones de la ecuación $xy=z$ 
con $x\in C_i$, $y\in C_j$ y $z\in C_k$. 

\begin{theorem}[Burnside]
\index{Teorema!de Burnside}
Si $G$ es un grupo finito y $C_1,\dots,C_s$ son sus clases de conjugación, entonces 
\[
a_{ijk}=\frac{|C_i||C_j|}{|G|}\sum_{\chi\in\Irr(G)}\frac{\chi(C_i)\chi(C_j)\overline{\chi(C_k)}}{\chi(1)}.
\]
\end{theorem}

\begin{proof}
Sabemos que 
\[
    \left(\frac{|C_i|}{\chi(1)}\chi(C_i)\right)
    \left(\frac{|C_j|}{\chi(1)}\chi(C_j)\right)
    =\sum_{k=1}^sa_{ijk}\left(\frac{|C_k|}{\chi(1)}\chi(C_k)\right),
\]
que puede reescribirse como
\[
\frac{|C_i||C_j|\chi(C_i)\chi(C_j)}{\chi(1)}=\sum_{k=1}^sa_{ijk}|C_k|\chi(C_k).
\]
Al multiplicar esta igualdad por $\overline{\chi(C_l)}$ y luego sumar sobre
todos los $\chi\in\Irr(G)$ tenemos 
\begin{align*}
    |C_i||C_j|\sum_{\chi\in\Irr(G)}\frac{\overline{\chi(C_l)}}{\chi(1)}\chi(C_i)\chi(C_j)
    &=\sum_{\chi\in\Irr(G)}\sum_{k=1}^sa_{ijk}|C_k|\chi(C_k)\overline{\chi(C_l)}\\
    &=\sum_{k=1}^sa_{ijk}|C_k|\sum_{\chi\in\Irr(G)}\chi(C_k)\overline{\chi(C_l)}\\
    &=a_{ijl}|G|,
\end{align*}
ya que, gracias a la segunda relación de ortogonalidad de Schur, sabemos que
\[
\sum_{\chi\in\Irr(G)}\chi(C_k)\overline{\chi(C_l)}=\begin{cases}
\frac{|G|}{|C_l|} & \text{si $k=l$},\\
0 & \text{en otro caso}.
\end{cases}\qedhere
\]
\end{proof}

Veamos ahora algunos corolarios sobre conmutadores. 

\begin{theorem}[Burnside]
Sea $G$ un grupo finito y sean $g,x\in G$. Entonces $g$ y $[x,y]$ son conjugados para algún $y\in G$ si y sólo si 
\[
\sum_{\chi\in\Irr(G)}\frac{|\chi(x)|^2\chi(g)}{\chi(1)}>0.
\]
\end{theorem}

\begin{proof}
Sean $C_1,\dots,C_s$ las clases de conjugación de $G$. Supongamos que $x\in C_i$ y que $g\in C_k$. 
Para $i\in\{1,\dots,s\}$ sea $C_i^{-1}=\{z^{-1}:z\in C_i\}$. El teorema de Burnside 
en el caso $C_j=C_i^{-1}$ implica entonces que
\[
a_{ijk}=\frac{|C_i|^2}{|G|}\sum_{\chi\in\Irr(G)}\frac{|\chi(C_i)|^2\overline{\chi(C_k)}}{\chi(1)}.
\]
Para cada $i\in\{1,\dots,s\}$ sea $g_i\in C_i$. 

Demostremos primero la implicación $\impliedby$. Como en este caso 
$a_{ijk}>0$, existen $u\in C_i$ y $v\in C_j$ tales que $g=uv$. Si $x$ y $u$ son conjugados, entonces
$x^{-1}$ y $v$ también, digamos
\[
u=zxz^{-1},\quad
v=z_1x^{-1}z_1^{-1}
\]
para ciertos $z,z_1\in G$. 
Si $y=z^{-1}z_1$, entonces $y^{-1}z^{-1}=z_1^{-1}$ y luego 
$g$ y $[x,y]$ son conjugados, pues 
\[
g=uv=(zxz^{-1})(z_1x^{-1}z_1^{-1})=zxyx^{-1}y^{-1}z^{-1}. 
\]

Demostremos ahora la implicación $\implies$. Si existe $y\in G$ tal que 
$g$ y $[x,y]$ son conjugados, entonces $g=z(xyx^{-1}y^{-1})z^{-1}$ para algún $z\in G$. Si $v=yxy^{-1}$, entonces
$v^{-1}=yx^{-1}y^{-1}$ y luego $g$ y 
$[x,y]=xyx^{-1}y^{-1}=xv^{-1}$ son conjugados. En particular, $g\in C_iC_i^{-1}=C_iC_j$ y luego $a_{ijk}>0$.  
\end{proof}

%\begin{lemma}[Weil]
\begin{exercise}
    Si $G$ es un grupo finito, $g,h\in G$ y $\chi\in\Irr(G)$, entonces 
    \[
    \chi(g)\chi(h)=\frac{\chi(1)}{|G|}\sum_{z\in G}\chi(zgz^{-1}h).
    \]
\end{exercise}

\begin{exercise}
Si $G$ es un grupo finito, $g,h\in G$ y $\chi\in\Irr(G)$, entonces 
\[
\sum_{h\in G}\chi([g,h])=\frac{|G|}{\chi(1)}|\chi(g)|^2.
\]
\end{exercise}

Sea $G$ un grupo finito. Para $g\in G$ sea 
\[
\tau(g)=|\{(x,y)\in G\times G:[x,y]=g\}|.
\]
Vamos a demostrar una fórmula descubierta por Frobenius que nos permite calcular
el valor de $\tau(g)$ a partir de la tabla de caracteres de $G$. 

% \begin{proof}
% Fijemos $z_1\in G$. Como $\chi$ es una función de clases, 
% \[
% \sum_{z\in G}\chi(zgz^{-1}h)=\sum_{z\in G}\chi((z_1zgz^{-1}z_1^{-1})(z_1hz_1^{-1}))
% =\sum_{z\in G}\chi((zgz^{-1})(z_2hz_2^{-1})),
% \]
% para un cierto $z_2\in G$, pues como $z$ recorre todo $G$ y $z_1$ 
% está fijo, entonces $z_2=z_1z$ recorre todo $G$. 

% Sean $C_1,\dots,C_s$ las clases de conjugación de $G$ y para cada $i\in\{1,\dots,s\}$ sea
% $g_i\in C_i$. Entonces, la fórmula anterior, en particular, implica que 
% \begin{align*}
%     \sum_{z\in G}\chi(zg_iz^{-1}g_j)=\sum_{z\in G}\chi((zg_iz^{-1})(z_2g_jz_2^{-1})).    
% \end{align*}
% Al sumar estas expresiones sobre todo $z_2\in G$ obtenemos
% \begin{align*}
% |G|\sum_{z\in G}\chi(zg_iz^{-1}g_j)&=\sum_{z_2\in G}\sum_{z\in G}\chi((zg_iz^{-1})(z_2g_jz_2^{-1}))\\
% &=|C_G(g_i)||C_G(g_j)|\sum_{x\in C_i}\sum_{y\in C_j}\chi(xy)\\
% &=|C_G(g_i)||C_G(g_j)|\sum_{k=1}^s|C_k|a_{ijk}\chi(C_k)\\
% &=|C_G(g_i)||C_G(g_j)|\frac{|C_i||C_j|}{\chi(1)}\chi(C_i)\chi(C_j).
% \end{align*}
% Luego 
% \[
% \sum_{z\in G}\chi(zg_iz^{-1}g_j)=
% \frac{|G|}{\chi(1)}\chi(C_i)\chi(C_j),
% \]
% que es equivalente a la primera fórmula que queríamos demostrar. 
% En particular, si $g_j=g_i^{-1}$, se concluye que 
% \[
% \sum_{z\in G}\chi([z,g_i])=\frac{|G|}{\chi(1)}|\chi(g_i)|^2.\qedhere
% \]
% \end{proof}

\begin{theorem}[Frobenius]
    \index{Teorema!de los conmutadores de Frobenius}
    Si $G$ es un grupo finito, entonces
    \[
    \tau(g)=|G|\sum_{\chi\in\Irr(G)}\frac{\chi(g)}{\chi(1)}.
    \]
\end{theorem}

\begin{proof}
    Si $\chi$ es irreducible, entonces
    \begin{equation}
    \label{eq:chi}
    \begin{aligned}
        1=\langle \chi,\chi\rangle &= \frac{1}{|G|}\sum_{z\in G}\chi(z)\overline{\chi(z)}\\
        &=\frac{1}{|G|}\sum_{C}|C|\chi(C)\chi(C^{-1}),
    \end{aligned}
    \end{equation}
    donde la última suma se hace sobre todas las clases de conjugación de $G$. 

    Sea $g\in G$ y sea $C$ la clase de conjugación de $g$ en $G$. Sabemos
    que la ecuación $xu^{-1}=g$, donde $x\in C$ y $u\in C^{-1}$ tiene 
    \[
    \frac{|C||C^{-1}|}{|G|}\sum_{\chi\in\Irr(G)}\frac{\chi(C)\chi(C^{-1})\chi(g^{-1})}{\chi(1)}.
    \]
    Si $(x,u)$ es una solución, entonces existen $|C_G(x)|$ elementos $y$ tales que 
    $yxy^{-1}=u$. La ecuación $[x,y]=g$ tiene entonces 
    \[
    |C|\sum_{\chi\in\Irr(G)}\frac{\chi(C)\chi(C^{-1})\chi(g^{-1})}{\chi(1)}.
    \]
    soluciones. Al sumar sobre todas las clases de conjugación y utilizar 
    la fórmula~\eqref{eq:chi} obtenemos
    \begin{align*}
    \sum_C|C|\sum_{\chi\in\Irr(G)}\frac{\chi(C)\chi(C^{-1})\chi(g^{-1})}{\chi(1)}
    &=\sum_{\chi\in\Irr(G)}\left(\sum_C|C|\chi(C)\chi(C^{-1})\right)\frac{\chi(g^{-1})}{\chi(1)}\\
    &=|G|\sum_{\chi\in\Irr(G)}\frac{\chi(g^{-1})}{\chi(1)}.
    \end{align*}
    Como este número es un entero (pues cuenta la cantidad de soluciones de una cierta ecuación), 
    es en particular un número real. En consecuencia, al conjugar obtenemos
    el resultado que queríamos demostrar.
    % Como $\tau$ es una función de clases, podemos escribir 
    % \[
    % \tau=\sum_{\chi\in\Irr(G)}\langle\tau,\chi\rangle\chi.
    % \]
    % El lema de Weil en el caso $g=h$ implica que
    % \[
    % |\chi(g)|^2=\chi(g^{-1})\chi(g)=\frac{\chi(1)}{|G|}\sum_{z\in G}\chi([g,z]).
    % \]
    % Por la primera relación de ortogonalidad de Schur, entonces
    % \[
    % |G|=\sum_{g\in G}|\chi(g)|^2=\frac{\chi(1)}{|G|}\sum_{x,y\in G}\chi([x,y])=\frac{\chi(1)}{|G|}\sum_{g\in G}\tau(g)\chi(g),
    % \]
    % por la definición de $\tau$. Por otro lado, como 
    % \[
    % \overline{\langle \tau,\chi\rangle}=\frac{1}{|G|}\sum_{g\in G}\tau(g)\chi(g)=\frac{|G|}{\chi(1)},
    % \]
    % entonces $\langle\tau,\chi\rangle=|G|/\chi(1)$ y luego 
    % \[
    % \tau(g)=\sum_{\chi\in\Irr(G)}\langle\tau,\chi\rangle\chi(g)=|G|\sum_{\chi\in\Irr(G)}\frac{\chi(g)}{\chi(1)}.\qedhere
    % \]
\end{proof}

El teorema de Frobenius obviamente implica 
el siguiente resultado demostrado en forma independiente 
por Burnside: Si $G$ es un 
grupo finito y $g\in G$, entonces $g$ 
es un conmutador si y sólo si 
\[
\sum_{\chi\in\Irr(G)}\frac{\chi(g)}{\chi(1)}\ne 0.
\]

\index{Conjetura!de Ore}
En 1951 Ore conjeturó que todo elemento de un grupo simple finito no abeliano es un conmutador. 
El resultado fue demostrado en 2010:

\begin{theorem}[Liebeck--O'Brien--Shalev--Tiep]
\index{Teorema!de Liebeck--O'Brien--Shalev--Tiep}
Todo elemento de un grupo simple finito no abeliano es un conmutador. 
\end{theorem}

La demostración puede consultarse en~\cite{MR2654085}. Ocupa unas 70 páginas y 
utiliza la clasificación de grupos simples finitos y teoría de caracteres, 
en particular los teoremas que presentamos en este capítulo. Para más información sobre 
el teorema de Liebeck--O'Brien--Shalev--Tiep
referimos a~\cite{MR3289286}. Sin embargo, podemos dar una demostración computacional
para algunos casos particulares:

\begin{proposition}
La conjetura de Ore es verdadera para todo grupo simple esporádico. 
\end{proposition}

\begin{proof}
Sea $G$ un grupo simple finito. 
Sabemos que $g\in G$ es un conmutador si y sólo si 
\[
\sum_{\chi\in\Irr(G)}\frac{\chi(g)}{\chi(1)}\ne 0.
\]
La función que exponemos a continuación nos permite determinar 
si todo elemento de un grupo es un conmutador. La función recibe como parámetro una tabla de caracteres y 
devuelve \lstinline{true} si todo elemento del grupo es un conmutaador o \lstinline{false} en caso contrario. 
\begin{lstlisting}
gap> Ore := function(char) 
> local s,f,k;
> for k in [1..NrConjugacyClasses(char)] do
> s := 0;
> for f in Irr(char) do
> s := s+f[k]/Degree(f);  
> od;
> if s<=0 then
> return false;
> fi;
> od;
> return true;
> end;
function( char ) ... end
\end{lstlisting}
Verificamos entonces la conjetura de Ore para los cinco grupos de Mathieu:
\begin{lstlisting}
gap> Ore(CharacterTable("M11"));
true
gap> Ore(CharacterTable("M12"));
true
gap> Ore(CharacterTable("M22"));
true
gap> Ore(CharacterTable("M23"));
true
gap> Ore(CharacterTable("M24"));
true
\end{lstlisting}
Dejamos como ejercicio demostrar la conjetura de Ore para el resto de los grupos esporádicos. 
\end{proof}

Para otras aplicaciones de la teoría de caracteres a los grupos simples finitos referiremos a 
\cite{MR3821142}. 
