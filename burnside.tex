\chapter{Algunos teoremas de Burnside}

Recordemos cómo actúa la representación natural del grupo simétrico.  Sea
$n\in\N$ y sea $\{e_1,\dots,e_n\}$ la base canónica de $\C^n$.  La
\textbf{representación natural} de $\Sym_n$ es la representación 
\[
	\rho\colon\Sym_n\to\GL(n,\C),\quad
	\sigma\mapsto\rho_{\sigma},
\] 
donde $\rho_\sigma(e_j)=e_{\sigma(j)}$ para todo $j\in\{1,\dots,n\}$. 
La matriz de $\rho_\sigma$ en la base canónica está dada por 
\begin{equation}
    \label{eq:Sn_natural}
    (\rho_\sigma)_{ij}=\begin{cases}
      1 & \text{si $i=\sigma(j)$},\\
      0 & \text{en otro caso}.
    \end{cases}
\end{equation}

\begin{lemma}
	\label{lem:permutaciones}
	Sea $n\in\N$ y sea $\rho\colon\Sym_n\to\GL(n,\C)$ la representación
	natural del grupo simétrico. Si $A\in\C^{n\times n}$ y $\sigma\in\Sym_n$ entonces 
	\[
		A_{ij}=(\rho_{\sigma}A)_{\sigma(i)j}=(A\rho_{\sigma})_{i\sigma^{-1}(j)}
	\]
	para todo $i,j\in\{1,\dots,n\}$.
\end{lemma}

\begin{proof}
	Con la fórmula~\eqref{eq:Sn_natural} calculamos
	\[
		(A\rho_{\sigma})_{ij}=\sum_{k=1}^n A_{ik}(\rho_{\sigma})_{kj}=A_{i\sigma(j)},
		\quad
		(\rho_\sigma A)_{ij}=\sum_{k=1}^n (\rho_\sigma)_{ik}A_{kj}=A_{\sigma^{-1}(i)j}.\qedhere
	\]
\end{proof}

\begin{definition}
  \index{Caracter!real}
  Sea $G$ un grupo finito. Un caracter $\chi$ de $G$ se dice \textbf{real} si
  $\chi=\overline{\chi}$, es decir si $\chi(g)\in\R$ para todo $g\in G$. 
\end{definition}

\begin{exercise}
	\label{xca:chi_irreducible}
	Demuestre que si $\chi$ es un carácter irreducible de un grupo finito $G$
	entonces $\overline{\chi}$ es irreducible.
\end{exercise}

\begin{definition}
  \index{Clase de conjugación!real}
  Sea $G$ un grupo. Una case de conjugación $C$ de $G$ se dice \textbf{real} si
  para cada $g\in C$ se tiene $g^{-1}\in C$. 
\end{definition}

Utilizaremos la siguiente notación: si $C=\{xgx^{-1}:x\in G\}$ es una clase de
conjugación de un grupo $G$, entonces $C^{-1}=\{xg^{-1}x^{-1}:x\in G\}$.  

\begin{theorem}[Burnside]
  \index{Burnside!Teorema de}
  \index{Teorema de!Burnside}
  Sea $G$ un grupo finito. La cantidad de clases de conjugación reales es igual
  a la cantidad de caracteres irreducibles reales.
\end{theorem}

\begin{proof}
  Sea $r$ la cantidad de clases de conjugación de $G$. Sean $C_1,\dots,C_r$ las
  clases de conjugación de $G$ y sean $\chi_1,\dots,\chi_r$ los caracteres
  irreducibles de $G$. 
  Sean $\alpha,\beta\in\Sym_r$ dados por $\overline{\chi_i}=\chi_{\alpha(i)}$ y
  $C_i^{-1}=C_{\beta(i)}$ para todo $i\in\{1,\dots,r\}$. Observar que $\chi_i$
  es real si y sólo si $\alpha(i)=i$ y que $C_i$ es real si y sólo si
  $\beta(i)=i$. La cantidad $n$ de puntos fijos de $\alpha$ es igual a la cantidad
  de caracteres irreducibles de $G$ y la cantidad $m$ de puntos fijos de $\beta$ es
  igual a la cantidad de clases reales. 

  Sea $\rho\colon\Sym_r\to\GL(r,\C)$ la representación natural de $\Sym_r$. Entonces
  $\chi_\rho(\alpha)=n$ y $\chi_\rho(\beta)=m$. Veamos que 
  $\trace\rho_\alpha=\trace\rho_\beta$. 
  Sea $X\in\GL(r,\C)$ la matriz de caracteres de $G$. Por el lema~\ref{lem:permutaciones}, 
  \[
	\rho_\alpha X=\overline{X}=X\rho_\beta.
  \]
  Como $X$ es una matriz inversible, $\rho_{\alpha}=X\rho_{\beta}X^{-1}$. Luego
  \[
    n=\chi_{\rho}(\alpha)=\trace\rho_{\alpha}=\trace\rho_{\beta}=\chi_{\rho}(\beta)=m.
  \]
\end{proof}

\begin{corollary}
  \label{corollary:|G|impar}
  Sea $G$ un grupo finito. Entonces $|G|$ es impar si y sólo si el único
  $\chi\in\Irr(G)$ real es el caracter trivial. 
\end{corollary}

% explicar mejor

\begin{proof}
  Supongamos que $G$ tiene una clase de conjugación $C$ real no trivial y sea
  $g\in C$. Basta con demostrar que $G$ tiene un elemento de orden par.  Sea
  $h\in G$ tal que $hgh^{-1}=g^{-1}$. Entonces $h^2\in C_G(g)$ (pues
  $h^2gh^{-2}=g$). Si $h\in\langle h^2\rangle\subseteq C_G(g)$, $g$ tiene orden
  par pues $g^{-1}=g$. Si $h\not\in\langle h^2\rangle$ entonces $h^2$ no es un
  generador de $\langle h\rangle$ y luego $2$ divide a $|h|$ (pues
  $|h|\ne|h^2|=|h|/(|h|:2)$).  Recíprocamente, si $|G|$ es par, existe $g\in G$
  de orden dos y la clase de conjugación de $g$ es real. 
\end{proof}

\begin{theorem}[Burnside]
  \index{Burnside!Teorema de}
  \index{Teorema de!Burnside}
  Sea $G$ un grupo de orden impar y sea $r$ el número de clases de conjugación
  de $G$. Entonces 
  \[
	  r\equiv|G|\bmod{16}.
  \]
\end{theorem}

\begin{proof}
  Como $|G|$ es impar, todo $\chi\in\Irr(G)$ no trivial es no real por el
  corolario anterior. Los caracteres irreducibles de $G$ son entonces 
  \[
    1,\chi_1,\overline{\chi_1},\dots,\chi_k,\overline{\chi_k},
    \quad
    r=1+2k.
  \]
  Para cada $j\in\{1,\dots,k\}$ sea $d_j=\chi_j(1)$.   Como cada $d_j$ divide a
  $|G|$ por el teorema~\ref{theorem:chi(1)||G|} de Frobenius y $|G|$ es impar, los $d_j$ son
  números impares, digamos $d_j=1+2m_j$. Entonces 
  \begin{align*}
    |G|&=1+\sum_{j=1}^k 2d_j^2=1+\sum_{j=1}^k2(2m_j+1)^2\\
    &=1+\sum_{j=1}^k2(4m_j^2+4m_j+1)
    =1+2k+8\sum_{j=1}^km_j(m_j+1).
  \end{align*}
  Luego $|G|\equiv r\bmod{16}$ pues $r=1+2k$ y cada $m_j(m_j+1)$ es un número par. 
\end{proof}

% \begin{theorem}
%   Si $G$ es un grupo finito no abeliano tal que existe $\chi\in\Irr(G)$ con 
%   $\chi(1)=2$, entonces $G$ no es simple.
% \end{theorem}

% \begin{proof}
%   Supongamos que $G$ es simple. Si $\rho$ es una representación irreducible con
%   caracter $\chi$, entonces $\rho$ es una representación fiel pues $G$ es
%   simple.  Como $\chi(1)$ divide al orden de $G$ por el teorema de Frobenius,
%   sabemos que $|G|$ es un número par.  Existe entonces $g\in G$ de orden dos.
%   Como $\rho_g$ es diagonalizable, existe una base en la que la matriz de
%   $\rho_g$ es de la forma 
%   $\begin{pmatrix}
%     a & 0\\
%     0 & b
%   \end{pmatrix}\ne\id$. Pero $\rho_g^2=\id$ y entonces $a^2=b^2=1$.  
%   Veamos que $ab=1$. Como $G$ es simple, $G=[G,G]$. Existen entonces 
%   $x_1,\dots,x_n,y_1,\dots,y_n\in G$ tales que
%   $g=\prod_{i=1}^n[x_i,y_i]$. En particular,
%   \[
% 	  ab=\det\rho_g=\det\prod_{i=1}^n\rho_{[x_i,y_i]}=1.
%   \]
%   En conclusión $a=b=-1$ y luego $\rho_g=-\id$. Esto implica
%   que $1\ne g\in Z(G)$, una contradicción.
% \end{proof}

\index{Serie!derivada}
Si $G$ es un grupo se define 
\[
		G^{(0)}=G,\quad
		G^{(i+1)}=[G^{(i)},G^{(i)}]\quad i\geq0.
\]
La \textbf{serie derivada} de $G$ se define entonces como
\[
G=G^{(0)}\supseteq G^{(1)}\supseteq G^{(2)}\supseteq\cdots
\]
Cada $G^{(i)}$ es un subgrupo característico de $G$. Diremos que
$G$ es \textbf{resoluble} si existe $n\in\N$ tal que $G^{(n)}=1$. 

%\begin{example}
%	El grupo $\SL_2(3)$ es resoluble. La serie derivada de $\SL_2(3)$ es
%	$\SL_2(3)\supseteq Q_8\supseteq C_4\supseteq C_2\supseteq 1$. Veamos el código:
%	\begin{lstlisting}
%gap> IsSolvable(SL(2,3));
%true
%gap> List(DerivedSeries(SL(2,3)),StructureDescription);
%[ "SL(2,3)", "Q8", "C2", "1" ]
%	\end{lstlisting}
%\end{example}

\begin{example}
	Todo grupo abeliano es resoluble.
\end{example}

\begin{example}
	El grupo $\SL_2(3)$ es resoluble. La serie derivada de $\SL_2(3)$ es
	\[
	\SL_2(3)\supseteq Q_8\supseteq C_4\supseteq C_2\supseteq 1.
	\]
	Veamos el código:
\begin{lstlisting}
gap> IsSolvable(SL(2,3));
true
gap> List(DerivedSeries(SL(2,3)),StructureDescription);
[ "SL(2,3)", "Q8", "C2", "1" ]
\end{lstlisting}
\end{example}

\begin{example}
	Un grupo simple no abeliano no es resoluble. 
\end{example}

\begin{theorem}
	\label{theorem:resoluble}
	Sea $G$ un grupo. 
	\begin{enumerate}
		\item Todo subgrupo $H$ de $G$ es resoluble. 
		\item Sea $K$ es un subgrupo normal de $G$. Entonces $G$ es resoluble
			si y sólo si $K$ y $G/K$ son resolubles.
	\end{enumerate}
\end{theorem}

\begin{proof}
	La primera afirmación es fácil pues $H^{(i)}\subseteq G^{(i)}$ para
	todo $i\geq0$. Demostremos la segunda afirmación. Sean $Q=G/K$ y $\pi\colon G\to Q$ el
	morfismo canónico. Demostramos por inducción que $\pi(G^{(i)})=Q^{(i)}$ para todo
	$i\geq0$. El caso $i=0$ es trivial pues $\pi$ es sobreyectiva. Si el
	resultado es válido para algún $i\geq0$ entonces
	\[
		\pi(G^{(i+1)})=\pi([G^{(i)},G^{(i)}])=[\pi(G^{(i)}),\pi(G^{(i)})]=[Q^{(i)},Q^{(i)}]=Q^{(i+1)}.
	\]

	Supongamos que $Q$ y $K$ son resolubles. Como $Q$ es resoluble, 
	existe $n$ tal que $Q^{(n)}=1$.
	Como $\pi(G^{(n)})=Q^{(n)}=1$, se tiene que $G^{(n)}\subseteq K$. Como $K$
	es resoluble, existe $m$ tal que
	\[
		G^{(n+m)}\subseteq (G^{(n)})^{(m)}\subseteq K^{(m)}=1,
	\]
	y luego $G$ es resoluble. 

	Supongamos ahora que $G$ es resoluble. Existe $n\in\N$ tal que $G^{(n)}=1$.
	Luego $Q$ es resoluble pues $Q^{n}=f(G^{(n)})=f(1)=1$. Además $K$ es
	resoluble por ser un subgrupo de $G$. 
\end{proof}

\begin{example}
	Sea $n\geq5$. El grupo $\Sym_n$ no es resoluble pues $\Alt_n$ no es
	resoluble.
\end{example}

\begin{example}
	Si $H$ y $K$ son grupos resolubles entonces $H\times K$ es resoluble.
\end{example}

\begin{proposition}
	Sea $p$ un número primo y sea $G$ un $p$-grupo finito. Entonces $G$ es
	resoluble.
\end{proposition}

\begin{proof}
	Procederemos por inducción en $|G|$. Supongamos que el resultado es válido
	para todos los $p$-grupos de orden $<|G|$. Como $Z(G)\ne1$, por hipótesis
	inductvia $G/Z(G)$ es un $p$-grupo resoluble.  Como $Z(G)$ es resoluble por
	ser un grupo abeliano, $G$ es resoluble por el
	teorema~\ref{theorem:resoluble}. 
\end{proof}

%\begin{exercise}
%	\label{exercise:resoluble:eq}
%	Sea $G$ un grupo. Demuestre que las siguientes afirmaciones son equivalentes:
%	\begin{enumerate}
%		\item $G$ es resoluble.
%		\item $G$ admite una sucesión de subgrupos $G=G_0\supseteq
%			G_1\supseteq\cdots\supseteq G_n=1$ tal que cada $G_i$ es normal en
%			$G$ y cada $G_{i-1}/G_i$ es abeliano.
%		\item $G$ admite una sucesión de subgrupos $G=G_0\supseteq
%			G_1\supseteq\cdots\supseteq G_n=1$ tal que cada $G_i$ es normal en
%			$G_{i-1}$ y cada $G_{i-1}/G_i$ es abeliano.
%	\end{enumerate}
%\end{exercise}
%
%\begin{svgraybox}
%	Para demostrar $(1)\implies(2)$ basta considerar la serie derivada. La
%	implicación $(2)\implies(3)$ es trivial. Para demostrar $(3)\implies(1)$
%	hay que observar que, como $G^{(i)}=[G^{(i-1)},G^{(i-1)}]\subseteq G_i$
%	pues $G_{i-1}/G_i$ es abeliano, $G^{(n)}=1$.
%\end{svgraybox}

Antes de demostrar el teorema de resolubilidad de Burnside vamos a demostrar un
resultado auxiliar que resulta de interés. Necesitamos un resultado previo:

\begin{lemma}
	\label{lem:4Burnside}
	Sean $\epsilon_1,\dots,\epsilon_n$ raíces de la unidad tales que
	$(\epsilon_1+\cdots+\epsilon_n)/n\in\A$. Entonces
	$\epsilon_1=\cdots=\epsilon_n$ o bien $\epsilon_1+\cdots+\epsilon_n=0$.
\end{lemma}

% explicar mejor

\begin{proof}
	Sea $\alpha=(\epsilon_1+\cdots+\epsilon_n)/n$.
	Si los $\epsilon_j$ no son todos iguales, entonces $N(\alpha)<1$. Además 
	$N(\beta)<1$ para todo conjugado algebraico $\beta$ de $\alpha$. Como el
	producto de los conjugados algebraicos de $\alpha$ es un entero de módulo
	$<1$, se conluye que es cero.
\end{proof}

\begin{theorem}[Burnside]
	\index{Teorema!de Burnside}
	\label{thm:Burnside_auxiliar}
	Sea $G$ un grupo finito. Sea $\phi\colon G\to\GL(n,\C)$ una representación
	con carácter $\chi$ y sea $C$ es una clase de conjugación de $G$ tal que 
	$(|C|:n)=1$. Para cada $g\in C$ se tiene que $\chi(g)=0$ o bien que $\phi_g$ es una matriz
	escalar. 
\end{theorem}

\begin{proof}
	Sean $\epsilon_1,\dots,\epsilon_n$ los autovalores de $\phi_g$. Como
	$(|C|:n)=1$, existen $a,b\in\Z$ tales que $a|C|+bn=1$.  Como
	$|C|\chi(g)/n\in\A$, al multiplicar por $\chi(g)/n$ obtenemos 
	\[
		a|C|\frac{\chi(g)}{n}+b\chi(g)=\frac{\chi(g)}{n}=\frac{1}{n}(\epsilon_1+\cdots+\epsilon_n)\in\A.
	\]
	El lema anterior nos dice que entonces hay dos posbilidades:
	$\epsilon_1=\cdots=\epsilon_n$ o bien $\epsilon_1+\cdots+\epsilon_n=0$. En
	el primer caso, como $\phi_g$ es diagonalizable, $\phi_g$ es una matriz
	escalar. El segundo caso dice exactamente que $\chi(g)=0$.
\end{proof}

\begin{theorem}[Burnside]
	\index{Teorema!de Burnside}
  Sea $p$ un número primo. Si $G$ es un grupo finito y $C$ es una clase de
  conjugación de $G$ con $p^k>1$ elementos, entonces $G$ no es simple.
\end{theorem}

\begin{proof}
	Sea $g\in C\setminus\{1\}$. Por la ortogonalidad de las columnas, 
	\begin{equation}
	\label{eq:Burnside}
	\begin{aligned}
		0&=\sum_{\chi\in\Irr(G)}\chi(1)\chi(g)\\
		&=\sum_{p\mid\chi(1)}\chi(1)\chi(g)+\sum_{p\nmid\chi(1)}\chi(1)\chi(g)+1,
	\end{aligned}
	\end{equation}
	donde el uno corresponde a la representación trivial de $G$. Al
	mirar~\eqref{eq:Burnside} módulo $p$ vemos que existe una representación no
	trivial irreducible $\phi$ con carácter $\chi$ tal que $p$ no divide a
	$\chi(1)$ y además $\chi(g)\ne0$. Por el teorema anterior, $\phi_g$ es una
	matriz escalar. Si $\phi$ es fiel, entonces $g$ es un elemento central no
	trivial, una contradicción pues $|C|>1$. En caso contrario, $G$ no es simple pues
	$\ker\phi$ es un subgrupo propio no trivial de $G$.
%	Si $p$ divide a $\deg\phi$, entonces
%	$\frac1p(\deg\phi)\chi_\phi(g)\in\A$ y luego
%	\[
%		\alpha=\sum_{p\mid\deg\phi}\frac1p(\deg\phi)\chi_\phi(g)\in\A.
%	\]
%	La fórmula~\eqref{eq:Burnside} queda entonces 
%	\[
%		0=1+p\alpha+\sum_{p\nmid\deg\phi}(\deg\phi)\chi_{\phi}(g).
%	\]
\end{proof}

\begin{theorem}[Burnside]
  \index{Teorema!de Burnside}
  Sean $p,q$ primos. Si $G$ tiene orden $p^aq^b$ entonces $G$ es resoluble.
\end{theorem}

\begin{proof}
	Supongamos que el teorema no es cierto y sea $G$ un grupo de orden $p^aq^b$
	minimal con la propiedad de no ser resoluble. La minimalidad de $|G|$
	implica que $G$ es simple. Por el teorema anterior, $G$ no tiene clases de
	conjugación de tamaño $p^k$ ni clases de tamaño $q^l$ con $k,l\geq1$. El
	tamaño de toda clase de conjugación de $G$ es entonces igual a uno o es
	divisible por $pq$. Pero entonces la ecuación de clases, 
	\[
		|G|=1+\sum_{C:|C|>1}|C|,
	\]
	donde la suma se hace sobre todas las clases de conjugación que tienen más
	de un elemento, da una contradicción.
\end{proof}

Concluimos el capítulo con los enunciados de algunas 
generalizaciones del teorema de Burnside. 

\begin{theorem}[Kegel--Wielandt]
\index{Teorema!de Kegel--Wielandt}
Si $G$ es un grupo finito y existen subgrupos nilpotentes $A$ y $B$ de $G$ tales
que $G=AB$, entonces $G$ es resoluble. 
\end{theorem}

La demostración del teorema de Kegel--Wielandt puede consultarse en el segundo 
capítulo del libro~\cite{MR1211633}, más precisamente en el teorema 2.13. 

\begin{theorem}[Feit--Thompson]
\index{Teorema!de Feit--Thompson}
Todo grupo finito de orden impar es resoluble. 
\end{theorem}

La demostración del teorema de Feit--Thompson es extremadamente difícil y ocupa un volumen completo del 
\emph{Pacific Journal of Mathematics}~\cite{MR166261}. 
En~\cite{MR3111271} se anunció haber verificado formalmente 
demostración del teorema de Feit--Thompson con el 
sistema de ayuda para la demostración de teoremas Coq. 

\medskip
\index{Conjetura!de Feit--Thompson}
En los sesenta se sabía que la demostración del teorema de Feit--Thomson iba a poder simplificarse 
si la conjetura de Feit--Thompson es verdadera:

\begin{quote}
No existen primos distintos $p$ y $q$ tales que
$\frac {p^{q}-1}{p-1}$ divide a $\frac{q^{p} - 1}{q - 1}$. 
\end{quote}

Ya no es necesaria esa conjetura para simplificar la demostración, 
y la conjetura de Feit--Thompson permanece abierta. 
En~\cite{MR297686} 
Stephens demostró que la versión fuerte de la conjetura no es cierta, ya que 
los enteros $\frac {p^{q}-1}{p-1}$ y $\frac{q^{p} - 1}{q - 1}$ 
podrían tener factores en común. De hecho, si $p=17$ y $q=3313$, 
entonces 
\[
\gcd\left(\frac {p^{q}-1}{p-1},\frac{q^{p} - 1}{q - 1}\right)=112643.
\]
Hoy podemos reproducir los cálculos de 
Stephens con casi cualquier computadora de escritorio:
\begin{lstlisting}
gap> Gcd((17^3313-1)/16,(3313^17-1)/3312);
112643
\end{lstlisting}

Otra dirección en la que puede generalizarse el teorema de Burnside es con el uso de las funciones de palabra. 
Una \emph{función de palabra} de un grupo $G$ es una función 
\[
G^k\to G,\quad 
(x_1,\dots,x_k)\mapsto w(x_1,\dots,x_k)
\]
para alguna 
palabra $w(x_1,\dots,x_k)$ en el grupo libre $F_k$ de rango $k$. 
Algunas palabras son sobreyecticas en 
todo grupo o en cierta familia de grupos. Por ejemplo, 
la conjetura de Ore es la sobreyectividad de la función 
$(x,y)\mapsto [x,y]=xyx^{-1}y^{-1}$ en todo grupo finito simple no abeliano.

\begin{theorem}[Guralnick--Liebeck--O'Brien--Shalev--Tiep]
Sean $p$ y $q$ dos primos, $a,b\geq0$ y $N=p^aq^b$. La función $(x,y)\mapsto x^Ny^N$ es 
sobreyectiva en todo grupo simple.
\end{theorem}

El teorema fue demostrado en~\cite{MR3827208}. 

Veamos por qué implica el teorema de Burnside. Supongamos que $G$ es un grupo de orden $N=p^aq^b$ y que $G$ no es resoluble. 
Si fijamos una serie de composición de $G$, tenemos un factor $S$ no abeliano de orden que divide a $N$. Como entonces
$S$ es simple y no abeliano y $s^N=1$, se concluye que la función $(x,y)\mapsto x^Ny^N$ tiene imagen trivial en $S$, una contradicción al teorema. 

