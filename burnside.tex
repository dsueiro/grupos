\chapter{Algunos teoremas de Burnside}

Recordemos cómo actúa la representación natural del grupo simétrico.  Sea
$n\in\N$ y sea $\{e_1,\dots,e_n\}$ la base canónica de $\C^n$.  La
\textbf{representación natural} de $\Sym_n$ es la representación 
\[
	\rho\colon\Sym_n\to\GL(n,\C),\quad
	\sigma\mapsto\rho_{\sigma},
\] 
donde $\rho_\sigma(e_j)=e_{\sigma(j)}$ para todo $j\in\{1,\dots,n\}$. 
La matriz de $\rho_\sigma$ en la base canónica está dada por 
\begin{equation}
    \label{eq:Sn_natural}
    (\rho_\sigma)_{ij}=\begin{cases}
      1 & \text{si $i=\sigma(j)$},\\
      0 & \text{en otro caso}.
    \end{cases}
\end{equation}

\begin{lemma}
	\label{lem:permutaciones}
	Sea $n\in\N$ y sea $\rho\colon\Sym_n\to\GL(n,\C)$ la representación
	natural del grupo simétrico. Si $A\in\C^{n\times n}$ y $\sigma\in\Sym_n$ entonces 
	\[
		A_{ij}=(\rho_{\sigma}A)_{\sigma(i)j}=(A\rho_{\sigma})_{i\sigma^{-1}(j)}
	\]
	para todo $i,j\in\{1,\dots,n\}$.
\end{lemma}

\begin{proof}
	Con la fórmula~\eqref{eq:Sn_natural} calculamos
	\[
		(A\rho_{\sigma})_{ij}=\sum_{k=1}^n A_{ik}(\rho_{\sigma})_{kj}=A_{i\sigma(j)},
		\quad
		(\rho_\sigma A)_{ij}=\sum_{k=1}^n (\rho_\sigma)_{ik}A_{kj}=A_{\sigma^{-1}(i)j}.\qedhere
	\]
\end{proof}

\begin{definition}
  \index{Caracter!real}
  Sea $G$ un grupo finito. Un caracter $\chi$ de $G$ se dice \textbf{real} si
  $\chi=\overline{\chi}$, es decir si $\chi(g)\in\R$ para todo $g\in G$. 
\end{definition}

\begin{exercise}
	\label{xca:chi_irreducible}
	Demuestre que si $\chi$ es un carácter irreducible de un grupo finito $G$
	entonces $\overline{\chi}$ es irreducible.
\end{exercise}

\begin{definition}
  \index{Clase de conjugación!real}
  Sea $G$ un grupo. Una case de conjugación $C$ de $G$ se dice \textbf{real} si
  para cada $g\in C$ se tiene $g^{-1}\in C$. 
\end{definition}

Utilizaremos la siguiente notación: si $C=\{xgx^{-1}:x\in G\}$ es una clase de
conjugación de un grupo $G$, entonces $C^{-1}=\{xg^{-1}x^{-1}:x\in G\}$.  

\begin{theorem}[Burnside]
  \index{Burnside!Teorema de}
  \index{Teorema de!Burnside}
  Sea $G$ un grupo finito. La cantidad de clases de conjugación reales es igual
  a la cantidad de caracteres irreducibles reales.
\end{theorem}

\begin{proof}
  Sea $r$ la cantidad de clases de conjugación de $G$. Sean $C_1,\dots,C_r$ las
  clases de conjugación de $G$ y sean $\chi_1,\dots,\chi_r$ los caracteres
  irreducibles de $G$. 
  Sean $\alpha,\beta\in\Sym_r$ dados por $\overline{\chi_i}=\chi_{\alpha(i)}$ y
  $C_i^{-1}=C_{\beta(i)}$ para todo $i\in\{1,\dots,r\}$. Observar que $\chi_i$
  es real si y sólo si $\alpha(i)=i$ y que $C_i$ es real si y sólo si
  $\beta(i)=i$. La cantidad $n$ de puntos fijos de $\alpha$ es igual a la cantidad
  de caracteres irreducibles de $G$ y la cantidad $m$ de puntos fijos de $\beta$ es
  igual a la cantidad de clases reales. 

  Sea $\rho\colon\Sym_r\to\GL(r,\C)$ la representación natural de $\Sym_r$. Entonces
  $\chi_\rho(\alpha)=n$ y $\chi_\rho(\beta)=m$. Veamos que 
  $\trace\rho_\alpha=\trace\rho_\beta$. 
  Sea $X\in\GL(r,\C)$ la matriz de caracteres de $G$. Por el lema~\ref{lem:permutaciones}, 
  \[
	\rho_\alpha X=\overline{X}=X\rho_\beta.
  \]
  Como $X$ es una matriz inversible, $\rho_{\alpha}=X\rho_{\beta}X^{-1}$. Luego
  \[
    n=\chi_{\rho}(\alpha)=\trace\rho_{\alpha}=\trace\rho_{\beta}=\chi_{\rho}(\beta)=m.\qedhere
  \]
\end{proof}

\begin{corollary}
  \label{corollary:|G|impar}
  Sea $G$ un grupo finito. Entonces $|G|$ es impar si y sólo si el único
  $\chi\in\Irr(G)$ real es el caracter trivial. 
\end{corollary}

% explicar mejor

\begin{proof}
  Supongamos que $G$ tiene una clase de conjugación $C$ real no trivial y sea
  $g\in C$. Basta con demostrar que $G$ tiene un elemento de orden par.  Sea
  $h\in G$ tal que $hgh^{-1}=g^{-1}$. Entonces $h^2\in C_G(g)$ (pues
  $h^2gh^{-2}=g$). Si $h\in\langle h^2\rangle\subseteq C_G(g)$, $g$ tiene orden
  par pues $g^{-1}=g$. Si $h\not\in\langle h^2\rangle$ entonces $h^2$ no es un
  generador de $\langle h\rangle$ y luego $2$ divide a $|h|$ (pues
  $|h|\ne|h^2|=|h|/(|h|:2)$).  Recíprocamente, si $|G|$ es par, existe $g\in G$
  de orden dos y la clase de conjugación de $g$ es real. 
\end{proof}

\begin{theorem}[Burnside]
  \index{Burnside!Teorema de}
  \index{Teorema de!Burnside}
  Sea $G$ un grupo de orden impar y sea $r$ el número de clases de conjugación
  de $G$. Entonces 
  \[
	  r\equiv|G|\bmod{16}.
  \]
\end{theorem}

\begin{proof}
  Como $|G|$ es impar, todo $\chi\in\Irr(G)$ no trivial es no real por el
  corolario anterior. Los caracteres irreducibles de $G$ son entonces 
  \[
    \chi_1,\chi_2,\overline{\chi_2},\dots,\chi_k,\overline{\chi_k},
    \quad
    r=1+2k,
  \]
  donde $\chi_1$ representa al carácter trivial. 
  Para cada $j\in\{2,\dots,k\}$ sea $d_j=\chi_j(1)$.   Como cada $d_j$ divide a
  $|G|$ por el teorema~\ref{theorem:chi(1)||G|} de Frobenius y $|G|$ es impar, los $d_j$ son
  números impares, digamos $d_j=1+2m_j$. Entonces 
  \begin{align*}
    |G|&=1+\sum_{j=2}^k 2d_j^2=1+\sum_{j=2}^k2(2m_j+1)^2\\
    &=1+\sum_{j=2}^k2(4m_j^2+4m_j+1)
    =1+2k+8\sum_{j=2}^km_j(m_j+1).
  \end{align*}
  Luego $|G|\equiv r\bmod{16}$ pues $r=1+2k$ y cada $m_j(m_j+1)$ es un número par. 
\end{proof}

\begin{exercise}
Demuestre que todo grupo de orden 15 es abeliano.
\end{exercise}

Hirsch demostró una generaización de este teorema de Burnside. 

\begin{theorem}[Hirsch]
\index{Teorema!de Hirsch}
Sea $G$ un grupo finito con $r$ clases de conjugación y sean 
$p_1,\dots,p_k$ los primos que dividen al orden de $G$. Si 
$d$ es el máximo común divisor de los números $p_i^2-1$, donde $i\in\{1,\dots,k\}$, 
entonces
\[
|G|\equiv\begin{cases}
r\bmod 2d&\text{si $|G|$ es impar},\\
r\bmod 3&\text{si $|G|$ es par y $\gcd(3,|G|)=1$.}
\end{cases}
\]
\end{theorem}

La demostración de Hirsch es completamente elemental --no utiliza teoría de caracteres-- 
y puede consultarse en~\cite{MR36755}. 


% \begin{theorem}
%   Si $G$ es un grupo finito no abeliano tal que existe $\chi\in\Irr(G)$ con 
%   $\chi(1)=2$, entonces $G$ no es simple.
% \end{theorem}

% \begin{proof}
%   Supongamos que $G$ es simple. Si $\rho$ es una representación irreducible con
%   caracter $\chi$, entonces $\rho$ es una representación fiel pues $G$ es
%   simple.  Como $\chi(1)$ divide al orden de $G$ por el teorema de Frobenius,
%   sabemos que $|G|$ es un número par.  Existe entonces $g\in G$ de orden dos.
%   Como $\rho_g$ es diagonalizable, existe una base en la que la matriz de
%   $\rho_g$ es de la forma 
%   $\begin{pmatrix}
%     a & 0\\
%     0 & b
%   \end{pmatrix}\ne\id$. Pero $\rho_g^2=\id$ y entonces $a^2=b^2=1$.  
%   Veamos que $ab=1$. Como $G$ es simple, $G=[G,G]$. Existen entonces 
%   $x_1,\dots,x_n,y_1,\dots,y_n\in G$ tales que
%   $g=\prod_{i=1}^n[x_i,y_i]$. En particular,
%   \[
% 	  ab=\det\rho_g=\det\prod_{i=1}^n\rho_{[x_i,y_i]}=1.
%   \]
%   En conclusión $a=b=-1$ y luego $\rho_g=-\id$. Esto implica
%   que $1\ne g\in Z(G)$, una contradicción.
% \end{proof}

