\chapter{El teorema de It\^o}

\begin{definition}
	\index{Grupo!metabeliano}
	Un grupo $G$ se dice \textbf{metabeliano} si $[G,G]$ es abeliano. 
\end{definition}

\begin{exercise}
\label{xca:metabelian1}
	Demuestre que un grupo $G$ es metabeliano si y sólo si existe un subgrupo
	normal $K$ de $G$ tal que $K$ y $G/K$ son abelianos.
\end{exercise}


% \begin{remark}
% 	Los grupos metabelianos son resolubles pues 
% 	si $G$ es metabeliano entonces 
% 	$G\supseteq [G,G]\supseteq 1$ es una serie resoluble para $G$.
% \end{remark}

\begin{exercise}
\label{xca:metabelian2}
	Sea $G$ un grupo metabeliano. 
	\begin{enumerate}
		\item Si $H$ es un subgrupo de $G$ entonces $H$ es metabeliano.
		\item Si $f\colon G\to H$ es un morfismo entonces $f(H)$ es
			metabeliano.
	\end{enumerate}
\end{exercise}

\begin{lemma}
En un grupo valen las siguientes fórmulas:
\begin{enumerate}
	\item $[a,bc]=[a,b]b[a,c]b^{-1}$. 
	\item $[ab,c]=a[b,c]a^{-1}[a,c]$.
\end{enumerate}
\end{lemma}

\begin{proof}
Es un cálculo directo:
 	\begin{align*}
 	&[a,b]b[a,c]b^{-1}=aba^{-1}b^{-1}baca^{-1}c^{-1}b^{-1}=abca^{-1}c^{-1}b^{-1}=[a,bc],\\
 	&a[b,c]a^{-1}[a,c]=abcb^{-1}c^{-1}a^{-1}aca^{-1}c^{-1}=abcb^{-1}a^{-1}c^{-1}=[ab,c].\qedhere
 	\end{align*}
\end{proof}

\begin{example}
	El grupo $\Sym_3$ es metabeliano pues $\Alt_3\simeq C_3$ es un subgrupo normal de
	$\Sym_3$ tal que $\Sym_3/\Alt_3\simeq C_2$ es abeliano.
\end{example}

\begin{example}
	El grupo $\Alt_4$ es metabeliano pues el subgrupo 
	\[
	K=\{\id,(12)(34),(13)(24),(14)(23)\}
	\]
	es abeliano y
	normal en $\Alt_4$ y el cociente $\Alt_4/K\simeq C_3$ es abeliano.
\end{example}


\begin{example}
	El grupo $\SL_2(3)$ no es metabeliano pues $[\SL_2(3),\SL_2(3)]\simeq Q_8$ 
	no es un grupo abeliano. En efecto:
\begin{lstlisting}
gap> IsAbelian(DerivedSubgroup(SL(2,3)));
false
gap> StructureDescription(DerivedSubgroup(SL(2,3)));
"Q8"
\end{lstlisting}
\end{example}

El siguiente resultado considerado uno de los 
resultados más importantes de la teoría de factoriación 
de grupos. La demostración es sorprendentemente simple. 

% todo: después
\begin{theorem}[It\^o]
	\label{theorem:Ito}
	Sea $G=AB$ una factorización de $G$ con $A$ y $B$ subgrupos de $G$
	abelianos. Entonces $G$ es metabeliano.
\end{theorem}

\begin{proof}
	Como $G=AB$ entonces $AB=BA$. Veamos primero que $[A,B]$ es un subgrupo
	normal de $G$. Sean $a,\alpha\in A$, $b,\beta\in B$. Sean $a_1,a_2\in A$,
	$b_1,b_2\in B$ tales que $\alpha b\alpha^{-1}=b_1a_1$, $\beta
	a\beta^{-1}=a_2b_2$. Entonces, como
	\begin{align*}
		&\alpha[a,b]\alpha^{-1}=a(\alpha b\alpha^{-1})a^{-1}(\alpha b^{-1}\alpha^{-1})=ab_1a_1a^{-1}a_1^{-1}b_1^{-1}=[a,b_1]\in [A,B]\\
		&\beta[a,b]\beta^{-1}=(\beta a\beta^{-1})\beta b\beta^{-1}(\beta a^{-1}\beta^{-1})b^{-1}=a_2b_2bb_2^{-1}a_2^{-1}b^{-1}=[a_2,b]\in [A,B],
	\end{align*}
	se concluye que $[A,B]$ es normal en $G$. 

	Veamos ahora que $[A,B]$ es abeliano. Como 
	\begin{align*}
		&\beta\alpha[a,b]\alpha^{-1}\beta^{-1} = \beta[a,b_1]\beta^{-1}=(\beta a\beta^{-1})b_1(\beta a^{-1}\beta^{-1})b_1^{-1}=[a_2,b_1],\\
		&\alpha\beta[a,b]\beta^{-1}\alpha^{-1} = \alpha[a_2,b]\alpha^{-1}=a_2(\alpha b\alpha^{-1})a_2^{-1}(\alpha b\alpha^{-1})=[a_2,b_1],
	\end{align*}
	un cálculo directo muestra que 
	\[
		[\alpha^{-1},\beta^{-1}][a,b][\alpha^{-1},\beta^{-1}]^{-1}=[a,b].
	\]
	Como dos generadores arbitrarios de $[A,B]$ conmutan, el grupo $[A,B]$ es
	abeliano. 
	
	Para completar la demostración observamos $[G,G]=[A,B]$ pues 
	\[
	[a_1b_1,a_2b_2]=a_1[a_2,b_1]^{-1}a_1^{-1}a_2[a_1,b_2]a_2^{-1}\subseteq [A,B],
	\]
	ya que $[A,B]$ es normal en $G$. 
\end{proof}

En 1988 apareció la siguiente generalización 
del teorema de It\^o. 

\begin{theorem}[Sysak]
\index{Teorema!de Sysak}
    Sean $A$ y $B$ subgrupos abelianos de $G$. Si $H$ es un
    subgrupo de $G$ tal que $H$ está contenido en el 
    conjunto $AB$, entonces $H$ es meta-abeliano. 
\end{theorem}

Para la demostración referimos a \cite{MR988177}.