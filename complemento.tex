\chapter{El teorema del complemento normal}

\begin{lemma}
	\label{lemma:normal_complement}
	Sea $G$ un grupo finito y sea $p$ un primo que divide al orden de $G$. Sea
	$P\in\Syl_p(G)$. Si $g,h\in C_G(P)$ son conjugados en $G$ entonces son
	conjugados en $N_G(P)$.
\end{lemma}

\begin{proof}
	Sea $x\in G$ tal que $g=xhx^{-1}$. Entonces $g\in C_G(xPx^{-1})$. Luego $P$
	y $xPx^{-1}$ son subgrupos de Sylow de $C_G(g)$. Por el teorema de Sylow,
	existe $c\in C_G(g)$ tal que $P=cxP(cx)^{-1}$. Tenemos $cx\in N_G(P)$ y 
	\[
	(cx)h(cx)^{-1}=c(xhx^{-1})c^{-1}=cgc^{-1}=g.
	\]
\end{proof}

\begin{definition}
	\index{Complemento normal}
	Sea $G$ un grupo finito y sea $p$ un primo que divide al orden de $G$. Un
	\textbf{$p$-complemento normal} es un subgrupo normal $N$ de orden coprimo
	con $p$ y tal que $(G:N)$ es una potencia de $p$.
\end{definition}

\begin{definition}
	\index{$p$-nilpotente}
	Un grupo finito se dice \textbf{$p$-nilpotente} si tiene un $p$-complemento
	normal. 
\end{definition}

\begin{proposition}
	Si $G$ tiene un $p$-complemento normal $N$, entonces $N$ es un subgrupo
	característico de $G$.
\end{proposition}

\begin{proof}
	Supongamos que $|G|=p^\alpha n$, donde $n$ es coprimo con $p$, y sea
	$\pi\colon G\to G/N$ el morfismo canónico.  Por hipótesis, $N$ tiene orden
	$n$. Vamos a demostrar que $N$ es el único subgrupo de $G$ de orden $n$. Si
	$K$ es un subgrupo de $G$ de orden $n$, entonces  $\pi(K)\simeq K/K\cap N$
	y luego el orden de $\pi(K)$ divide a $m$. Pero además el orden de $\pi(K)$
	divide al primo $p$ pues $\pi(K)\leq G/N$. Luego $\pi(K)$ es trivial y
	entonces $K=N$ y luego $G$ tiene un único subgrupo de orden $n$. En
	particular, $N$ es un subgrupo característico de $G$.
\end{proof}



\begin{theorem}[Burnside]
	\index{Burnside!teorema del complemento normal de}
	\index{Teorema del complemento normal}
	\label{theorem:Burnside:normal_complement}
	Sea $G$ un grupo finito y sea $p$ un primo que divide a $|G|$. Sea
	$P\in\Syl_p(G)$ tal que $P\subseteq Z(N_G(P))$. Entonces $G$ es
	$p$-nilpotente.
\end{theorem}

\begin{proof}
	Como $P$ es abeliano, sea $\nu\colon G\to P$ el morfismo de transferencia.
	Sea $g\in P$.  Por el lema~\ref{lemma:evaluation} existen $s_1,\dots,s_m\in
	G$ y existen $n_1,\dots,n_m$ tales que $n_1+\cdots+n_m=n$,
	$s_i^{-1}g^{n_i}s_i\in P$ y 
	\[
		v(g)=\prod_{i=1}^m s_i^{-1}g^{n_i}s_i.
	\]
	Como $P$ es abeliano, $P\subseteq C_G(P)$. Por 
	el lema~\ref{lemma:normal_complement}, existe $c_i\in N_G(P)$ tal
	que 
	\[
	s_i^{-1}g^{n_i}s_i=c_i^{-1}g^{n_i}c_i,
	\]
	y luego $s_i^{-1}g^{n_i}s_i=g_i^{n_i}$ pues $P\subseteq Z(N_G(P))$. Tenemos
	entonces que $\nu(g)=g^n$, donde $n=(G:P)$. Como $n$ y $|P|$ son coprimos,
	existen $r,s\in\Z$ tales que $rn+s|P|=1$. Esto implica que $\nu|_P$ es
	sobreyectiva pues
	\[
	g=(g^r)^n=\nu(g^r).
	\]
	Por el teorema de isomorfismos, $P/\ker\nu\cap P\simeq\nu(P)=P$. 
	Luego $\ker\nu\cap P=1$. Además $\nu(G)=\nu(P)$ pues 
	$P\supseteq \nu(G)\supseteq \nu(P)=P$.
	
	Veamos que $\ker\nu$ es un $p$-complemento normal en $G$. Es claro que $\ker\nu$ es normal en $G$. 
	Como $(G:\ker\nu)=|\nu(G)|=|P|$ y $P$ es un $p$-subgrupo de Sylow, se concluye que $\ker\nu$ tiene orden coprimo con $p$.
%	Sea $K=\ker\nu$. Por el primer teorema de isomorfismo, $G/K\simeq P$. El
%	subgrupo $K$ es un complemento normal de $P$ en $G$ pues $G=KP$ y $K\cap
%	P=1$ (pues $|K|=n$ y $|P|$ son coprimos).
\end{proof}

% $P\subseteq Z(N_G(P))$ si y sólo si $N_G(P)=C_G(P)$.


\begin{lemma}
	\label{lemma:NC}
	Sea $G$ un grupo y $H$ un subgrupo de $G$. Entonces $C_G(H)$ es un subgrupo
	normal de $N_G(H)$ y $N_G(H)/C_G(H)$ es isomorfo a un subgrupo de
	$\Aut(H)$.
\end{lemma}

\begin{proof}
	Sea $\phi\colon N_G(H)\to\Aut(H)$,  $\phi(g)=c_g|H$, donde
	$c_g(h)=ghg^{-1}$.  La función $\phi$ está bien definida (pues su dominio
	es $N_G(H)$) y es morfismo de grupos. Como $\ker\phi=C_G(H)$, se tiene que
	$C_G(H)$ es normal en $N_G(H)$. Por el primer teorema de isomorfismo,
	$N_G(H)/C_G(H)\simeq\phi(N_G(H))\leq\Aut(H)$.
\end{proof}

\begin{corollary}
	\label{corollary:Sylow_ciclico}
	Sea $G$ un grupo finito y sea $p$ el menor primo que divide a $|G|$. Si
	algún $P\in\Syl_p(G)$ es cíclico, $G$ es $p$-nilpotente.
\end{corollary}

\begin{proof}
	Supongamos que $|P|=p^m$.  Por el lema~\ref{lemma:NC}, $N_G(P)/C_G(P)$ es
	isomorfo a un subgrupo de $\Aut(P)$. Como $P$ es cíclico, $|N_G(P)/C_G(P)|$ divide a 
	\[
		|\Aut(P)|=\phi(|P|)=p^{m-1}(p-1).
	\]
	Como $P\subseteq C_G(P)$ por ser $P$ abeliano, $p$ es coprimo con
	$|N_G(P)/C_G(P)|$.  Luego $|N_G(P)/C_G(P)|$ divide a $p-1$. Pero $p-1$ y
	$|G|$ son coprimos, pues $p$ es el menor primo que divide a $|G|$.  Como
	además $|N_G(P)/C_G(P)|$ divide al orden de $G$, se concluye que
	$|N_G(P)/C_G(P)|=1$, es decir: $N_G(P)=C_G(P)$. 

	Como $P$ es abeliano, $P\subseteq Z(C_G(P))=Z(N_G(P))$. El teorema de
	Burnside~\ref{theorem:Burnside:normal_complement} implica entonces que $G$
	es $p$-nilpotente. 
\end{proof}

\begin{exercise}
	Sea $G$ un grupo finito tal que todos sus subgrupos de Sylow son cíclicos.
	Entonces $G$ es resoluble.
\end{exercise}

Vamos a demostrar algo más fuerte:

\begin{corollary}
	\label{corollary:Sylow_ciclicos:resoluble}
	Sea $G$ un grupo finito tal que todos sus subgrupos de Sylow son cíclicos.
	Entonces $G$ es superresoluble.
\end{corollary}

%\begin{proof}
%	Supongamos que $G$ es no trivial y hagamos inducción en $|G|$. Si $p$ es el
%	menor primo que divide a $|G|$, por el
%	corolario~\ref{corollary:Sylow_ciclico} el grupo $G$ tiene un
%	$p$-complemento normal $N$. Por hipótesis inductiva, $N$ es super
%	resoluble, y entonces existe una sucesión
%	\[
%		1=N_0\subseteq N_1\subseteq\cdots\subseteq N_k=N
%	\]
%	de subgrupos normales de $N$ tal que cada cociente $N_{i+1}/N_i$ es cíclico
%	de orden primo.  Como $K$ es un complemento normal, es un subgrupo
%	característico de $G$ y luego cada $N_i$ es normal en $G$. Sea $P\in\Syl_p(G)$ 
%	y supongamos que $|G|=p^{\alpha}m$ con $m$ no divisible por $p$. 
%	Sea $\pi\colon G\to G/K\simeq P$ el morfismo canónico. 
%	Sabemos que para cada $j\in\{1,\dots,\alpha-1\}$, existe un subgrupo $H_h$ tal que $K\subseteq H_j$ y $|\pi(H_j)|=p^j$. 
%	Como $P$ es cíclico, todo subgrupo es característico en $P$ y luego 
%\end{proof}
%\begin{example}
%	Si los Sylows de un grupo finito $G$ son cíclicos y $p$ es el menor primo que
% 	divide a $|G|$, sabemos que existe un $p$-complemento. Este subgrupo podría no ser cíclico!
%\end{example}

\begin{proof}
	Supongamos que $G$ es no trivial y hagamos inducción en $|G|$. Si $p$ es el
	menor primo que divide a $|G|$, por el
	corolario~\ref{corollary:Sylow_ciclico} el grupo $G$ tiene un
	$p$-complemento normal $N$. Por hipótesis inductiva, $N$ es resoluble. Como
	$G/N$ es un $p$-grupo, es resoluble. Luego $G$ es resoluble.
\end{proof}

\begin{corollary}
	Sea $G$ un grupo finito cuyo orden es libre de cuadrados. Entonces $G$ es
	resoluble.
\end{corollary}

\begin{proof}
	Es consecuencia del corolario~\ref{corollary:Sylow_ciclicos:resoluble} pues
	en este caso todo subgrupo de Sylow es cíclico.  
\end{proof}

\begin{corollary}
	Sea $G$ un grupo finito simple no abeliano y sea $p$ el menor primo que
	divide a $|G|$. Entonces $p^3$ divide a $|G|$ o bien $12$ divide a $|G|$.
\end{corollary}

\begin{proof}
	Sea $P\in\Syl_p(G)$. Por el corolario~\ref{corollary:Sylow_ciclico}, $P$ no
	es cíclico, y entonces $|P|\geq p^2$. Si $p^3$ no divide a $|G|$, $P\simeq
	C_p\times C_p$ es un $\F_p$-espacio vectorial de dimensión dos. Como
	$|N_G(P)/C_G(P)|$ divide al orden de $G$, los divisores primos de
	$|N_G(P)/C_G(P)|$ son $\geq p$. Además, como $N_G(P)/C_G(P)$ es isomorfo a
	un subgrupo de $\Aut(P)$ por el lema~\ref{lemma:NC} y
	$\Aut(P)\simeq\GL_2(p)$ tiene orden $(p^2-1)(p^2-p)=p(p+1)(p-1)^2$,
	$|N_G(P)/C_G(P)|$ divide a $p(p+1)(p-1)^2$.  Como $P$ es abeliano,
	$P\subseteq C_G(P)$. Entonces $|N_G(P)/C_G(P)|$ es coprimo con $p$ y luego
	$|N_G(P)/C_G(P)|$ divide a $(p+1)(p-1)^2$. Como $p$ es el menor primo que
	divide a $|G|$, los números $p-1$ y $|N_G(P)/C_G(P)|$ son coprimos, y
	entonces $|N_G(P)/C_G(P)|$ divide a $p+1$.  Además, por el teorema de
	Burnside~\ref{theorem:Burnside:normal_complement}, $|N_G(P)/C_G(P)|\ne1$.
	Esto implica que $p=2$ pues si $p$ es impar el menor primo que divide a
	$|N_G(P)/C_G(P)|$ es $\geq p+2$.  Como entonces $p=2$, se concluye que
	$|N_G(P)/C_G(P)|=3$ y luego $|G|$ es divisible por $12=2^23$. 
\end{proof}

\begin{theorem}
	\label{theorem:[GG]PZNG(P)=1}
	Sea $G$ un grupo finito y sea $P$ un subgrupo de Sylow abeliano. Entonces 
	$[G,G]\cap P\cap Z(N_G(P))=1$.
\end{theorem}

\begin{proof}
	Sea $x\in [G,G]\cap P\cap Z(N_G(P))$ y sea $\nu\colon G\to P$ el morfismo
	de transferencia.  Por el lema~\ref{lemma:evaluation} existen
	$s_1,\dots,s_m\in G$ y existen $n_1,\dots,n_m$ tales que
	$n_1+\cdots+n_m=(G:P)$, $s_i^{-1}g^{n_i}s_i\in P$ y 
	\[
		v(x)=\prod_{i=1}^m s_i^{-1}x^{n_i}s_i.
	\]
	Como $P$ es abeliano, $P\subseteq C_G(P)$. Entonces $x^{n_i}$ y
	$s_i^{-1}x^{n_i}s_i$ son conjugados en $N_G(P)$ por el
	lema~\ref{lemma:normal_complement}. Como $x^{n_i}$ es central en $N_G(P)$ y
	$[G,G]\subseteq\ker\nu$, se concluye que $x=1$ pues $1=\nu(x)=x^{(G:P)}$ y
	$x\in P$.
\end{proof}

\begin{corollary}
	Sea $G$ un grupo finito no abeliano y sea $P\in\Syl_2(G)$ tal que $P\simeq
	C_{a_1}\times\cdots\times C_{a_k}$ con $a_1>a_2\geq a_3\geq\cdots\geq
	a_k\geq 2$.  Entonces $G$ no es simple. 
\end{corollary}

\begin{proof}
	Sea $S=\{x^{n/2}:x\in P\}$. Es fácil ver que $S$ es un subgrupo de $P$ y
	que $S$ es característico en $P$, es decir: $f(S)\subseteq S$ para todo
	$f\in\Aut(P)$. Como $S\simeq C_2$, podemos escribir $S=\{1,s\}$. Entonces
	$s\in Z(N_G(P))$ pues $gsg^{-1}\in S$ para todo $g\in N_G(P)$. Por el
	teorema~\ref{theorem:[GG]PZNG(P)=1}, $s\not\in[G,G]$ y luego $[G,G]\ne G$.
	Si $G$ fuera simple, $G$ sería abeliano pues $[G,G]=1$.
\end{proof}

Vimos en el corolario~\ref{corollary:Sylow_ciclicos:resoluble} que todo grupo
tal que todos sus subgrupos de Sylow son cíclicos es resoluble.

\begin{definition}
	\index{Z-grupo}
	Un Z-grupo es un grupo finito $G$ tal que 
	odos sus subgrupos de Sylow son
	cíclicos.
\end{definition}

\index{Grupo!meta-cíclico}
Un grupo $G$ se dice \emph{meta-cíclico} si $G$ tiene un subgrupo normal $N$
cíclico tal que $G/N$ es cíclico.

\begin{lemma}
	Si $G$ es un grupo resoluble, entonces $C_G(F(G))=F(G)$.
%	https://groupprops.subwiki.org/wiki/Solvable_implies_Fitting_subgroup_is_self-centralizing		
\end{lemma}

\begin{proof}
	
\end{proof}

\begin{theorem}
	\label{theorem:Z=>metacyclic}
	Todo Z-grupo es meta-cíclico.
\end{theorem}

\begin{proof}
	Sea $G$ un Z-grupo.	Por el
	corolario~\ref{corollary:Sylow_ciclicos:resoluble}, $G$ es resoluble y
	entonces, por el lema, el subgrupo de Fitting $F(G)$ satisface
	$C_G(F(G))\subseteq F(G)$. 
	
	Demostremos que $F(G)$ es cíclico. En efecto, como $F(G)$ es nilpotente,
	$F(G)$ es producto directo de sus subgrupos de Sylow. Como todo subgrupo de
	Sylow de $F(G)$ es un $p$-subgrupo de $G$, todo Sylow de $F(G)$ es cíclico
	(por estar contenido en algún subgrupo de Sylow de $G$). 

	Como $F(G)$ es cíclico, $F(G)$ es en particular abeliano y luego
	$F(G)\subseteq C_G(F(G))$. Si $G$ actúa en $F(G)$ por conjugación, se tiene
	un morfismo $\gamma\colon G\to\Aut(F(G))$ tal que
	$\ker\gamma=C_G(F(G))=F(G)$ (pues $\gamma_g(x)=gxg^{-1}$). En particular,
	$G/F(G)$ es abeliano por ser isomorfo a un subgrupo del grup abeliano
	$\Aut(F(G))$. Como además los subgrupos de Sylow de $G/F(G)$ son cíclicos (pues
	son cocientes de los subgrupos de Sylow de $G$), $G/F(G)$ es cíclico.
\end{proof}

