\chapter{El subgrupo de Frattini}

\begin{definition}
	\index{Frattini!subgrupo de}
	\index{Subgrupo!de Frattini}
	Sea $G$ un grupo. Si $G$ posee grupos maximales, se define el
	\textbf{subgrupo de Frattini} $\Phi(G)$ como la intersección de los
	subgrupos maximales de $G$. En caso contrario, se define $\Phi(G)=G$.
\end{definition}

\begin{exercise}
	Demuestre que $\Phi(G)$ es un subgrupo característico de $G$.
\end{exercise}

\begin{svgraybox}
	Si $M$ es un subgrupo maximal de $G$ y $f\in\Aut(G)$ entonces $f(M)$ es
	también un subgrupo maximal de $G$. La colección de subgrupos maximales de
	$G$ es invariante por automorfismos de $G$ y luego $f(\Phi(G))\subseteq
	\Phi(G)$.
\end{svgraybox}

\begin{example}
	Sea $G=\Sym_3$.  Los subgrupos maximales de $G$ son 
	\[
	M_1=\langle (123)\rangle,
	\quad
	M_2=\langle (12)\rangle,
	\quad
	M_3=\langle (23)\rangle,
	\quad
	M_4=\langle (13)\rangle.
	\]
	Luego
	$\Phi(G)=1$. 
\end{example}

\begin{example}
	Sea $G=\langle g\rangle\simeq C_{12}$. Como $G$ es cíclico, los subgrupos de $G$ son 
	\[
	1,\quad
	\langle g^6\rangle\simeq C_2,\quad
	\langle g^4\rangle\simeq C_3,\quad
	\langle g^3\rangle\simeq C_4,\quad
	\langle g^2\rangle\simeq C_6,\quad
	G.
	\]
	Como los únicos subgrupos maximales son $\langle g^3\rangle\simeq C_4$ y $\langle
	g^2\rangle\simeq C_6$, obtenemos que $\Phi(G)=\langle g^3\rangle\cap \langle
	g^2\rangle=\langle g^6\rangle\simeq C_2$.
\end{example}

\begin{example}
	Sea $G=\SL_2(3)$. El siguiente código muestra que $\Phi(G)\simeq C_2$: 
\begin{lstlisting}
gap> StructureDescription(FrattiniSubgroup(SL(2,3)));
"C2"
\end{lstlisting}
\end{example}

\begin{lemma}[Dedekind]
	\label{lemma:Dedekind}
	\index{Lema!de Dedekind}
	\index{Dedekind!lema de}
	Sean $H,K,L$ subgrupos de $G$ tales que $H\subseteq L\subseteq G$. Entonces
	$HK\cap L=H(K\cap L)$.
\end{lemma}

\begin{proof}
	Demostraremos que $HK\cap L\subseteq H(K\cap L)$ pues la otra inclusión es
	trivial. Si $x=hk\in HK\cap L$, donde $x\in L$, $h\in H$, $k\in K$,
	entonces $k=h^{-1}x\in L\cap K$ pues $H\subseteq L$. Luego $x=hk\in H(L\cap
	K)$.
\end{proof}

\begin{lemma}
	\label{lemma:G=HPhi(G)}
	Sea $G$ un grupo finito. Si $H$ es un subgrupo de $G$ tal que $G=H\Phi(G)$
	entonces $H=G$.
\end{lemma}

\begin{proof}
	Supongamos que $H\ne G$ y sea $M$ un subgrupo maximal de $G$ tal que
	$H\subseteq M$. Como $\Phi(G)\subseteq M$, $G=H\Phi(G)\subseteq M$, una
	contradicción.
\end{proof}

\begin{proposition}
	\label{proposition:phi(N)phi(G)}
	Sea $N$ un subgrupo normal de un grupo finito $G$. Entonces
	$\Phi(N)\subseteq\Phi(G)$.
\end{proposition}

\begin{proof}
	Como $\Phi(N)$ es característico en $N$ y $N$ es normal en $G$, $\Phi(N)$
	es normal en $G$.  Sea $M$ un subgrupo maximal de $G$ tal que
	$\Phi(N)\not\subseteq M$.  La maximalidad de $M$ implica que $\Phi(N)M=G$
	pues de lo contrario $M=\Phi(N)M\supseteq\Phi(N)$.  Por el lema de
	Dedekind~\ref{lemma:Dedekind} (con $H=\Phi(N)$, $K=M$ y $L=N$), 
	\[
		N=G\cap N=(\Phi(N)M)\cap N=\Phi(N)(M\cap N).
	\]
	Por el lema~\ref{lemma:G=HPhi(G)} esto implica que $\Phi(N)\subseteq
	N\subseteq M$, una contradicción. Luego todo subgrupo maximal de $G$
	contiene a $\Phi(N)$ y por lo tanto $\Phi(G)\supseteq\Phi(N)$. 
\end{proof}

\begin{lemma}
	\label{lemma:nongenerators}
	Sea $G$ un grupo finito. Entonces 
	\[
	\Phi(G)=\{x\in G:\text{si $G=\langle x,Y\rangle$ para algún $Y\subseteq G$ entonces $G=\langle Y\rangle$}\}.
	\]
%	$\Phi(G)$ es el conjunto de $x\in G$ tales que 
%	si $G=\langle x,Y\rangle$ para algún subconjunto $Y$ de $G$ entonces
%	$G=\langle Y\rangle$. 
\end{lemma}

\begin{proof}
	Veamos primero la inclusión $\supseteq$.  Sea $x\in G$ y sea $M$ un
	subgrupo maximal de $G$.  Si $x\not\in M$ entonces, como $G=\langle
	x,M\rangle$, se tiene $G=\langle M\rangle=M$, absurdo. Luego $x\in M$ para
	todo subgrupo maximal $M$ y entonces $x\in \Phi(G)$. 

	Veamos ahora la inclusión $\subseteq$. Sea $x\in\Phi(G)$ tal que $G=\langle
	x,Y\rangle$ para algún subconjunto $Y$ de $G$. Si $G\ne \langle Y\rangle$,
	existe un subgrupo maximal $M$ tal que $\langle Y\rangle\subseteq M$. Como
	$x\in M$, $G=\langle x,Y\rangle\subseteq M$, una contradicción.
\end{proof}

\begin{example}
	Sea $p$ un número primo. Sea $G$ un $p$-grupo elemental abeliano, es decir
	$G\simeq C_p^m$ para algún $m\in\N$.  Supongamos además que
	$G=\langle x_1\rangle\times\cdots\times\langle x_m\rangle$ con $\langle x_j\rangle\simeq C_p$.  
	Veamos que $\Phi(G)$ es trivial. 
	Sea $j\in\{1,\dots,m\}$ y sea $n_j\in\{1,\dots,p-1\}$. Como el conjunto
	\[
	\{x_1,\dots,x_{j-1},x_j^{n_j},x_{j+1},\dots,x_m\}
	\]
	genera al grupo $G$ y $\{x_1,\dots,x_{j-1},x_{j+1},\dots,x_m\}$ no lo hace,
	entonces $x_j^{n_j}\not\in\Phi(G)$ por el lema~\ref{lemma:nongenerators}.
	Luego $\Phi(G)=1$.
\end{example}

\begin{theorem}[Frattini]
	\label{theorem:Frattini}
	\index{Teorema!de Frattini}
	\index{Frattini!teorema de}
	Sea $G$ un grupo finito. Entonces $\Phi(G)$ es nilpotente.
\end{theorem}

\begin{proof}
	Sea $P\in\Syl_p(\Phi(G))$ para algún primo $p$. Como $\Phi(G)$ es normal en
	$G$, gracias al argumento de Frattini, lema~\ref{lemma:Frattini_argument},
	podemos escribir $G=\Phi(G)N_G(P)$. Por el lema~\ref{lemma:G=HPhi(G)},
	$G=N_G(P)$. Como todo subgrupo de Sylow de $\Phi(G)$ es normal en $G$,
	$\Phi(G)$ es nilpotente.
\end{proof}

\begin{exercise}
	\label{exercise:G/M}
	Sea $G$ un grupo y sea $M$ un subgrupo normal de $G$ maximal. Demuestre que
	$G/M$ es cíclico de orden primo. 
\end{exercise}

\begin{svgraybox}
	Por el teorema de la correspondencia, $G/M$ no tiene subgrupos no trivales.
	Luego $G/M\simeq C_p$ para algún primo $p$.
\end{svgraybox}

\begin{theorem}[Gasch\"utz]
	\label{theorem:Gaschutz}
	\index{Teorema!de Gasch\"utz}
	\index{Gasch\"utz!teorema de}
	Si $G$ es un grupo finito entonces 
	\[
	[G,G]\cap Z(G)\subseteq\Phi(G).
	\]
\end{theorem}

\begin{proof}
	Sea $D=[G,G]\cap Z(G)$. Supongamos que $D$ no está contenido en $\Phi(G)$.
	Como $\Phi(G)$ está contenido en todo subgrupo maximal de $G$, existe un
	subgrupo maximal $M$ de $G$ tal que $D$ no está contenido en $M$.  Esto
	implica que $G=MD$. Como $D\subseteq Z(G)$, $M$ es normal en $G$ pues si
	$g=md\in G=MD$ entonces
	\[
		gMg^{-1}=(md)Md^{-1}m^{-1}=mMm^{-1}=M.
	\]
	El ejercicio~\ref{exercise:G/M} implica que $G/M$ es cíclico de orden
	primo. Como en particular $G/M$ es abeliano, $[G,G]\subseteq M$. Luego
	$D\subseteq [G,G]\subseteq M$, una contradicción.
\end{proof}

\begin{lemma}
	\label{lemma:N_G(H)=H}
	Sea $G$ un grupo finito y sea $P\in\Syl_p(G)$. Sea $H$ un subgrupo de $G$
	tal que $N_G(P)\subseteq H$. Entonces $N_G(H)=H$.
\end{lemma}

\begin{proof}
	Sea $x\in N_G(H)$. Como $P\in\Syl_p(H)$ y $Q=xPx^{-1}\in\Syl_p(H)$, existe
	$h\in H$ tal que $hQh^{-1}=(hx)P(hx)^{-1}=P$. Entonces $hx\in
	N_G(P)\subseteq H$ y luego $x\in H$. 
\end{proof}

\begin{theorem}[Wielandt]
	\label{theorem:Wielandt}
	\index{Teorema!de Wielandt}
	\index{Wielandt!teorema de}
	Sea $G$ un grupo finito. Entonces $G$ es nilpotente si y sólo si
	$[G,G]\subseteq\Phi(G)$.
\end{theorem}

\begin{proof}
	Supngamos que $[G,G]\subseteq\Phi(G)$. Sea $P\in\Syl_p(G)$. Si $N_G(P)\ne
	G$ entonces $N_G(P)\subseteq M$ para algún subgrupo maximal $M$ de $G$. Si
	$g\in G$ y $m\in M$ entonces, como 
	\[
		gmg^{-1}m^{-1}=[g,m]\in [G,G]\subseteq\Phi(G)\subseteq M,
	\]
	$M$ es normal en $G$. Como además $N_G(P)\subseteq M$, el
	lema~\ref{lemma:N_G(H)=H} implica que 
	\[
	G=N_G(M)=M,
	\]
	una contradicción.
	Luego $N_G(P)=G$. Todo subgrupo de Sylow de $G$ es normal en $G$ y entonces
	$G$ es nilpotente.

	Supongamos ahora que $G$ es nilpotente. Sea $M$ un subgrupo maximal de $G$.
	Como $M$ es normal en $G$ y maximal, $G/M$ no tiene subgrupos propios.
	Luego $G/M\simeq C_p$ para algún primo $p$. En particular $G/M$ es abeliano
	y luego $[G,G]\subseteq M$. Como $[G,G]$ está contenido en
	todo subgrupo maximal de $G$, $[G,G]\subseteq\Phi(G)$.
\end{proof}

\begin{theorem}
	\label{theorem:G/phi(G)}
	Sea $G$ un grupo finito. Entonces $G$ es nilpotente si y sólo si
	$G/\Phi(G)$ es nilpotente.
\end{theorem}
%%% TODO: la demostración no está bien explicada!
\begin{proof}
	Si $G$ es nilpotente, entonces $G/\Phi(G)$ es nilpotente por el
	teorema~\ref{theorem:nilpotente}. Supongamos que $G/\Phi(G)$ es
	nilpotente. Sea $P\in\Syl_p(G)$. Como
	$\Phi(G)P/\Phi(G)\in\Syl_p(G/\Phi(G))$ y $G/\Phi(G)$ es nilpotente,
	$\Phi(G)P/\Phi(G)$ es un subgrupo normal de $G/\Phi(G)$. Luego, por el
	teorema de la correspondencia, $\Phi(G)P$ es un subgrupo normal de $G$.
	Como $P\in\Syl_p(\Phi(G)P)$, el argumento de Frattini del
	lemma~\ref{lemma:Frattini_argument} implica que
	\[
		G=\Phi(G)PN_G(P)=\Phi(G)N_G(P)
	\]
	pues $P\subseteq N_G(P)$. Luego $G=N_G(P)$ por el
	lema~\ref{lemma:nongenerators} y entonces $P$ es normal en $G$. Esto
	implica que $G$ es nilpotente.
\end{proof}

\begin{theorem}[Hall]
	\index{Teorema!de Hall}
	\index{Hall!teorema de}
	\label{theorem:Hall_nilpotente}
	Sea $G$ un grupo finito y sea $N$ un subgrupo normal de $G$. Si $N$ y
	$G/[N,N]$ son nilpotentes, entonces $G$ es nilpotente.
\end{theorem}

\begin{proof}
	Como $N$ es nilpotente, $[N,N]\subseteq\Phi(N)$ por el
	teorema~\ref{theorem:Wielandt}.	
	Por la proposición~\ref{proposition:phi(N)phi(G)},
	$[N,N]\subseteq\Phi(N)\subseteq\Phi(G)$. 
	Por propiedad universal, existe un morfismo
	$G/[N,N]\to G/\Phi(G)$ sobreyectivo que hace conmutar el diagrama
    \[
	\xymatrix{
	G
	\ar[d]
	\ar[r]
	& G/\Phi(G)
	\\
	G/[N,N]\ar@{-->}[ur]
	}
%    \xymatrix{ & P\ar[d]^f\ar@{-->}[ld]_h\\ M\ar[r]^g & N\ar[r] & 0 }
    \]
	Como por hipótesis $G/[N,N]$ es nilpotente, $G/\Phi(G)$ es nilpotente por
	el teorema~\ref{theorem:nilpotente}. Luego $G$ es nilpotente por el
	teorema~\ref{theorem:G/phi(G)}.
\end{proof}

\begin{definition}
	Un \textbf{conjunto minimal de generadores} de un grupo $G$ es un conjunto
	$X$ de generadores de $G$ tal que ningún subconjunto propio de $X$ genera a
	$G$.
\end{definition}

\begin{remark}
	Es importante observar que un conjunto minimal de generadores puede no
	tener cardinal mínimo. Sea $G=\langle g\rangle\simeq C_6$.  Si $a=g^2$ y
	$b=g^3$ entonces $\{a,b\}$ es un conjunto minimal de generadores de $G$,
	aunque no tiene cardinal mínimo pues por ejemplo $G=\langle ab\rangle$.
\end{remark}

\begin{lemma}
	\label{lemma:Burnside:minimal}
	Sea $p$ un número primo y sea 
	$G$ un $p$-grupo finito. Entonces $G/\Phi(G)$ es un espacio vectorial
	sobre $\F_p$.
\end{lemma}

\begin{proof}
	Sea $K$ un subgrupo maximal de $G$. Como $G$ es nilpotente por la
	proposición~\ref{proposition:pgrupo_nilpotente}, $K$ es normal en $G$
	(ejercicio~\ref{exercise:normalizadora}). Luego $G/K\simeq C_p$ por ser un $p$-grupo
	simple. 
	
	Basta ver que $G/\Phi(G)$ es $p$-grupo elemental abeliano. En un $p$-grupo
	pues $G$ es un $p$-grupo.  Sean $K_1,\dots,K_m$ son los subgrupos maximales
	de $G$. Si $x\in G$ entonces $x^p\in K_j$ para todo $j\in\{1,\dots,m\}$ y
	luego $x^p\in\Phi(G)=\cap_{j=1}^m K_j$. Además $G/\Phi(G)$ es abeliano pues
	$[G,G]\subseteq \Phi(G)$ por ser $G$ nilpotente
	(teorema~\ref{theorem:Wielandt}). 
\end{proof}

\begin{theorem}[Burnside]
	\label{theorem:Burnside:basis}
	Sea $p$ un número primo y sea $G$ un $p$-grupo finito. Si $X$ es un
	conjunto minimal de generadores entonces $|X|=\dim G/\Phi(G)$. 
\end{theorem}

%%% TODO: explicar mejor la demostración

\begin{proof}
	Vimos en el lema~\ref{lemma:Burnside:minimal} que $G/\Phi(G)$ es un espacio
	vectorial sobre $\F_p$. Sea $\pi\colon G\to G/\Phi(G)$ el morfismo canónico
	y sea $\{x_1,\dots,x_n\}$ un conjunto minimal de generadores de $G$.
	Veamos que $\{\pi(x_1),\dots,\pi(x_n)\}$ es un conjunto linealmente
	independiente de $G/\Phi(G)$.  Supongamos sin perder generalidad que
	$\pi(x_1)\in\langle \pi(x_2),\dots,\pi(x_n)\rangle$. Existe entonces $y\in
	\langle x_2,\dots,x_n\rangle$ tal que $x_1y^{-1}\in\Phi(G)$. Como $G$ está
	generado por $\{x_1y^{-1},x_2,\dots,x_n\}$ y $x_1y^{-1}\in\Phi(G)$, el
	lema~\ref{lemma:nongenerators} implica que $G$ también está generado por
	$\{x_2,\dots,x_n\}$, una contradicción a la minimalidad. Luego $n=\dim
	G/\Phi(G)$.
\end{proof}

% TODO: agregar una aplicación (teorema de Hall). Ver Passman permutation groups, 11.7, pag 47

%\begin{corollary}
%	Sea $p$ un número primo y sea $G$ un $p$-grupo finito.  Todo elemento de
%	$\Phi(G)$ pertenece a algún conjunto minimal de generadores.
%\end{corollary}
