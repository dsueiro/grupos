\chapter{Super resolubilidad}

%\section{Grupos súper-resolubles}
\label{super_resoluble}

\begin{definition}
	\index{Grupo!súper-resoluble}
	Un grupo $G$ se dice \textbf{súper-resoluble} si existe una sucesión de
	subgrupos normales
	\[
		G=G_0\supseteq G_1\supseteq\cdots\supseteq G_n=\{1\}
	\]
	tal que cada cociente $G_{i-1}/G_i$ es cíclico.
\end{definition}

Observemos que en la definición anterior no se pide que $G$ sea un grupo finito. Los cocientes
pueden ser entonces grupos cíclicos de orden $n$ o grupos cíclicos infinitos, es decir grupos 
isomorfos a $\Z$. 

\begin{example}
	El grupo diedral $\D_{n}$ de orden $2n$ es súper-resoluble pues 
    \[	
    \D_{n}\supseteq \langle
	r\rangle\supseteq \{1\}
	\]
	es una sucesión de subgrupos normales con factores
	cíclicos.
\end{example}

Todo grupo súper-resoluble es resoluble, ver ejercicio~\ref{exercise:resoluble}.

\begin{example}
	El grupo $\Alt_4$ es resoluble pero no es súper-resoluble. El único 
	subgrupo propio no trivial normal de $\Alt_4$ es
	\[
	\{\id,(12)(34),(13)(24),(14)(23)\}\simeq C_2\times C_2.
	\]
	Luego $\Alt_4$ no posee una sucesión de subgrupos normales con factores
	cíclicos.	
\end{example}

\begin{exercise}
    Demuestre que el grupo $\Aff(\Z)$ es super resoluble. 
\end{exercise}
% aff(Z) es súper-resoluble

\begin{example}
	El grupo $\SL_2(3)$ es resoluble pero no es súper-resoluble:
\begin{lstlisting}
gap> IsSolvable(SL(2,3));
true
gap> IsSupersolvable(SL(2,3));
false
\end{lstlisting}
\end{example}

\begin{exercise}
	\label{xca:super}
	Demuestre las siguientes afirmaciones:
	\begin{enumerate}
		\item Todo subgrupo de un grupo súper-resoluble es súper-resoluble.
		\item Todo cociente de un grupo súper-resoluble es súper-resoluble.
	\end{enumerate}
\end{exercise}

% \begin{svgraybox}
% 	Sea $G$ un grupo súper-resoluble y sea 			
% 	\[ 
% 	G=G_0\supseteq G_1\supseteq \cdots\supseteq G_n=1 
% 	\] 
% 	una sucesión de subgrupos normales
% 	donde cada cociente $G_{i-1}/G_{i}$ es cíclico. 
% 	\begin{enumerate}
% 		\item Sea $H$ un subgrupo de $G$. Como $G$ es
% 			súper-resoluble, Sea 
% 			\[
% 			H=H\cap G_0\supseteq H\cap G_1\supseteq\cdots\supseteq H\cap G_n=1
% 			\]
% 			una sucesión de subgrupos de $H$. Cada $H\cap G_i$ es normal en $H$
% 			pues $G_i$ es normal en $G$. Fijemos $i\in\{1,\dots,n\}$ y sea
% 			$\pi_{i-1}\colon G_{i-1}\to G_{i-1}/G_{i}$ el morfismo canónico. La
% 			restricción de $\pi_{i-1}$ al subgrupo $H\cap G_{i-1}$ es un morfismo con
% 			núcleo $G_{i}\cap H$.  Al usar el teorema de isomorfismos vemos que 
% 			\[
% 			\frac{H\cap G_{i-1}}{H\cap G_{i}}\simeq \pi_{i-1}(H\cap G_i)\subseteq G_{i-1}/G_i
% 			\]
% 			es un grupo cíclico por ser subgrupo de un grupo cíclico. 
% 		\item Sea $K$ un subgrupo normal de $G$ y sea $\pi\colon G\to G/K$ el
% 			morfismo canónico. Para cada $i$ sea $Q_i=\pi(G_i)$. Cada $Q_i$ es
% 			normal en $Q_n=\pi(G_n)=G/K$ pues $G_i$ es normal en $G$. Como
% 			$G_{i-1}K=G_{i-1}(G_iK)$ para todo $i$, 
% 			el grupo
% 			\begin{align*}
% 			Q_{i-1}/Q_i
% 			&\simeq\frac{G_{i-1}/G_{i-1}\cap K}{G_i/G_i\cap K}
% 			\simeq \frac{G_{i-1}K/K}{G_{i}K/K}\\
% 			&\simeq\frac{ G_{i-1}K}{G_iK}
% 			\simeq\frac{ G_{i-1}(G_iK)}{G_iK}
% 			\simeq\frac{ G_{i-1}}{G_iK\cap G_{i-1}}
% 			\simeq\frac{ G_{i-1}/G_i}{G_iK\cap G_{i-1}/G_i}
% 			\end{align*}
% 			es cíclico por ser un cociente de un grupo cíclico.
% 	\end{enumerate}
% \end{svgraybox}

\begin{exercise}
	\label{exercise:directosuper}
	Demuestre que el producto directo de grupos súper-resolubles es
	súper-resoluble.
\end{exercise}

% \begin{svgraybox}
% 	Supongamos que $G$ admite una sucesión $G=G_0\supseteq G_1\supseteq
% 	\cdots\supseteq G_n=1$ de de subgrupos normales tales que cada cociente
% 	$G_{i-1}/G_i$ es cíclico, y que $H$ admite una sucesión $H=H_0\supseteq
% 	H_1\supseteq \cdots\supseteq H_m=1$ de subgrupos normales donde cada
% 	$H_{i-1}/H_i$ es cíclico. Consideramos la sucesión 
% 	\[
% 		1=G_0\times H_0\supseteq G_1\times H_0\supseteq\cdots\supseteq G_n\times H_0\supseteq G_n\times H_1\supseteq \cdots\supseteq G_n\times H_m=G\times H
% 	\]
% 	tiene factores cíclicos pues 
% 	cada $G_{i-1}\times H_0/G_i\times H_0\simeq G_{i-1}/G_i$ es cíclico y cada 
% 	$G_n\times H_{j-1}/G_n\times H_j$ también pues
% 	\[
% 	G_n\times H_{j-1}/G_n\times H_j
% 	\simeq \frac{GH_{j-1}/G}{GH_j/G}
% 	\simeq \frac{H_{j-1}/H_{j-1}\cap G}{H_j/H_j\cap G}\simeq H_{j-1}/H_j.
% 	\]
% \end{svgraybox}

\begin{exercise}
	Sean $H$ y $K$ subgrupos normales de un grupo $G$ tales que $G/K$ y $G/H$
	son súper-resolubles. Demuestre que $G/H\cap K$ es súper-resoluble.
\end{exercise}

% \begin{svgraybox}
% 	El producto directo $G/H\times G/K$ es súper-resoluble. Sea $\partial\colon
% 	G\to G/H\times G/K$, $g\mapsto (gH,gK)$.  Como $\ker\partial=H\cap K$, se
% 	tiene que $G/H\cap K\simeq\partial(G)$, que es súper-resoluble por ser un
% 	subgrupo de un grupo súper-resoluble.
% \end{svgraybox}

\begin{proposition}
	\label{proposition:Nciclico}
	Sea $N$ un subgrupo normal cíclico de un grupo $G$. Si $G/N$ es
	súper-resoluble entonces $G$ es súper-resoluble.
\end{proposition}

% todo: arreglar 

\begin{proof}
	Sea $\pi\colon G\to G/N$ el morfismo canónico y sea $Q=G/N$. Como $Q$ es
	súper-resoluble, tenemos una sucesión
	\[
		Q=Q_0\supseteq Q_1\supseteq \cdots\supseteq Q_n=\{1\}
	\]
	de subgrupos normales de $Q$ tales que cada cociente $Q_{i-1}/Q_i$ es
	cíclico. Cada elemento de la sucesión
	\[
	G=\pi^{-1}(Q)\supseteq\pi^{-1}(Q_1)\supseteq\cdots\supseteq \pi^{-1}(Q_n)=N\supseteq \{1\}
	\]
	es normal en $G$ (por la correspondencia) y dejamos como 
	ejercicio demostrar que cada cociente es cíclico. 
% 	cada cociente es cíclico $N$ es cíclico. 
% 	Queda como ejercicio demostrar 
% 	y cada 
% 	\[
% 	\frac{\pi^{-1}(Q_j)}{\pi^{-1}(Q_{j+1})}
% 		=\frac{Q_jN}{Q_{j+1}N}
% 		\simeq\frac{Q_jN/N}{Q_{j+1}N/N}
% 		\simeq\frac{Q_j(Q_{j+1}N)}{Q_{j+1}N}
% 		\simeq\frac{Q_j/Q_{j+1}}{Q_{j+1}N\cap Q_j}
% 	\]
% 	es cíclico por ser cociente de un grupo cíclico.
\end{proof}

\begin{theorem}
	\label{theorem:ZorCp}
	Sea $G$ un grupo súper-resoluble no trivial. Entonces $G$ posee una sucesión
	de subgrupos $G=G_0\supseteq G_1\supseteq\cdots\supseteq G_n=\{1\}$ de
	subgrupos normales tales que cada cociente $G_{i-1}/G_i$ es cíclico de
	orden primo o isomorfo a $\Z$.
\end{theorem}

\begin{proof}
	Sea $G=G_0\supseteq G_1\supseteq\cdots\supseteq G_n=\{1\}$ una sucesión de
	subgrupos normales tal que cada cociente $G_{i-1}/G_i$ es cíclico. Sea
	$i\in\{1,\dots,n\}$ tal que el cociente $G_{i-1}/G_i\simeq C_n$ para algún
	$n$ que no es primo y sea $\pi\colon G_{i-1}\to G_{i-1}/G_i$ el morfismo
	canónico.  Sea $p$ un primo que divide a $n$ y sea $H$ un subgrupo de $G$
	tal que $\pi(H)$ es un subgrupo de $G_{i-1}/G_i$ de orden $p$. Por el
	teorema de la correspondencia, $G_{i}\subseteq H\subseteq G_{i-1}$. 
	
	Veamos que $H$ es normal en $G$. Sea $g\in G$. Como $\pi(gHg^{-1})$ es un
	subgrupo de orden $p$ del cíclico $G_{i-1}/G_i$, $\pi(gHg^{-1})=\pi(H)$. Luego
	$gHg^{-1}=G_{i}H\subseteq H$ y en conclusión $gHg^{-1}=H$. 
% 	\[
% 	\frac{gHg^{-1}}{G_i}=\frac{G_{i}H}{G_{i}}\simeq \frac{H}{G_i\cap H}=\frac{H}{G_i}
% 	\]
% 	y entonces $gHg^{-1}\subseteq H$.  

    Observemos que $H/G_i$ es cíclico de orden
	primo pues 
	\[
		H/G_i=H/H\cap G_i\simeq \pi(H)\simeq C_p
	\]
	y que $G_{i-1}/H$ también es cíclico pues 
	\[
	G_{i-1}/H\simeq\frac{G_{i-1}/G_i}{H/G_i}
	\]
	es cociente de un grupo cíclico.
	
	Demostramos que al insertar $H$ en la
	sucesión obtenemos una nueva sucesión con factores cíclicos y donde
	$H/G_{i}$ es cíclico de orden primo. Al repetir este proceso se obtiene el
	resultado deseado.
\end{proof}

Una aplicación inmediata:

\begin{corollary}
	Un grupo finito súper-resoluble admite una sucesión de subgrupos 
	$G=G_0\supseteq G_1\supseteq\cdots\supseteq G_n=\{1\}$ 
	normales donde cada cociente $G_{i-1}/G_i$ es cíclico de orden primo.
\end{corollary}

% \begin{proof}
% 	Es consecuencia inmediata del teorema~\ref{theorem:ZorCp}.
% \end{proof}

Veamos otras propiedades importantes de grupos súper-resolubles. 

\begin{theorem}
	\label{theorem:super_structure}
	Sea $G$ un grupo súper-resoluble. 
	\begin{enumerate}
		\item Si $N$ es minimal-normal en $G$ entonces $N\simeq C_p$ para algún primo $p$.
		\item Si $M$ es maximal en $G$ entonces $(G:M)=p$ para algún primo $p$.
		\item El conmutador $[G,G]$ es nilpotente.
		\item Si $G$ es no abeliano existe un subgrupo normal $N\ne G$ tal que
			$Z(G)\subsetneq N$.
	\end{enumerate}
\end{theorem}

\begin{proof}
	Demostremos la primera afirmación. 
	%Como $G$ es súper-resoluble, $N$ es
	%resoluble. El conmutador $[N,N]$ es característico en $N$ y luego $[N,N]$
	%es normal en $G$. Por la minimalidad, $N$ entonces un grupo abeliano. 
	Como $G$ es súper-resoluble, existe una sucesión $G=G_0\supseteq G_1\supseteq
	G_2\supseteq\cdots\supseteq G_n=\{1\}$ de subgrupos normales con factores
	$G_{i-1}/G_i$ cíclicos. Como cada $G_i\cap N$ es un subgrupo normal de $G$ contenido en $N$, la
	minimalidad de $N$ implica que 
	cada $G_i\cap N$ es trivial o igual a $N$. Además $N\cap G_0=N$ y $N\cap
	G_n=\{1\}$. Sea $j$ el mínimo entero tal que $N\cap G_j=\{1\}$. Como $N\subseteq
	G_{j-1}$ (pues $N\cap G_{j-1}=N$), la composición
	\[
	N\hookrightarrow G_{j-1}\to G_{j-1}/G_j
	\]
	es un morfismo inyectivo, pues tiene núcleo igual a $N\cap G_{j}=\{1\}$. Luego $N$ es
	cíclico por ser isomorfo a un subgrupo del cíclico $G_{i-1}/G_i$. Si
	$G_{i-1}/G_i\simeq\Z$ entonces $N\simeq\Z$ pero no sería minimal-normal ya
	que por ejemplo $2\Z$ es un subgrupo característico de $\Z$ y por lo tanto
	es normal en $G$. Luego $N$ es cíclico y finito y entonces $N\simeq C_p$. 

	Demostremos la segunda afirmación. Sea $M$ un subgrupo maximal de $G$. 
	Si $M$ es normal en $G$ entonces $G/M$ no contiene subgrupos propios no
	triviales. 
	%Como $G/M$ es súper-resoluble, la sucesión $1\supseteq
	%G/M$ tiene factores isomorfos cíclicos de orden primo (por el
	%teorema~\ref{theorem:ZorCp}) y 
	Luego $G/M\simeq C_p$ para algún
	primo $p$.  Supongamos entonces que $M$ no es normal en $G$. Sea $H=\cap_{g\in
	G}gMg^{-1}$ y sea $\pi\colon G\to G/H$.  Como $\pi(M)$ es maximal en
	$\pi(G)=G/H$ y además
	\[
		(G:M)=(G/H:M/H)=(G/H:M/H\cap M)=(\pi(G):\pi(M)),
	\]
	podemos suponer que $M$ no contiene subgrupos normales no triviales de $G$, ya 
	que en vez de trabajar con $G$ lo hacemos con el cociente $G/H$. 
	Como $G$ es súper-resoluble
	existe una sucesión $G=G_0\supseteq G_1\supseteq\cdots\supseteq G_n=\{1\}$ de
	subgrupos normales de $G$ con factores isomorfos a $\Z$ o cíclicos de orden primo.
	Sea $N=G_{n-1}$. Como $N$ es cíclico, todo subgrupo de $N$ es característico en
	$N$ y por lo tanto es normal en $G$. En particular, $M\cap N$ es normal en
	$G$ y luego $M\cap N=\{1\}$. Como $M\subseteq
	NM\subseteq G$, entonces, por la maximalidad de $M$, $M=NM$ o bien $G=NM$.
	Pero como $N\subseteq NM=M$ es una contradicción, se concluye que $G=NM$.
	Si $N\simeq C_p$ para algún primo $p$, entonces $(G:M)=p$ y la afirmación
	queda demostrada.  Si $N\simeq\Z$ sea $H$ un subgrupo propio de $N$. Como
	$H$ es característico en $N$, $H$ es normal en $G$ y luego, como
	$M\subseteq HM\subseteq NM=G$, la maximalidad de $M$ implica que $HM=M$ o
	bien $HM=G$. Como el caso $HM=M$ implica que $H\subseteq M\cap N=\{1\}$,
	podemos suponer que $HM=G$. Si $n\in N\setminus H$ entonces $n=hm$ para
	algún $h\in H$, $m\in M$. Luego $h=n$ pues $h^{-1}n\in N\cap M=\{1\}$, una
	contradicción.

	Demostremos ahora la tercera afirmación. Como $G$ es súper-resoluble, existe
	una sucesión
	\[
	G=G_0\supseteq G_1\supseteq\cdots\supseteq G_n=1
	\]
	de subgrupos normales tal que cada $G_i/G_{i+1}$ es cíclico. Para cada
	$i\in\{0,\dots,n\}$ sea $H_i=[G,G]\cap G_i$. Como $[G,G]$ y los $G_i$ son
	normales en $G$, se tiene una sucesión
	\[
	[G,G]=H_0\supseteq H_1\supseteq\cdots\supseteq H_n=1
	\]
	de subgrupos normales de $G$. Como $H_i$ y $H_{i+1}$ es normal en $G$, el
	grupo $G$ actúa por conjugación en $H_i/H_{i+1}$. Esto induce un morfismo
	$\gamma\colon G\to\Aut(H_i/H_{i+1})$. Como $H_i/H_{i+1}$ es cíclico, 
	$\Aut(H_i/H_{i+1})$ es abeliano y luego $[G,G]\subseteq\ker \gamma$. Luego
	$[G,G]$ actúa trivialmente por conjugación en $H_{i}/H_{i+1}$ y entonces
	\[
	H_i/H_{i+1}\subseteq Z([G,G]/H_{i+1}).
	\]
	%%% TODO: explicar mejor

	Por último demostremos la cuarta afirmación. Como $G$ es no abeliano,
	$Z(G)\ne G$. Sea $\pi\colon G\to G/Z(G)$ el morfismo canónico.  El cociente
	$G/Z(G)$ es súper-resoluble y la sucesión
	\[
	G/Z(G)=\pi(G)\supseteq \pi(G_1)\supseteq\cdots\supseteq \pi(1)=1
	\]
	es una sucesión de subgrupos normales de $G/Z(G)$ con cocientes cíclicos.
	En particular, $1\ne \pi(G_1)$ es propio y normal en $G/Z(G)$.  Por el
	teorema de la correspondencia, $\pi^{-1}(\pi(G_1))\ne G$ es un subgrupo normal
	de $G$ que contiene propiamente a $Z(G)$. 
\end{proof}

\begin{example}
	Si $G$ es un grupo resoluble, no necesariamente $[G,G]$ es un grupo nilpotente. El grupo
	$\Sym_4$ es resoluble pero $[\Sym_4,\Sym_4]=\Alt_4$ no es nilpotente.
\end{example}

\begin{proposition}
	\label{proposition:psuper}
	Sea $p$ un número primo.  Todo $p$-grupo finito es súper-resoluble.
\end{proposition}

\begin{proof}
	Sea $G$ un contraejemplo de orden minimal. Podemos suponer que $|G|=p^n$
	con $n>1$ (pues si $n=1$ el grupo $G$ es trivialmente súper-resoluble).
	Como $G$ es un $p$-grupo, es nilpotente  y existe un subgrupo normal $N$ de
	orden $p$. El cociente $G/N$ tiene orden $p^{n-1}$ entonces es
	súper-resoluble pues $|G/N|<|G|$. Como $N$ es cíclico y $G/N$ es
	súper-resoluble, $G$ es súper-resoluble por la
	proposición~\ref{proposition:Nciclico}.
\end{proof}

\begin{corollary}
	Todo grupo finito nilpotente es súper-resoluble.
\end{corollary}

\begin{proof}
	Todo grupo finito nilpotente es producto directo (finito) de subgrupos de
	Sylow. Como cada $p$-grupo es súper-resoluble por la
	proposición~\ref{proposition:psuper}, el resultado se obtiene
	inmediatamente del ejercicio~\ref{exercise:directosuper}.
\end{proof}

\begin{theorem}
	Todo grupo súper-resoluble tiene subgrupos maximales.	
\end{theorem}

\begin{proof}
	Procederemos por inducción en la longitud de la sucesión de
	superresolubilidad. Si la longitud es uno, el teorema es cierto pues en
	este caso el grupo es cíclico. Supongamos entonces que $G$ admite una
	sucesión
	\[
		G=G_0\supseteq\cdots\supseteq G_k=1
	\]
	y que la afirmación es cierta para grupos súper-resolubles con sucesiones 
	de longitud $<k$. Como $G_{k-1}$ es normal en $G$, sea $\pi\colon G\to
	G/G_{k-1}$ el morfismo canónico. Entonces la sucesión
	\[
		G/G_{k-1}=\pi(G)\supseteq \pi(G_1)\supseteq\cdots\supseteq\pi(G_{k-1})=1
	\]
	prueba la resolubilidad de $\pi(G)$ y tiene longitud $<k$. Por hipótesis
	inductiva, $G/G_{k-1}$ admite subgrupos maximales y luego, por el teorema
	de la correspondencia, $G$ también admite subgrupos maximales.
\end{proof}

Los grupos resolubles o nilpotentes no siempre admiten
subgrupos maximales, ver por ejemplo $\Q$.

\begin{definition}
	\index{Grupo!que satisface la condición maximal para subgrupos}
	Se dice que un grupo $G$ satisface la \textbf{condición maximal para
	subgrupos} si 
	todo subconjunto $\mathcal{S}$ no vacío de subgrupos tiene un subgrupo
	maximal (es decir, no contenido en ningún otro subgrupo de $\mathcal{S}$). 
	%toda sucesión creciente
	%$S_1\subseteq S_2\subseteq S_3\subseteq\cdots$
	%de subgrupos es finita. 
	%%si todo subconjunto $\mathcal{S}$ 
	%%no vacío de subgrupos tiene un elemento maximal, es decir: existe
	%$M\in\mathcal{S}$ tal que $S\subseteq M$ para todo $S\in\mathcal{S}$.
\end{definition}

%\begin{lemma}
%	Un grupo $G$ satisface la la condición maximal para subgrupos si y sólo si
%	todo subconjunto $\mathcal{S}$ no vacío de subgrupos tiene un subgrupo
%	maximal (es decir, no contenido en ningún otro subgrupo de $\mathcal{S}$). 
%\end{lemma}

\begin{lemma}
	\label{lemma:MAX=fg}
	Sea $G$ un grupo. Entonces $G$ satisface la condición maximal para
	subgrupos si y sólo si todo subgrupo de $G$ es finitamente generado.
\end{lemma}

\begin{proof}
	Supongamos que $G$ satisface la condición maximal para subgrupos y sea $H$
	un subgrupo de $G$.  Sea $\mathcal{S}$ el conjunto de subgrupos de $H$
	finitamente generados. Como $\mathcal{S}$ es no vacío (pues
	$1\in\mathcal{S}$), existe un elemento maximal $M\in\mathcal{S}$.  Sea
	$x\in H$. Como $\langle M,x\rangle\in\mathcal{S}$, $M=\langle M,x\rangle$ y
	luego $x\in M$. Como entonces $H=M$, $H$ es finitamente generado.
	%Supongamos que $G$ no es finitamente generado y satisface la condición maximal para subgrupos. Sea $1\ne g\in G$
	%y sea $S_1=\langle g_1\rangle$. Como $S_1\ne G$, existe $g_2\in G\setminus S_1$, y entonces 
	%$S_1\subseteq S_2=\langle x_1,x_2\rangle$. 

	Supongamos ahora que todo subgrupo de $G$ es finitamente generado. Si
	$\mathcal{S}$ es un subconjunto no vacío de subgrupos de $G$ sin elemento
	maximal, podemos construir una sucesión de subgrupos $S_1\subseteq
	S_2\subseteq\cdots$ que no se estabiliza (acá necesitamos utilizar el
	axioma de elección). Como la unión 
	\[
		S=\bigcup_{j\geq1}S_j 
	\]
	es un subgrupo de $G$, es finitamente generado y luego $S\subseteq S_k$
	para algún $k$ suficientemente grande, una contradicción.
\end{proof}

\begin{proposition}
	\label{proposition:max:N}
	Sea $G$ un grupo y sea $H$ un subgrupo de $G$.  Si $G$ satisface la
	condición maximal para subgrupos entonces $H$ también. 
\end{proposition}

\begin{proof}
	Es consecuencia inmediata del lemma~\ref{lemma:MAX=fg}.
\end{proof}

\begin{proposition}
	\label{proposition:max:G/N}
	Sea $G$ un grupo y sea $N$ un subgrupo normal de $G$.  Si $G/N$ y $N$
	satisfacen la condición maximal para subgrupos entonces $G$ también. 
\end{proposition}

\begin{proof} 
	Sea $\pi\colon G\to G/N$ el morfismo canónico.  Sea $\mathcal{S}$ un
	subconjunto no vacío de subgrupos de $G$. El conjunto $\{S\cap
	N:S\in\mathcal{S}\}$ tiene un elemento maximal $A$ y el conjunto
	$\{\pi(S):S\in\mathcal{S},S\cap N=A\}$ tiene un elemento maximal $B$. Sea
	$S\in\mathcal{S}$ tal que $\pi(S)=B$ y $S\cap N=A$. Si $S$ no es maximal en
	$\mathcal{S}$, existe $T\in\mathcal{S}$ tal que $S\subseteq T$, $N\cap T=A$
	y $\pi(T)=B$. Sea $x\in T\setminus S$. Como $\pi(xN)=\pi(x)\in\pi(T)=B$,
	existe $y\in S$ tal que $xN=yN$. Luego $y^{-1}x\in N\cap T=A=N\cap S$, una
	contradicción pues $x\not\in S$. 
\end{proof}

% TODO: agregar teorema de Huppert (ver por ejemplo Robinson, p. 268)
% corolario: G super si y sólo G/\Phi(G) super
% teorema de Iwasawa, Hall 342-345, 19.3
% teorema de Zappa-Ore, Duke 5 (1939), 431-460, Duke 6 (1940), 511-512

%\begin{definition}
%	\index{Grupo!que satisface la condición minimal para subgrupos}
%	Se dice que un grupo $G$ satisface la \textbf{condición minimal para
%	subgrupos} si todo subconjunto no vacío de subgrupos tiene un elemento
%	minimal.
%\end{definition}
%
%\begin{example}
%	El grupo $\Z$ no satisface la condición minimal para subgrupos pues
%	el conjunto $\{2^n\Z:n\in\N\}$ no posee elemento minimal. 
%\end{example}
%
%\begin{proposition}
%	Sea $G$ un grupo que satisface la condición minimal sobre subgrupos.
%	Entonces todo elemento de $G$ tiene orden finito.
%\end{proposition}
%
%\begin{proof}
%	Si existe $x\in G$ de orden infinito, la sucesión $\mathcal{S}$ de subgrupos 
%	\[
%	\langle x\rangle\supsetneq\langle x^2\rangle\supsetneq\langle
%	x^4\rangle\supsetneq\cdots\supsetneq\langle x^{2^k}\rangle\supsetneq\cdots
%	\]
%	tiene infinitos elementos y luego no posee un elemento minimal. 
%\end{proof}
%
%\begin{exercise}
%	\label{exercise:min:N}
%	Sea $G$ un grupo y sea $H$ un subgrupo de $G$.  Si $G$ satisface la
%	condición minimal para subgrupos entonces $H$ también. 
%\end{exercise}
%
%\begin{svgraybox}
%	Si $\mathcal{S}$ es un subconjunto no vacío de subgrupos de $H$, entonces
%	$\mathcal{S}$ posee un elemento minimal por ser un subconjunto no vacío de
%	subgrupos de $G$.
%\end{svgraybox}
%
%\begin{proposition}
%	\label{proposition:min:G/N}
%	Sea $G$ un grupo y sea $N$ un subgrupo normal de $G$.  Si $G/N$ y $N$
%	satisfacen la condición minimal para subgrupos entonces $G$ también. 
%\end{proposition}
%
%\begin{proof}
%	
%\end{proof}

\begin{proposition}
	\label{proposition:superfg}
	Todo grupo súper-resoluble satisface la condición maximal para subgrupos. En
	particular, todo grupo súper-resoluble es finitamente generado.
\end{proposition}

\begin{proof}
	Procederemos por inducción en la longitud $n$ de la sucesión de
	superresolubilidad.  El caso $n=1$ es trivial pues entonces $G$ es cíclico.
	Supongamos entonces que el resultado vale para grupos súper-resolubles con
	serie de longitud $\leq n-1$.  Sea $G$ un grupo súper-resoluble no trivial y sea 
	\[
	G=G_0\supsetneq
	G_1\supsetneq\cdots\supsetneq G_n=1
	\]
	una sucesión de subgrupos normales de $G$ con factores cíclicos. Como
	$G_{n-1}$ es súper-resoluble por el ejercicio~\ref{xca:super},
	$G_{n-1}$ satisface la condición maximal para subgrupos por hipótesis
	inductiva.  Luego, por la proposición~\ref{proposition:max:G/N}, $G$ satisface la condición maximal para subgrupos porque
	$G/G_{n-1}$ es un grupo cíclico.
\end{proof}

%\begin{proposition}\
%	\begin{enumerate}
%		\item Si un grupo súper-resoluble admite una serie de composición,
%			entonces es finito. 
%		\item Si un grupo súper-resoluble satisface la condición de minimal en
%			subgrupos entonces es finito.
%	\end{enumerate}
%\end{proposition}
%
%\begin{proof}
%	%Para probar la segunda afirmación obsevemos que todo cociente de $G$ es súper-resoluble 
%	%y que por el teorema~\ref{theorem:ZorCp} todo factor de la serie debe ser finito pues
%	%$\Z$ no satisface la condición minimal para subgrupos.
%\end{proof}

\begin{example}
	El grupo abeliano $\Q$ es nilpotente pero no es súper-resoluble
	porque no es finitamente generado.
\end{example}

%\begin{example}
%	El grupo $\Sym_3$ es súper-resoluble pero no es nilpotente. 
%\end{example}

Si $G$ es un grupo y $x_1,\dots,x_{n+1}\in G$ se define 
\[
[x_1,\dots,x_{n+1}]=\left[ [x_1,\dots,x_n],x_{n+1} \right],\quad
n\geq1.
\]

\begin{lemma}
	\label{lemma:G_n}
	Sea $G$ un grupo finitamente generado, digamos $G=\langle X\rangle$ con $X$
	finito. Para cada $n\geq2$ se define
	\[
		G_n=\langle g[x_1,\dots,x_n]g^{-1}:x_1,\dots,x_n\in X,\,g\in G\rangle.
	\]
	Entonces $G_n=\gamma_n(G)$ para todo $n\geq2$. 
\end{lemma}

\begin{proof}
	Observemos que cada $G_n$ es normal en $G$.  Procederemos por inducción en
	$n$. El caso $n=2$ es trivial. Supongamos entonces que
	$\gamma_{n-1}(G)=G_{n-1}$. Sean $x_1,\dots,x_n\in X$. Como
	$[x_1,\dots,x_n]\in\gamma_{n}(G)$, $G_{n-1}\subseteq\gamma_n(G)$. Sea
	$N=G_n$ y sea $\pi\colon G\to G/N$ el morfismo canónico. El grupo $G/N$ es
	finitamente generado. Como
	\[
	[\pi([x_1,\dots,x_{n-1}]),\pi(x_n)]=\pi([x_1,\dots,x_n])=1,
	\]
	se tiene que $\pi([x_1,\dots,x_{n-1}])\in Z(G/N)$. Luego
	$\pi(g[x_1,\dots,x_n]g^{-1})=1$ para todo $g\in G$ y, por hipótesis
	inductiva, 
	se concluye que 
	\[
	\pi(\gamma_{n-1}(G))=\pi(G_{n-1})\subseteq Z(G/N).
	\]
	Como entonces 
	\[
	\pi(\gamma_{n}(G))=\pi([\gamma_{n-1}(G),G])=[\pi(\gamma_{n-1}(G)),\pi(G)]=1,
	\]
	se concluye que $\gamma_n(G)\subseteq N=G_n$.
\end{proof}

\begin{lemma}
	\label{lemma:gamma_n/gamma_n+1}
	Sea $G$ un grupo finitamente generado.  Entonces
	$\gamma_n(G)/\gamma_{n+1}(G)$ es finitamente generado. 
\end{lemma}

\begin{proof}
	Supongamos que $G=\langle X\rangle$ con $X$ finito. 
	Al escribir 
	\[
	g[x_1,\dots,x_n]g^{-1}=[g,[x_1,\dots,x_n]][x_1,\dots,x_n]
	\]
	y usar el lema~\ref{lemma:G_n} para obtener 
	que $[g,[x_1,\dots,x_n]]\in \gamma_{n+1}(G)=G_{n+1}$, 
	\[
	g[x_1,\dots,x_n]g^{-1}\equiv [x_1,\dots,x_n]\bmod \gamma_{n+1}(G). 
	\]
	Luego $\gamma_{n}(G)/\gamma_{n+1}(G)$ está generado por 
	el conjunto finito 
	\[
	\{[x_1,\dots,x_n]\gamma_{n+1}(G):x_1,\dots,x_n\in X\}. 
	\]
\end{proof}

\begin{theorem}
	\label{theorem:super=fg}
	Sea $G$ un grupo nilpotente. Entonces $G$ es súper-resoluble si y sólo si
	$G$ es finitamente generado.
\end{theorem}

\begin{proof}
	Si $G$ es súper-resoluble, es finitamente generado por la
	proposición~\ref{proposition:superfg}.  Supongamos que $G$ es finitamente
	generado y nilpotente. Como por el lema~\ref{lemma:gamma_n/gamma_n+1} cada
	$\gamma_{n}(G)/\gamma_{n+1}(G)$ es finitamente generado, digamos por
	$y_1,\dots,y_m$. Sea $\pi\colon G\to G/\gamma_{n+1}(G)$ el morfismo
	canónico.  Para cada $j\in\{1,\dots,m\}$ sea 
	\[
	K_j=\langle \gamma_{n+1}(G),y_1,\dots,y_j\rangle.
	\]
	Como
	$[K_j,G]\subseteq [\gamma_n(G),G]=\gamma_{n+1}(G)$, 
	se tiene que $\pi(K_j)$ es central en $\pi(G)$. Luego $\pi(K_j)$ es normal
	en $\pi(G)$ y por lo tanto $K_j$ es normal en $G$. Como cada $K_j/K_{j-1}$
	es cíclico generado por $y_jK_{j-1}$, entre $\gamma_n(G)$ y
	$\gamma_{n+1}(G)$ pudimos construir una sucesión de subgrupos normales de
	$G$ con factores cíclicos. Como $G$ es nilpotente, existe $c$ tal que
	$\gamma_{c+1}(G)=1$ y luego $G$ es súper-resoluble.
\end{proof}

\begin{corollary}
	\label{corollary:nilpotente=>max}
	Todo grupo nilpotente finitamente generado satisface la condición maximal
	en subgrupos.
\end{corollary}

\begin{proof}
	Es consecuencia del teorema~\ref{theorem:super=fg} y la
	proposición~\ref{proposition:superfg}.
\end{proof}

\begin{theorem}
	Sea $G$ un grupo nilpotente y finitamente generado. Entonces $T(G)$ es
	finito.
\end{theorem}

\begin{proof}
	Como $G$ es nilpotente, $G$ satisface la condición maximal para subgrupos
	por el corolario~\ref{corollary:nilpotente=>max} y entonces
	todo subgrupo de $G$ es finitamente generado. Como $T(G)$ es un subgrupo por el teorema~\ref{theorem:T(nilpotent)}, 
	es finitamente generado y de torsión. Luego $T(G)$ es finito por el
	teorema~\ref{theorem:T(G)finito}.
\end{proof}




