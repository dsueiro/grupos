\chapter*{Algunas soluciones}

\begin{sol}{xca:n+3mgeq4k}
     Sabemos que $G$ tiene $k$ clases de isomorfismos de módulos simples y que exactamente 
     $m$ son de dimensión uno. Luego $n=\sum_{i=1}^kn_i^2\geq m+4(m-k)$. 
\end{sol}

\section*{Grupos nilpotentes}

\begin{sol}{xca:nilpotente=>resoluble}
	Por inducción se demuestra que $G^{(i)}\subseteq\gamma_i(G)$ para todo
	$i\geq1$. Luego si existe $c$ tal que $\gamma_{c+1}(G)$ entonces $G$ es
	resoluble pues $G^{(c+1)}=\{1\}$.
\end{sol}

\begin{sol}{xca:gamma}
	Todas las afirmaciones se demuestran fácilmente por inducción. El paso
	inductivo para la primera es el siguiente: si $f\in\Aut(G)$ entonces
	\[
		f(\gamma_{i+1}(G))
		=f([G,\gamma_i(G)])
		=[f(G),f(\gamma_i(G))]\subseteq [G,\gamma_i(G)]
		=\gamma_{i+1}(G).
	\]
	Para la segunda:
	\[
		\gamma_{i+1}(G)=[G,\gamma_i(G)]\subseteq [G,\gamma_{i-1}(G)]=\gamma_{i}(G).
	\]
	Similarmente, el paso inductivo para demostrar la tercera afirmación:
	\[
		f(\gamma_{i+1}(G))=f([G,\gamma_i(G)])=[f(G),f(\gamma_i(G))]=[H,\gamma_i(H)]=\gamma_{i+1}(H).
	\]
\end{sol}

\begin{sol}{xca:HxK_nilpotente}
	Por inducción se demuestra fácilmente que $\gamma_i(H\times
	K)\subseteq\gamma_i(H)\times\gamma_i(K)$ para todo $i\geq1$. 
\end{sol}

\begin{sol}{xca:nilpotente_central}
	Sea $\pi\colon G\to G/K$. Usar $\pi^{-1}$ para levantar una serie central de $G/K$ y obtener una serie central para $G$.
\end{sol}

\begin{sol}{xca:nilpotente_minimalnormal1}
 	Como $M\cap Z(G)$ es normal en $G$, la minimalidad de $M$ implica que hay
 	dos posibilidades: $M\cap Z(G)$ es trivial o bien $M=M\cap Z(G)\subseteq Z(G)$.
 	Por el teorema~\ref{theorem:Z(nilpotent)}, $M\cap Z(G)\ne \{1\}$.
\end{sol}

\begin{sol}{xca:nilpotente_minimalnormal2}
	Sea $x\in N_G(M)$. Como $P\subseteq M$ y $M$ es normal en $N_G(M)$,
	$xPx^{-1}\subseteq M$.  Como $P$ y $xPx^{-1}$ son $p$-subgrupos de Sylow de
	$M$, existe $m\in M$ tal que 
	\[
	mPm^{-1}=xPx^{-1}.
	\]
	Luego $x\in M$ pues
	$m^{-1}x\in N_G(P)\subseteq M$. 
\end{sol}


\begin{sol}{xca:normalizadora}
	Para demostrar que $(1)\implies(2)$ simplemente usamos el
	lema~\ref{lemma:normalizadora}. Para demostrar que $(2)\implies(3)$ hacemos
	lo siguiente: si $M$ es un subgrupo maximal, como $M\subsetneq N_G(M)$ por
	hipótesis, $N_G(M)=G$ por maximalidad. Finalmente demostremos que
	$(3)\implies(1)$.  Sea $P\in\Syl_p(G)$. Si $P$ no es normal en $G$,
	$N_G(P)\ne G$ y entonces existe un subgrupo maximal $M$ tal que
	$N_G(P)\subseteq M$. Como $M$ es normal en $G$, el
	ejercicio~\ref{exercise:truco} implica que $M=N_G(M)=G$, una contradicción.
	Luego $P$ es normal en $G$ y entonces $G$ es nilpotente por el
	teorema~\ref{theorem:nilpotente:eq}.
\end{sol}

% ejercicio: G finito. Es nilpotente si y solo si dos elementos de ordenes coprimos conmnutan
% 5.41 rotman


\begin{sol}{xca:pgrupos}
	\begin{enumerate}
		\item Sabemos que $Z(G)\ne1$. Sea $g\in Z(G)$ tal que $g\ne 1$.
			Supongamos que el orden de $g$ es $p^k$ para algún $k\geq1$.
			Entonces $g^{p^{k-1}}$ tiene orden $p$ y luego genera un subgrupo
			central de orden $p$. 
		\item Procederemos por inducción en $n$. Si $n=1$ el resultado es
			trivial.  Supongamos entonces que el resultado vale para un cierto
			$n\geq2$. Por el punto anterior, $G$ posee un subgrupo normal $N$
			de orden $p$. Luego $G/N$ tiene orden $p^{n-1}$. Sea $\pi\colon G\to G/N$ el morfismo canónico. 
			Por hipótesis
			inductiva, para cada $j\in\{0,\dots,n-1\}$. Por el teorema de la
			correspondecia, cada subgrupo normal $S_j$ de $G/N$ de orden $p^j$ se
			corresponde con un subgrupo $\pi^{-1}(S_j)$ de $G$ de orden $p^{j+1}$ pues, como
			$\pi$ es sobreyectiva, se tiene $\pi(\pi^{-1}(S_j))=S_j$, y luego
			\[
			p^j=|S_j|=|\pi(\pi^{-1}(S_j))|=\frac{|\pi^{-1}(S_j)|}{|\pi^{-1}(S_j)\cap N|}=\frac{|\pi^{-1}(S_j)|}{|N|}=\frac{|\pi^{-1}(S_j)|}{p}.
			\]
	\end{enumerate}
\end{sol}

\begin{sol}{xca:nilpotente_equivalencia}
	Veamos que $(1)\implies(2)$. Sabemos que $G$ es producto directo de sus
	subgrupos de Sylow, digamos $G=\prod_{i=1}^k S_i$, donde los $S_i$ son los
	distintos subgrupos de Sylow de $G$.  Sean
	$x=(x_1,\dots,x_k),y=(y_1,\dots,y_k)\in G$. Como $|x|$ y $|y|$ son
	coprimos, para cada $i\in\{1,\dots,k\}$ se tiene $x_i=1$ o $y_i=1$. Luego
	\[
		[x,y]=([x_1,y_1],[x_2,y_2],\dots,[x_k,y_k])=1. 
	\]
	Demostremos ahora que $(2)\implies(1)$. Supongamos que
	$|G|=p_1^{\alpha_1}\cdots p_k^{\alpha_k}$, donde los $p_j$ son primos
	distintos y para cada $j$ sea $P_j\in\Syl_{p_j}(G)$. Como elementos de
	órdenes coprimos conmutan, la función $P_1\times\cdots\times P_k\to G$,
	$(x_1,\dots,x_k)\mapsto x_1\cdots x_k$, es un morfismo inyectivo de grupos.
	Como entonces $G\simeq P_1\times\cdots P_k$, y cada $P_j$ es nilpotente,
	$G$ es nilpotente. 

	Para demostrar que $(1)\implies(3)$ simplemente hay que observar que todo
	cociente de $G$ es nilpotente y luego utilizar el
	teorema~\ref{theorem:Z(nilpotent)}. Demostremos que $(3)\implies(1)$. Como
	todo cociente no trivial de $G$ tiene centro no trivial, en particular
	$Z_1=Z(G)$ es no trivial. Si $Z_1=G$ entonces $G$ es abeliano y no hay nada
	para demostrar. Si $Z_1\ne G$ entonces $G/Z_1\ne 1$; luego $Z(G/Z_1)\ne 1$.
	Si $\pi_1\colon G\to G/Z_1$ es el morfismo canónico,
	$Z_2=\pi_1^{-1}(Z(G/Z_1))$. Inductivamente, si tenemos construido el
	subgrupo $Z_i$, $Z_i\ne G$ y  $\pi_i\colon G\to G/Z_{i}$ es el morfismo
	canónico, se define el subgrupo $Z_{i+1}=\pi_i^{-1}(Z(G/Z_i))$. Por
	construcción, $Z_i\subseteq Z_{i+1}$ para todo $i$. Como $G$ es finito,
	existe $k$ tal que $Z_k=G$ y luego $G$ es nilpotente.

	Demostremos que $(1)\implies(4)$. Esta implicación es consecuencia
	inmediata del ejercicio~\ref{exercise:pgrupos}. 
	Como $G$ es nilpotente, $G$ producto
	directo de sus $p$-grupos de Sylow. Si $d=p_1^{\alpha_1}\cdots
	p_k^{\alpha_k}$ es un divisor del orden de $G$, basta tomar
	$H=H_1\times\cdots\times H_k$, 
	donde cada $H_j$ es un subgrupo normal del $p_j$-subgrupo de Sylow de $G$
	de orden $p_j^{\alpha_j}$. Para demostrar que $(4)\implies(1)$ simplemente
	se aplica la hipótesis a cada $p$-subgrupo de $G$ de orden maximal.
\end{sol}
