\preface

Estas notas corresponden a un curso dictado el primer cuatrimestre de 2021 en 
el Departamento de Matemática de la FCEN, Universidad de Buenos Aires. 

La dos primeras partes --álgebras semisimples y representaciones de grupos-- 
constituye un curso sobre representaciones complejas 
de grupos finitos, principalmente teoría de caracteres. El material 
puede exponerse en detalle aproximadamente en unas treinta horas 
de clase, sin contar el capítulo \ref{Hurwitz}  
que es opcional. 

Los requisitos para seguir este curso son básicamente los
que se obtienen al cursar la materia Álgebra II del 
Departamento de Matemática de la FCEN. El material 
necesario puede encontrarse en cualquier libro de álgebra, 
por ejemplo \cite{MR600654}.
También en \url{https://github.com/vendramin/estructuras}.

Dependiendo
de los conocimientos de las personas que sigan el curso, 
algunos capítulos podrán omitirse. Ejemplos: 

\begin{itemize}
    \item En los primeros dos capítulos están dedicados a la teoría de álgebras semisimples de dimensión finita. Se incluye la demostración del teorema de Wedderburn y propiedades básicas del radical de Jacobson. 
    \item En el capítulo \ref{Maschke} se demuestra el teorema de Maschke. 
    \item En el capítulo \ref{Representaciones} hay varias páginas dedicadas a las nociones básicas sobre
    productos tensoriales de espacios vectoriales. 
\end{itemize}

%Estas notas pueden usarse como guía para distintos cursos posibles. 
%Ejemplos:  
%\begin{enumerate}
%    \item Curso 1 (aproximadamente 30 horas). 
%    Representaciones complejas de grupos finitos, capítulos 1--16. 
%    Posibles proyectos finales: Análisis de Fourier en grupos, 
%    el teorema de Suzuki sobre grupos con centralizadores abelianos, tabla de caracteres de
%    $\SL_2(q)$. 
%    \item Curso 2, dictado virtualmente en la FCEN durante primer cuatrimestre de 2021. 
%    Representaciones de grupos finitos: Capítulos 1--15 (aproximadamente 30 horas). 
%    Teoría clásica de grupos, todos los capítulos de la tercera parte. La cuarta parte
%    está formada por temas opcionales que no se dieron en clase y pueden utilizarse como
%    tópicos para proyectos finales. 
%\end{enumerate}

Quiero agradecer a todas las personas que leyeron las notas y me enviaron 
correcciones y comentarios. Agradecimientos especialmente van para
Carlos Miguel Soto, 
Martín Mereb, 
Santiago Varela
.

\medskip
Versión compilada el \today~a las~\currenttime.

\begin{flushright} 
Leandro Vendramin\\Buenos Aires, Argentina\par
\end{flushright}